\chapter{Overview and Motivation} \label{sec:overview}

% TODO: Remember to mentioned that the languages are also referred to as "source" and "target"

\section{Dependent Types}

As a simple but expressive foundation for dependent types,
I begin with the Generalized Calculus of Constructions (\GCC)\index{Generalized Calculus of Constructions},
originally proposed by \citet{GCC-Coquand},
proven strongly normalizing by \citet{GCC-Luo},
and extended to include definitions by \citet{universes}.
It has the following features:

\begin{itemize}
  \item \textbf{Dependent function types}, as is standard in the Calculus of Constructions (CC)~\citep{CoC}\index{Calculus of Constructions};
  \item \textbf{Definitions}, \ie locally-named expressions;
  \item \textbf{Universes \ala Russell}\index{universes \ala Russell}, where the types of types are themselves terms
    (as opposed to universes \ala Tarski\index{universes \ala Tarski}, where their \emph{encodings} are terms);
  \item An \textbf{impredicative universe}\index{impredicativity} $\Prop$ such that function types into types in $\Prop$
    are themselves in $\Prop$;
  \item A \textbf{cumulative hierarchy of universes}\index{cumulativity} such that $\Type{i}: \Type{i+1}$,
    any term in $\Type{i}$ is also in $\Type{j}$ given \emph{universe levels}\index{universe level} $i \leq j$,
    and there is a subtyping relation on types induced by this inclusion; and
  \item \textbf{Untyped conversion}\index{conversion} stating when two terms are judgementally equal to one another.
\end{itemize}

These features cover many modern proof assistants.
To name a few, in terms of universes,
Coq and Arend have all of the above;
Lean lacks cumulativity; and
Agda and \Fstar lack cumulativity and impredicative universe.
On the other hand, these proof assistants all have some form of
\emph{universe level polymorphism},
but this is much more complex and largely orthogonal to sized types
and the syntactic model.

Perhaps the most contentious design decision so far is the use of untyped conversion,
as can be found in Coq, over typed conversion or \emph{equivalence}\index{equivalence}%
\footnote{From this point onwards, I will use \emph{conversion}\index{conversion}
to refer to the untyped judgemental equality,
and \emph{equivalence}\index{equivalence} to refer to the typed judgemental equality.},
as can be found in Agda.
Certain proponents%
\footnote{This would be a fun place to link to someone's tweets.}
will argue that equivalence is ``more correct'',
since one never deals with ill-typed terms,
but this means that the equivalence judgement depends on the typing judgement.
Meanwhile, since types can depend on terms,
typing itself depends on equivalence to check whether one type can be used in place of another.
As we'll see in \TODO, these mutually-defined judgements would greatly complicate the proofs,
so I settle for conversion instead.

On top of \GCC, I add two inductive definitions featured in Martin--L\"of type theory (MLTT)~\citep{mltt}\index{Martin--L\"of type theory}:
the \emph{Peano naturals}\index{naturals} and \emph{wellfounded trees}\index{wellfounded trees}, augmented with sizes.
As the simplest nontrivial inductive,
the naturals make it easy to demonstrate intuitive uses of sized inductives.
On the other hand, wellfounded trees are an example of \emph{generalized} inductives,
with recursive arguments that are functions that return wellfounded trees,
and can in fact encode all (nonmutual, nonnested) inductives---%
although not so much their proper induction principles~\citep{W-types}.
I don't add inductive types in general in their place
because the syntactic baggage that comes with handling the generalization
obscures the intuition behind sized inductive types and the syntactic model,
while it's easy to see how one \emph{could} go from the naturals and from wellfounded trees
to inductive types in general.

Finally, I add a standard \emph{homogeneous propositional equality type}\index{propositional equality},
which isn't particularly difficult to deal with
and will allow for some more interesting examples.

\section{Sized Types}

Sized types are introduced with explicit size quantification $\Funtype{\alpha}{\tau}$,
size abstraction $\Fun{\alpha}{e}$, and size application $\App{e}{s}$.
Sizes themselves consist of size variables as well as a \emph{base size}
and a \emph{size successor operator}.
This is the standard for size expressions in sized type systems.
% TODO: forward ref to related work
Some such as \Fhattimes~\citep{F-hat-times} and \CIChatl~\citep{CIC-hat-l}
add addition of size variables and multiplication by a natural, respectively,
increasing expressivity at the expense of complexity.
To keep things simple, I don't include these features and stick to successor sizes.
Note that size expressions are \emph{not} terms,
and their quantifications, abstractions, and applications
are syntactically distinct from those of terms,
similar to how, in nondependent polymorphic type systems,
types are distinct from terms.

Having explicit sizes differs from some sized type systems such as
\lambdahat~\citep{lambda-hat, lambda-hat-diss},
\Fhat~\citep{F-hat}, \CIChat~\citep{cic-hat},
\CIChatminus~\citep{cic-hat-minus-nat, cic-hat-minus},
\CChatomega~\citep{cc-hat-omega}, and \CIChatstar~\citep{CIC-hat-star} where,
extending the type polymorphism analogy,
there is only implicit \emph{rank-1} or
\emph{prenex}\index{polymorphism!prenex/rank-1 \textasciitilde} size quantification:
size quantifications never appear inside of a type,
and in fact all size abstractions and applications are fully inferred.
Explicit sizes, in contrast, let us express
\emph{higher-rank}\index{polymorphism!higher-rank \textasciitilde} size quantification,
which is also found in \Fhatomega~\citep{Abel-diss},
\Fcopomega~\citep{F-omega-cop},
and in a MLTT-style sized dependent type theory~\cite{NbE}.
Higher-rank sizes allow for more expressiveness:
for instance, supposing we have cons lists parametrized over some sized type $\tau$,
one could write a size-preserving mapping function over a list
that leaves the sizes untouched.
The type of such a function might be
$$\Funtype{\alpha}{\arr*{(\Funtype{\beta}{\arr*{\App{\tau}{\beta}}{\App{\tau}{\beta}}})}{\List{(\App{\tau}{\alpha})}}{\List{(\App{\tau}{\alpha})}}}.$$

Along with higher-rank sizes, I also include \emph{bounded} size quantification $\Funtype<{\alpha}{s}{\tau}$
and abstraction $\Fun<{\alpha}{s}{e}$.
An order on sizes is induced by these bound instantiations and the successor operator;
this order has nothing to do with subtyping,
and in particular we do \emph{not} have subtyping relations between
$\N{\alpha}$ and $\N{\sss{\alpha}}$, for instance.
% TODO: forward ref to where we define the cast between the two
Fixpoint expressions recur on smaller sizes according to the order,
summarized by the below typing rule.
%
\begin{mathpar}
\inferrule[]{
  \check{\Phi, \alpha; \Gamma, f: \Funtype<{\beta}{\alpha}{\subst{\tau}{\alpha}{\beta}}}{e}{\tau}
}{
  \infer{\Phi; \Gamma}{\fix{}{f}{\alpha}{\tau}{e}}{\Funtype{\alpha}{\tau}}
}
\end{mathpar}

Bounded sizes were originally introduced for MiniAgda~\citep{MiniAgda, flationary}
to avoid complex \emph{semi-continuity}\index{semi-continuity} or approximative \emph{polarity}\index{polarity}
requirements on fixpoints' types in the presence of an \emph{infinite size}\index{infinite size}
that is strictly larger than all sizes.
Although I don't include an infinite size,
this style of recursion is more elegant because it corresponds neatly to well-founded induction on sizes.

In summary, the sized type features I include are:

\begin{itemize}
  \item \textbf{Explicit size} quantification $\Funtype{\alpha}{\tau}$,
    abstraction $\Fun{\alpha}{e}$, and
    application $\App{e}{s}$;
  \item A \textbf{simple size grammar} with size variables $\alpha$, a base size $\circ$, and successors $\sss{s}$;
  \item \textbf{Higher-rank sizes}, \eg $\arr*{(\Funtype{\alpha}{\tau})}{\sigma}$;
  \item \textbf{Bounded size} quantification $\Funtype<{\alpha}{s}{\tau}$ and
  abstraction $\Fun<{\alpha}{s}{e}$.
\end{itemize}

Notably, these are all features found in Agda's implementation of sized types.
The only missing feature is the infinite size,
since its properties and its use are known to be inconsistent in Agda%
\footnote{See issue \#2820 of the Agda repo,
\textit{Equality is incompatible with sized types},
URL \url{https://github.com/agda/agda/issues/2820}.}.
% TODO: forward ref to discussion on the infinite size

\section{Syntactic Model}