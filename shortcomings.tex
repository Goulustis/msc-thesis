\chapter{Design Shortcomings}\label{sec:shortcomings}

With \lang and its syntactic model come shortcomings in the design decisions that make it work.
Most notable are the lack of an infinite size
and the fact that the type of naturals and well-founded trees live in a universe
higher than that in which their corresponding unsized types in, say, pCIC would live.
Another shortcoming is the lack of coinductive types,
which is a common use case of sized types in the Agda standard library.
In this chapter, I elaborate on these features and the problems that arise from them, if any.

\section{The Infinite Size}

In prior sized type systems, the infinite size $\infty$ is applied to sized inductive types
to represent a ``full'' inductive type encompassing that inductive at all sizes,
which essentially corresponds to the usual unsized inductive.
Whereas an inductive of some size $s$ can be thought of as the type of elements
with at most $s$ many layers of constructors,
the full inductive can be thought of as the type of elements with any number of layers of constructors.

The infinite size is characterized by its subsizing behaviour:
$\subsize*{s}{\infty}$ holds for \emph{any} $s$.
This includes its own successor, \ie $\subsize*{\sss{\infty}}{\infty}$,
leading to non--well-founded sequences of strictly ``decreasing'' sizes:
$\dots < \infty < \infty < \infty$.
Naturally, there's no way to model the infinite size as an element of $\SizeT$
given that I've shown that $\SizeT$s \emph{are} well founded.
If there were, then it'd be possible to prove an inconsistency.
Let $\inftyT$ be the translation of $\infty$,
and let $\inftyltinfty$ be the translation of $\subsize*{\sss{\infty}}{\infty}$.
\begin{align*}
&\LetT{\tg{{\neg}wf\inftyT}}{\arrT*{\app{\AccT}{\inftyT}}{\botT}}{\funT{\mathit{acc}}{\app{\AccT}{\inftyT}}{\matchT*{\mathit{acc}}{\app{\accT}{p} \RightarrowT \app{p}{\inftyT}{\inftyltinfty}}}} \\
&\LetT{\tg{false}}{\botT}{\app{\tg{{\neg}wf\inftyT}}{(\app{\accessible}{\inftyT})}}
\end{align*}

In set-theoretic models of sized type systems with an infinite size,
sizes are modelled as set-theoretic (transfinite) ordinals,
the infinite size isn't modelled as a single ordinal;
instead, for each use of the infinite size,
its set-theoretic interpretation is an ordinal that is ``large enough'' in that context.
For instance, the interpretation of the infinite size of $\N{\infty}$
is the first limit ordinal $\omega$.

This strategy doesn't adapt well to \lang with its size abstractions and syntactic model,
since it requires a non-local translation of sizes.
For instance, given the size application $\App{e}{\infty}$,
what $\infty$ translates to would hypothetically depend on what $e$ translates to,
and likely require further static analysis of $\compile{e}$ beyond a simple translation
over typing derivations.

Since the motivation for having the infinite size is specifically for representing full inductives,
one alternative could be to define the full inductive separately
and provide functions to and from the corresponding sized inductive.
The following does so for $\W*$.
\begin{align*}
& \data{\App{\app{\W*}{(\annot{A}{\Type{i}})}{(\annot{B}{\arr*{A}{\Type{i}}})}}{\infty}}{\Type{i+1}} \\
& \quad \annot{\constr{sup\infty}}{\arr{x}{A}{\arr*{(\arr*{\app{B}{x}}{\app{\App{\W*}{\infty}}{A}{B}})}{\App{\app{\W*}{A}{B}}{\infty}}}}
\end{align*}

Defining a function from $\W{x}{A}{B}{s}$ to $\W{x}{A}{B}{\infty}$ is trivial,
since we're discarding size information.
What about going from $\W{x}{A}{B}{\infty}$ to $\W{x}{A}{B}{s}$?
What should $s$ be?
The size algebra could be augmented to be able to represent transfinite ordinals
so that $s$ is again a size that is ``large enough'',
but we can hardly expect programmers to be able to manipulate ordinals,
and I conjecture that we would lose any hope of deciding $\subsize*{}{}$
without any user intervention.

The key insight is that what's important about an element of a full inductive
isn't its precise size and depth of constructors,
but merely that it has \emph{some} unknown size.
Another alternative to the infinite size, then, could be to represent a full inductive
as an existentially-quantified sized inductive,
\ie $\Pairtype{\alpha}{\N{\alpha}}$ and $\Pairtype{\alpha}{\W{x}{A}{B}{\alpha}}$.
We've already seen existential sizes in action: they're in the return types of $\qsort$ and $\msort$.

There is still a limitation similar to that in \cref{sec:examples:limitations}
when trying to represent the constructors of full inductives.
For $\Pairtype{\alpha}{\W{x}{A}{B}{\alpha}}$, we need a ``constructor'' of the following type.
\begin{align*}
\annot{\const{sup'}}{\arr{x}{A}{\arr*{(\arr*{B}{\Pairtype{\alpha}{\W{x}{A}{B}{\alpha}}})}{\Pairtype{\alpha}{\W{x}{A}{B}{\alpha}}}}}
\end{align*}
All we need is a function
$$\annot{\const{ac}}{\arr{x}{A}{\arr*{(\arr*{B}{\Pairtype{\alpha}{\W{x}{A}{B}{\alpha}}})}{(\Pairtype{\alpha}{\arr*{B}{\W{x}{A}{B}{\alpha}}})}}}$$
and we're good to go.
\begin{align*}
\app{\const{sup'}}{x}{f} =
\unpair*{\alpha}{f'}{\app{\const{ac}}{x}{f}}{\Pair{\sss{\alpha}}{\sup{x}{A}{B}{\sss{\alpha}}{\alpha}{x}{f'}}}
\end{align*}

Unfortunately, as the name might suggest,
$\const{ac}$ is an instance of the axiom of choice,
which for weak existentials\index{weak dependent pair} (whose elements we can't project out)
is nonconstructive,
so there's no hope of implementing $\const{ac}$.
However, if we take weak existentials as a primitive of \lang
rather than being defined as an encoding,
and model them by strong dependent pairs in \CICE,
then the translation of $\const{ac}$ \emph{can} be implemented as a function.
In other words, the syntactic model justifies the axiom $\const{ac}$ in the source.
The mechanization of $\const{ac}$ in Agda and Coq are given in
\cref{app:mechanization:agda:W} and \cref{app:mechanization:coq:W}, respectively.
