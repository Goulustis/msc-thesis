%% For double-blind review submission, w/o CCS and ACM Reference (max submission space)
\documentclass[acmsmall,review,anonymous]{acmart}\settopmatter{printfolios=true,printccs=false,printacmref=false}
%% For double-blind review submission, w/ CCS and ACM Reference
%\documentclass[acmsmall,review,anonymous]{acmart}\settopmatter{printfolios=true}
%% For single-blind review submission, w/o CCS and ACM Reference (max submission space)
%\documentclass[acmsmall,review]{acmart}\settopmatter{printfolios=true,printccs=false,printacmref=false}
%% For single-blind review submission, w/ CCS and ACM Reference
%\documentclass[acmsmall,review]{acmart}\settopmatter{printfolios=true}
%% For final camera-ready submission, w/ required CCS and ACM Reference
%\documentclass[acmsmall]{acmart}\settopmatter{}


%% Journal information
%% Supplied to authors by publisher for camera-ready submission;
%% use defaults for review submission.
\acmJournal{PACMPL}
\acmVolume{7}
\acmNumber{POPL}
\acmArticle{0}
\acmYear{2023}
\acmMonth{1}
\acmDOI{} % \acmDOI{10.1145/nnnnnnn.nnnnnnn}
\startPage{1}

%% Copyright information
%% Supplied to authors (based on authors' rights management selection;
%% see authors.acm.org) by publisher for camera-ready submission;
%% use 'none' for review submission.
\setcopyright{none}
%\setcopyright{acmcopyright}
%\setcopyright{acmlicensed}
%\setcopyright{rightsretained}

%% Bibliography style
\bibliographystyle{ACM-Reference-Format}
\citestyle{acmauthoryear}

%% Packages
\usepackage{style}

%% Title information
\title{Sized Dependent Types via Extensional Type Theory}

%% Author information
\author{Jonathan Chan}
\orcid{0000-0003-0830-3180}
\affiliation{
  %\position{}
  \department{Department of Computer Science}
  \institution{University of British Columbia}
  \streetaddress{2329 West Mall}
  \city{Vancouver}
  \state{British Columbia}
  \postcode{V6T 1Z4}
  \country{Canada}
}
\email{jcxz@cs.ubc.ca}
\author{William J. Bowman}
\orcid{0000-0002-6402-4840}
\affiliation{
  \position{Assistant Professor}
  \department{Department of Computer Science}
  \institution{University of British Columbia}
  \streetaddress{2329 West Mall}
  \city{Vancouver}
  \state{British Columbia}
  \postcode{V6T 1Z4}
  \country{Canada}
}
\email{wjb@williamjbowman.com}

%% Abstract
%% Note: \begin{abstract}...\end{abstract} environment must come
%% before \maketitle command
\begin{abstract}
Contemporary proof assistants such as Coq and Agda that are based on dependent type theories
must forbid nonterminating recursive functions to ensure logical consistency of the type theory
and decidability of type checking.
This is currently done via guard predicates that only allow functions
that are structurally recursive and that recur only on syntactically smaller arguments.
However, these guard predicates are sometimes too restrictive and reject functions
that aren't structurally recursive but are obviously terminating by inspection,
creating extra work for the programmer to convince the guard checker.

An alternative is to use sized types, a type-based termination checking method
where inductively-defined types are annotated with sizes.
Successful type checking guarantees that functions recur only on arguments whose types have smaller sizes,
rather than merely on arguments that syntactically appear smaller,
allowing more terminating functions to be accepted.
There exist many formal models of sized dependent type theories,
but none simultaneously feature higher-rank size quantification,
which allows for passing around size-preserving functions,
and bounded size quantification,
which eliminates the need for complex monotonicity checks required by prior sized type systems.

In this paper, we design a sized dependent type theory with higher-rank and bounded sizes (\lang),
and prove its logical consistency with a syntactic model:
by compiling \lang into the Extensional Calculus of Inductive Constructions (\CICE),
a variant of the type theory on which Coq is based,
and showing that this translation is type preserving,
the consistency of \lang follows from the consistency of \CICE.
\end{abstract}

%% 2012 ACM Computing Classification System (CSS) concepts
%% Generate at http://dl.acm.org/ccs/ccs.cfm
\begin{CCSXML}
<ccs2012>
   <concept>
       <concept_id>10003752.10003790.10011740</concept_id>
       <concept_desc>Theory of computation~Type theory</concept_desc>
       <concept_significance>500</concept_significance>
       </concept>
   <concept>
       <concept_id>10003752.10010124.10010125.10010130</concept_id>
       <concept_desc>Theory of computation~Type structures</concept_desc>
       <concept_significance>500</concept_significance>
       </concept>
 </ccs2012>
\end{CCSXML}

\ccsdesc[500]{Theory of computation~Type theory}
\ccsdesc[500]{Theory of computation~Type structures}
%% End of generated code

%% Keywords
%% comma separated list
\keywords{type theory, dependent types, sized types, consistency}

\begin{document}
\maketitle

\section{Introduction} \label{sec:introduction}

\TODO: Adapt introduction.

In the sections to follow, initial appearances in the text of new notation is \new{highlighted in grey}.
Furthermore, we use ellipses \new{$\seq$} as metanotation for denoting a repeated sequence of some syntactic construct,
overlines \new{$\vec{\phantom{I}}$} for sequences of variables or terms specifically (\eg $\vec{z}$, $\vec{p}$),
and \new{$\mt$} for denoting an empty sequence (particularly for environments).
Irrelevant constructs are omitted using an underscore \new{$\any$}.
Metafunctions are introduced in the prose as needed.

\section{Sized Dependent Types} \label{sec:sized-dep-types}

\newcommand{\FigSyntax}[1]{
  \begin{figure}[h]
  \centering
  \begin{align*}
  i, j, k, m, n &\Coloneqq \meta{\textsf{naturals}} &
  \Gamma &\Coloneqq \mt \mid \Gamma, \annot{x}{\tau} \mid \Gamma, \define{x}{e} &
  r, s &\Coloneqq \alpha \mid \sss{s} \mid \circ \\
  f, g, x, y, z &\Coloneqq \meta{\textsf{term variables}} &
  \Delta &\Coloneqq \mt \mid \Delta, \annot{x}{\tau} &
  U &\Coloneqq \Prop \mid \Type{i} \\
  \alpha, \beta &\Coloneqq \meta{\textsf{size variables}} &
  \Phi &\Coloneqq \mt \mid \Phi, \alpha \mid \Phi, \bound{\alpha}{s} \\
  e, p, P, \tau, \sigma &\Coloneqq \mathrlap{x \mid U \mid \letin{x}{\tau}{e}{e} \mid \eq{e}{\tau}{e} \mid \refl{e} \mid \J{P}{d}{p}} \\
  &\mid \mathrlap{\funtype{x}{\tau}{\tau} \mid \fun{x}{\tau}{e} \mid \app{e}{e} \mid \Funtype{\alpha}{\tau} \mid \Funtype<{\alpha}{s}{\tau} \mid \Fun{\alpha}{e} \mid \Fun<{\alpha}{s}{e} \mid \App{e}{s}} \\
  \end{align*}
  \caption{Syntax (base \lang)}
  \label{#1}
  \end{figure}
}

\newcommand{\FigRed}[1]{
  \begin{figure}[h]
  \centering
  \begin{mathpar}
  \fbox{$\red{\Phi; \Gamma}{e}{e}$} \qquad
  \fbox{$\red*{\Phi; \Gamma}{e}{e}$} \hfill \\
  \inferrule[]{
    (\annot{x}{\tau}) \in \Gamma \\\\
    (\define{x}{e}) \in \Gamma
  }{
    \red{\Phi; \Gamma}{x}{e}
  }
  \and \inferrule[]{~}{\red{\Phi; \Gamma}{\app{(\fun{x}{\tau}{e})}{e'}}{\subst{e}{x}{e'}}}
  \and \inferrule[]{~}{\red{\Phi; \Gamma}{\letin{x}{\tau}{e'}{e}}{\subst{e}{x}{e'}}}
  \and \inferrule[]{~}{\red{\Phi; \Gamma}{\App{(\Fun{\alpha}{e})}{s}}{\subst{e}{\alpha}{s}}}
  \and \inferrule[]{~}{\red{\Phi; \Gamma}{\App{(\Fun<{\alpha}{r}{e})}{s}}{\subst{e}{\alpha}{s}}}
  \and \inferrule[]{~}{\red{\Phi; \Gamma}{\J{}{d}{\refl{}}}{d}}
  \\\\
  \inferrule[\rlabel{$\rhd^*$-refl}{red*-refl}]{~}{\red*{\Phi; \Gamma}{e}{e}}
  \and
  \inferrule[\rlabel{$\rhd^*$-trans}{red*-trans}]{
    \red{\Phi; \Gamma}{e_1}{}{e_2} \\
    \red*{\Phi; \Gamma}{e_2}{e_3}
  }{
    \red*{\Phi; \Gamma}{e_1}{e_3}
  }
  \and
  \inferrule[\rlabel{$\rhd^*$-cong}{red*-cong}]{
    \text{For every $1 \leq i \leq n$:} \and
    \red*{\Phi'; \Gamma'}{e_i}{e'_i}
  }{
    \red*{\Phi; \Gamma}{\subst{e}{x_1, \seq, x_n}{e_1, \seq, e_n}}{\subst{e}{x_1, \seq, x_n}{e'_1, \seq, e'_n}}
  }
  \end{mathpar}
  \caption{Reduction rules (base \lang)}
  \label{#1}
  \end{figure}
}

\newcommand{\FigSubtype}[1]{
  \begin{figure}
  \centering
  \begin{mathpar}
  \fbox{$\subtype{\Phi; \Gamma}{\tau}{\tau}$} \qquad
  \fbox{$\acum{e}{e}$} \hfill \\
  \inferrule[$\preccurlyeq$-red]{
    \red*{\Phi; \Gamma}{\tau_1}{\sigma_1} \\
    \red*{\Phi; \Gamma}{\tau_2}{\sigma_2} \\
    \acum{\sigma_1}{\sigma_2}
  }{
    \subtype{\Phi; \Gamma}{\tau_1}{\tau_2}
  }
  \and
  \inferrule[\rlabel{$\sqsubseteq$-prop}{acum-prop}]{~}{\acum{\Prop{}}{\Type{i}}}
  \and
  \inferrule[\rlabel{$\sqsubseteq$-type}{acum-type}]{i \leq j}{\acum{\Type{i}}{\Type{j}}}
  \and
  \inferrule[\rlabel{$\sqsubseteq$-refl}{acum-refl}]{
  }{
      \acum{e}{e}
  }
  \and
  \inferrule[\rlabel{$\sqsubseteq$-pi}{acum-pi}]{
    \acum{\tau_1}{\tau_2}
  }{
    \acum{\fun{x}{\sigma}{\tau_1}}{\fun{x}{\sigma}{\tau_2}}
  }
  \and
  \inferrule[\rlabel{$\sqsubseteq$-forall}{acum-forall}]{
    \acum{\tau_1}{\tau_2}
  }{
    \acum{\Funtype{\alpha}{\tau_1}}{\Funtype{\alpha}{\tau_2}}
  }
  \and
  \inferrule[\rlabel{$\sqsubseteq$-forall$<$}{acum-forall<}]{
    \acum{\tau_1}{\tau_2}
  }{
    \acum{\Funtype<{\alpha}{s}{\tau_1}}{\Funtype<{\alpha}{s}{\tau_2}}
  }
  \end{mathpar}
  \caption{Subtyping and $\alpha$-cumulativity rules}
  \label{#1}
  \end{figure}
}

\newcommand{\FigSubsize}[1]{
  \begin{figure}
  \centering
  \begin{mathpar}
  \fbox{$\wf{\Phi}{s}$} \qquad
  \fbox{$\subsize{\Phi}{s}{s}$} \hfill \\
  \inferrule[]{
    \wf{}{\Phi} \\
    \alpha \in \Phi
    \textit{ or }
    (\bound{\alpha}{s}) \in \Phi
  }{
    \wf{\Phi}{\alpha}
  }
  \and
  \inferrule[]{\wf{}{\Phi}}{\wf{\Phi}{\circ}}
  \and
  \inferrule[]{
    \wf{\Phi}{s}
  }{
    \wf{\Phi}{\sss{s}}
  }
  \and
  \inferrule[]{
    \wf{}{\Phi} \\
    (\bound{\alpha}{s}) \in \Phi
  }{
    \subsize{\Phi}{\sss{\alpha}}{s}
  }
  \\\\
  \inferrule[]{
    \wf{\Phi}{s}
  }{
    \subsize{\Phi}{\circ}{s}
  }
  \and
  \inferrule[]{
    \wf{\Phi}{s}
  }{
    \subsize{\Phi}{s}{s}
  }
  \and
  \inferrule[]{
    \wf{\Phi}{s}
  }{
    \subsize{\Phi}{s}{\sss{s}}
  }
  \and
  \inferrule[]{
    \subsize{\Phi}{r}{s}
  }{
    \subsize{\Phi}{\sss{r}}{\sss{s}}
  }
  \and
  \inferrule[]{
    \subsize{\Phi}{s_1}{s_2} \\\\
    \subsize{\Phi}{s_2}{s_3}
  }{
    \subsize{\Phi}{s_1}{s_3}
  }
  % \and \inferrule[]{~}{\subsize{\Gamma}{s}{\infty}}
  \end{mathpar}
  \caption{Size and subsizing rules}
  \label{#1}
  \end{figure}
}

\newcommand{\FigWF}[1]{
  \begin{figure}
  \centering
  \begin{mathpar}
  \fbox{$\wf{}{\Phi}$} \qquad
  \fbox{$\wf{\Phi}{\Gamma}$} \hfill \\
  \inferrule[\rlabel{nil}{nil}]{~}{\wf{}{\mt}}
  \and
  \inferrule[\rlabel{cons-size}{cons-size}]{
    \wf{}{\Phi}
  }{
    \wf{}{\Phi, \alpha}
  }
  \and
  \inferrule[\rlabel{cons-size$<$}{cons-size<}]{
    \wf{}{\Phi} \\\\
    \wf{\Phi}{s}
  }{
    \wf{}{\Phi, \bound{\alpha}{s}}
  }
  \and
  \inferrule[nil]{\wf{}{\Phi}}{\wf{\Phi}{\mt}}
  \and
  \inferrule[\rlabel{cons-ass}{cons-ass}]{
    \wf{\Phi}{\Gamma} \\\\
    \infer{\Phi; \Gamma}{\tau}{U}
  }{
    \wf{\Phi}{\Gamma, \annot{x}{\tau}}
  }
  \and
  \inferrule[\rlabel{cons-def}{cons-def}]{
    \wf{\Phi}{\Gamma} \\\\
    \infer{\Phi; \Gamma}{e}{\tau}
  }{
    \wf{\Phi}{\Gamma, \define{x}{e}}
  }
  \end{mathpar}
  \caption{Wellformedness rules}
  \label{#1}
  \end{figure}
}
\newcommand{\FigSyntaxInd}[1]{
  \begin{figure}[h]
  \centering
  \begin{align*}
  c &\Coloneqq \zero* \mid \succ* \mid \sup* &&&&& \\
  %t &\Coloneqq e \mid \mathopen{[} s \mathclose{]} &
  %\Theta &\Coloneqq \mt \mid \Theta, \annot{x}{\tau} \mid \Theta, \alpha \mid \Theta, \bound{\alpha}{s} & \\
  e &\Coloneqq \mathrlap{\cdots \mid \N{s} \mid \zero{s}{s} \mid \succ{s}{s}{e} \mid \W{x}{\tau}{\tau}{s} \mid \sup{x}{\tau}{\tau}{s}{s}{e}{e}} \\
  &\mid \mathrlap{\match{e}{\fun*{x}{P}}{(\app{\App{c}{\alpha}}{z_1 \seq z_m} \Rightarrow e) \seq} \mid \fix{}{f}{\alpha}{\tau}{e}}
  \end{align*}
  \caption{Syntax (naturals, well-founded trees, \\ and $\kw{case}$ and fixpoint expressions)}
  \label{#1}
  \end{figure}
}

\newcommand{\FigRedInd}[1]{
  \begin{figure}[h]
  \centering
  \begin{mathpar}
  \vspace{-2ex}
  \fbox{$\red{\Phi; \Gamma}{e}{e}$} \quad \cdots \hfill \\
  \inferrule[]{~}{
    \red[]{\Phi; \Gamma}{\match*{\App{\zero*_{\any}}{s}}{(\App{\zero*}{\alpha} \Rightarrow e_z) (\app{\App{\succ*}{\any}}{\any} \Rightarrow \any)}}{\subst{e_z}{\alpha}{s}}
  }
  \and
  \inferrule[]{~}{
    \red[]{\Phi; \Gamma}{\match*{\app{\App{\succ*_{\any}}{s}}{e}}{(\App{\zero*}{\any} \Rightarrow \any) (\app{\App{\succ*}{\alpha}}{z} \Rightarrow e_s)}}{\subst{e_s}{\alpha, z}{s, e}}
  }
  \and
  \inferrule[]{~}{
    \red[]{\Phi; \Gamma}{\match*{\app{\App{\sup*_{\any}}{s}}{e_1}{e_2}}{(\app{\App{\sup*}{\alpha}}{z_1}{z_2} \Rightarrow e)}}{\subst{e}{\alpha, z_1, z_2}{s, e_1, e_2}}
  }
  \and
  \inferrule[]{
    \subsize{\Phi}{\sss{r}}{s}
  }{
    \red[]{\Phi; \Gamma}{
      \App{(\fix{}{f}{\alpha}{\sigma}{e})}{s}
    }{
      \subst{e}{\alpha, f}{s, \Fun<{\beta}{s}{\App{(\fix{}{f}{\alpha}{\sigma}{e})}{\beta}}}
    }
  }
  \end{mathpar}
  \caption{Reduction rules ($\kw{case}$ and fixpoint expressions)}
  \label{#1}
  \end{figure}
}

\newcommand{\FigTypingInd}[1]{
  \begin{figure}[h]
  \begin{mathpar}
  \fbox{$\type{\Phi; \Gamma}{e}{\tau}$} \quad \cdots \hfill \\
  \inferrule[\rlabel*{nat}]{
    \wf{\Phi}{\Gamma} \\
    \wf{\Phi}{s}
  }{
    \infer{\Phi; \Gamma}{\N{s}}{\Type{1}}
  }
  \and
  \inferrule[\rlabel*{zero}]{
    \wf{\Phi}{\Gamma} \\
    \subsize{\Phi}{\hat{r}}{s}
  }{
    \infer{\Phi; \Gamma}{\zero{s}{r}}{\N{s}}
  }
  \and
  \inferrule[\rlabel*{succ}]{
    \subsize{\Phi}{\hat{r}}{s} \\
    \check{\Phi; \Gamma}{e}{\N{r}}
  }{
    \infer{\Phi; \Gamma}{\succ{s}{r}{e}}{\N{s}}
  }
  \and
  \inferrule[\rlabel*{wft}]{
    \wf{\Phi}{s} \\
    \infer{\Phi; \Gamma}{\sigma}{U} \\\\
    \infer{\Phi; \Gamma, \annot{x}{\sigma}}{\tau}{U}
  }{
    \infer{\Phi; \Gamma}{\W{x}{\sigma}{\tau}{s}}{\axioms{U}}
  }
  \and
  \inferrule[\rlabel*{sup}]{
    \subsize{\Phi}{\hat{r}}{s} \\
    \check{\Phi; \Gamma}{e_1}{\sigma} \\\\
    \check{\Phi; \Gamma}{e_2}{\arr*{\subst{\tau}{x}{e_1}}{\W{x}{\sigma}{\tau}{r}}}
  }{
    \infer{\Phi; \Gamma}{\sup{x}{\sigma}{\tau}{s}{r}{e_1}{e_2}}{\W{x}{\sigma}{\tau}{s}}
  }
  \and
  \inferrule[\rlabel*{case-nat}]{
    \infer{\Phi; \Gamma}{e}{\N{s}} \\
    z \notin \FV{P} \\
    \infer{\Phi; \Gamma, \annot{x}{\N{s}}}{P}{U} \\\\
    \check{\Phi, \bound{\alpha}{s}; \Gamma}{e_z}{\subst{P}{x}{\zero{s}{\alpha}}} \\
    \check{\Phi, \bound{\beta}{s}; \Gamma, \annot{z}{\N{\beta}}}{e_s}{\subst{P}{x}{\succ{s}{\beta}{z}}}
  }{
    \infer{\Phi; \Gamma}{\match{e}{\fun*{x}{P}}{(\App{\zero*}{\alpha} \Rightarrow e_z)(\app{\App{\succ*}{\beta}}{z} \Rightarrow e_s)}}{\subst{P}{x}{e}}
  }
  \and
  \inferrule[\rlabel*{case-sup}]{
    \infer{\Phi; \Gamma}{e}{\W{y}{\sigma}{\tau}{s}} \\
    z_1, z_2 \notin \FV{P} \\
    \infer{\Phi; \Gamma, \annot{x}{\W{y}{\sigma}{\tau}{s}}}{P}{U} \\
    \check{\Phi, \bound{\alpha}{s}; \Gamma, \annot{z_1}{\sigma}, \annot{z_2}{\arr*{\subst{\tau}{y}{z_1}}{\W{y}{\tau}{\sigma}{\alpha}}}}{e_s}{\subst{P}{x}{\sup{y}{\sigma}{\tau}{s}{\alpha}{z_1}{z_2}}}
  }{
    \infer{\Phi; \Gamma}{\match{e}{\fun*{x}{P}}{(\app{\App{\sup*}{\alpha}}{z_1}{z_2} \Rightarrow e_s)}}{\subst{P}{x}{e}}
  }
  \and
  \inferrule[\rlabel*{fix}]{
    \infer{\Phi, \alpha; \Gamma}{\sigma}{U} \\
    \fresh{\beta} \\
    \check{\Phi, \alpha; \Gamma, \annot{f}{\Funtype<{\beta}{\alpha}{\subst{\sigma}{\alpha}{\beta}}}}{e}{\sigma}
  }{
    \infer{\Phi; \Gamma}{\fix{}{f}{\alpha}{\sigma}{e}}{\Funtype{\alpha}{\sigma}}
  }
  \end{mathpar}
  \caption{Typing rules (naturals, well-founded trees, \\ and $\kw{case}$ and fixpoint expressions)}
  \label{#1}
  \end{figure}
}

Although sized types are quite pointless without any inductive types to be sized,
we present first the sublanguage of \lang without any inductives
to not only get the preliminary details out of the way first,
but also to show that the sublanguage is independent of the inductives chosen.
After that, naturals and well-founded trees are added as representative inductive types,
followed by examples of programming with sized dependent types.

\subsection{Base \lang}

\FigSyntax{fig:syntax}
\cref{fig:syntax} gives its syntax, which consists of universes $U$,
sizes $s$, and terms $e$, which includes term functions and size abstractions.
Note that the hierarchy of universes above $\Prop$ start at $\Type{1}$, not $\Type{0}$.
Most judgements use two environments: a term environment $\Gamma$ with assumptions $\annot{x}{\tau}$
and type-annotated definitions $\define{x}{\tau}{e}$,
and a size environment $\Phi$ with unbounded and bounded size variables.
We also use the assumption environment%
$\Delta$ as a shorthand when writing nested expressions with assumptions;
in particular, letting for instance $\Delta_{xy} = \annot{x}{\sigma_1}, \annot{y}{\sigma_2}$,
We use \new{$\arr*{\Delta_{xy}}{\tau}$} to mean $\funtype{x}{\sigma_1}{\funtype{y}{\sigma_2}{\tau}}$.
Similarly, we use \new{$\arr*{\sigma}{\tau}$} to mean the nondependent function type $\funtype{\any}{\sigma}{\tau}$
As a loose convention, we use $\tau, \sigma$ for type-like terms,
$P$ for the motive of eliminators,
$z$ for variables representing constructor arguments, and
$f, g$ for variables representing functions.

As mentioned in \cref{sec:introduction}, the judgement forms of \lang include
reduction, its various closures, $\alpha$-cumulativity, subtyping, and typing.
On top of those, there are also judgement forms for subsizing, sizes,
and well-formedness of size and term environments.

\begin{figure}[h]
\centering
\begin{mathpar}
\fbox{$\red{\Phi; \Gamma}{e}{e}$} \qquad
\fbox{$\red*{\Phi; \Gamma}{e}{e}$} \hfill
\inferrule[]{
  (\define{x}{\tau}{e}) \in \Gamma
}{
  \red{\Phi; \Gamma}{x}{e}
}
\and \inferrule[]{~}{\red{\Phi; \Gamma}{\app{(\fun{x}{\tau}{e})}{e'}}{\subst{e}{x}{e'}}}
%\and \inferrule[]{~}{\red{\Phi; \Gamma}{\J{}{d}{\refl{}}}{d}}
\and \inferrule[]{~}{\red{\Phi; \Gamma}{\App{(\Fun{\alpha}{e})}{s}}{\subst{e}{\alpha}{s}}}
\and \inferrule[]{~}{\red{\Phi; \Gamma}{\App{(\Fun<{\alpha}{r}{e})}{s}}{\subst{e}{\alpha}{s}}}
\and \inferrule[]{~}{\red{\Phi; \Gamma}{\letin{x}{\tau}{e'}{e}}{\subst{e}{x}{e'}}}
%\and \inferrule[]{~}{\red{\Phi; \Gamma}{\unpair{\alpha}{x}{\Pair{s}{e'}}{e}}{\subst{e}{\alpha, x}{s, e'}}}
\and \inferrule[\rlabel{$\rhd$-cong}{red-cong}]{
  \red{\Phi'; \Gamma'}{e'}{e''}
}{
  \red{\Phi; \Gamma}{\subst{e}{x}{e'}}{\subst{e}{x}{e''}}
}
\\\\
\inferrule[\rlabel{$\rhd^*$-once}{red*-once}]{
  \red{\Phi; \Gamma}{e_1}{e_2}
}{
  \red*{\Phi; \Gamma}{e_1}{e_2}
}
\and
\inferrule[\rlabel{$\rhd^*$-refl}{red*-refl}]{~}{\red*{\Phi; \Gamma}{e}{e}}
\and
\inferrule[\rlabel{$\rhd^*$-trans}{red*-trans}]{
  \red*{\Phi; \Gamma}{e_1}{e_2} \\
  \red*{\Phi; \Gamma}{e_2}{e_3}
}{
  \red*{\Phi; \Gamma}{e_1}{e_3}
}
\end{mathpar}
\caption{Reduction rules (base \lang)}
\label{fig:reduction}
\end{figure}

The reduction rules and their reflexive, transitive closure are described in \cref{fig:reduction}.
\new{$\subst{e}{x}{e'}$} denotes capture-avoiding substitution of $x$ for $e'$ within $e$,
and correspondingly \new{$\subst{e}{x_1, \seq, x_n}{e_1, \seq, e_n}$} denotes simultaneous substitution.
For every syntactic form of a composite term there is a corresponding congruence rule,
which is summarized by the last reduction rule using substitution;
the full set of rules can be found in \cref{app:cong:red}.
By convention, the reduction rules for function applications are also referred to as $\beta$-reduction,
for $\kw{let}$ expressions as $\zeta$-reduction,
and for defined variables as $\delta$-reduction.

\FigSubtype{fig:subtyping}
Rather than a single subtyping judgement like in
\GCC,
We use the same presentation as MetaCoq
and split it into a subtyping judgement
and a separate $\alpha$-cumulativity judgement,
listed in \cref{fig:subtyping}.
This overcomes some technical proof complications that appear in
the single-judgement presentation due to the transitivity rule.
Aside from the expected cumulativity of universes,
the function type and size quantification are covariant in the codomain
(\ie are $\alpha$-cumulative when their codomains are),
while remaining invariant in the domain.
All other types are invariant, as reflected by \rref{acum-refl}.
A term is then a subtype of another if they are confluent up to $\alpha$-cumulativity.

Notably, \lang does \emph{not} have a notion of $\eta$-conversion
in either the reduction rules or in the subtyping rules,
which would otherwise allow conversion between $\fun{x}{\tau}{\app{f}{x}}$ and $f$.
Mixing $\eta$-conversion and untyped conversion is notoriously difficult~\citep{eta},
and remains to date an unresolved problem in MetaCoq, so we exclude it here.

\FigSubsize{fig:subsizing}

\cref{fig:subsizing} describes a preorder on sizes such that
the successor operator is monotonic with respect to the order.
The base size $\circ$ is smaller than all sizes,
and the strict preorder $\bound{\alpha}{s}$ arising from bounded quantification or abstraction
is defined as $\sss{\alpha} \mathrel{\leqslant} s$.
An additional size judgement ensures well-scopedness of sizes.
The size environment must be well formed as well;
its rules are listed in \cref{fig:wf},
along with those for well-formedness of term environments.

\FigWF{fig:wf}

\FigTyping[h]{fig:typing}

Finally, the typing rules for the base \lang are given in \cref{fig:typing}.
They use two metafunctions defined by
$\axioms{\Prop} = \Type{1}$ and $\axioms{\Type{i}} = \Type{i+1}$;
and by $\rules{U}{\Prop} = \Prop$, $\rules{\Prop}{U} = U$, and $\rules{\Type{i}}{\Type{j}} = \Type{\maximum{i, j}}$.

\rref{var, univ, let} are the usual rules for variables, universes, and $\kw{let}$ expressions in \GCC,
while \rref{pi, lam, app} are the usual ones for functions.
\rref{conv} uses the subtyping judgement and essentially allows casting a term
from one type to a supertype as needed.
\rref{forall, forall<, slam, slam<, sapp, sapp<} are the new rules relevant to sized types,
describing bound and unbound size quantification, abstraction, and application,
which work similarly to functions.
Of note is the bounded size application rule,
which only allows applications to smaller sizes following the subsizing judgement.

\subsection{Inductive Types: Naturals and Well-Founded Trees} \label{subsec:ind-types}

\FigSyntaxInd{fig:syntax-ind}
\cref{fig:syntax-ind} extends the grammar with sized naturals, sized well-founded trees,
$\kw{case}$ expressions, and fixpoint expressions.
Informally, borrowing syntax from the definition of general inductives,
sized naturals and well-founded trees can be thought of as being defined by the following:
%
\begin{align*}
&\data{\N{\alpha}}{\Type{1}} && \data{\App{\app{\W*}{(\annot{A}{\Type{i}})}{(\annot{B}{\arr*{A}{\Type{i}}})}}{\alpha}}{\Type{i+1}} \\
&\quad \annot{\zero*}{\Funtype<{\beta}{\alpha}{\N{\alpha}}} && \quad \annot{\sup*}{\Funtype<{\beta}{\alpha}{\arr{x}{A}{\arr*{(\arr*{\app{B}{x}}{\app{\App{\W*}{\beta}}{A}{B}})}{\App{\app{\W*}{A}{B}}{\alpha}}}}} \\
&\quad \annot{\succ*}{\Funtype<{\beta}{\alpha}{\arr*{\N{\beta}}{\N{\alpha}}}}
\end{align*}

The types of naturals and well-founded trees can then be considered to be (nonuniformly) parametrized by a size,
and constructing an element of that type requires providing a strictly smaller size,
which is the size%
\footnote{The ``size of'' some construction is more precisely the size by which its \emph{inductive type} is parametrized.}
of the constructor's recursive arguments.
Constructors therefore always construct elements whose sizes are larger than their arguments'.
In \lang, the constructors are annotated with their types
since the parameter-like sizes cannot otherwise be synthesized.
Additionally, W types have explicit binders for convenience:
the variable $x$ is bound within $\tau$ in the type $\W{x}{\sigma}{\tau}{s}$.

The expression
$\match{e}{\fun*{x}{P}}{(\app{\App{c}{\alpha}}{z_1}{\seq}{z_m} \Rightarrow e_c) \seq}$
contains three parts:
the \emph{target} $e$ it destructs,
the \emph{motive} $P$ denoting the return type of the expression abstracted over a target,
and the \emph{branches} $e_c$, one for each constructor of the target's type,
abstracted over the constructor's size and term arguments.
Its reduction rules for each constructor are given in \cref{fig:reduction-ind},
along with the reduction rule for fixpoint expressions.
By convention, the reduction rules for $\kw{case}$ expressions
are also referred to as $\iota$-reduction,
and for fixpoint expressions as $\mu$-reduction.
\FigRedInd{fig:reduction-ind}

Fixpoints reduce when applied to some size $s$ by substitution of itself into its own body.
Fixpoints' bodies are well typed when recursive applications occur only on smaller sizes,
so the substitution wraps itself in a bound size abstraction.
Most importantly, they reduce only when there exists some size strictly smaller than $s$;
intuitively, this restriction prevents fixpoints from reducing indefinitely
because subsizing is well founded, and there cannot be an infinite chain of smaller sizes.

This reduction strategy supersedes the usual restriction that fixpoints only reduce
when applied to a constructor,
since all sized constructors carry a smaller size argument
that will statisfy the subsizing premise.
Furthermore, reduction can also occur when the fixpoint is applied to a successor size
by reflexivity of subsizing,
making the rule a strict supersession.

\FigTypingInd{fig:typing-ind}

The typing rules for all new constructs are given in \cref{fig:typing-ind}.
There are two additional metafunctions:
$\FV{\mt}$, which produces the free variables in the given term,
and $\fresh{\seq}$, which asserts the freshness of the given variables.
The type of naturals and well-founded trees are well typed
when the sizes they are applied to are well formed, and
their constructors are well typed when applied to smaller sizes.
$\kw{case}$ expressions match on these size arguments in addition to the usual term arguments,
asserting within their branches that they are strictly smaller than the target's size.
The motive is dependent on the target,
and the type of a branch is the motive with its target as the constructor being destructed.
As discussed, the body of a fixpoint is only well typed
when the fixpoint is recursively applied to a smaller size,
as enforced by its type in the environment when type checking the body.

Sizes aside, the only difference from regular, unsized naturals and well-founded trees
is that their types live in a universe one level higher than they usually are.
$\N{s}$ lives in $\TypeT{1}$ rather than in $\Type{0}$,
and $\W{x}{\sigma}{\tau}{s}$ lives in $\axioms{U}$ rather than in $U$.
This is a direct consequence of the way that the translation is defined,
and is necessary to maintain its type preservation properties.
While not incorrect, inductive types living in the ``wrong'' universe is aesthetically unpleasant
and removes some of the impredicativity of their parameters
by preventing them from quantifying over the inductive types themselves.
Potential methods of circumventing this undesirable trait to varying degrees of success
is discussed later in \cref{subsec:infinity}.

\subsection{Examples} \label{subsec:examples}

Now that the rules of \lang have been established,
this section presents examples of using \lang
for programming and proving.
Although it only has naturals and well-founded trees,
We also use other sized inductive types as examples,
informally defining them similarly to \cref{subsec:ind-types}.
Additionally, we omit the type annotation of $\kw{let}$-bound expressions
when the type is evident or deducible from context.

\subsubsection{Size-preserving functions}

One of the most important uses of sized types is the ability to define
\emph{size-preserving} functions,
where the sizes of the input and output are the same.
This guarantees that the output is never larger than the input,
and size-preserving functions can be used in recursive calls of fixpoints.
For instance, the predecessor function $\pred$ which computes
$\maximum{0, n - 1}$ for some number $n$ is size preserving.
%
\begin{align*}
&\Let{\pred}{\Funtype{\alpha}{\arr*{\N{\alpha}}{\N{\alpha}}}}{ \\
&\Fun{\alpha}{\fun{n}{\N{\alpha}}{\match*{n}{ \\
&\quad \App{\zero*}{\beta} \Rightarrow \zero{\alpha}{\beta} \\
&\quad \app{\App{\succ*}{\beta}}{m} \Rightarrow \app{\App{\liftN}{\alpha}{\beta}}{m}}}}}
\end{align*}

The $\monus$ function, which computes $\maximum{0, n - m}$ given numbers $n, m$,
is similarly size preserving in its first argument,
since $n - m$ is never greater than $n$.
%
\begin{align*}
&\Let{\monus}{\Funtype{\alpha}{\Funtype{\beta}{\arr*{\N{\beta}}{\N{\alpha}}{\N{\beta}}}}}{ \\
&\fix{\monus*}{\alpha}{\Funtype{\beta}{\arr*{\N{\beta}}{\N{\alpha}}{\N{\beta}}}}{ \\
&\quad \Fun{\beta}{\fun{n}{\N{\beta}}{\fun{m}{\N{\alpha}}{\match*{m}{ \\
&\qquad \App{\zero*}{\gamma} \Rightarrow n \\
&\qquad \app{\App{\succ*}{\gamma}}{k} \Rightarrow \app{\App{\monus*}{\gamma}{\beta}}{(\app{\App{\pred}{\beta}}{n})}{k}}}}}}}
\end{align*}

We see the benefit of size preservation with $\divv$,
which computes Euclidean division of $n$ by $m$, or $\left\lceil\frac{n}{m+1}\right\rceil$.
This is computed recursively by subtracting $m$ from the numerator using $\monus$
until $\zero*$ is reached, and counting the number of times the subtraction is performed.
The recursive call is done on the result of $\monus$;
$\divv$ then only type checks because $\monus$ is size preserving.
Because the first argument of the recursive call to $\divv$ isn't \emph{structurally}
a subterm with the call to $\monus$ in the way,
a guardedness check wouldn't accept the corresponding unsized function,
as is the case for CIC.
%
\begin{align*}
&\Let{\divv}{\Funtype{\alpha}{\Funtype{\beta}{\arr*{\N{\alpha}}{\N{\beta}}{\N{\alpha}}}}}{ \\
&\fix{\divv*}{\alpha}{\Funtype{\beta}{\arr*{\N{\alpha}}{\N{\beta}}{\N{\alpha}}}}{ \\
&\quad \Fun{\beta}{\fun{n}{\N{\alpha}}{\fun{m}{\N{\beta}}{\match*{n}{ \\
&\qquad \App{\zero*}{\gamma} \Rightarrow \zero{\alpha}{\gamma} \\
&\qquad \app{\App{\succ*}{\gamma}}{k} \Rightarrow
\succ{\alpha}{\gamma}{
  (\app{\App{\divv*}{\gamma}{\beta}}{
    (\app{\App{\monus}{\beta}{\gamma}}{k}{
      (\app{\App{\pred}{\beta}}{m})})}{m})}}}}}}}
\end{align*}

\subsubsection{Higher-rank sizes}

This example, which demonstrates higher-rank size quantification,
is a traversal of a well-founded tree with a size-preserving function,
adapted from the rose tree traversal example by \citet{NbE}.
Given some size-preserving function on a well-founded tree,
the function is applied to subtrees prior to traversal.
Suppose that $B$ here is some type operator on $A$.
%
\begin{align*}
&\Let{\traverseW}{\arr*{(\Funtype{\gamma}{\arr*{\W{x}{A}{\app{B}{x}}{\gamma}}{\W{x}{A}{\app{B}{x}}{\gamma}}})}{ \\
&\phantom{\kw{let} \: \traverseW \mathrel{:} \phantom{}} \Funtype{\alpha}{\arr*{\W{x}{A}{\app{B}{x}}{\alpha}}{\W{x}{A}{\app{B}{x}}{\alpha}}}}}{ \\
&\quad \fun{f}{\Funtype{\gamma}{\arr*{\W{x}{A}{\app{B}{x}}{\gamma}}{\W{x}{A}{\app{B}{x}}{\gamma}}}}{ \\
&\quad \fix{\traverse}{\alpha}{\arr*{\W{x}{A}{\app{B}{x}}{\alpha}}{\W{x}{A}{\app{B}{x}}{\alpha}}}{ \\
&\qquad \fun{w}{\W{x}{A}{\app{B}{x}}{\alpha}}{\match*{w}{ \\
&\qquad \quad \app{\App{\sup*}{\beta}}{a}{g} \Rightarrow \sup{x}{A}{\app{B}{x}}{\alpha}{\beta}{a}{(\fun{b}{\app{B}{a}}{\app{\App{\traverse}{\beta}}{(\app{\App{f}{\beta}}{(\app{g}{b})})}})}}}}}}
\end{align*}

\subsubsection{Limitations} \label{subsec:examples:limitations}

As expressive as bounded, higher-rank sized types are,
there still exist limitations to what can be expressed in comparison to ordinary inductive types
or to sized type theories which have an infinite size,
such as the ones listed in \TODO.
Limitations typically arise when dealing with inductive definitions where
recursive arguments appear as the return type of a function.
Consider the following inductive definition
representing the Brouwer notation for ordinals (see \eg \citet{ordinals})
with zero, successor, and limit ordinals:
%
\begin{align*}
&\data{\Ord{\alpha}}{\Type{1}} \\
&\quad \annot{\zord*}{\Funtype<{\beta}{\alpha}{\Ord{\alpha}}} \\
&\quad \annot{\sord*}{\Funtype<{\beta}{\alpha}{\arr*{\Ord{\beta}}{\Ord{\alpha}}}} \\
&\quad \annot{\lord*}{\Funtype<{\beta}{\alpha}{\arr*{(\Funtype<{\gamma}{\beta}{\arr*{\N{\gamma}}{\Ord{\beta}}})}{\Ord{\alpha}}}}
\end{align*}

The limit ordinal, taking some function on naturals returning an ordinal,
constructs an ordinal meant to be ``larger'' than any of the returned ordinals.
Such a function should be able to return an ordinal far larger than any natural,
hence the bounded size quantification in its domain.
Conversely, the naturals embed quite naturally within the ordinals.
%
\begin{align*}
&\Let{\natOrd}{\Funtype{\alpha}{\arr*{\N{\alpha}}{\Ord{\alpha}}}}{ \\
&\fix{\natOrd*}{\alpha}{\arr*{\N{\alpha}}{\Ord{\alpha}}}{ \\
&\quad \fun{n}{\N{\alpha}}{\match*{n}{ \\
&\qquad \App{\zero*}{\beta} \Rightarrow \zord{\alpha}{\beta} \\
&\qquad \app{\App{\succ*}{\beta}}{m} \Rightarrow \sord{\alpha}{\beta}{(\app{\App{\natOrd*}{\beta}}{m})}}}}}
\end{align*}

In an unsized type theory, supposing that $\const{natOrd'}$ is an unsized version of $\natOrd$,
the first limit ordinal $\const{\omega'}$ is easily defined as
$\Let{\const{\omega'}}{\const{Ord}}{\app{\lord*}{\const{natOrd'}}}$.
However, in \lang, things are not so easy; the na\"ive attempt yields the following:
%
\begin{align*}
&\Let{\liftO}{\Funtype{\alpha}{\Funtype<{\beta}{\alpha}{\arr*{\Ord{\beta}}{\Ord{\alpha}}}}}{\seq} \\
&\Let{\omegaOrd}{\Ord{\sss{\hole}}}{\lord{\sss{\hole}}{\hole}{(\Fun<{\gamma}{\hole}{\fun{n}{\N{\gamma}}{\app{\App{\liftO}{\hole}{\gamma}}{(\app{\App{\natOrd}{\gamma}}{n})}}})}}
\end{align*}

Aside from the additional bounds and size lifting, there is one crucial problem: what size fills in the hole \new{$\hole$}?
Intuitively, this size must be larger than the size of \emph{any} natural.
Therefore, a corresponding ``limit size'' is needed;
the infinite size in other sized type theories can fill this role since it's larger than all sizes
and therefore the limit of \emph{all} sizes.
We discuss more about the infinite size and its absence from \lang in \cref{subsec:infinity}.

\section{Syntactic Model of \lang}

\newcommand{\FigSyntaxCIC}[1]{
  \begin{figure}[h]
  \centering
  \begin{align*}
  \iT, \jT, \kT, \mT, \nT &\Coloneqq \meta{\textrm{naturals}} &
  \GammaT &\Coloneqq \mt \mid \GammaT, \annotT{\xT}{\tauT} \mid \GammaT, \define{\xT}{\tauT}{\eT} \\
  \fT, \gT, \wT, \xT, \yT, \zT, \alphaT, \betaT, \gammaT &\Coloneqq \meta{\textrm{term variables}} &
  \DeltaT &\Coloneqq \mt \mid \DeltaT, \annotT{\xT}{\tauT} \mid \DeltaT, \annotT{\cT}{\tauT} \\
  \XT &\Coloneqq \meta{\textrm{inductive type names}} &
  \UT &\Coloneqq \PropT \mid \TypeT{\iT} \\
  \cT &\Coloneqq \meta{\textrm{inductive constructor names}} &
  \DT &\Coloneqq \app{\dataT{\app{\XT}{\DeltaT}}{\arrT*{\DeltaT}{\UT}}}{\DeltaT} \\
  \eT, \aT, \dT, \pT, \PT, \rT, \sT, \tauT, \sigmaT &\Coloneqq
    \mathrlap{\xT \mid \UT \mid \funtypeT{\xT}{\tauT}{\tauT} \mid \funT{\xT}{\tauT}{\eT} \mid \app{\eT}{\eT} \mid \letinT{\xT}{\tauT}{\eT}{\eT} \mid \fixT{\iT}{\fT}{\tauT}{\eT}} \\
  &\mid \mathrlap{\eqT{\eT}{\tauT}{\eT} \mid \reflT{\eT} \mid \JT{\PT}{\dT}{\pT} \mid \matchT{\eT}{\funT*{\vec{\yT}}{\xT}{\PT}}{(\app{\cT}{\vec{\zT}} \RightarrowT \eT) \seq}}
  \end{align*}
  \caption{Syntax (\CICE)}
  \label{#1}
  \end{figure}
}

\newcommand{\FigEquiv}[1]{
  \begin{figure}[p]
  \centering
  \begin{mathpar}
  \fbox{$\defeq{\GammaT}{\eT}{\eT}{\tauT}$} \hfill \\
  \inferrule[\rlabel{$\equiv$-refl}{equiv-refl}]{
    \type{\GammaT}{\eT}{\tauT}
  }{
    \defeq{\GammaT}{\eT}{\eT}{\tauT}
  }
  \and
  \inferrule[\rlabel{$\equiv$-sym}{equiv-sym}]{
    \defeq{\GammaT}{\eT_2}{\eT_1}{\tauT}
  }{
    \defeq{\GammaT}{\eT_1}{\eT_2}{\tauT}
  }
  \and
  \inferrule[\rlabel{$\equiv$-trans}{equiv-trans}]{
    \defeq{\GammaT}{\eT_1}{\eT_2}{\tauT} \\\\
    \defeq{\GammaT}{\eT_2}{\eT_3}{\tauT}
  }{
    \defeq{\GammaT}{\eT_1}{\eT_3}{\tauT}
  }
  \and
  \inferrule[\rlabel{$\equiv$-conv}{equiv-conv}]{
    \subtype{\GammaT}{\sigmaT}{\tauT} \\\\
    \defeq{\GammaT}{\eT_1}{\eT_2}{\sigmaT}
  }{
    \defeq{\GammaT}{\eT_1}{\eT_2}{\tauT}
  }
  \and
  \inferrule[\rlabel{$\equiv$-reflect}{equiv-reflect}]{
    \type{\GammaT}{\pT}{\eqT{\eT_1}{\tauT}{\eT_2}}
  }{
    \defeq{\GammaT}{\eT_1}{\eT_2}{\tauT}
  }
  \and
  \inferrule[\rlabel{$\equiv$-cong}{equiv-cong}]{
    \textrm{For every $1 \leq i \leq n$:} \\
    \defeq{\GammaT'}{\eT_i}{\eT'_i}{\tauT'}
  }{
    \defeq{\GammaT}
      {\subst{\eT}{\xT_1, \seq, \xT_n}{\eT_1, \seq, \eT_n}}
      {\subst{\eT}{\xT_1, \seq, \xT_n}{\eT'_1, \seq, \eT'_n}}
      {\tauT}
  }
  \and
  \inferrule[\rlabel{$\equiv$-$\delta$}{equiv-delta}]{
    (\defineT{\xT}{\tauT}{\eT}) \in \GammaT
  }{
    \defeq{\GammaT}{\xT}{\eT}{\tauT}
  }
  \and
  \inferrule[\rlabel{$\equiv$-$\beta$}{equiv-beta}]{
    \type{\GammaT}{\sigmaT}{\UT} \\
    \type{\GammaT, \annotT{\xT}{\sigmaT}}{\eT}{\tauT} \\
    \type{\GammaT}{\eT'}{\sigmaT}
  }{
    \defeq{\GammaT}{\app{(\funT{\xT}{\sigmaT}{\eT})}{\eT'}}{\subst{\eT}{\xT}{\eT'}}{\subst{\tauT}{\xT}{\eT'}}
  }
  \and
  \inferrule[\rlabel{$\equiv$-$\eta$}{equiv-eta}]{
    \defeq{\GammaT, \annotT{\xT}{\sigmaT}}{\app{\eT_1}{\xT}}{\app{\eT_2}{\xT}}{\tauT}
  }{
    \defeq{\GammaT}{\eT_1}{\eT_2}{\funtypeT{\xT}{\sigmaT}{\tauT}}
  }
  \and
  \inferrule[\rlabel{$\equiv$-$\zeta$}{equiv-zeta}]{
    \type{\GammaT}{\sigmaT}{\UT} \\
    \type{\GammaT}{\eT'}{\sigmaT} \\
    \type{\GammaT, \defineT{\xT}{\sigmaT}{\eT'}}{\eT}{\tauT}
  }{
    \defeq{\GammaT}{\letinT{\xT}{\sigmaT}{\eT'}{\eT}}{\subst{\eT}{\xT}{\eT'}}{\subst{\tauT}{\xT}{\eT'}}
  }
  \and
  \inferrule[\rlabel{$\equiv$-$\rho$}{equiv-rho}]{
    \type{\GammaT}{\eT}{\tauT} \\
    \type{\GammaT}{\PT}{\funtypeT{\yT}{\tauT}{\funtypeT{\zT}{\eqT{\eT}{\tauT}{\yT}}{\UT}}} \\
    \type{\GammaT}{\dT}{\app{\PT}{\eT}{\reflT{\eT}}}
  }{
    \defeq{\GammaT}{\JT{\PT}{\dT}{\reflT{\eT}}}{\dT}{\app{\PT}{\eT}{\reflT{\eT}}}
  }
  \iffalse
  \and
  \inferrule[\rlabel{$\equiv$-$\iota$}{equiv-iota}]{
    \app{\dataT{\app{\XT}{(\annotT{\vec{\wT}}{\vec{\vphantom{\wT}\any}})}}{\arrT*{\DeltaT_I}{\UT}}}{\DeltaT_c} \\
    \annotT{\vec{\yT}}{\vec{\sigmaT}} = \subst{\DeltaT_I}{\vec{\wT}}{\vec{\pT}} \\
    (\annotT{\cT}{\arrT*{(\annotT{\vec{\zT}}{\vec{\tauT}})}{\app{\XT}{\vec{\pT}}{\vec{\aT}}}}) \in \subst{\DeltaT_c}{\vec{\wT}}{\vec{\pT}} \\
    \vec{\aT}' = \subst{\vec{\aT}}{\vec{\zT}}{\vec{\eT}} \\
    \type{\GammaT}{\app{\cT}{\vec{\pT}}{\vec{\eT}}}{\app{\XT}{\vec{\pT}}{\vec{\aT}'}} \\
    \type{\GammaT, \annotT{\vec{\yT}}{\vec{\sigmaT}}, \annotT{\xT}{\app{\XT}{\vec{\pT}}{\vec{\yT}}}}{\PT}{\UT} \\
    \type{\GammaT, \annotT{\vec{\zT}}{\vec{\tauT}}}{\eT}{\subst{\PT}{\vec{\yT}, \vec{\xT}}{\vec{\aT}, \app{\cT}{\vec{\pT}}{\vec{\zT}}}}
  }{
    \defeq{\GammaT}{\matchT{\app{\cT}{\vec{\pT}}{\vec{\eT}}}{\funT*{\vec{\yT}}{\xT}{\PT}}{\seq(\app{\cT}{\vec{\zT}} \RightarrowT \eT)\seq}}{\subst{\eT}{\vec{\zT}}{\vec{\eT}}}{\subst{\PT}{\vec{\yT}, \xT}{\vec{\aT}', \app{\cT}{\vec{\pT}}{\vec{\eT}}}}
  }
  \fi
  \and
  \inferrule[\rlabel{$\equiv$-$\iota$}{equiv-iota}]{
    \app{\dataT{\app{\XT}{(\annotT{\vec{\wT}}{\vec{\sigmaT}_P})}}{\arrT*{\DeltaT_I}{\UT}}}{\DeltaT_c} \\
    \type{\GammaT}{\vec{\pT}}{\vec{\sigmaT}_P} \\
    \annotT{\vec{\yT}}{\vec{\sigmaT}_I} = \subst{\DeltaT_I}{\vec{\wT}}{\vec{\pT}} \\
    \type{\GammaT, \annotT{\vec{\yT}}{\vec{\sigmaT}_I}, \annotT{\xT}{\app{\XT}{\vec{\pT}}{\vec{\yT}}}}{\PT}{\UT'} \\
    \elim{\XT}{\UT}{\UT'} \\\\
    \card{\DeltaT_c} = n \\
    \textrm{For every $1 \leq i \leq n$:} \\
    (\annotT{\cT_i}{\arrT*{(\annotT{\vec{\zT}_i}{\vec{\tauT}_i})}{\app{\XT}{\vec{\pT}}{\vec{\aT}_i}}}) \in \subst{\DeltaT_c}{\vec{\wT}}{\vec{\pT}} \\
    \vec{\zT}_i \notin \FV{\PT} \\
    \type{\GammaT, \annotT{\vec{\zT}_i}{\vec{\tauT}_i}}{\eT_i}{\subst{\PT}{\vec{\yT}, \vec{\xT}}{\vec{\aT}_i, \app{\cT_i}{\vec{\pT}}{\vec{\zT}_i}}} \\\\
    j \in 1 \seq n \\
    \type{\GammaT}{\vec{\eT}}{\vec{\tauT}_j} \\
    \vec{\aT}'_j = \subst{\vec{\aT}_j}{\vec{\zT}_j}{\vec{\eT}}
  }{
    \defeq{\GammaT}{\matchT{\app{\cT_j}{\vec{\pT}}{\vec{\eT}}}{\funT*{\vec{\yT}}{\xT}{\PT}}{(\app{\cT_1}{\vec{\zT}_1} \RightarrowT \eT_1)\seq(\app{\cT_n}{\vec{\zT}_n} \RightarrowT \eT_n)}}{\subst{\eT_j}{\vec{\zT}_j}{\vec{\eT}}}{\subst{\PT}{\vec{\yT}, \xT}{\vec{\aT}'_j, \app{\cT_j}{\vec{\pT}}{\vec{\eT}}}}
  }
  \and
  \inferrule[\rlabel{$\equiv$-$\mu$}{equiv-mu}]{
    \type{\GammaT}{\tauT}{\UT} \\
    \defeq{\GammaT}{\tauT}{\arr*{\DeltaT}{\funtypeT{\any}{\app{\XT}{\vec{\pT}}{\vec{\aT}}}{\tauT'}}}{\UT} \\
    \card{\DeltaT} + 1 = \nT \\
    \type{\GammaT, \annotT{\fT}{\tauT}}{\eT}{\tauT}
  }{
    \defeq{\GammaT}{\fixT{\nT}{\fT}{\tauT}{\eT}}{\subst{\eT}{\fT}{\fixT{\nT}{\fT}{\tauT}{\eT}}}{\tauT}
  }
  \end{mathpar}
  \caption{Equivalence rules}
  \label{#1}
  \end{figure}
}

\newcommand{\FigTypingCIC}[1]{
  \begin{figure}[p]
  \centering
  \begin{mathpar}
  \fbox{$\wf{}{\GammaT}$} \qquad
  \fbox{$\type{\GammaT}{\eT}{\tauT}$} \hfill \\
  \inferrule[\rlabel*{nil*}]{~}{\wf{}{\mt}}
  \and
  \inferrule[\rlabel*{cons*-ass}]{
    \wf{}{\GammaT} \\\\
    \type{\GammaT}{\tauT}{\UT}
  }{
    \wf{}{\GammaT, \annotT{\xT}{\tauT}}
  }
  \and
  \inferrule[\rlabel*{cons*-def}]{
    \wf{}{\GammaT} \\\\
    \type{\GammaT}{\eT}{\tauT}
  }{
    \wf{}{\GammaT, \defineT{\xT}{\tauT}{\eT}}
  }
  \and
  \inferrule[\rlabel*{conv*}]{
    \type{\GammaT}{\eT}{\sigmaT} \\
    \type{\GammaT}{\sigmaT}{\UT} \\\\
    \type{\GammaT}{\tauT}{\UT} \\
    \subtype{\GammaT}{\sigmaT}{\tauT}
  }{
    \type{\GammaT}{\eT}{\tauT}
  } 
  \and
  \inferrule[\rlabel*{var*}]{
    \wf{}{\GammaT} \\
    (\annotT{\xT}{\tauT}) \in \GammaT \\\\
    \textit{or } (\defineT{\xT}{\tauT}{\eT}) \in \Gamma
  }{
    \type{\GammaT}{\xT}{\tauT}
  }
  \and
  \inferrule[\rlabel*{univ*}]{
    \wf{}{\GammaT}
  }{
    \type{\GammaT}{\UT}{\axioms{\UT}}
  }
  \and
  \inferrule[\rlabel*{pi*}]{
    \type{\GammaT}{\sigmaT}{\UT_1} \\
    \type{\GammaT, \annotT{\xT}{\sigmaT}}{\tauT}{\UT_2}
  }{
    \type{\GammaT}{\funtypeT{\xT}{\sigmaT}{\tauT}}{\rules{\UT_1}{\UT_2}}
  }
  \and
  \inferrule[\rlabel*{lam*}]{
    \type{\GammaT}{\sigmaT}{\UT} \\
    \type{\GammaT, \annotT{\xT}{\sigmaT}}{\eT}{\tauT}
  }{
    \type{\GammaT}{\funT{\xT}{\sigmaT}{\eT}}{\funtypeT{\xT}{\sigmaT}{\tauT}}
  }
  \and
  \inferrule[\rlabel*{app*}]{
    \type{\GammaT}{\eT_1}{\funtypeT{\xT}{\sigmaT}{\tauT}} \\
    \type{\GammaT}{\eT_2}{\sigmaT}
  }{
    \type{\GammaT}{\app{\eT_1}{\eT_2}}{\subst{\tauT}{\xT}{\eT_1}}
  }
  \and
  \inferrule[\rlabel*{let*}]{
    \type{\GammaT}{\sigmaT}{\UT} \\
    \type{\GammaT}{\eT_1}{\sigmaT} \\
    \type{\GammaT, \defineT{\xT}{\sigmaT}{\eT_1}}{\eT_2}{\tauT}
  }{
    \type{\GammaT}{\letinT{\xT}{\sigmaT}{\eT_1}{\eT_2}}{\subst{\tauT}{\xT}{\eT_1}}
  }
  \and
  \inferrule[\rlabel*{eq}]{
    \type{\GammaT}{\tauT}{\UT} \\
    \type{\GammaT}{\eT_1}{\tauT} \\
    \type{\GammaT}{\eT_2}{\tauT}
  }{
    \type{\GammaT}{\eqT{\eT_1}{\tauT}{\eT_2}}{\UT}
  }
  \and
  \inferrule[\rlabel*{refl}]{
    \type{\GammaT}{\eT}{\tauT}
  }{
    \type{\GammaT}{\reflT{\eT}}{\eqT{\eT}{\tauT}{\eT}}
  }
  \and
  \inferrule[\rlabel*{J}]{
    \type{\GammaT}{\pT}{\eqT{\eT_1}{\tauT}{\eT_2}} \\
    \type{\GammaT}{\PT}{\funtypeT{\yT}{\tauT}{\funtypeT{\zT}{\eqT{\eT_1}{\tauT}{\yT}}{\UT}}} \\
    \type{\GammaT}{\dT}{\app{\PT}{\eT_1}{\reflT{\eT_1}}}
  }{
    \type{\GammaT}{\JT{\PT}{\dT}{\pT}}{\app{\PT}{\eT_2}{\pT}}
  }
  \and
  \inferrule[\rlabel*{ind}]{
    \wf{}{\GammaT} \\
    \app{\dataT{\app{\XT}{\DeltaT_P}}{\tauT}}{\any}
  }{
    \type{\GammaT}{\XT}{\arrT*{\DeltaT_P}{\tauT}}
  }
  \and
  \inferrule[\rlabel*{constr}]{
    \wf{}{\GammaT} \\
    \app{\dataT{\app{\XT}{\DeltaT_P}}{\any}}{\DeltaT_c} \\
    (\annotT{\cT}{\tauT}) \in \DeltaT_c
  }{
    \type{\GammaT}{\cT}{\arrT*{\DeltaT_P}{\tauT}}
  }
  \and
  \inferrule[\rlabel*{case}]{
    \app{\dataT{\app{\XT}{(\annotT{\vec{\wT}}{\vec{\vphantom{\wT}\any}})}}{\arrT*{\DeltaT_I}{\UT}}}{\DeltaT_c} \\
    \annotT{\vec{\yT}}{\vec{\sigmaT}} = \subst{\DeltaT_I}{\vec{\wT}}{\vec{\pT}} \\
    \type{\GammaT}{\eT}{\app{\XT}{\vec{\pT}}{\vec{\aT}}} \\
    \type{\GammaT, \annotT{\vec{\yT}}{\vec{\sigmaT}}, \annotT{\xT}{\app{\XT}{\vec{\pT}}{\vec{\yT}}}}{\PT}{\UT'} \\
    \elim{\XT}{\UT}{\UT'} \\\\
    \card{\DeltaT_c} = n \\
    \textrm{For every $1 \leq i \leq n$:} \\
    (\annotT{\cT_i}{\arrT*{(\annotT{\vec{\zT}_i}{\vec{\tauT}_i})}{\app{\XT}{\vec{\pT}}{\vec{\aT}_i}}}) \in \subst{\DeltaT_c}{\vec{\wT}}{\vec{\pT}} \\
    \vec{\zT}_i \notin \FV{\PT} \\
    \type{\GammaT, \annotT{\vec{\zT}_i}{\vec{\tauT}_i}}{\eT_i}{\subst{\PT}{\vec{\yT}, \vec{\xT}}{\vec{\aT}_i, \app{\cT_i}{\vec{\pT}}{\vec{\zT}_i}}}
  }{
    \type{\GammaT}{\matchT{\eT}{\funT*{\vec{\yT}}{\xT}{\PT}}{(\app{\cT_1}{\vec{\zT}_1} \RightarrowT \eT_1)\seq(\app{\cT_n}{\vec{\zT}_n} \RightarrowT \eT_n)}}{\subst{\PT}{\vec{\yT}, \xT}{\vec{\aT}, \eT}}
  }
  \and
  \inferrule[\rlabel*{fix*}]{
    \type{\GammaT}{\tauT}{\UT} \\
    \defeq{\GammaT}{\tauT}{\arr*{\DeltaT}{\funtypeT{\any}{\app{\XT}{\vec{\pT}}{\vec{\aT}}}{\tauT'}}}{\UT} \\
    \card{\DeltaT} + 1 = \nT \\
    \type{\GammaT, \annotT{\fT}{\tauT}}{\eT}{\tauT}
  }{
    \type{\GammaT}{\fixT{\nT}{\fT}{\tauT}{\eT}}{\tauT}
  }
  \end{mathpar}
  \caption{Typing and well-formedness rules (\CICE)}
  \label{#1}
  \end{figure}
}

\newcommand{\FigSubtypingCIC}[1]{
  \begin{figure}[h]
  \centering
  \begin{mathpar}
  \fbox{$\subtype{\GammaT}{\tauT}{\tauT}$} \hfill \\
  \inferrule[\rlabel{$\preccurlyeq$-conv}{subtype-conv}]{
    \defeq{\GammaT}{\tauT_1}{\tauT_2}{\UT}
  }{
    \subtype{\GammaT}{\tauT_1}{\tauT_2}
  }
  \and
  \inferrule[\rlabel{$\preccurlyeq$-trans}{subtype-trans}]{
    \subtype{\GammaT}{\tauT_1}{\tauT_2} \\
    \subtype{\GammaT}{\tauT_2}{\tauT_3}
  }{
    \subtype{\GammaT}{\tauT_1}{\tauT_3}
  }
  \and
  \inferrule[\rlabel{$\preccurlyeq$-prop}{subtype-prop}]{~}{
    \subtype{\GammaT}{\PropT}{\TypeT{\iT}}
  }
  \and
  \inferrule[\rlabel{$\preccurlyeq$-type}{subtype-type}]{
    \iT \leq \jT
  }{
    \subtype{\GammaT}{\TypeT{\iT}}{\TypeT{\jT}}
  }
  \and
  \inferrule[\rlabel{$\preccurlyeq$-pi}{subtype-pi}]{
    \defeq{\GammaT}{\sigmaT_1}{\sigmaT_2}{\UT} \\
    \subtype{\GammaT, \annotT{\xT}{\sigmaT_2}}{\tauT_1}{\tauT_2}
  }{
    \subtype{\GammaT}{\funtypeT{\xT}{\sigmaT_1}{\tauT_1}}{\funtypeT{\xT}{\sigmaT_2}{\tauT_2}}
  }
  \end{mathpar}
  \caption{Subtyping rules (\CICE)}
  \label{#1}
  \end{figure}
}
\newcommand{\FigData}[1]{
  \begin{figure}[h]
  \newlength{\xspacing}
  \setlength{\xspacing}{1.5ex}
  \centering
  \begin{align*}
  &\dataT{\botT}{\PropT} \\[\xspacing]
  &\dataT{\SizeT}{\TypeT{i+1}} \\
  &\quad \annotT{\sucT}{\arrT*{\SizeT}{\SizeT}} \\
  &\quad \annotT{\limT}{\funtypeT{A}{\TypeT{i}}{\arrT*{(\arrT*{A}{\SizeT})}{\SizeT}}} \\
  &\LetT{\baseT}{\SizeT}{\app{\limT}{\botT}{(\funT{\mathit{false}}{\botT}{\matchT{\mathit{false}}{\funT*{\mt}{\any}{\SizeT}}{\mt}})}} \\[\xspacing]
  &\dataT{\any \szleT \any}{\arrT*{\SizeT}{\SizeT}{\TypeT{i+1}}} \\
  &\quad \annotT{\monoT}{\funtypeT{\alpha, \beta}{\SizeT}{\arrT*{\alpha \szleT \beta}{\app{\sucT}{\alpha} \szleT \app{\sucT}{\beta}}}} \\
  &\quad \annotT{\coconeT}{\funtypeT{A}{\TypeT{i}}{\funtypeT{\beta}{\SizeT}{\funtypeT{f}{\arrT*{A}{\SizeT}}{\funtypeT{a}{A}{\arr*{\beta \szleT \app{f}{a}}{\beta \szleT \app{\limT}{A}{f}}}}}}} \\
  &\quad \annotT{\limitT}{\funtypeT{A}{\TypeT{i}}{\funtypeT{\beta}{\SizeT}{\funtypeT{f}{\arrT*{A}{\SizeT}}{\arr*{(\funtypeT{a}{A}{\app{f}{a} \szleT \beta})}{\app{\limT}{A}{f} \szleT \beta}}}}} \\
  &\LetT{\any \szltT \any}{\arrT*{\SizeT}{\SizeT}{\TypeT{i+1}}}{\funT{\alpha, \beta}{\SizeT}{\app{\sucT}{\alpha} \szltT \beta}} \\[\xspacing]
  &\dataT{\app{\AccT}{(\annot{\alpha}{\SizeT})}}{\PropT} \\
  &\quad \annot{\accT}{\arrT*{(\funtypeT{\beta}{\SizeT}{\arrT*{\beta \szltT \alpha}{\app{\AccT}{\beta}}})}{\app{\AccT}{\alpha}}} \\[\xspacing]
  &\dataT{\app{\NatT}{(\annot{\alpha}{\SizeT})}}{\TypeT{1}} \\
  &\quad \annot{\zeroT}{\funtypeT{\beta}{\SizeT}{\arrT*{\beta \szltT \alpha}{\app{\NatT}{\alpha}}}} \\
  &\quad \annot{\succT}{\funtypeT{\beta}{\SizeT}{\arrT*{\beta \szltT \alpha}{\app{\NatT}{\beta}}{\app{\NatT}{\alpha}}}} \\[\xspacing]
  &\dataT{\app{\WT}{(\annot{A}{\TypeT{i}})}{(\annot{B}{\arrT*{A}{\TypeT{i}}})}{(\annot{\alpha}{\SizeT})}}{\TypeT{i+1}} \\
  &\quad \annot{\supT}{\funtypeT{\beta}{\SizeT}{\arrT*{\beta \szltT \alpha}{\funtypeT{a}{A}{\arrT*{(\arrT*{\app{B}{a}}{\app{\WT}{A}{B}{\beta}})}{\app{\WT}{A}{B}{\alpha}}}}}}
  \end{align*}
  \caption{Inductive data definitions}
  \label{#1}
  \end{figure}
}
\newcommand{\FigTransSize}[1]{
  \begin{figure}[h]
  \centering
  \begin{multicols}{2}
  \begin{align*}
  \Aboxed{\compile{s} &= \eT} \\
  \compile{\alpha} &= \alphaT \\
  \compile{\circ} &= \baseT \\
  \compile{\sss{s}} &= \app{\sucT}{\compile{s}}
  \end{align*}

  \begin{align*}
  \Aboxed{\compile{\Phi} &= \GammaT} \\
  \compile{\mt} &= \mt \\
  \compile{\Phi, \alpha} &= \compile{\Phi}, \annot{\alphaT}{\SizeT} \\
  \compile{\Phi, \bound{\alpha}{s}} &= \compile{\Phi}, \annot{\alphaT}{\SizeT}, \annot{\alphaT^*}{\alphaT \szltT \compile{s}}
  \end{align*}
  \end{multicols}
  \caption{Translation of sizes and size environments}
  \label{#1}
  \end{figure}
}

\newcommand{\FigTransSubsize}[1]{
  \begin{figure}[h]
  \centering
  \begin{mathpar}
  \fbox{$\subsizeto{\Phi}{s}{s}{\eT}$} \hfill \\
  \inferrule{
    (\bound{\alpha}{s}) \in \Phi
  }{
    \subsizeto{\Phi}{\sss{\alpha}}{s}{\alphaT^*}
  }
  \and
  \inferrule{~}{
    \subsizeto{\Phi}{\circ}{s}{\app{\baseleq}{\compile{s}}}
  }
  \and
  \inferrule{~}{
    \subsizeto{\Phi}{s}{s}{\app{\reflleq}{\compile{s}}}
  }
  \and
  \inferrule{~}{
    \subsizeto{\Phi}{s}{\sss{s}}{\app{\sucleq}{\compile{s}}}
  }
  \and
  \inferrule{
    \subsizeto{\Phi}{r}{s}{\eT}
  }{
    \subsizeto{\Phi}{\sss{r}}{\sss{s}}{\app{\monoT}{\compile{r}}{\compile{s}}{\eT}}
  }
  \and
  \inferrule{
    \subsizeto{\Phi}{s_1}{s_2}{\eT_{12}} \\
    \subsizeto{\Phi}{s_2}{s_3}{\eT_{23}}
  }{
    \subsizeto{\Phi}{s_1}{s_3}{\app{\transleq}{\compile{s_1}}{\compile{s_2}}{\compile{s_3}}{\eT_{12}}{\eT_{23}}}
  }
  \end{mathpar}
  \caption{Translation of subsizing}
  \label{#1}
  \end{figure}
}

\newcommand{\FigTransTerm}[1]{
  \begin{figure}[ht]
  \centering
  \begin{multicols}{2}
  \begin{align*}
  \Aboxed{\compile{e} &= \eT} \\
  \compile{x} &= \xT \\
  \compile{U} &= \UT \\
  \compile{\funtype{x}{\sigma}{\tau}} &= \funtypeT{\xT}{\compile{\sigma}}{\compile{\tau}_{\annot{x}{\sigma}}} \\
  \compile{\fun{x}{\sigma}{e}} &= \funT{\xT}{\compile{\sigma}}{\compile{e}_{\annot{x}{\sigma}}} \\
  \compile{\app{e_1}{e_2}} &= \app{\compile{e_1}}{\compile{e_2}} \\
  \compile{\Funtype{\alpha}{\tau}} &= \funtypeT{\alphaT}{\SizeT}{\compile{\tau}_{\alpha}} \\
  \compile{\Funtype<{\alpha}{s}{\tau}} &= \funtypeT{\alphaT}{\SizeT}{\funtypeT{\alphaT^*}{\alphaT \szltT \compile{s}}{\compile{\tau}_{\bound{\alpha}{s}}}} \\
  \compile{\Fun{\alpha}{e}} &= \funT{\alphaT}{\SizeT}{\compile{e}_{\alpha}} \\
  \compile{\Fun<{\alpha}{s}{e}} &= \funT{\alphaT}{\SizeT}{\funT{\alphaT^*}{\alphaT \szltT \compile{s}}{\compile{e}_{\bound{\alpha}{s}}}} \\
  \compile{\App{e}{s}} &= \app{\compile{e}}{\compile{s}} \qquad (\textrm{\rref{sapp}}) \\
  \end{align*}

  \begin{align*}
  \Aboxed{\compile{\Gamma} &= \GammaT} \\
  \compile{\mt} &= \mt \\
  \compile{\Gamma, \annot{x}{\tau}} &= \compile{\Gamma}, \annotT{\xT}{\compile{\tau}} \\
  \compile{\Gamma, \define{x}{\tau}{e}} &= \compile{\Gamma}, \defineT{\xT}{\compile{\tau}}{\compile{e}}
  \end{align*}
  \end{multicols}
  \vspace{-2\baselineskip}
  \begin{mathpar}
  \fbox{$\typeto{\Phi; \Gamma}{e}{\tau}{\eT}$} \quad \cdots \hfill \\
  \inferrule[\nameref{conv}]{
    \cdots \\
    \typeto{\Phi; \Gamma}{e}{\sigma}{\eT} \\
    \subtype{\Phi; \Gamma}{\sigma}{\tau}
  }{
    \typeto{\Phi; \Gamma}{e}{\tau}{\eT}
  }
  \and
  \inferrule[\nameref{sapp<}]{
    \subsizeto{\Phi}{\hat{s}}{r}{\eT'} \\
    \typeto{\Phi; \Gamma}{e}{\Funtype<{\alpha}{r}{\tau}}{\eT}
  }{
    \typeto{\Phi; \Gamma}{\App{e}{s}}{\subst{\tau}{\alpha}{s}}{\app{\eT}{\compile{s}}{\eT'}}
  }
  \and
  \inferrule[\nameref{let}]{
    \typeto{\Phi; \Gamma}{\sigma}{\UT}{\sigmaT} \\
    \typeto{\Phi; \Gamma}{e_1}{\sigma}{\eT_1} \\
    \typeto{\Phi; \Gamma, \define{x}{\sigma}{e_1}}{e_2}{\tau}{\eT_2}
  }{
    \typeto{\Phi; \Gamma}{\letin{x}{\sigma}{e_1}{e_2}}{\subst{\tau}{x}{e_1}}{\letinT{\xT}{\sigmaT}{\eT_1}{\eT_2}}
  }
  \end{mathpar}
  \caption{Translation of terms (base \lang) and term environments}
  \label{#1}
  \end{figure}
}

\newcommand{\FigTransInd}[1]{
  \begin{figure}[p]
  \centering
  \begin{align*}
  \Aboxed{\compile{e} &= \eT} \quad \cdots \\
  \compile{\N{s}} &= \app{\NatT}{\compile{s}} \\
  \compile{\W{s}{x}{\sigma}{\tau}} &= \app{\WT}{\compile{\sigma}}{\compile{\fun{x}{\sigma}{\tau}}}{\compile{s}}
  \end{align*}
  \begin{mathpar}
  \setlength{\jot}{-1.5pt}
  \fbox{$\typeto{\Phi; \Gamma}{e}{\tau}{\eT}$} \quad \cdots \hfill \\
  \inferrule[\nameref{zero}]{
    \wf{\Phi}{\Gamma} \\
    \subsizeto{\Phi}{\hat{r}}{s}{\eT'}
  }{
    \typeto{\Phi; \Gamma}{\zero{s}{r}}{\N{s}}{\app{\zeroT}{\compile{s}}{\compile{r}}{\eT'}}
  }
  \and
  \inferrule[\nameref{succ}]{
    \subsizeto{\Phi}{\hat{r}}{s}{\eT'} \\
    \typeto{\Phi; \Gamma}{e}{\N{r}}{\eT}
  }{
    \typeto{\Phi; \Gamma}{\succ{s}{r}{e}}{\N{s}}{\app{\succT}{\compile{s}}{\compile{r}}{\eT'}{\eT}}
  }
  \and
  \inferrule[\nameref{sup}]{
    \subsizeto{\Phi}{\hat{r}}{s}{\eT'} \\
    \typeto{\Phi; \Gamma}{\sigma}{U}{\sigmaT} \\
    \typeto{\Phi; \Gamma, \annot{x}{\sigma}}{\tau}{U}{\tauT} \\\\
    \typeto{\Phi; \Gamma}{e_1}{\sigma}{\eT_1} \\
    \typeto{\Phi; \Gamma}{e_2}{\arr*{\subst{\tau}{x}{e_1}}{\W{x}{\sigma}{\tau}{r}}}{\eT_2}
  }{
    \typeto{\Phi; \Gamma}{\sup{x}{\sigma}{\tau}{s}{r}{e_1}{e_2}}{\W{x}{\sigma}{\tau}{s}}{\app{\supT}{\sigmaT}{(\funT{\xT}{\sigmaT}{\tauT})}{\compile{s}}{\compile{r}}{\eT'}{\eT_1}{\eT_2}}
  }
  \and
  \inferrule[\nameref{case-nat}]{
    \typeto{\Phi; \Gamma}{e}{\N{s}}{\eT} \\
    \typeto{\Phi; \Gamma, \annot{x}{\N{s}}}{P}{U}{\PT} \\\\
    \typeto{\Phi, \bound{\alpha}{s}; \Gamma}{e_z}{\subst{P}{x}{\zero{s}{\alpha}}}{\eT_z} \\
    \typeto{\Phi, \bound{\beta}{s}; \Gamma, \annot{z}{\N{\beta}}}{e_s}{\subst{P}{x}{\succ{s}{\beta}{z}}}{\eT_s}
  }{
    \typeto{\Phi; \Gamma}{
      \begin{aligned}
      &\match{e}{\fun*{x}{P}}{ \\
      &\quad \App{\zero*}{\alpha} \Rightarrow e_z \\
      &\quad \app{\App{\succ*}{\beta}}{z} \Rightarrow e_s}
      \end{aligned}
    }{\subst{P}{x}{e}}{
      \begin{aligned}
      &\matchT{\eT}{\funT*{\mt}{\xT}{\PT}}{\\
      &\quad \app{\zeroT}{\alphaT}{\alphaT^*} \RightarrowT \eT_z \\
      &\quad \app{\succT}{\betaT}{\betaT^*}{\zT} \RightarrowT \eT_s}
      \end{aligned}
    }
  }
  \and
  \inferrule[\nameref{case-wft}]{
    \typeto{\Phi; \Gamma}{e}{\W{y}{\sigma}{\tau}{s}}{\eT} \\
    \typeto{\Phi; \Gamma, \annot{x}{\W{y}{\sigma}{\tau}{s}}}{P}{U}{\PT} \\
    \typeto{\Phi, \bound{\alpha}{s}; \Gamma, \annot{z_1}{\sigma}, \annot{z_2}{\arr*{\subst{\tau}{y}{z_1}}{\W{y}{\tau}{\sigma}{\alpha}}}}{e_s}{\subst{P}{x}{\sup{y}{\sigma}{\tau}{s}{\alpha}{z_1}{z_2}}}{\eT_s}
  }{
    \typeto{\Phi; \Gamma}{
      \begin{aligned}
      &\match{e}{\fun*{x}{P}}{ \\
      &\quad \app{\App{\sup*}{\alpha}}{z_1}{z_2} \Rightarrow e_s}
      \end{aligned}
    }{\subst{P}{x}{e}}{
      \begin{aligned}
      &\matchT{\eT}{\funT*{\mt}{\xT}{\PT}}{ \\
      &\quad \app{\supT}{\alphaT}{\alphaT^*}{\zT_1}{\zT_2} \RightarrowT \eT_s}
      \end{aligned}
    }
  }
  \and
  \inferrule[\nameref{fix}]{
    \typeto{\Phi, \alpha; \Gamma}{\sigma}{U}{\sigmaT} \\
    \typeto{\Phi, \alpha; \Gamma, \annot{f}{\Funtype<{\beta}{\alpha}{\subst{\sigma}{\alpha}{\beta}}}}{e}{\sigma}{\eT}
  }{
    \typeto{\Phi; \Gamma}{\fix{f}{\alpha}{\sigma}{e}}{\Funtype{\alpha}{\sigma}}{
      \begin{aligned}
      &\app{\wfind}{(\funT{\alphaT}{\SizeT}{\sigmaT})}{ \\
      &\quad (\funT{\alphaT}{\SizeT}{\funT{\fT}{\funtypeT{\betaT}{\SizeT}{\arrT*{\betaT \szltT \alphaT}{\subst{\sigmaT}{\alphaT}{\betaT}}}}{\eT}})}
      \end{aligned}
    }
  }
  \end{mathpar}
  \caption[Translation of naturals, well-founded trees, $\kw{case}$, fixpoints]{Translation of naturals, well-founded trees, \\
    and $\kw{case}$ and fixpoint expressions}
  \label{#1}
  \end{figure}
}

\lang is modelled in \CICE, as briefly described in \cref{sec:introduction}.
The key idea is that sizes in \lang can themselves be represented as a inductive type in \CICE,
and naturals and well-founded trees are then inductives with an additional size parameter.
Sizes are represented as a (generalization of) the Brouwer notation for ordinals in type theory,
and their order as an inductive type indexed by sizes.
The order is \emph{well-founded}:
there is no infinite sequence of ever-smaller sizes,
and there is always a ``smallest'' size (or many of them).
This property allows for \emph{well-founded induction},
where to prove some property on sizes, one supposes that it holds for all strictly smaller sizes.

Every fixpoint expression in \lang is modelled as an instance of well-founded induction in \CICE.
To prove well-foundedness and in turn the induction principle,
we show that sizes satisfy an \emph{accessiblity predicate}~\citep{accessibility}.
For the type preservation proof to go through,
\emph{definitional proof irrelevance}
of accessibility predicates is required---that is,
every proof of accessibility is definitionally equal to one another.
This holds in extensional type theory via equality reflection
but not in intensional type theory,
which is why an extensional CIC is used.

The first half of this section provides the syntax and judgements of \CICE.
In addition to the notation used in \cref{sec:sized-dep-types},
given variables $\vec{\xT} = \xT_1 \seq \xT_n$,
terms $\vec{\eT} = \eT_1 \seq \eT_n$,
and types $\vec{\tauT} = \tauT_1 \seq \tauT_n$,
\new{$\annotT{\vec{\xT}}{\vec{\tauT}}$} denotes the assumption environment
$\annotT*{\xT_1}{\tauT_1}, \seq, \annotT*{\xT_n}{\tauT_n}$,
\new{$\subst{\eT}{\vec{\xT}}{\vec{\eT}}$} denotes the simultaneous substitution
$\subst{\eT}{\xT_1, \seq, \xT_n}{\eT_1, \seq, \eT_n}$, 
\new{$\funT{\vec{\xT}}{\vec{\tauT}}{\eT}$} denotes the $n$-ary function
$\funT{\xT_1}{\tauT_1}{\seq \funT{\xT_n}{\tauT_n}{\eT}}$, and
\new{$\type{\GammaT}{\vec{\eT}}{\vec{\tauT}}$} denotes the $n$ typing judgements
$(\type{\GammaT}{\eT_1}{\tauT_1})$, \seq, $(\type{\GammaT, \annotT{\eT_1}{\tauT_1}, \seq, \annotT{\eT_{n-1}}{\tauT_{n-1}}}{\eT_n}{\tauT_n})$.

The second half then describes the translation from \lang to \CICE,
which is a metafunction from typing derivations of \lang to terms of \CICE.
Therefore, the translation is only defined for well-typed \lang terms,
but the type preservation theorem only applies to well-typed terms anyway.

\subsection{Target Type Theory}

\FigSyntaxCIC{fig:syntax-cic}
The syntax of \CICE is given in \cref{fig:syntax-cic};
differences from \lang include a 1-based index for the recursive argument of fixpoint expressions,
$\tg{case}$ expression motives abstracted over the target's inductive type indices,
and a homogeneous propositional equality type with the reflexivity constructor and $\JT*$ eliminator.
New inductive types are defined using data definitions $\DT$,
whose syntax resembles the informal presentation used in \cref{sec:sized-dep-types}.
Metavariable usage convention is roughly the same as for \lang,
with the addition of $\pT$ for inductive type parameters or proofs of equality
and $\aT$ for inductive type indices.

The well-formedness conditions on inductive data definitions,
such as well-typedness and \emph{strict positivity},
are entirely standard, so they aren't detailed here;
see pCIC~\citep{pCIC} for instance for a full description.
Inductive definitions in their full generality are not needed,
and nonmutual, nonnested inductives suffice.
Indeed, only six inductive definitions are used for the translation,
for representing sizes, their order, their well-foundedness,
and the empty type, naturals, and well-founded trees.

The typed equivalence, subtyping, and typing judgements are defined mutually:
equivalence depends on typing and subtyping,
subtyping depends on equivalence,
and typing depends on subtyping and equivalence.
The mutual dependence is due to typed equivalence,
since with untyped conversion as seen in \lang has no typing premises.
We present first the equivalence rules in \cref{fig:equivalence},
with the subtyping and typing rules to follow.

Equivalence is, by definition, an equivalence relation,
satisfying reflexivity, symmetry, and transitivity.
Equivalence is also congruent, using the same summary of congruence rules as for \lang via \rref{equiv-cong};
the full set of rules can similarly be found in \cref{app:cong:equiv}.
An equivalence judgement can be converted to one annotated by a supertype via \rref{equiv-conv}.
The key rule for extensionality is equality reflection in \rref{equiv-reflect},
which definitionally equates two terms whenever there exists some proof of their propositional equality.

\FigEquiv[h]{fig:equivalence}

The remaining equivalence rules are typed versions of the usual reduction rules,
with typing premises to ensure well-typedness of both sides.
Functions have both a $\beta$-equivalence rule and an $\eta$-equivalence rule,
and the rules for $\tg{let}$ expressions are exactly the same as in \lang.
The $\JT*$ eliminator and $\tg{case}$ expressions reduce when applied to
reflexivity and inductive constructors, respectively.

\rref{equiv-mu} for fixpoint expressions is \emph{unguarded},
meaning that fixpoints are equivalent to the substitution of itself into its own body
regardless of what they are applied to.
To maintain normalization,
the usual guarded reduction rule in intensional CIC reduces fixpoints
only when applied to a literal constructor in the recursive argument position.
\begin{mathpar}
\inferrule[\rlabel{$\equiv$-$\mu$-guarded}{equiv-mu-guarded}]{\cdots \\ \card{\vec{\eT}'} + 1 = \nT}{
  \defeq{\GammaT}{\app{(\fixT{\nT}{\fT}{\tauT}{\eT})}{\vec{\eT}'}{(\app{\cT}{\vec{\aT}})}}{\app{(\subst{\eT}{\fT}{\fixT{\nT}{\fT}{\tauT}{\eT}})}{\vec{\eT}'}{(\app{\cT}{\vec{\aT}})}}{\tauT}
}
\end{mathpar}

Evidently \rref{equiv-mu-guarded} can be derived from \rref{equiv-mu} by congruence.
On the other hand,
\rref{equiv-mu} can be derived from \rref{equiv-mu-guarded}
by reflecting a proof of their propositional equality,
which can be constructed transitively using applications of \rref{equiv-eta, equiv-cong}.
Since \rref{equiv-mu} and \rref{equiv-mu-guarded} are metatheoretically equivalent,
we choose to use \rref{equiv-mu} for its simplicity.

\begin{figure}[h]
\centering
\begin{mathpar}
\fbox{$\subtype{\GammaT}{\tauT}{\tauT}$} \hfill
\inferrule[\rlabel{$\preccurlyeq$-conv}{subtype-conv}]{
  \defeq{\GammaT}{\tauT_1}{\tauT_2}{\UT}
}{
  \subtype{\GammaT}{\tauT_1}{\tauT_2}
}
\and
\inferrule[\rlabel{$\preccurlyeq$-trans}{subtype-trans}]{
  \subtype{\GammaT}{\tauT_1}{\tauT_2} \\
  \subtype{\GammaT}{\tauT_2}{\tauT_3}
}{
  \subtype{\GammaT}{\tauT_1}{\tauT_3}
}
\and
\inferrule[\rlabel{$\preccurlyeq$-prop}{subtype-prop}]{~}{
  \subtype{\GammaT}{\PropT}{\TypeT{\iT}}
}
\and
\inferrule[\rlabel{$\preccurlyeq$-type}{subtype-type}]{
  \iT \leq \jT
}{
  \subtype{\GammaT}{\TypeT{\iT}}{\TypeT{\jT}}
}
\and
\inferrule[\rlabel{$\preccurlyeq$-pi}{subtype-pi}]{
  \defeq{\GammaT}{\sigmaT_1}{\sigmaT_2}{\UT} \\
  \subtype{\GammaT, \annotT{\xT}{\sigmaT_2}}{\tauT_1}{\tauT_2}
}{
  \subtype{\GammaT}{\funtypeT{\xT}{\sigmaT_1}{\tauT_1}}{\funtypeT{\xT}{\sigmaT_2}{\tauT_2}}
}
\end{mathpar}
\caption{Subtyping rules (\CICE)}
\label{fig:subtyping-cic}
\end{figure}

As opposed to \lang, for \CICE we use pCIC's presentation of subtyping in \cref{fig:subtyping-cic},
which has a typed equivalence premise in \rref{subtype-conv}.
It also has an explicit rule for transitivity of subtyping since judgements such as
$\subtype{\mt}{\app{(\funT{P}{\TypeT{\tg{1}}}{P})}{\PropT}}{\TypeT{\tg{0}}}$ would fail to hold otherwise.
Like \rref{acum-pi}, \rref{subtype-pi} is invariant in the domain of function types.

The typing and environment well-formedness rules are in \cref{fig:typing-cic}.
Except for \rref{fix*}, the starred rules are the same as for \lang,
with metafunctions $\axioms{\mt}$ and $\rules{\mt}{\mt}$ operating similarly on universes $\UT$
as for $U$.
An additional premise to \rref{fix*} ensures that the $\nT$th argument is indeed an inductive type.

As previously mentioned, fixpoints must also be guarded:
recursive calls can only occur on structurally smaller arguments of elements of inductives.
The guard condition is well studied \citep{guard, guard-relax, Coq} and so omitted here.
To justify uses of fixpoint expressions in the translation,
we will provide either a mechanization or present a brief argument of guardedness.

\FigTypingCIC[h]{fig:typing-cic}

The new \rref{eq, refl, J} are for the propositional equality type,
its constructor, and its eliminator.
The proof of type preservation eliminates propositional equalities mostly through
reflection rather than using $\JT*$,
but we retain $\JT*$ in \CICE for completeness.

\rref{ind, constr, case} assign types to inductive types, their constructors,
and $\tg{case}$ expressions, under the premise that
the relevant inductive data definition exists and is well formed.
Here, the difference between the parameters and the indices of inductive types becomes apparent:
the motive of a $\tg{case}$ expression is abstracted over the indices by $\vec{\yT}$
in addition to the target by $\xT$, while the parameters $\vec{\pT}$ are fixed throughout.
Therefore, when dealing with the types of indices and constructor arguments,
the parameters are first substituted in place of $\vec{\wT}$.

A $\tg{case}$ expression is well typed if its target is,
if its motive is for any indices and target with those indices,
and if each branch is well typed for that branch's constructor arguments,
where its type is the motive with the appropriate indices and reconstructed target.
For notational simplicity, the rule assumes that the binding variable names
$\vec{\yT}$ and $\vec{\zT}$ are those found in the data definition,
but of course these can be renamed at the expense of additional renaming substitutions.

Additionally, the motive of a $\tg{case}$ expression is restricted by the metarelation $\elim{\mt}{\mt}{\mt}$,
which indicates when \emph{large elimination} is allowed.
$\elim{\any}{\TypeT{\iT}}{\any}$ and $\elim{\any}{\PropT}{\PropT}$ always hold,
so that inductives in $\TypeT{}$ can be eliminated to any universe
and inductives in $\PropT$ can be eliminated to $\PropT$,
while $\elim{\XT}{\PropT}{\TypeT{\iT}}$ holds if the inductive $\XT$ in $\PropT$ satisfies further conditions.
For the purposes of the translation, the only relevant conditions are that $\XT$ either have no constructors
or have a single constructor whose arguments are all in $\PropT$.
They can be loosened while still retaining consistency (see \eg \citet{SProp}).

The typing premises of \rref{equiv-beta, equiv-zeta, equiv-rho, equiv-iota, equiv-mu}
corresponding to some of the reduction rules duplicate the premises found in many of the typing rules,
trivially ensuring that both sizes of these equivalences are well-typed with the same type
(\ie \emph{subject equivalence}).

\subsection{Preliminary Definitions} \label{subsec:prelim}

Before defining the translation from \lang terms to \CICE terms,
in this section we describe how the \CICE terms are constructed,
which comprises the aforementioned six inductive data definitions
and well-founded induction principle
as well as the various properties which the order on sizes satisfies.

\FigData{fig:data-defns}

The inductive definitions are listed in \cref{fig:data-defns}
along with some basic definitions we treat as global.
$\botT$ is the usual empty type.
$\SizeT$ is a generalization of the $\Ord*$ type introduced in \cref{subsec:examples},
with the domain of the function passed to the limit size $\limT$ replaced by some arbitrary type.
Although limit sizes aren't strictly necessary for the translation,
since \lang only has successor sizes and size variables,
we include them so that various solutions to the problem of the infinite size
can be explored in \cref{subsec:infinity}.
Furthermore, this allows for a simplification of the definition,
since the zero size $\baseT$ can be defined as a limit size rather than as another constructor.

The order on sizes $\mt \szleT \mt$ is defined by the three properties that must hold~\citep{ordinals}:
\begin{itemize}[noitemsep]
  \item $\monoT$: The successor operator $\sucT$ is monotone with respect to the order;
  \item $\coconeT$: The limit operator $\limT$ constructs an upper bound in that
    given some function $f$ returning a size,
    any size smaller than any size returned by $f$ is also smaller than the limit of $f$;
  \item $\limitT$: The limit operator on $f$ constructs a \emph{least} upper bound such that
    if a size is larger than \emph{all} sizes returned by $f$
    then it must also be larger than the limit of $f$.
\end{itemize}

Other properties of the order can be derived by induction from these constructors alone.
A corresponding strict order $\mt \szltT \mt$ is also defined,
and an accessibility predicate $\AccT$ is specialized to sizes and the strict order.
Note that it lives in $\PropT$, as does the argument of its sole constructor $\accT$,
so accessibility predicates are intended to be interpreted as proof irrelevant,
and their large elimination is allowed.

\FigDefns{fig:defns}

Before moving on to naturals and well-founded trees,
\cref{fig:defns} lists the names and types of a number of provable definitions.
First is \emph{function extensionality},
asserting that two functions $\annot{f, g}{\funtypeT{x}{A}{\app{B}{x}}}$ are propositionally equal if they are pointwise equal.
In \CICE, they are in fact equivalent (\ie definitionally equal):
given some proof $h$ of their pointwise equality,
under the assumption $\annot{x}{A}$,
the propositional equality $\eq{\app{f}{x}}{}{\app{g}{x}}$ by $\app{h}{x}$
can be reflected into the corresponding definitional equality,
which by \rref{equiv-eta} is then a definitional equality of $f$ and $g$.

The remaining definitions have been mechanized in either Agda and Coq in
\cref{app:mechanization:agda:prelim} and \cref{app:mechanization:coq:prelim},
respectively, under the assumption of function extensionality as an axiom
(which cannot be proven in intensional CIC).
The mechanizations don't use any additional type-theoretic features beyond CIC,
and the proofs could theoretically be written in plain CIC,
but they would be far less comprehensible without the ergonomics provided by the proof assistants.
As an example, the Coq proof for $\accessible$ consists of a dozen lines of tactics,
while the full proof term generated from the tactics is 129 lines long.

The definitions themselves describe properties of the order on sizes
and of the accessibility predicate:
\begin{itemize}[noitemsep]
  \item $\baseleq$, $\reflleq$, $\transleq$, and $\sucleq$:
    The order is reflexive and transitive (\ie a preorder)
    such that $\baseT$ is smaller than or equal to all sizes
    and the successor of a size is greater or equal to itself.
  \item $\accIsProp$: Accessibility predicates are \emph{mere propositions}:
    any two proofs of accessibility of a size are propositionally equal.
  \item $\accleq$: Any size smaller or equal to an accessible size is itself accessible.
  \item $\accessible$: All sizes are accessible; in other words, the order on sizes is well founded.
  \item $\wfind$ and $\wfacc$: The well-founded induction principle on sizes with respect to the order,
    proven by structural induction on the accessibility of sizes.
\end{itemize}

The only time equality reflection is needed is to prove $\accIsProp$ (via $\funext$),
which in turn is the only other equality that is reflected for proving type preservation.
Since $\AccT$ is in already $\PropT$, \CICE could be replaced by an intensional CIC
with a universe of \emph{strict propositions} $\SPropT$
of types whose elements are definitionally equal~\citep{SProp},
then placing $\AccT$ in $\SPropT$.
This is disallowed by \opcit because it breaks normalization;
however, it doesn't break consistency,
so it would remain suitable as the target language of a syntactic model.
In any case, we use \CICE because equality reflection is more established in the literature
and it allows us to use \rref{equiv-mu} in place of \rref{equiv-mu-guarded}.

Finally, the naturals and the well-founded trees in \CICE are parametrized by a $\SizeT$.
Their definitions respect the translation in the upcoming section,
so that the types of $\NatT$ and $\WT$ and their constructors are preserved.

\subsection{Translation}

The key type preservation theorem states that well-typed terms of \lang translate to
corresponding well-typed terms of CICE.
However, terms are not the only thing requiring translation:
well-typedness holds under some environment, so term environments need translations;
sizes and their environments translate to terms and term environments as well;
and derivations of size orders translate to terms which represent them.

\FigTransSize{fig:trans:size}
We begin with the translation of sizes and their environments in \cref{fig:trans:size},
which are straightforward recursive metafunctions over their syntax.
We use an asterisk superscript \new{$\mt^\ast$} on a variable $\alphaT$ to represent
a fresh variable uniquely associated with $\alphaT$.
Given some bound size variable $\alphaT$,
$\alphaT^\ast$ represents the proof that the translated size is strictly smaller
than its size bound.

\begin{figure}[h]
\centering
\begin{mathpar}
\fbox{$\subsizeto{\Phi}{s}{s}{\eT}$} \hfill
\inferrule{
  (\bound{\alpha}{s}) \in \Phi
}{
  \subsizeto{\Phi}{\sss{\alpha}}{s}{\alphaT^*}
}
\and
\inferrule{\wf{\Phi}{s}}{
  \subsizeto{\Phi}{\circ}{s}{\app{\baseleq}{\compile{s}}}
}
\and
\inferrule{\wf{\Phi}{s}}{
  \subsizeto{\Phi}{s}{s}{\app{\reflleq}{\compile{s}}}
}
\and
\inferrule{\wf{\Phi}{s}}{
  \subsizeto{\Phi}{s}{\sss{s}}{\app{\sucleq}{\compile{s}}}
}
\hfill
\inferrule{
  \subsizeto{\Phi}{r}{s}{\eT}
}{
  \subsizeto{\Phi}{\sss{r}}{\sss{s}}{\app{\monoT}{\compile{r}}{\compile{s}}{\eT}}
}
\hfill
\inferrule{
  \subsizeto{\Phi}{s_1}{s_2}{\eT_{12}} \\
  \subsizeto{\Phi}{s_2}{s_3}{\eT_{23}}
}{
  \subsizeto{\Phi}{s_1}{s_3}{\app{\transleq}{\compile{s_1}}{\compile{s_2}}{\compile{s_3}}{\eT_{12}}{\eT_{23}}}
}
\end{mathpar}
\caption{Translation of subsizing}
\label{fig:trans:subsize}
\end{figure}

The translation of subsizing judgements, on the other hand,
is a recursive metafunction over the \emph{subsizing derivation}.
\cref{fig:trans:subsize} defines the translation to \CICE terms
by induction on the subsizing rules, which recursively translates subderivations
(omitting irrelevant size well-scopedness premises).

The translation of terms and term environments are similarly defined
as recursive metafunctions over the typing and well-formedness derivations,
denoted by
\mbox{$\typeto{\Phi; \Gamma}{e}{\tau}{\eT}$} and \mbox{$\wfto{\Phi}{\Gamma}{\GammaT}$}
respectively.
However, for concision, we use $\compile{e}$ to mean the translation of $e$
when well typed under the current implicit environments,
$\compile{e}_{\Phi}$ or $\compile{e}_{\Gamma}$ to mean the translation of $e$
with the current implicit environments extended with $\Phi$ or $\Gamma$,
and $\compile{\Gamma}$ to mean the translation of $\Gamma$
when well formed under the current implicit size environment.

\FigTransTermSquash{fig:trans:term}

The translation for base \lang without inductives is given in \cref{fig:trans:term}
in the more concise notation translated terms follow directly from subderivations
and in the usual notation otherwise.
Terms not involving sizes are translated in a straightforward recursive manner.
Unbounded size quantifications and abstractions translate to quantifications and abstractions over $\SizeT$,
while bounded ones have additional quantifications and abstractions over a proof of $\szltT$.
Size applications translate to an additional application to a proof of $\szltT$
when the size abstraction applied is bounded.

Finally, \cref{fig:trans:ind} gives the translation for naturals, well-founded trees,
$\kw{case}$ expressions, and fixpoint expressions from their typing derivations
to a \CICE term (again omitting irrelevant premises).
Aside from fixpoints, the translations are fairly straightforward,
with an additional proof term from subsizing for constructors
and dually an additional abstraction over such terms in the branches of $\tg{case}$ expressions.

\FigTransInd[h]{fig:trans:ind}

Fixpoints in \lang are not translated as fixpoints in \CICE.
Doing so while preserving types would mean that \lang fixpoints
need to be subject to the same guard conditions as \CICE fixpoints,
which is clearly undesirable and not the case in the examples from \cref{subsec:examples}.
Instead, every single \lang fixpoint, regardless of the inductive on which they recur,
is translated to well-founded induction,
which is defined via a \CICE fixpoint on accessibility predicates.
Intuitively, the return type of a fixpoint corresponds to the motive of well-founded induction,
while recursion on a strictly smaller size corresponds to the induction hypothesis,
where the motive holds for all strictly smaller sizes.

The main challenge with this translation is showing that the translations of a fixpoint
and of its $\mu$-reduction are equivalent in \CICE,
because this requires showing that the computational behaviour of well-founded induction
is equivalent to the computational behaviour of the fixpoint.
Once that has been established,
showing type preservation of the translation is more or less going through the motions of the proof,
since the remaining \lang terms translate almost directly to their syntactically corresponding terms.

\section{Metatheory and Type Preservation}

In this section, we outline the architecture of the type preservation proof,
highlighting the interesting cases.
The final proof is done by induction on the typing derivations of \lang,
but since typing depends on subtyping,
subtyping on $\alpha$-cumulativity and closure of reduction,
closure of reduction on reduction, and reduction on substitution,
we need to first prove that they too are all preserved by the translation.
More precisely, we need to show that translation of substitution satisfies a \emph{compositionality} principle,
that the translation of reduced terms are equivalent in \CICE,
and that translated $\alpha$-cumulative terms and subtypes are correspondingly subtypes in \CICE.

These proofs in turn require several well-known metatheoretical properties of \lang and \CICE.
We begin first with these properties before moving on to the preservation proofs.

\subsection{Metatheory}

The metatheoretical properties we need of \lang are
\emph{confluence} (two different reducts of a term further reduce to a coinciding term),
transitivity of $\alpha$-cumulativity and subtyping (since it's not an explicit derivation rule),
\emph{regularity} (well-typedness of the type of a well-typed term),
and \emph{subject reduction} (well-typed terms reduce to well-typed terms).
Significant lemmas are introduced as needed.

\begin{theorem}[Confluence] \label{thm:confluence}
If $\red*{\Phi; \Gamma}{e}{e_1}$ and $\red*{\Phi; \Gamma}{e}{e_2}$
then there is some term $e'$ such that
$\red*{\Phi; \Gamma}{e_1}{e'}$ and $\red*{\Phi; \Gamma}{e_2}{e'}$.
\end{theorem}

\begin{proof}
We use the \emph{Z property}~\citep{Z, confluence} technique to prove confluence,
first defining a \emph{complete developement} judgement $\develop{\Phi; \Gamma}{e}{e'}$,
then showing that it satisfies the Z property:
if $\red{\Phi; \Gamma}{e_1}{e_2}$ and $\red{\Phi; \Gamma}{e_i}{e'_i}$ for $i = 1, 2$,
then $\red*{\Phi; \Gamma}{e_2}{e'_1}$ and $\red*{\Phi; \Gamma}{e'_1}{e'_2}$.
Confluence then follows from the Z property by \opcit.
Complete development is in essence the deterministic, simultaneous reduction
of all visible redexes in a term from the inside out,
and its complete definition is given in \cref{app:develop}.
\end{proof}

\begin{theorem}[Confluence up to $\alpha$-cumulativity] \label{thm:confluence-acum}
If $\acum{e_1}{e_2}$ and $\red*{\Phi; \Gamma}{e_i}{e'_i}$ for $i = 1, 2$,
then $\red*{\Phi; \Gamma}{e'_i}{e''_i}$ for $i = 1, 2$ and $\acum{e''_1}{e''_2}$.
\end{theorem}

\begin{proof}
By induction on the derivation of $\acum{e_1}{e_2}$,
making use of \nameref{thm:confluence}.
\end{proof}

\begin{lemma}[Transitivity of $\alpha$-cumulativity] \label{lem:acum-trans}
If $\acum{e_1}{e_2}$ and $\acum{e_2}{e_3}$ then $\acum{e_1}{e_3}$.
\end{lemma}

\begin{proof}
By nested induction on the derivations of $\acum{e_1}{e_2}$ and $\acum{e_2}{e_3}$.
\end{proof}

\begin{theorem}[Transitivity of subtyping] \label{thm:subtyping-trans}
If $\subtype{\Phi; \Gamma}{\tau_1}{\tau_2}$ and $\subtype{\Phi; \Gamma}{\tau_2}{\tau_3}$
then $\subtype{\Phi; \Gamma}{\tau_1}{\tau_3}$.
\end{theorem}

\begin{proof}
By cases on the derivation of
$\subtype{\Phi; \Gamma}{\tau_1}{\tau_2}$ and
$\subtype{\Phi; \Gamma}{\tau_2}{\tau_3}$,
making use of \nameref{thm:confluence}, \nameref{thm:confluence-acum},
and \nameref{lem:acum-trans}.
\end{proof}

\begin{lemma} \label{lem:wf-subderiv}
If $\wf{\Phi}{\Gamma_1, \define{x}{\tau}{e}, \Gamma_2}$
then $\type{\Phi; \Gamma_1}{e}{\tau}$ is a subderivation.
\end{lemma}

\begin{proof}
By induction on the derivation of $\wf{\Phi}{\Gamma_1, \define{x}{\tau}{e}, \Gamma_2}$.
\end{proof}

\begin{theorem}[Regularity] \label{thm:regularity}
If $\type{\Phi; \Gamma}{e}{\tau}$, then $\wf{}{\Phi}$, $\wf{\Phi}{\Gamma}$,
and $\type{\Phi; \Gamma}{\tau}{U}$.
\end{theorem}

\begin{proof}
By induction on the derivation of $\type{\Phi; \Gamma}{e}{\tau}$,
using \cref{lem:wf-subderiv} for \rref{var} when the variable refers to a definition.
\end{proof}

\begin{lemma}[Subject reduction] \label{lem:SR}
If $\type{\Phi; \Gamma}{e}{\tau}$
and $\red{\Phi; \Gamma}{e}{e'}$
then $\type{\Phi; \Gamma}{e'}{\tau}$.
\end{lemma}

\begin{proof}
By induction on the derivation of $\red{\Phi; \Gamma}{e}{e'}$
and inversion on the derivation of $\type{\Phi; \Gamma}{e}{\tau}$,
using \nameref{thm:subtyping-trans} and \nameref{thm:regularity} as needed.
\end{proof}

\begin{theorem}[Subject reduction] \label{thm:SR}
If $\type{\Phi; \Gamma}{e}{\tau}$
and $\red*{\Phi; \Gamma}{e}{e'}$
then $\type{\Phi; \Gamma}{e'}{\tau}$.
\end{theorem}

\begin{proof}
By induction on the derivation of $\red*{\Phi; \Gamma}{e}{e'}$,
using \nameref{lem:SR} for \rref{red*-once}.
\end{proof}

The metatheory we need of \CICE are \emph{subject equivalence},
or that the two equivalent terms are well typed,
and \emph{consistency}, which take as a postulate.

\begin{theorem}[Subject equivalence] \label{thm:SE}
If $\defeq{\GammaT}{\eT_1}{\eT_2}{\tauT}$
then $\type{\GammaT}{\eT_1}{\tauT}$ and $\type{\GammaT}{\eT_2}{\tauT}$.
\end{theorem}

\begin{postulate}[Consistency] \label{fact:consistency}
There exists no term $\eT$ such that
$\type{\mt}{\eT}{\funtypeT{\PT}{\PropT}{\PT}}$.
\end{postulate}

\subsection{Type Preservation} \label{subsec:preservation}

In the following proofs, we assume well-typedness of the preliminary definitions of \cref{subsec:prelim}.
The first few lemmas we need are compositionality of substitution.
Because there are a number of different notions of substitution
(term substitution into a term,
size substitution into a size,
bounded size substitution into a term,
unbounded size substitution into a term),
we need compositionality lemmas for them all.
The proofs are tedious but straightforward.

\begin{lemma}[Term compositionality] \label{lem:compos-term}
If $\type{\Phi; \Gamma_1, \annot{x}{\tau'}, \Gamma_2}{e}{\tau}$
and $\type{\Phi; \Gamma_1}{e'}{\tau'}$ then
$\subst{\compile{e}}{\xT}{\compile{e'}} = \compile{\subst{e}{x}{e'}}$.
\end{lemma}

\begin{proof}
By induction on the derivation of $\type{\Phi; \Gamma_1, \annot{x}{\tau'}, \Gamma_2}{e}{\tau}$.
\end{proof}

\begin{lemma}[Size compositionality] \label{lem:compos-size}
$\subst{\compile{s}}{\alpha}{\compile{r}} = \compile{\subst{s}{\alpha}{r}}$.
\end{lemma}

\begin{proof}
By induction on the structure of $s$.
\end{proof}

\begin{lemma}[Bounded compositionality of subsizing] \label{lem:compos-subsizing-bounded}
If $\subsizeto{\Phi_1, \bound{\alpha}{r'}, \Phi_2}{s}{r}{\eT}$
and $\subsizeto{\Phi_1}{\sss{s}'}{r'}{\eT'}$
then $\subsizeto{\Phi_1, \subst{\Phi_2}{\alpha}{s'}}{\subst{s}{\alpha}{s'}}{\subst{r}{\alpha}{s'}}{\subst{\eT}{\alphaT, \alphaT^*}{\compile{s'}, \eT'}}$.
\end{lemma}

\begin{proof}
By induction on the derivation of $\subsizeto{\Phi_1, \bound{\alpha}{r'}, \Phi_2}{s}{r}{\eT}$,
making use of \cref{lem:compos-size}.
\end{proof}

\begin{lemma}[Unbounded compositionality of subsizing] \label{lem:compos-subsizing-unbounded}
If $\subsizeto{\Phi_1, \alpha, \Phi_2}{s}{r}{\eT}$ and $\wf{\Phi_1}{s'}$ then
$\subsizeto{\Phi_1, \subst{\Phi_2}{\alpha}{s'}}{\subst{s}{\alpha}{s'}}{\subst{r}{\alpha}{s'}}{\subst{\eT}{\alphaT}{\compile{s'}}}$.
\end{lemma}

\begin{proof}
By induction on the derivation of $\subsizeto{\Phi_1, \alpha, \Phi_2}{s}{r}{\eT}$,
making use of \cref{lem:compos-size}.
\end{proof}

\begin{lemma}[Bounded size compositionality] \label{lem:compos-bounded}
If $\type{\Phi_1, \bound{\alpha}{r}, \Phi_2; \Gamma}{e}{\tau}$
and $\subsizeto{\Phi_1}{\sss{s}}{r}{\eT'}$ then
$\subst{\compile{e}}{\alphaT, \alphaT^*}{\compile{s}, \eT'} = \compile{\subst{e}{\alpha}{s}}$.
\end{lemma}

\begin{proof}
By induction on the derivation of $\type{\Phi_1, \bound{\alpha}{r}, \Phi_2; \Gamma}{e}{\tau}$,
making use of \cref{lem:compos-subsizing-bounded}.
\end{proof}

\begin{lemma}[Unbounded size compositionality] \label{lem:compos-unbounded}
If $\type{\Phi_1, \alpha, \Phi_2; \Gamma}{e}{\tau}$
and $\wf{\Phi_1}{s}$ then
$\subst{\compile{e}}{\alphaT}{\compile{s}} = \compile{\subst{e}{\alpha}{s}}$.
\end{lemma}

\begin{proof}
By induction on the derivation of $\type{\Phi_1, \alpha, \Phi_2; \Gamma}{e}{\tau}$,
making use of \cref{lem:compos-subsizing-unbounded}.
\end{proof}

Now we are able to prove preservation of reduction,
making use of various compositionality lemmas where substitution is involved.

\begin{lemma}[Preservation of reduction] \label{lem:pres-red}
If $\red{\Phi; \Gamma}{e}{e'}$,
$\type{\Phi; \Gamma}{e}{\tau}$, and
$\type{\compile{\Phi}, \compile{\Gamma}}{\compile{e}}{\tauT}$,
then $\defeq{\compile{\Phi}, \compile{\Gamma}}{\compile{e}}{\compile{e'}}{\tauT}$,
where $\type{\Phi; \Gamma}{e'}{\tau}$ is given by \nameref{lem:SR}.
\end{lemma}

\begin{proof}
By induction on the derivation of $\red{\Phi; \Gamma}{e}{e'}$,
using \nameref{lem:compos-term}, \nameref{lem:compos-bounded}, and \nameref{lem:compos-unbounded}
where term and (un)bounded size substitution are involved.
Inversion on the derivation of $\type{\compile{\Phi}, \compile{\Gamma}}{\compile{e}}{\tauT}$
is used to derive the typing premises needed to construct some of the equivalence derivations.
The majority of cases are straightforward; the interesting case is that of fixpoints,
since they reduce to applications of well-founded induction on sizes.
We therefore cover only that case here.
\begin{itemize}[noitemsep, label=\textbf{Case}, leftmargin=*, labelindent=\parindent]
  \item $\red[]{\Phi; \Gamma}{
      \App{(\fix{f}{\alpha}{\sigma}{e})}{s}
    }{
      \subst{e}{\alpha, f}{s, \Fun<{\beta}{s}{\App{(\fix{f}{\alpha}{\sigma}{e})}{\beta}}}
    }$.\\
    By definition of the translation, we have
    $$\type{\compile{\Phi}, \compile{\Gamma}}{
      \begin{aligned}
      & \app{\wfind}{(\funT{\alphaT}{\SizeT}{\compile{\sigma}})}{\\
      & \quad (\funT{\alphaT}{\SizeT}{\funT{\fT}{\funtypeT{\betaT}{\SizeT}{\arrT*{\betaT \szltT \alphaT}{\subst{\compile{\sigma}}{\alphaT}{\betaT}}}}{\compile{e}}})}{\compile{s}}
      \end{aligned}
    }{\tauT},$$
    where $\wfind$ (and $\wfacc$) are defined in \cref{fig:defns}.
    Because of the sheer volume of the terms involved in the translation, $\wfind$, and $\wfacc$,
    Well-typedness premises when applying equivalence rules are omitted,
    especially since all of these terms are known to be well typed.
    Do note, however, that by repeated applications of inversion, we have
    $\subtype{\compile{\Phi}, \compile{\Gamma}}{\app{(\funT{\alphaT}{\SizeT}{\compile{\sigma}})}{\compile{s}}}{\tauT}$. \\[\baselineskip]
    Let $\sigmaT$ be $\funT{\alphaT}{\SizeT}{\compile{\sigma}}$,
    and let $\eT$ be $\funT{\alphaT}{\SizeT}{\funT{\fT}{\funtypeT{\betaT}{\SizeT}{\arrT*{\betaT \szltT \alphaT}{\subst{\compile{\sigma}}{\alphaT}{\betaT}}}}{\compile{e}}}$.
    Liberally using \rref{equiv-trans}, for the left-hand side we then have
    \begin{align*}
    \compile{\Phi}, \compile{\Gamma} &\vdash \app{\wfind}{\sigmaT}{\eT}{\compile{s}} \\
    &\equiv \app{\wfacc}{\sigmaT}{\eT}{\compile{s}}{(\app{\accessible}{\compile{s}})} \qquad \textrm{by $\wfind$ and \rref*{equiv-beta}} \\
    &\equiv \app{(\fixT{1}{\wfacc*}{\funtypeT{\alphaT}{\SizeT}{\arrT*{\app{\AccT}{\alphaT}}{\app{\sigmaT}{\alphaT}}}}{ \\
      &\phantom{\equiv} \quad \funT{\alphaT}{\any}{\funT{\access}{\any}{\app{\eT}{\alphaT}{(\funT{\betaT}{\SizeT}{\funT{\betaT^*}{\betaT \szltT \alphaT}{ \\
      &\phantom{\equiv} \qquad \app{\wfacc*}{\betaT}{(\matchT*{\access}{(\app{\accT}{\pT} \Rightarrow \app{\pT}{\betaT}{\betaT^*})})}}})}}}})}{\compile{s}}{(\app{\accessible}{\compile{s}})} \\
      &\phantom{\equiv} \textrm{by $\wfacc$ and \rref*{equiv-beta}} \\
    &\equiv \subst{\compile{e}}{\alphaT, \fT}{\compile{s}, \funT{\betaT^*}{\betaT \szltT \compile{s}}{ \\
      &\phantom{\equiv} \quad \app{\wfacc}{\sigmaT}{\eT}{\beta}{(\matchT*{\app{\accessible}{\compile{s}}}{(\app{\accT}{\pT} \Rightarrow \app{\pT}{\betaT}{\betaT^*})})}}} \\
      &\phantom{\equiv} \textrm{by \rref*{equiv-mu}, \rref*{equiv-beta}, and $\wfacc$} \\
    &: \app{(\funT{\alphaT}{\SizeT}{\compile{\sigma}})}{\compile{s}}
    \end{align*}
    Meanwhile, for the right-hand side we have
    \begin{align*}
    &\compile{\subst{e}{\alpha, f}{s, \Fun<{\beta}{s}{\App{(\fix{f}{\alpha}{\sigma}{e})}{\beta}}}} \\
    &= \subst{\compile{e}}{\alphaT, \fT}{\compile{s}, \funT{\betaT}{\SizeT}{\funT{\betaT^*}{\betaT \szltT \compile{s}}{\app{\compile{\fix{f}{\alpha}{\sigma}{e}}}{\betaT}}}} \\
      &\phantom{=} \textrm{by \cref{lem:compos-unbounded} and \cref{lem:compos-term}} \\
    &= \subst{\compile{e}}{\alphaT, \fT}{\compile{s}, \funT{\betaT}{\SizeT}{\funT{\betaT^*}{\betaT \szltT \compile{s}}{\app{\wfind}{\sigmaT}{\eT}{\betaT}}}} \textrm{by translation of $\kw{fix}$} \\
    &= \subst{\compile{e}}{\alphaT, \fT}{\compile{s}, \funT{\betaT}{\SizeT}{\funT{\betaT^*}{\betaT \szltT \compile{s}}{\app{\wfacc}{\sigmaT}{\eT}{\betaT}{(\app{\accessible}{\betaT})}}}} \qquad \textrm{by $\wfind$}
    \end{align*}
    Note that the only difference between the left- and right-hand sides now
    is the proof of $\app{\AccT}{\betaT}$ for some $\betaT \szltT \compile{s}$,
    but we know that such proofs are propositionally equal to one another.
    From inversion, we know that $\compile{s}$ is well typed with type $\SizeT$.
    Then we can show that
    \begin{align*}
    &\type{\compile{\Phi}, \compile{\Gamma}, \annotT{\betaT}{\SizeT}, \annotT{\betaT^*}{\betaT \szltT \compile{s}}}{\app{\accIsProp}{\betaT}{(\matchT*{\app{\accessible}{\compile{s}}}{(\app{\accT}{\pT} \Rightarrow \app{\pT}{\betaT}{\betaT^*})})}{(\app{\accessible}{\betaT})}}{\\
    &\phantom{\type{\compile{\Phi}, \compile{\Gamma}, \annotT{\betaT}{\SizeT}, \annotT{\betaT^*}{\betaT \szltT \compile{s}}}{}{}}
    \eq{\matchT*{\app{\accessible}{\compile{s}}}{(\app{\accT}{\pT} \Rightarrow \app{\pT}{\betaT}{\betaT^*})}}{\app{\AccT}{\betaT}}{\app{\accessible}{\betaT}}}.
    \end{align*}
    By \rref{equiv-reflect}, we have
    \begin{align*}
    &\defeq{\compile{\Phi}, \compile{\Gamma}, \annotT{\betaT}{\SizeT}, \annotT{\betaT^*}{\betaT \szltT \compile{s}}}{\matchT*{\app{\accessible}{\compile{s}}}{(\app{\accT}{\pT} \Rightarrow \app{\pT}{\betaT}{\betaT^*})}}{\app{\accessible}{\betaT}}{\app{\AccT}{\betaT}}.
    \end{align*}
    Finally, by \rref{equiv-cong}, we can equate the left- and right-hand sides,
    and by \rref{equiv-conv}, we obtain our goal.
    \begin{align*}
    \defeq{\compile{\Phi}, \compile{\Gamma}}{\compile{\App{(\fix{f}{\alpha}{\sigma}{e})}{s}}}{\compile{\subst{e}{\alpha, f}{s, \Fun<{\beta}{s}{\App{(\fix{f}{\alpha}{\sigma}{e})}{\beta}}}}}{\tauT} \qedhere
    \end{align*}
\end{itemize}
\end{proof}

The remaining preservation proofs are rather straightforward in comparison.
We therefore don't cover any of the cases in detail,
only mentioning where prior lemmas are used in later ones.

\begin{lemma}[Preservation of reflexive, transitive closure of reduction] \label{lem:pres-red*}
If $\red*{\Phi; \Gamma}{e}{e'}$,
$\type{\Phi; \Gamma}{e}{\tau}$, and
$\type{\compile{\Phi}, \compile{\Gamma}}{\compile{e}}{\tauT}$,
then $\defeq{\compile{\Phi}, \compile{\Gamma}}{\compile{e}}{\compile{e'}}{\tauT}$,
where $\type{\Phi; \Gamma}{e'}{\tau}$ is given by \nameref{thm:SR}.
\end{lemma}

\begin{proof}
By induction on the derivation of $\red*{\Phi; \Gamma}{e}{e'}$.
\nameref{lem:pres-red} is used for \rref{red*-once},
while \nameref{thm:SR} and \nameref{thm:SE} are used for \rref{red*-trans}
to ensure that the middle term in transitivity and its translation are well typed.
\end{proof}

\begin{lemma}[Preservation of $\alpha$-cumulativity] \label{lem:pres-acum}
If $\acum{\tau_1}{\tau_2}$,
$\type{\Phi; \Gamma}{\tau_i}{U_i}$ for $i = 1, 2$, and
$\type{\compile{\Phi}, \compile{\Gamma}}{\compile{\tau_i}}{\UT_i}$ for $i = 1, 2$,
then $\subtype{\compile{\Phi}, \compile{\Gamma}}{\compile{\tau_1}}{\compile{\tau_2}}$.
\end{lemma}

\begin{proof}
By induction on the derivation of $\acum{\tau_1}{\tau_2}$
inverting on the derivations of $\type{\Phi; \Gamma}{\tau_i}{U_i}$
and $\type{\compile{\Phi}, \compile{\Gamma}}{\compile{\tau_i}}{\UT_i}$
to apply the induction hypotheses.
\end{proof}

\begin{lemma}[Preservation of subtyping] \label{lem:pres-subtyping}
If $\subtype{\Phi; \Gamma}{\tau_1}{\tau_2}$,
$\type{\Phi; \Gamma}{\tau_i}{U}$ for $i = 1, 2$, and
$\type{\compile{\Phi}, \compile{\Gamma}}{\compile{\tau_i}}{\compile{U}}$ for $i = 1, 2$,
then $\subtype{\compile{\Phi}, \compile{\Gamma}}{\compile{\tau_1}}{\compile{\tau_2}}$.
\end{lemma}

\begin{proof}
By cases on the derivation of $\subtype{\Phi; \Gamma}{\tau_1}{\tau_2}$,
using \nameref{lem:pres-red*}, \nameref{lem:pres-acum},
\nameref{thm:SR}, and \nameref{thm:SE}.
\end{proof}

\begin{lemma}[Preservation of sizes] \label{lem:pres-size}
If $\wf{}{\Phi}$ then $\wf{}{\compile{\Phi}}$; and
if $\wf{\Phi}{s}$ then $\type{\compile{\Phi}}{\compile{s}}{\SizeT}$.
\end{lemma}

\begin{proof}
By mutual induction on the derivations of $\wf{}{\Phi}$ and $\wf{\Phi}{s}$.
\end{proof}

\begin{lemma}[Preservation of subsizing] \label{lem:pres-subsizing}
If $\subsizeto{\Phi}{r}{s}{\eT}$
then $\type{\compile{\Phi}}{\eT}{\compile{r} \szleT \compile{s}}$.
\end{lemma}

\begin{proof}
By induction on the derivation of $\subsizeto{\Phi}{r}{s}{\eT}$.
\end{proof}

\begin{theorem}[Type preservation] \label{thm:type-pres}
If $\wf{\Phi}{\Gamma}$ then $\wf{}{\compile{\Phi}, \compile{\Gamma}}$; and
if $\type{\Phi; \Gamma}{e}{\tau}$ then $\type{\compile{\Phi}, \compile{\Gamma}}{\compile{e}}{\compile{\tau}}$.
\end{theorem}

\begin{proof}
By mutual induction on the derivations of $\wf{\Phi}{\Gamma}$ and $\type{\Phi; \Gamma}{e}{\tau}$,
using \nameref{lem:pres-size} and \nameref{lem:pres-subsizing}
where the translations of sizes, size environments, and subsizing are involved.
\nameref{lem:pres-subtyping} is used for \rref{conv}.
\end{proof}

Finally, the consistency of \lang follows from the consistency of \CICE and type preservation.

\begin{theorem}[Consistency of \lang] \label{thm:consistency}
There exists no term $e$ such that $\type{\mt \mathbin{;} \mt}{e}{\funtypeT{\PT}{\PropT}{\PT}}$.
\end{theorem}

\begin{proof}
Suppose that there were such a term $e$.
By \nameref{thm:type-pres}, we would have that
$\type{\mt}{\compile{e}}{\funtypeT{\PT}{\PropT}{\PT}}$ holds.
However, this contradicts \cref{fact:consistency},
so there must not be such a term.
\end{proof}

\section{Discussion and Conclusions} \label{sec:discussion}

\lang is neither perfect nor complete:
the nature of the syntatic model requires that inductives live in a universe
higher than that in which their corresponding unsized types would live,
and it's missing features such as an infinite size.
In this chapter, we discuss some of these shortcomings
and give some possible directions for future work based on them.

\subsection{The Infinite Size} \label{subsec:infinity}

In prior sized type systems, the infinite size $\infty$ is applied to sized inductive types
to represent a ``full'' inductive type encompassing that inductive at all sizes,
which essentially corresponds to the usual unsized inductive.
Whereas an inductive of some size $s$ can be thought of as the type of elements
with at most $s$ many layers of constructors,
the full inductive can be thought of as the type of elements with any number of layers of constructors.

The infinite size is characterized by its subsizing behaviour:
$\subsize*{s}{\infty}$ holds for \emph{any} $s$.
This includes its own successor, \ie $\subsize*{\sss{\infty}}{\infty}$,
leading to non--well-founded sequences of strictly ``decreasing'' sizes:
$\dots < \infty < \infty < \infty$.
Naturally, there's no way to model the infinite size as an element of $\SizeT$
given that we've shown that $\SizeT$s \emph{are} well founded.
If there were, then it'd be possible to prove an inconsistency.
Let $\inftyT$ be the translation of $\infty$,
and let $\inftyltinfty$ be the translation of $\subsize*{\sss{\infty}}{\infty}$.
\begin{align*}
&\LetT{\tg{{\neg}wf\inftyT}}{\arrT*{\app{\AccT}{\inftyT}}{\botT}}{\funT{\mathit{acc}}{\app{\AccT}{\inftyT}}{\matchT*{\mathit{acc}}{\app{\accT}{p} \RightarrowT \app{p}{\inftyT}{\inftyltinfty}}}} \\
&\LetT{\tg{false}}{\botT}{\app{\tg{{\neg}wf\inftyT}}{(\app{\accessible}{\inftyT})}}
\end{align*}

In set-theoretic models of sized type systems with an infinite size,
sizes are modelled as set-theoretic (transfinite) ordinals,
the infinite size isn't modelled as a single ordinal;
instead, for each use of the infinite size,
its set-theoretic interpretation is an ordinal that is ``large enough'' in that context.
For instance, the interpretation of the infinite size of $\N{\infty}$
is the first limit ordinal $\omega$.

This strategy doesn't adapt well to \lang with its size abstractions and syntactic model,
since it requires a non-local translation of sizes.
For instance, given the size application $\App{e}{\infty}$,
what $\infty$ translates to would hypothetically depend on what $e$ translates to,
and likely require further static analysis of $\compile{e}$ beyond a simple translation
over typing derivations.

Since the motivation for having the infinite size is specifically for representing full inductives,
one alternative could be to define the full inductive separately
and provide functions to and from the corresponding sized inductive,
such as the following for $\W*$.
\begin{align*}
& \data{\App{\app{\W*}{(\annot{A}{\Type{i}})}{(\annot{B}{\arr*{A}{\Type{i}}})}}{\infty}}{\Type{i+1}} \\
& \quad \annot{\constr{sup\infty}}{\arr{x}{A}{\arr*{(\arr*{\app{B}{x}}{\app{\App{\W*}{\infty}}{A}{B}})}{\App{\app{\W*}{A}{B}}{\infty}}}}
\end{align*}

Defining a function from $\W{x}{A}{B}{s}$ to $\W{x}{A}{B}{\infty}$ is trivial,
since we're discarding size information.
What about going from $\W{x}{A}{B}{\infty}$ to $\W{x}{A}{B}{s}$?
What should $s$ be?
The size algebra could be augmented to be able to represent transfinite ordinals
so that $s$ is again a size that is ``large enough'',
but we can hardly expect programmers to manipulate ordinals,
and we conjecture that we would lose any hope of deciding $\subsize*{}{}$
without any user intervention.

The key insight is that what's important about an element of a full inductive
isn't its precise size and depth of constructors,
but merely that it has \emph{some} unknown size.
Another alternative to the infinite size, then, could be to represent a full inductive
as an existentially-quantified sized inductive,
\ie $\Pairtype{\alpha}{\N{\alpha}}$ and $\Pairtype{\alpha}{\W{x}{A}{B}{\alpha}}$.
Existential quantification is defined using an encoding for \emph{weak dependent pairs},
whose components can't be projected out, since the first component is a size and not a term.
We omit type annotations when deducible from context.
\begin{align*}
\Pairtype{\alpha}{\tau} &= \funtype{\sigma}{\Type{i}}{\arr*{(\Funtype{\alpha}{(\arr*{\tau}{\sigma})})}{\sigma}} \\
\Pair{s}{e}_{\Pairtype{\alpha}{\tau}} &= \fun{\sigma}{\Type{i}}{\fun{f}{\Funtype{\alpha}{(\arr*{\tau}{\sigma})}}{\app{\App{f}{s}}{e}}} \\
\unpair{\alpha}{x}{\alpha}{\tau}{e_1}{\annot{e_2}{\sigma}} &= \app{e_1}{\sigma}{(\Fun{\alpha}{\fun{x}{\tau}{e_2}})}
\end{align*}

There is still a limitation similar to that in \cref{subsec:examples:limitations}
when trying to represent the constructors of full inductives.
For $\Pairtype{\alpha}{\W{x}{A}{B}{\alpha}}$, we need a ``constructor'' of the following type.
\begin{align*}
\annot{\const{sup'}}{\arr{x}{A}{\arr*{(\arr*{B}{\Pairtype{\alpha}{\W{x}{A}{B}{\alpha}}})}{\Pairtype{\alpha}{\W{x}{A}{B}{\alpha}}}}}
\end{align*}
All we need is a function
$$\annot{\const{ac}}{\arr{x}{A}{\arr*{(\arr*{B}{\Pairtype{\alpha}{\W{x}{A}{B}{\alpha}}})}{\Pairtype{\alpha}{(\arr*{B}{\W{x}{A}{B}{\alpha}})}}}}$$
and we're good to go.
\begin{align*}
\app{\const{sup'}}{x}{f} =
\unpair*{\alpha}{f'}{\app{\const{ac}}{x}{f}}{\Pair{\sss{\alpha}}{\sup{x}{A}{B}{\sss{\alpha}}{\alpha}{x}{f'}}}
\end{align*}

Unfortunately, as the name might suggest,
$\const{ac}$ is an instance of the axiom of choice\punctstack{,}%
\footnote{The ``choice'' made here is the existentially-quantified size in the consequent:
the axiom asserts that there's always a way to choose a size such that
all of the well-founded trees returned have that size.}
which for weak existentials is nonconstructive,
so there's no hope of implementing $\const{ac}$.
However, if we take weak existentials as a primitive of \lang
rather than being defined as an encoding,
and model them by strong dependent pairs in \CICE,
then the translation of $\const{ac}$ \emph{can} be implemented as a function,
also making use of the limit operator.
In other words, the syntactic model justifies the axiom $\const{ac}$ in the source.
The mechanization of $\compile{\const{ac}}$ in Agda and Coq are given in
\cref{app:mechanization:agda:W} and \cref{app:mechanization:coq:W}, respectively.

Nevertheless, $\const{ac}$ remains a noncomputing axiom
unless the size algebra is augmented to include a limit operator
and size expressions consequently treated as regular terms.
There is no way to represent generalized full inductives as
sized inductives existentially quantified by sizes
and faithfully define their constructors without losing either
decidability of subsizing from exposing the underlying ordinal representation of sizes,
or canonicity from the inclusion of a noncomputing axiom.
Representing ordinary full inductives (namely naturals) this way,
however, has been previously explored~\citep{guarded, modal-sizes}.

\subsection{Universe Levels of Inductives} \label{subsec:universe-levels}

The universes in which the type of natural numbers and W types in \lang live
are, in a sense, one level higher than is usually expected;
their typing rules are reproduced below.
$\N*$ is typically in $\Type{0}$,
while $\W*$ is typically in $U$ rather than $\axioms{U}$.
\begin{mathpar}
\inferrule[\rref*{nat}]{
  \wf{\Phi}{\Gamma} \\
  \wf{\Phi}{s}
}{
  \infer{\Phi; \Gamma}{\N{s}}{\Type{1}}
}
\and
\inferrule[\rref*{wft}]{
  \wf{\Phi}{s} \\
  \infer{\Phi; \Gamma}{\sigma}{U} \\\\
  \infer{\Phi; \Gamma, \annot{x}{\sigma}}{\tau}{U}
}{
  \infer{\Phi; \Gamma}{\W{x}{\sigma}{\tau}{s}}{\axioms{U}}
}
\end{mathpar}

This is due to how the translation is defined:
$\NatT$ and $\WT$ have a $\SizeT$ as parameter,
the $\limT$ operator quantifies over the type of the domain of its function argument,
and that type must be ``large'' enough to accomodate the correct type.
For $\app{\WT}{\sigmaT}{\tauT}{\sT}$,
the domain is $\app{\tauT}{\aT}$ for some $\aT$\punctstack{,}%
\footnote{For generalized inductives, given a recursive argument in the form of a function,
the domain of $\limT$ should emcompass the domain of that function.}
as is the case for the definition of $\compile{\const{ac}}$,
so if its universe is $\TypeT{\iT}$,
then the universe of the type of $\sT$ must be $\TypeT{\tg{i+1}}$,
according to the definition of $\SizeT$, reproduced below.
\begin{align*}
&\dataT{\SizeT}{\TypeT{\tg{i+1}}} \\
&\quad \annotT{\sucT}{\arrT*{\SizeT}{\SizeT}} \\
&\quad \annotT{\limT}{\funtypeT{A}{\TypeT{\iT}}{\arrT*{(\arrT*{A}{\SizeT})}{\SizeT}}}
\end{align*}

This is a nonnegotiable condition:
defining $\SizeT$ to be in the same universe as that which $\limT$ quantifies over
leads to $\szltT$ no longer being well founded,
since in this hypothetical scenario $\SizeT$ itself could be applied to $\limT$
to define an $\inftyT$ size.
\begin{align*}
&\LetT{\inftyT}{\SizeT}{\app{\limT}{\SizeT}{(\funT{\xT}{\SizeT}{\xT})}} \\
&\LetT{\inftyltinfty}{\inftyT \szltT \inftyT}{\app{\coconeT}{\SizeT}{(\app{\sucT}{\inftyT})}{(\funT{\xT}{\SizeT}{\xT})}{(\app{\sucT}{\inftyT})}{(\app{\reflleq}{(\app{\sucT}{\inftyT})})}}
\end{align*}

One way to ``shrink'' the universe of $\SizeT$, so to speak,
could be to parametrize it over the type over which $\limT$ currently quantifies over,
yielding the following inductive definition.
\begin{align*}
&\dataT{\app{\SizeT}{(\annotT{A}{\TypeT{\iT}})}}{\TypeT{\iT}} \\
&\quad \annotT{\sucT}{\arrT*{\SizeT}{\SizeT}} \\
&\quad \annotT{\limT}{\arrT*{(\arrT*{A}{\SizeT})}{\SizeT}}
\end{align*}

The problem with this alternative is that for $\app{\WT}{\sigmaT}{\tauT}$,
the parameter of its size parameter would be $\app{\tauT}{\aT}$,
where $\annotT{\aT}{\sigmaT}$ is the third formal argument to its constructor $\supT$,
but this argument is only part of the constructor, not the type.
The intuition is that the parametrized $\SizeT$ is too restrictive
and there aren't ``enough'' sizes to cover all well-founded trees of any particular type.

In a sense, it's reasonable to expect that $\SizeT$ needs to live in a larger universe.
The role of $\SizeT$ is to represent all of the possible sizes of a given full inductive,
so we would expect $\SizeT$ to be just as large as the inductive itself.
Aside from situations where the inductive is impredicative, \ie in $\PropT$,
parametrizing over $\SizeT$ then necessarily requires moving up a universe
so that it's not essentially quantifying over itself.

If having inductive types in the correct universe were more important than
the potential expressibility of infinitude,
we could eliminate $\limT$ altogether and model sizes as ordinary naturals.
Then $\SizeT$ would live in $\Type{0}$, $\NatT$ and $\WT$ would live in the correct universes,
as well as $\N*$ and $\W*$.
Although this would affect the judgements of both \lang and \CICE,
this wouldn't affect the proofs, since the changes are the same for both languages.
However, this does mean giving up $\const{ac}$
and likely any hope of defining any infinitary constructs like $\const{\omega}$
when using this model.

\section{Future Directions}

There is still a long way to go towards a sized dependent type system
that is expressive, consistent, and useable.
We've seen from \cref{subsec:examples} that working without an infinite size
or full inductives is verbose at best,
requiring tedious manipulation of existentially-quantified sizes,
and impossible at worse, in the case of $\const{\omega}$,
while attempts at including them either involve inconsistency, undecidability, or noncanonicity.
Alternate approaches need to be investigated to circumvent these problems.

While the issue with the universe levels of inductive types appears to be cosmetic
and not affect any important significant metatheoretical properties,
it might be a mere symptom of a larger issue with how sizes are modelled,
especially since prior sized type systems shown to be consistent
haven't required any similar restrictions.
One possibility is to take inspiration from Agda and
make $\SizeT$ live in a universe $\SizeUnivT$ independent of $\TypeT{}$,
but the consistency of arbitrarily adding a new universe is questionable,
especially when $\SizeUnivT$ needs to be slightly impredicative:
$\arrT*{\TypeT{}}{\SizeT}$ would remain in $\SizeUnivT$
(as would be the case for $\limT$)
and $\funtypeT{\alpha}{\SizeT}{\N{\alpha}}$ would remain in $\TypeT{0}$
(as would be the case for $\zeroT$).

Additionally, \lang is missing some important but comparatively simple features
from the perspective of the syntactic model, namely inductive types in general.
They can be added directly to \lang as a generalization of the existing inductives,
and we conjecture that the type preservation proofs would be a straightforward extension.
Alternatively, since there already are well-founded trees,
\lang could instead be augmented with dependent pairs and an equality type,
both of which we also conjecture to be straightforward;
then mutual inductives, indexed inductives \citep{whynotW}, some nested inductives \citep{barras},
and even some inductive--inductive types \citep{ind-ind}
can be encoded, along with their induction principles, using W types and propositional equality.

Dually, \lang is also missing coinductive types.
The main barrier to a syntactic model of \lang with any coinductive types
is the lack of an established target type theory with coinductive types
with nice properties.
Whereas pCIC and pCuIC lack them,
MetaCoq mechanizes Coq-style coinductive types,
but necessarily cannot prove its own consistency.
Meanwhile, older CICs with coinductive types and cofixpoints guarded by constructors
(\eg \citet{guard}) have issues with subject reduction~\citep{coind-SR},
which is a necessary property for proving type preservation in \cref{subsec:preservation}.
Another potential issue is the translation of cofixpoints into applications of
a well-founded induction principle, which is suspicious and also requires further investigation.

\hfill

In terms of metatheory, \lang is missing proof of two desirable properties:
strong normalization and decidability of type checking.
Because \nameref{lem:pres-red} only preserves reduction up to equivalence in \CICE,
not reduction, normalization doesn't hold immediately from the translation.
In particular, the proof of \nameref{lem:pres-red} uses equality reflection
to obtain an equivalence between two proofs of accessibility of a size,
and \citet{SProp} show that this equivalence can lead to nonnormalization in
particular inconsistent environments,
as is the case for $\tg{false}$.

Even so, we conjecture that translation of a \lang term will never lead to
a \CICE term with a subterm in an appropriate inconsistent environment that leads to nonormalization,
because the translation will never yield an assumption of the form $\sT \szltT \sT$.
By inspection of the translation, the only fitting subsizing relation possible is of the form $\alphaT \szltT \alphaT$,
and the only possible sources $\Funtype<{\alpha}{\alpha}{\tau}$, $\Fun<{\alpha}{\alpha}{e}$ are ill-scoped.
Since fixpoints only reduce when applied to sizes that have a subsize
and we can't obtain infinitely decreasing chains of subsizes,
they can't unfold endlessly and must stop eventually.

Decidability of type checking, meanwhile, rests not only on strong normalization,
but also on decidability of subsizing, \ie deciding whether $\bound{r}{s}$ holds.
The trickiest rule is transitivity of subsizing,
but we conjecture that subsizing, too, is decidable,
as the size environment only contains a finite number of bounded size variables,
and sizes can only have a finite number of successor operators applied.
Moreover, Agda has transitive subsizing, and so far hasn't had issues with its decidability.

\hfill

As for implementation, the consistency of \lang shows that sized types for inductives in Agda
are likely to be consistent were the infinite size removed.
This immediately poses a significant problem to Agda's standard library,
as 20 of the 33 files that use sized types in version 1.7.1 of the standard library
make use of the infinite size.
Furthermore, 30 of those files deal with coinductives rather than inductives,
and it remains to be seen whether \lang with coinductives really is consistent as well.
The addition of the $\const{ac}$ axiom to recover some of the expressivity of the infinite size
isn't ideal for a proof assistant either, since it would block computation.

\section{Conclusion}

In this paper, we introduced \lang, a sized dependent type theory
with higher-rank size quantification and bounded size quantification,
but without an infinite size that is strictly greater than itself,
along with sized naturals and well-founded trees.
we gave examples of programming with sized inductives to write recursive programs
that would have otherwise not passed the usual syntactic guard checks,
as well as limitations of programming without the infinite size.
We then reduced the consistency of \lang to that of \CICE,
an extensional dependent type theory, by translating \lang terms to \CICE terms
and proving that the translation is type preserving.
Finally, we discussed the tradeoffs in the design of \lang for this syntactic model to work,
namely the lack of an infinite size and raising the universe levels of the inductives,
as well as some potential extensions and future work.

%% Acknowledgments
\begin{acks}
%\grantsponsor{id}{name}{url}
%\grantnum{id}{num}
This research was supported by the Canada Graduate Scholarships -- Master’s (CGS M) programme.
Cette recherche a \'et\'e financ\'ee par le Programme de bourses d'\'etudes sup\'erieures
du Canada au niveau de la maitrise (BESC M).
\end{acks}

%% Bibliography
\bibliography{biblio}

%% Appendix
\appendix
\section{Appendix}

\newcommand{\FigCongRed}[1]{
  \small
  \begin{mathpar}
  \inferrule{
    \red{\Phi; \Gamma}{\sigma}{\sigma'}
  }{
    \red{\Phi; \Gamma}{\funtype{x}{\sigma}{\tau}}{\funtype{x}{\sigma'}{\tau}}
  }
  \and
  \inferrule{
    \red{\Phi; \Gamma, \annot{x}{\sigma}}{\tau}{\tau'}
  }{
    \red{\Phi; \Gamma}{\funtype{x}{\sigma}{\tau}}{\funtype{x}{\sigma}{\tau'}}
  }
  \and
  \inferrule{
    \red{\Phi; \Gamma}{\sigma}{\sigma'}
  }{
    \red{\Phi; \Gamma}{\fun{x}{\sigma}{e}}{\fun{x}{\sigma'}{e}}
  }
  \and
  \inferrule{
    \red{\Phi; \Gamma, \annot{x}{\sigma}}{e}{e'}
  }{
    \red{\Phi; \Gamma}{\fun{x}{\sigma}{e}}{\fun{x}{\sigma}{e'}}
  }
  \and
  \inferrule{
    \red{\Phi; \Gamma}{e_1}{e'_1}
  }{
    \red{\Phi; \Gamma}{\app{e_1}{e_2}}{\app{e'_1}{e_2}}
  }
  \and
  \inferrule{
    \red{\Phi; \Gamma}{e_2}{e'_2}
  }{
    \red{\Phi; \Gamma}{\app{e_1}{e_2}}{\app{e_1}{e'_2}}
  }
  \and
  \inferrule{
    \red{\Phi; \Gamma}{\sigma}{\sigma'}
  }{
    \red{\Phi; \Gamma}{\letin{x}{\sigma}{e_1}{e_2}}{\letin{x}{\sigma'}{e_1}{e_2}}
  }
  \and
  \inferrule{
    \red{\Phi; \Gamma}{e_1}{e'_1}
  }{
    \red{\Phi; \Gamma}{\letin{x}{\sigma}{e_1}{e_2}}{\letin{x}{\sigma}{e'_1}{e_2}}
  }
  \and
  \inferrule{
    \red{\Phi; \Gamma, \define{x}{\sigma}{e_1}}{e_2}{e'_2}
  }{
    \red{\Phi; \Gamma}{\letin{x}{\sigma}{e_1}{e_2}}{\letin{x}{\sigma}{e_1}{e'_2}}
  }
  \\
  \inferrule{
    \red{\Phi, \alpha; \Gamma}{\tau}{\tau'}
  }{
    \red{\Phi; \Gamma}{\Funtype{\alpha}{\tau}}{\Funtype{\alpha}{\tau'}}
  }
  \and
  \inferrule{
    \red{\Phi, \alpha; \Gamma}{e}{e'}
  }{
    \red{\Phi; \Gamma}{\Fun{\alpha}{e}}{\Fun{\alpha}{e'}}
  }
  \and
  \inferrule{
    \red{\Phi; \Gamma}{e}{e'}
  }{
    \red{\Phi; \Gamma}{\App{e}{s}}{\App{e'}{s}}
  }
  \and
  \inferrule{
    \red{\Phi, \bound{\alpha}{s}; \Gamma}{\tau}{\tau'}
  }{
    \red{\Phi; \Gamma}{\Funtype<{\alpha}{s}{\tau}}{\Funtype<{\alpha}{s}{\tau'}}
  }
  \and
  \inferrule{
    \red{\Phi, \bound{\alpha}{s}; \Gamma}{e}{e'}
  }{
    \red{\Phi; \Gamma}{\Fun<{\alpha}{s}{e}}{\Fun<{\alpha}{s}{e'}}
  }
  \and
  \inferrule{
    \red{\Phi, \alpha; \Gamma}{\sigma}{\sigma'}
  }{
    \red{\Phi; \Gamma}{\fix{f}{\alpha}{\sigma}{e}}{\fix{f}{\alpha}{\sigma'}{e}}
  }
  \and
  \inferrule{
    \red{\Phi, \alpha; \Gamma, \annot{f}{\Funtype<{\beta}{\alpha}{\subst{\sigma}{\alpha}{\beta}}}}{e}{e'}
  }{
    \red{\Phi; \Gamma}{\fix{f}{\alpha}{\sigma}{e}}{\fix{f}{\alpha}{\sigma}{e'}}
  }
  \and
  \inferrule{
    \red{\Phi; \Gamma}{\sigma}{\sigma'}
  }{
    \red{\Phi; \Gamma}{\W{x}{\sigma}{\tau}{s}}{\W{x}{\sigma'}{\tau}{s}}
  }
  \and
  \inferrule{
    \red{\Phi; \Gamma, \annot{x}{\sigma}}{\tau}{\tau'}
  }{
    \red{\Phi; \Gamma}{\W{x}{\sigma}{\tau}{s}}{\W{x}{\sigma}{\tau'}{s}}
  }
  \and
  \inferrule{
    \red{\Phi; \Gamma}{e}{e'}
  }{
    \red{\Phi; \Gamma}{\succ{r}{s}{e}}{\succ{r}{s}{e'}}
  }
  \end{mathpar}
}

\newcommand{\FigCongRedCase}[1]{
  \small
  \begin{mathpar}
  \setlength{\jot}{-1.5pt}
  \inferrule{
    \red{\Phi; \Gamma}{\sigma}{\sigma'}
  }{
    \red{\Phi; \Gamma}{\sup{x}{\sigma}{\tau}{s}{r}{e_1}{e_2}}{\sup{x}{\sigma'}{\tau}{s}{r}{e_1}{e_2}}
  }
  \quad
  \inferrule{
    \red{\Phi; \Gamma, \annot{x}{\sigma}}{\tau}{\tau'}
  }{
    \red{\Phi; \Gamma}{\sup{x}{\sigma}{\tau}{s}{r}{e_1}{e_2}}{\sup{x}{\sigma}{\tau'}{s}{r}{e_1}{e_2}}
  }
  \and
  \inferrule{
    \red{\Phi; \Gamma}{e_1}{e'_1}
  }{
    \red{\Phi; \Gamma}{\sup{x}{\sigma}{\tau}{s}{r}{e_1}{e_2}}{\sup{x}{\sigma}{\tau}{s}{r}{e'_1}{e_2}}
  }
  \quad
  \inferrule{
    \red{\Phi; \Gamma}{e_2}{e'_2}
  }{
    \red{\Phi; \Gamma}{\sup{x}{\sigma}{\tau}{s}{r}{e_1}{e_2}}{\sup{x}{\sigma}{\tau}{s}{r}{e_1}{e'_2}}
  }
  \and
  \inferrule{
    \red{\Phi; \Gamma}{e}{e'}
  }{
    \red{\Phi; \Gamma}{
      \begin{aligned}
      &\match{e}{\fun{x}{\N{s}}{P}}{ \\
      &\quad \App{\zero*}{\alpha} \Rightarrow e_z \\
      &\quad \app{\App{\succ*}{\beta}}{z} \Rightarrow e_s}
      \end{aligned}
    }{
      \begin{aligned}
      &\match{e'}{\fun{x}{\N{s}}{P}}{ \\
      &\quad \App{\zero*}{\alpha} \Rightarrow e_z \\
      &\quad \app{\App{\succ*}{\beta}}{z} \Rightarrow e_s}
      \end{aligned}
    }
  }
  \and
  \inferrule{
    \red{\Phi; \Gamma, \annot{x}{\N{s}}}{P}{P'}
  }{
    \red{\Phi; \Gamma}{
      \begin{aligned}
      &\match{e}{\fun{x}{\N{s}}{P}}{ \\
      &\quad \App{\zero*}{\alpha} \Rightarrow e_z \\
      &\quad \app{\App{\succ*}{\beta}}{z} \Rightarrow e_s}
      \end{aligned}
    }{
      \begin{aligned}
      &\match{e}{\fun{x}{\N{s}}{P'}}{ \\
      &\quad \App{\zero*}{\alpha} \Rightarrow e_z \\
      &\quad \app{\App{\succ*}{\beta}}{z} \Rightarrow e_s}
      \end{aligned}
    }
  }
  \and
  \inferrule{
    \red{\Phi, \bound{\alpha}{s}; \Gamma}{e_z}{e'_z}
  }{
    \red{\Phi; \Gamma}{
      \begin{aligned}
      &\match{e}{\fun{x}{\N{s}}{P}}{ \\
      &\quad \App{\zero*}{\alpha} \Rightarrow e_z \\
      &\quad \app{\App{\succ*}{\beta}}{z} \Rightarrow e_s}
      \end{aligned}
    }{
      \begin{aligned}
      &\match{e}{\fun{x}{\N{s}}{P}}{ \\
      &\quad \App{\zero*}{\alpha} \Rightarrow e'_z \\
      &\quad \app{\App{\succ*}{\beta}}{z} \Rightarrow e_s}
      \end{aligned}
    }
  }
  \and
  \inferrule{
    \red{\Phi, \bound{\beta}{s}; \Gamma, \annot{z}{\N{\beta}}}{e_s}{e'_s}
  }{
    \red{\Phi; \Gamma}{
      \begin{aligned}
      &\match{e}{\fun{x}{\N{s}}{P}}{ \\
      &\quad \App{\zero*}{\alpha} \Rightarrow e_z \\
      &\quad \app{\App{\succ*}{\beta}}{z} \Rightarrow e_s}
      \end{aligned}
    }{
      \begin{aligned}
      &\match{e}{\fun{x}{\N{s}}{P}}{ \\
      &\quad \App{\zero*}{\alpha} \Rightarrow e_z \\
      &\quad \app{\App{\succ*}{\beta}}{z} \Rightarrow e'_s}
      \end{aligned}
    }
  }
  \and
  \inferrule{
    \red{\Phi; \Gamma}{e}{e'}
  }{
    \red{\Phi; \Gamma}{
      \begin{aligned}
      &\match{e}{\fun{x}{\W{y}{\sigma}{\tau}{s}}{P}}{ \\
      &\quad \app{\App{\sup*}{\alpha}}{z_1}{z_2} \Rightarrow e_s}
      \end{aligned}
    }{
      \begin{aligned}
      &\match{e'}{\fun{x}{\W{y}{\sigma}{\tau}{s}}{P}}{ \\
      &\quad \app{\App{\sup*}{\alpha}}{z_1}{z_2} \Rightarrow e_s}
      \end{aligned}
    }
  }
  \and
  \inferrule{
    \red{\Phi; \Gamma, \annot{x}{\W{y}{\sigma}{\tau}{s}}}{P}{P'}
  }{
    \red{\Phi; \Gamma}{
      \begin{aligned}
      &\match{e}{\fun{x}{\W{y}{\sigma}{\tau}{s}}{P}}{ \\
      &\quad \app{\App{\sup*}{\alpha}}{z_1}{z_2} \Rightarrow e_s}
      \end{aligned}
    }{
      \begin{aligned}
      &\match{e}{\fun{x}{\W{y}{\sigma}{\tau}{s}}{P'}}{ \\
      &\quad \app{\App{\sup*}{\alpha}}{z_1}{z_2} \Rightarrow e_s}
      \end{aligned}
    }
  }
  \and
  \inferrule{
    \red{\Phi, \bound{\alpha}{s}; \Gamma, \annot{z_1}{\sigma}, \annot{z_2}{\arr*{\subst{\tau}{y}{z_1}}{\W{y}{\sigma}{\tau}{\alpha}}}}{e_s}{e'_s}
  }{
    \red{\Phi; \Gamma}{
      \begin{aligned}
      &\match{e}{\fun{x}{\W{y}{\sigma}{\tau}{s}}{P}}{ \\
      &\quad \app{\App{\sup*}{\alpha}}{z_1}{z_2} \Rightarrow e_s}
      \end{aligned}
    }{
      \begin{aligned}
      &\match{e}{\fun{x}{\W{y}{\sigma}{\tau}{s}}{P}}{ \\
      &\quad \app{\App{\sup*}{\alpha}}{z_1}{z_2} \Rightarrow e'_s}
      \end{aligned}
    }
  }
  \end{mathpar}
}

\newcommand{\FigCongTakahashi}[1]{
  \small
  \begin{mathpar}
  \setlength{\jot}{-1.5pt}
  \inferrule{
    \develop{\Phi; \Gamma}{\sigma}{\sigma'} \\
    \develop{\Phi; \Gamma, \annot{x}{\sigma}}{\tau}{\tau'}
  }{
    \develop{\Phi; \Gamma}{\funtype{x}{\sigma}{\tau}}{\funtype{x}{\sigma'}{\tau}}
  }
  \and
  \inferrule{
    \develop{\Phi; \Gamma}{\sigma}{\sigma'} \\
    \develop{\Phi; \Gamma, \annot{x}{\sigma}}{e}{e'}
  }{
    \develop{\Phi; \Gamma}{\fun{x}{\sigma}{e}}{\fun{x}{\sigma'}{e}}
  }
  \and
  \inferrule{
    \develop{\Phi; \Gamma}{e_1}{e'_1} \\
    \develop{\Phi; \Gamma}{e_2}{e'_2}
  }{
    \develop{\Phi; \Gamma}{\app{e_1}{e_2}}{\app{e'_1}{e'_2}}
  }
  \and
  \inferrule{
    \develop{\Phi, \alpha; \Gamma}{\tau}{\tau'}
  }{
    \develop{\Phi; \Gamma}{\Funtype{\alpha}{\tau}}{\Funtype{\alpha}{\tau'}}
  }
  \and
  \inferrule{
    \develop{\Phi, \alpha; \Gamma}{e}{e'}
  }{
    \develop{\Phi; \Gamma}{\Fun{\alpha}{e}}{\Fun{\alpha}{e'}}
  }
  \and
  \inferrule{
    \develop{\Phi; \Gamma}{e}{e'}
  }{
    \develop{\Phi; \Gamma}{\App{e}{s}}{\App{e'}{s}}
  }
  \and
  \inferrule{
    \develop{\Phi, \bound{\alpha}{s}; \Gamma}{\tau}{\tau'}
  }{
    \develop{\Phi; \Gamma}{\Funtype<{\alpha}{s}{\tau}}{\Funtype<{\alpha}{s}{\tau'}}
  }
  \and
  \inferrule{
    \develop{\Phi, \bound{\alpha}{s}; \Gamma}{e}{e'}
  }{
    \develop{\Phi; \Gamma}{\Fun<{\alpha}{s}{e}}{\Fun<{\alpha}{s}{e'}}
  }
  \and
  \inferrule{
    \develop{\Phi, \alpha; \Gamma}{\tau}{\tau'} \\\\
    \develop{\Phi, \alpha; \Gamma, \annot{f}{\Funtype<{\beta}{\alpha}{\subst{\tau}{\alpha}{\beta}}}}{e}{e'}
  }{
    \develop{\Phi; \Gamma}{\fix{f}{\alpha}{\tau}{e}}{\fix{f}{\alpha}{\tau'}{e}}
  }
  \and
  \inferrule{
    \develop{\Phi; \Gamma}{\sigma}{\sigma'} \\
    \develop{\Phi; \Gamma, \annot{x}{\sigma}}{\tau}{\tau'}
  }{
    \develop{\Phi; \Gamma}{\W{x}{\sigma}{\tau}{s}}{\W{x}{\sigma'}{\tau'}{s}}
  }
  \and
  \inferrule{
    \develop{\Phi; \Gamma}{e}{e'}
  }{
    \develop{\Phi; \Gamma}{\succ{r}{s}{e}}{\succ{r}{s}{e'}}
  }
  \and
  \inferrule{
    \develop{\Phi; \Gamma}{\sigma}{\sigma'} \\
    \develop{\Phi; \Gamma, \annot{x}{\sigma}}{\tau}{\tau'} \\\\
    \develop{\Phi; \Gamma}{e_1}{e'_1} \\
    \develop{\Phi; \Gamma}{e_2}{e'_2}
  }{
    \develop{\Phi; \Gamma}{\sup{x}{\sigma}{\tau}{r}{s}{e_1}{e_2}}{\sup{x}{\sigma'}{\tau'}{r}{s}{e'_1}{e'_2}}
  }
  \and
  \inferrule{
    \develop{\Phi; \Gamma}{e}{e'} \\
    \develop{\Phi; \Gamma, \annot{x}{\N{s}}}{P}{P'} \\\\
    \develop{\Phi, \bound{\alpha}{s}; \Gamma}{e_z}{e'_z} \\
    \develop{\Phi, \bound{\beta}{s}; \Gamma, \annot{z}{\N{\beta}}}{e_s}{e'_s}
  }{
    \develop{\Phi; \Gamma}{
      \begin{aligned}
      &\match{e}{\fun{x}{\N{s}}{P}}{ \\
      &\quad \App{\zero*}{\alpha} \Rightarrow e_z \\
      &\quad \app{\App{\succ*}{\beta}}{z} \Rightarrow e_s}
      \end{aligned}
    }{
      \begin{aligned}
      &\match{e'}{\fun{x}{\N{s}}{P'}}{ \\
      &\quad \App{\zero*}{\alpha} \Rightarrow e'_z \\
      &\quad \app{\App{\succ*}{\beta}}{z} \Rightarrow e'_s}
      \end{aligned}
    }
  }
  \and
  \inferrule{
    \develop{\Phi; \Gamma}{e}{e'} \\
    \develop{\Phi; \Gamma, \annot{x}{\W{y}{\sigma}{\tau}{s}}}{P}{P'} \\\\
    \develop{\Phi, \bound{\alpha}{s}; \Gamma, \annot{z_1}{\sigma}, \annot{z_2}{\arr*{\subst{\tau}{y}{z_1}}{\W{y}{\sigma}{\tau}{\alpha}}}}{e_s}{e'_s}
  }{
    \develop{\Phi; \Gamma}{
      \begin{aligned}
      &\match{e}{\fun{x}{\W{y}{\sigma}{\tau}{s}}{P}}{ \\
      &\quad \app{\App{\sup*}{\alpha}}{z_1}{z_2} \Rightarrow e_s}
      \end{aligned}
    }{
      \begin{aligned}
      &\match{e'}{\fun{x}{\W{y}{\sigma}{\tau}{s}}{P'}}{ \\
      &\quad \app{\App{\sup*}{\alpha}}{z_1}{z_2} \Rightarrow e'_s}
      \end{aligned}
    }
  }
  \and
  \inferrule{~}{
    \develop{\Phi; \Gamma}{e}{e}
  }
  \end{mathpar}
}

\newcommand{\FigCongEquiv}[1]{
  \begin{mathpar}
  \setlength{\jot}{-4pt}
  \inferrule{
    \defeq{\GammaT}{\sigmaT}{\sigmaT'}{\UT_1} \\
    \defeq{\GammaT, \annotT{\xT}{\sigmaT}}{\tauT}{\tauT'}{\UT_2}
  }{
    \defeq{\GammaT}{\funtypeT{\xT}{\sigmaT}{\tauT}}{\funtypeT{\xT}{\sigmaT'}{\tauT'}}{\rules{\UT_1}{\UT_2}}
  }
  \and
  \inferrule{
    \defeq{\GammaT}{\sigmaT}{\sigmaT'}{\UT} \\
    \defeq{\GammaT, \annotT{\xT}{\sigmaT}}{\eT}{\eT'}{\tauT}
  }{
    \defeq{\GammaT}{\funT{\xT}{\sigmaT}{\eT}}{\funT{\xT}{\sigmaT}{\eT'}}{\funtypeT{\xT}{\sigmaT}{\tauT}}
  }
  \and
  \inferrule{
    \defeq{\GammaT}{\eT_1}{\eT'_1}{\funtypeT{\xT}{\sigmaT}{\tauT}} \\
    \defeq{\GammaT}{\eT_2}{\eT'_2}{\sigmaT}
  }{
    \defeq{\GammaT}{\app{\eT_1}{\eT_2}}{\app{\eT'_1}{\eT'_2}}{\subst{\tauT}{\xT}{\eT_2}}
  }
  \and
  \inferrule{
    \defeq{\GammaT}{\eT}{\eT'}{\tauT}
  }{
    \defeq{\GammaT}{\reflT{\eT}}{\reflT{\eT'}}{\eqT{\eT}{\tau}{\eT}}
  }
  \and
  \inferrule{
    \defeq{\GammaT}{\sigmaT}{\sigmaT'}{\UT} \\
    \defeq{\GammaT}{\eT_1}{\eT'_1}{\sigmaT} \\
    \defeq{\GammaT, \defineT{\xT}{\sigmaT}{\eT_1}}{\eT_2}{\eT'_2}{\tauT}
  }{
    \defeq{\GammaT}{\letinT{\xT}{\sigmaT}{\eT_1}{\eT_2}}{\letinT{\xT}{\sigmaT'}{\eT'_1}{\eT'_2}}{\subst{\tauT}{\xT}{\eT_1}}
  }
  \and
  \inferrule{
    \defeq{\GammaT}{\tauT}{\tauT'}{\UT} \\\\
    \defeq{\GammaT}{\eT_1}{\eT'_1}{\tauT} \\
    \defeq{\GammaT}{\eT_2}{\eT'_2}{\tauT}
  }{
    \defeq{\GammaT}{\eqT{\eT_1}{\tauT}{\eT_2}}{\eqT{\eT'_1}{\tauT'}{\eT'_2}}{\UT}
  }
  \and
  \inferrule{
    \defeq{\GammaT}{\PT}{\PT'}{\funtypeT{\yT}{\tauT}{\funtypeT{\zT}{\eqT{\eT_1}{\tauT}{\yT}}{\UT}}} \\\\
    \defeq{\GammaT}{\dT}{\dT'}{\app{\PT}{\eT_1}{\reflT{\eT_1}}} \\
    \defeq{\GammaT}{\pT}{\pT'}{\eqT{\eT_1}{\tauT}{\eT_2}}
  }{
    \defeq{\GammaT}{\JT{\PT}{\dT}{\pT}}{\JT{\PT'}{\dT'}{\pT'}}{\app{\PT}{\eT_2}{\pT}}
  }
  \and
  \inferrule{
    \app{\dataT{\app{\XT}{(\annotT{\vec{\wT}}{\vec{\vphantom{\wT}\any}})}}{\arrT*{\DeltaT_I}{\UT}}}{\DeltaT_c} \\
    \annotT{\vec{\yT}}{\vec{\sigmaT}} = \subst{\DeltaT_I}{\vec{\wT}}{\vec{\pT}} \\
    \defeq{\GammaT}{\eT}{\eT'}{\app{\XT}{\vec{\pT}}{\vec{\aT}}} \\
    \defeq{\GammaT, \annotT{\vec{\yT}}{\vec{\sigmaT}}, \annotT{\xT}{\app{\XT}{\vec{\pT}}{\vec{\aT}}}}{\PT}{\PT'}{\UT'} \\
    \textrm{For every $1 \leq i \leq n$:} \\
    (\annotT{\cT_i}{\arrT*{(\annotT{\vec{\zT}_i}{\vec{\tauT}_i})}{\app{\XT}{\vec{\pT}}{\vec{\aT}_i}}}) \in \subst{\DeltaT_c}{\vec{\wT}}{\vec{\pT}} \\
    \defeq{\GammaT, \annotT{\vec{\zT}_i}{\vec{\tauT}_i}}{\eT_i}{\eT'_i}{\subst{\PT}{\vec{\yT}, \xT}{\vec{\aT}_i, \app{\cT_i}{\vec{\pT}}{\vec{\zT}_i}}}
  }{
    \defeq{\GammaT}{
      \begin{aligned}
      &\matchT{\eT}{\funT*{\vec{\yT}}{\xT}{\PT}}{ \\
      &\quad \app{\cT_1}{\vec{\zT}_1} \RightarrowT \eT_1 \\
      &\quad \seq \\
      &\quad \app{\cT_n}{\vec{\zT}_n} \RightarrowT \eT_n}
      \end{aligned}
    }{
      \begin{aligned}
      &\matchT{\eT'}{\funT*{\vec{\yT}}{\xT}{\PT'}}{ \\
      &\quad \app{\cT_1}{\vec{\zT}_1} \RightarrowT \eT'_1 \\
      &\quad \seq \\
      &\quad \app{\cT_n}{\vec{\zT}_n} \RightarrowT \eT'_n}
      \end{aligned}
    }{\subst{\PT}{\vec{\yT}, \xT}{\vec{\aT}, \eT}}
  }
  \and
  \inferrule{
    \defeq{\GammaT}{\tauT}{\tauT'}{\UT} \\
    \defeq{\GammaT, \annotT{\fT}{\tauT}}{\eT}{\eT'}{\tauT}
  }{
    \defeq{\GammaT}{\fixT{\nT}{\fT}{\tauT}{\eT}}{\fixT{\nT}{\fT}{\tauT'}{\eT'}}{\tauT}
  }
  \end{mathpar}
}
\iffalse
\newcommand{\FigTakahashi}[1]{
  \begin{figure}[h]
  \centering
  \begin{align*}
  x^\bullet_{\Phi; \Gamma} &= e^\bullet_{\Phi; \Gamma} \qquad \textit{if $(\define{x}{\tau}{e}) \in \Gamma$} \\
  (\app{(\fun{x}{\tau}{e})}{e'})^\bullet_{\Phi; \Gamma} &= \subst{e^\bullet_{\Phi; \Gamma}}{x}{e'^\bullet_{\Phi; \Gamma}} \\
  (\App{(\Fun{\alpha}{e})}{s})^\bullet_{\Phi; \Gamma} &= \subst{e^\bullet_{\Phi, \alpha; \Gamma}}{\alpha}{s} \\
  (\App{(\Fun<{\alpha}{r}{e})}{s})^\bullet_{\Phi; \Gamma} &= \subst{e^\bullet_{\Phi, \bound{\alpha}{r}; \Gamma}}{\alpha}{s} \\
  (\letin{x}{\tau}{e'}{e})^\bullet_{\Phi; \Gamma} &= \subst{e^\bullet_{\Phi; \Gamma}}{x}{e'^\bullet_{\Phi; \Gamma}} \\
  (\match*{\App{\zero*_{\any}}{s}}{(\App{\zero*}{\alpha} \Rightarrow e)\any})^\bullet_{\Phi; \Gamma}
    &= \subst{e^\bullet_{\Phi; \Gamma}}{\alpha}{s} \\
  (\match*{\app{\App{\succ*_{\any}}{s}}{e'}}{\any(\app{\App{\succ*}{\alpha}}{z} \Rightarrow e)})^\bullet_{\Phi; \Gamma}
    &= \subst{e^\bullet_{\Phi; \Gamma}}{\alpha, z}{s, e'^\bullet_{\Phi; \Gamma}} \\
  (\match*{\app{\App{\sup*_{\any}}{s}}{e_1}{e_2}}{(\app{\App{\sup*}{\alpha}}{z_1}{z_2} \Rightarrow e)})^\bullet_{\Phi; \Gamma}
    &= \subst{e^\bullet_{\Phi; \Gamma}}{\alpha, z_1, z_2}{s, {e_1}^\bullet_{\Phi; \Gamma}, {e_2}^\bullet_{\Phi; \Gamma}} \\
  (\App{(\fix{f}{\alpha}{\tau}{e})}{s})^\bullet_{\Phi; \Gamma}
    &= \subst{e^\bullet_{\Phi, \alpha; \Gamma}}{\alpha, f}{s, \Fun<{\beta}{s}{\App{(\fix{f}{\alpha}{\tau^\bullet_{\Phi, \alpha; \Gamma}}{e^\bullet_{\Phi, \alpha; \Gamma}})}{\beta}}} \\
    &\mathrel{\phantom{=}} \textit{if $\subsize{\Phi}{\sss{r}}{s}$} \\
  (\funtype{x}{\sigma}{\tau})^\bullet_{\Phi; \Gamma} &= \funtype{x}{\sigma^\bullet_{\Phi; \Gamma}}{\tau^\bullet_{\Phi; \Gamma}} \\
  (\fun{x}{\sigma}{e})^\bullet_{\Phi; \Gamma} &= \fun{x}{\sigma^\bullet_{\Phi; \Gamma}}{e^\bullet_{\Phi; \Gamma}} \\
  (\app{e}{e_2})^\bullet_{\Phi; \Gamma} &= \app{{e_1}^\bullet_{\Phi; \Gamma}}{{e_2}^\bullet_{\Phi; \Gamma}} \\
  (\Funtype{\alpha}{\tau})^\bullet_{\Phi; \Gamma} &= \Funtype{\alpha}{\tau^\bullet_{\Phi, \alpha; \Gamma}} \\
  (\Funtype<{\alpha}{s}{\tau})^\bullet_{\Phi; \Gamma} &= \Funtype<{\alpha}{s}{\tau^\bullet_{\Phi, \bound{\alpha}{s}; \Gamma}} \\
  (\Fun{\alpha}{e})^\bullet_{\Phi; \Gamma} &= \Fun{\alpha}{e^\bullet_{\Phi, \alpha; \Gamma}} \\
  (\Fun<{\alpha}{s}{e})^\bullet_{\Phi; \Gamma} &= \Fun<{\alpha}{s}{e^\bullet_{\Phi, \bound{\alpha}{s}; \Gamma}} \\
  (\App{e}{s})^\bullet_{\Phi; \Gamma} &= \App{e^\bullet_{\Phi; \Gamma}}{s} \\
  (\succ{s}{r}{e})^\bullet_{\Phi; \Gamma} &= \succ{s}{r}{e^\bullet_{\Phi; \Gamma}} \\
  (\W{x}{\sigma}{\tau}{s})^\bullet_{\Phi; \Gamma} &= \W{x}{\sigma^\bullet_{\Phi; \Gamma}}{\tau^\bullet_{\Phi; \Gamma}}{s} \\
  (\sup{x}{\sigma}{\tau}{s}{r}{e_1}{e_2})^\bullet_{\Phi; \Gamma} &= \sup{x}{\sigma^\bullet_{\Phi; \Gamma}}{\tau^\bullet_{\Phi; \Gamma}}{s}{r}{{e_1}^\bullet_{\Phi; \Gamma}}{{e_2}^\bullet_{\Phi; \Gamma}} \\
  (\match{e'}{\fun{x}{P}}{(\app{\App{c}{\alpha}}{\vec{z}} \Rightarrow e) \seq})^\bullet_{\Phi; \Gamma}
    &= \match{e'^\bullet_{\Phi; \Gamma}}{\fun{x}{P^\bullet_{\Phi; \Gamma}}}{(\app{\App{c}{\alpha}}{\vec{z}} \Rightarrow e^\bullet_{\Phi; \Gamma}) \seq} \\
  (\fix{f}{\alpha}{\tau}{e})^\bullet_{\Phi; \Gamma} &= \fix{f}{\alpha}{\tau^\bullet_{\Phi, \alpha; \Gamma}}{e^\bullet_{\Phi, \alpha; \Gamma}} \\
  e^\bullet_{\Phi; \Gamma} &= e
  \end{align*}
  \caption[Complete development for \lang]{Extension of Takahashi's complete development for \lang}
  \label{#1}
  \end{figure}
}
\fi

\newcommand{\FigTakahashi}[1]{
  \begin{figure}[h]
  \centering
  \begin{mathpar}
  \fbox{$\develop{\Phi; \Gamma}{e}{e}$} \hfill \\
  \inferrule[]{
    (\define{x}{\tau}{e}) \in \Gamma \\
    \develop{\Phi; \Gamma}{e}{e'}
  }{
    \develop{\Phi; \Gamma}{x}{e'}
  }
  \and
  \inferrule[]{
    \develop{\Phi; \Gamma, \annot{x}{\tau}}{e_1}{e'_1} \\
    \develop{\Phi; \Gamma}{e_2}{e'_2}
  }{
    \develop{\Phi; \Gamma}{\app{(\fun{x}{\tau}{e_1})}{e_2}}{\subst{e'_1}{x}{e'_2}}
  }
  \and
  \inferrule[]{
    \develop{\Phi, \alpha; \Gamma}{e}{e'}
  }{
    \develop{\Phi; \Gamma}{\App{(\Fun{\alpha}{e})}{s}}{\subst{e'}{\alpha}{s}}
  }
  \and
  \inferrule[]{
    \develop{\Phi, \bound{\alpha}{r}; \Gamma}{e}{e'}
  }{
    \develop{\Phi; \Gamma}{\App{\Fun<{\alpha}{r}{e}}{s}}{\subst{e'}{\alpha}{s}}
  }
  \and
  \inferrule[]{
    \develop{\Phi; \Gamma}{e_1}{e'_1} \\
    \develop{\Phi; \Gamma, \define{x}{\tau}{e_1}}{e_2}{e'_2}
  }{
    \develop{\Phi; \Gamma}{\letin{x}{\tau}{e_1}{e_2}}{\subst{e'_2}{x}{e'_1}}
  }
  \and
  \inferrule[]{
    \develop{\Phi, \bound{\alpha}{r}; \Gamma}{e_z}{e_z'}
  }{
    \develop{\Phi; \Gamma}{
      \match*{\zero{r}{s}}{(\App{\zero*}{\alpha} \Rightarrow e_z) \any}
    }{\subst{e'_z}{\alpha}{s}}
  }
  \and
  \inferrule[]{
    \develop{\Phi; \Gamma}{e}{e'} \\
    \develop{\Phi, \bound{\alpha}{r}; \Gamma, \annot{z}{\N{\alpha}}}{e_s}{e'_s}
  }{
    \develop{\Phi; \Gamma}{
      \match*{\succ{r}{s}{e}}{\any (\app{\App{\succ*}{\alpha}}{z} \Rightarrow e_s)}
    }{\subst{e'_s}{\alpha, z}{s, e'}}
  }
  \and
  \inferrule[]{
    \develop{\Phi; \Gamma}{e_1}{e'_1} \\
    \develop{\Phi; \Gamma}{e_2}{e'_2} \\
    \develop{\Phi, \bound{\alpha}{r}; \Gamma, \annot{z_1}{\sigma}, \annot{z_2}{\subst{\tau}{x}{z_1}}}{e}{e'}
  }{
    \develop{\Phi; \Gamma}{
      \match*{\sup{x}{\sigma}{\tau}{r}{s}{e_1}{e_2}}{(\app{\App{\sup*}{\alpha}}{z_1}{z_2} \Rightarrow e)}
    }{\subst{e'}{\alpha, z_1, z_2}{s, e'_1, e'_2}}
  }
  \and
  \inferrule[]{
    \subsize{\Phi}{\sss{r}}{s} \\
    \develop{\Phi, \alpha; \Gamma}{\tau}{\tau'} \\
    \develop{\Phi, \alpha; \Gamma, \annot{f}{\Funtype<{\beta}{\alpha}{\subst{\tau}{\alpha}{\beta}}}}{e}{e'}
  }{
    \develop{\Phi; \Gamma}{\App{\fix{f}{\alpha}{\tau}{e}}{s}}{\subst{e'}{\alpha, f}{s, \Fun<{\beta}{s}{\App{(\fix{f}{\alpha}{\tau'}{e'})}{\beta}}}}
  }
  \end{mathpar}
  \caption[Complete development for \lang]{Extension of Takahashi's complete development for \lang \\ (excerpt excl. congruence)}
  \label{#1}
  \end{figure}
}

\newcommand{\FigTakahashiFull}[1]{
  \begin{mathpar}
  \setlength{\jot}{-1.5pt}
  \fbox{$\develop{\Phi; \Gamma}{e}{e}$} \hfill
  \inferrule[]{
    (\define{x}{\tau}{e}) \in \Gamma \\
    \develop{\Phi; \Gamma}{e}{e'}
  }{
    \develop{\Phi; \Gamma}{x}{e'}
  }
  \and
  \inferrule[]{
    \develop{\Phi; \Gamma, \annot{x}{\tau}}{e_1}{e'_1} \\
    \develop{\Phi; \Gamma}{e_2}{e'_2}
  }{
    \develop{\Phi; \Gamma}{\app{(\fun{x}{\tau}{e_1})}{e_2}}{\subst{e'_1}{x}{e'_2}}
  }
  \and
  \inferrule[]{
    \develop{\Phi, \alpha; \Gamma}{e}{e'}
  }{
    \develop{\Phi; \Gamma}{\App{(\Fun{\alpha}{e})}{s}}{\subst{e'}{\alpha}{s}}
  }
  \and
  \inferrule[]{
    \develop{\Phi, \bound{\alpha}{r}; \Gamma}{e}{e'}
  }{
    \develop{\Phi; \Gamma}{\App{\Fun<{\alpha}{r}{e}}{s}}{\subst{e'}{\alpha}{s}}
  }
  \and
  \inferrule[]{
    \develop{\Phi; \Gamma}{e_1}{e'_1} \\
    \develop{\Phi; \Gamma, \define{x}{\tau}{e_1}}{e_2}{e'_2}
  }{
    \develop{\Phi; \Gamma}{\letin{x}{\tau}{e_1}{e_2}}{\subst{e'_2}{x}{e'_1}}
  }
  \and
  \inferrule[]{
    \develop{\Phi, \bound{\alpha}{r}; \Gamma}{e_z}{e_z'}
  }{
    \develop{\Phi; \Gamma}{
      \match*{\zero{r}{s}}{(\App{\zero*}{\alpha} \Rightarrow e_z) \any}
    }{\subst{e'_z}{\alpha}{s}}
  }
  \and
  \inferrule[]{
    \develop{\Phi; \Gamma}{e}{e'} \\
    \develop{\Phi, \bound{\alpha}{r}; \Gamma, \annot{z}{\N{\alpha}}}{e_s}{e'_s}
  }{
    \develop{\Phi; \Gamma}{
      \match*{\succ{r}{s}{e}}{\any (\app{\App{\succ*}{\alpha}}{z} \Rightarrow e_s)}
    }{\subst{e'_s}{\alpha, z}{s, e'}}
  }
  \and
  \inferrule[]{
    \develop{\Phi; \Gamma}{e_1}{e'_1} \\
    \develop{\Phi; \Gamma}{e_2}{e'_2} \\
    \develop{\Phi, \bound{\alpha}{r}; \Gamma, \annot{z_1}{\sigma}, \annot{z_2}{\subst{\tau}{x}{z_1}}}{e}{e'}
  }{
    \develop{\Phi; \Gamma}{
      \match*{\sup{x}{\sigma}{\tau}{r}{s}{e_1}{e_2}}{(\app{\App{\sup*}{\alpha}}{z_1}{z_2} \Rightarrow e)}
    }{\subst{e'}{\alpha, z_1, z_2}{s, e'_1, e'_2}}
  }
  \and
  \inferrule[]{
    \subsize{\Phi}{\sss{r}}{s} \\
    \develop{\Phi, \alpha; \Gamma}{\tau}{\tau'} \\
    \develop{\Phi, \alpha; \Gamma, \annot{f}{\Funtype<{\beta}{\alpha}{\subst{\tau}{\alpha}{\beta}}}}{e}{e'}
  }{
    \develop{\Phi; \Gamma}{\App{\fix{f}{\alpha}{\tau}{e}}{s}}{\subst{e'}{\alpha, f}{s, \Fun<{\beta}{s}{\App{(\fix{f}{\alpha}{\tau'}{e'})}{\beta}}}}
  }
  \and
  \inferrule{
    \develop{\Phi; \Gamma}{\sigma}{\sigma'} \\
    \develop{\Phi; \Gamma, \annot{x}{\sigma}}{\tau}{\tau'}
  }{
    \develop{\Phi; \Gamma}{\funtype{x}{\sigma}{\tau}}{\funtype{x}{\sigma'}{\tau}}
  }
  \and
  \inferrule{
    \develop{\Phi; \Gamma}{\sigma}{\sigma'} \\
    \develop{\Phi; \Gamma, \annot{x}{\sigma}}{e}{e'}
  }{
    \develop{\Phi; \Gamma}{\fun{x}{\sigma}{e}}{\fun{x}{\sigma'}{e}}
  }
  \and
  \inferrule{
    \develop{\Phi; \Gamma}{e_1}{e'_1} \\
    \develop{\Phi; \Gamma}{e_2}{e'_2}
  }{
    \develop{\Phi; \Gamma}{\app{e_1}{e_2}}{\app{e'_1}{e'_2}}
  }
  \and
  \inferrule{
    \develop{\Phi, \alpha; \Gamma}{\tau}{\tau'}
  }{
    \develop{\Phi; \Gamma}{\Funtype{\alpha}{\tau}}{\Funtype{\alpha}{\tau'}}
  }
  \and
  \inferrule{
    \develop{\Phi, \alpha; \Gamma}{e}{e'}
  }{
    \develop{\Phi; \Gamma}{\Fun{\alpha}{e}}{\Fun{\alpha}{e'}}
  }
  \and
  \inferrule{
    \develop{\Phi; \Gamma}{e}{e'}
  }{
    \develop{\Phi; \Gamma}{\App{e}{s}}{\App{e'}{s}}
  }
  \and
  \inferrule{
    \develop{\Phi, \bound{\alpha}{s}; \Gamma}{\tau}{\tau'}
  }{
    \develop{\Phi; \Gamma}{\Funtype<{\alpha}{s}{\tau}}{\Funtype<{\alpha}{s}{\tau'}}
  }
  \and
  \inferrule{
    \develop{\Phi, \bound{\alpha}{s}; \Gamma}{e}{e'}
  }{
    \develop{\Phi; \Gamma}{\Fun<{\alpha}{s}{e}}{\Fun<{\alpha}{s}{e'}}
  }
  \and
  \inferrule{
    \develop{\Phi, \alpha; \Gamma}{\tau}{\tau'} \\\\
    \develop{\Phi, \alpha; \Gamma, \annot{f}{\Funtype<{\beta}{\alpha}{\subst{\tau}{\alpha}{\beta}}}}{e}{e'}
  }{
    \develop{\Phi; \Gamma}{\fix{f}{\alpha}{\tau}{e}}{\fix{f}{\alpha}{\tau'}{e}}
  }
  \and
  \inferrule{
    \develop{\Phi; \Gamma}{\sigma}{\sigma'} \\
    \develop{\Phi; \Gamma, \annot{x}{\sigma}}{\tau}{\tau'}
  }{
    \develop{\Phi; \Gamma}{\W{x}{\sigma}{\tau}{s}}{\W{x}{\sigma'}{\tau'}{s}}
  }
  \and
  \inferrule{
    \develop{\Phi; \Gamma}{e}{e'}
  }{
    \develop{\Phi; \Gamma}{\succ{r}{s}{e}}{\succ{r}{s}{e'}}
  }
  \and
  \inferrule{
    \develop{\Phi; \Gamma}{\sigma}{\sigma'} \\
    \develop{\Phi; \Gamma, \annot{x}{\sigma}}{\tau}{\tau'} \\\\
    \develop{\Phi; \Gamma}{e_1}{e'_1} \\
    \develop{\Phi; \Gamma}{e_2}{e'_2}
  }{
    \develop{\Phi; \Gamma}{\sup{x}{\sigma}{\tau}{r}{s}{e_1}{e_2}}{\sup{x}{\sigma'}{\tau'}{r}{s}{e'_1}{e'_2}}
  }
  \and
  \inferrule{
    \develop{\Phi; \Gamma}{e}{e'} \\
    \develop{\Phi; \Gamma, \annot{x}{\N{s}}}{P}{P'} \\\\
    \develop{\Phi, \bound{\alpha}{s}; \Gamma}{e_z}{e'_z} \\
    \develop{\Phi, \bound{\beta}{s}; \Gamma, \annot{z}{\N{\beta}}}{e_s}{e'_s}
  }{
    \develop{\Phi; \Gamma}{
      \begin{aligned}
      &\match{e}{\fun{x}{\N{s}}{P}}{ \\
      &\quad \App{\zero*}{\alpha} \Rightarrow e_z \\
      &\quad \app{\App{\succ*}{\beta}}{z} \Rightarrow e_s}
      \end{aligned}
    }{
      \begin{aligned}
      &\match{e'}{\fun{x}{\N{s}}{P'}}{ \\
      &\quad \App{\zero*}{\alpha} \Rightarrow e'_z \\
      &\quad \app{\App{\succ*}{\beta}}{z} \Rightarrow e'_s}
      \end{aligned}
    }
  }
  \and
  \inferrule{
    \develop{\Phi; \Gamma}{e}{e'} \\
    \develop{\Phi; \Gamma, \annot{x}{\W{y}{\sigma}{\tau}{s}}}{P}{P'} \\\\
    \develop{\Phi, \bound{\alpha}{s}; \Gamma, \annot{z_1}{\sigma}, \annot{z_2}{\arr*{\subst{\tau}{y}{z_1}}{\W{y}{\sigma}{\tau}{\alpha}}}}{e_s}{e'_s}
  }{
    \develop{\Phi; \Gamma}{
      \begin{aligned}
      &\match{e}{\fun{x}{\W{y}{\sigma}{\tau}{s}}{P}}{ \\
      &\quad \app{\App{\sup*}{\alpha}}{z_1}{z_2} \Rightarrow e_s}
      \end{aligned}
    }{
      \begin{aligned}
      &\match{e'}{\fun{x}{\W{y}{\sigma}{\tau}{s}}{P'}}{ \\
      &\quad \app{\App{\sup*}{\alpha}}{z_1}{z_2} \Rightarrow e'_s}
      \end{aligned}
    }
  }
  \and
  \inferrule{~}{
    \develop{\Phi; \Gamma}{e}{e}
  }
  \end{mathpar}
}

\allowdisplaybreaks

\subsection{Congruence Rules for Reduction in \lang} \label{app:cong:red}

\FigCongRed{fig:cong:red*}
\FigCongRedCase{fig:cong:red*-case}

\subsection{Congruence Rules for Equivalence in \CICE} \label{app:cong:equiv}

\FigCongEquiv{fig:cong:equiv}

\subsection{Rules for Complete Development of \lang} \label{app:develop}

\FigTakahashiFull{fig:develop}

\setmonofont{iosevka.ttc}

\subsection{Mechanized \CICE Definitions in Coq}

The following type check on Coq 8.15.1.

\subsubsection{Preliminary definitions} \label{app:mechanization:coq:prelim}

\begin{minted}{coq}
From Equations Require Import Equations.
Require Import Coq.Program.Equality.
Require Import Coq.Unicode.Utf8_core.

Reserved Notation "r ≤ s" (at level 70, no associativity).

Inductive Size : Type :=
| suc : Size → Size
| lim : ∀ {A : Type}, (A → Size) → Size.

Inductive Leq : Size → Size → Type :=
| mono : ∀ {r s}, r ≤ s → suc r ≤ suc s
| cocone : ∀ {s A f} (a : A), s ≤ f a → s ≤ lim f
| limiting : ∀ {s A f}, (∀ (a : A), f a ≤ s) → lim f ≤ s
where "r ≤ s" := (Leq r s).

Definition Lt r s : Type := suc r ≤ s.
Notation "r < s" := (Lt r s).

Definition base : Size := lim (False_rect Size).

Definition baseLeq s : base ≤ s :=
  limiting (λ a, (False_rect (_ ≤ s) a)).

Fixpoint reflLeq {s} : s ≤ s :=
  match s with
  | suc s => mono reflLeq
  | lim f => limiting (λ a, cocone a reflLeq)
  end.

Property transLeq {r s t} (rs : r ≤ s) (st : s ≤ t) : r ≤ t.
Admitted.

(*
Derive NoConfusion for Size.
Equations transLeq {r s t : Size} (rs : r ≤ s) (st : s ≤ t) : r ≤ t :=
  transLeq (mono rs) (mono st) := mono (transLeq rs st);
  transLeq rs (cocone a sfa) := cocone a (transLeq rs sfa);
  transLeq (limiting fas) st := limiting (λ a, transLeq (fas a) st);
  transLeq (cocone a rfa) (limiting fat) := transLeq rfa (fat a).
*)

Fixpoint sucLeq s : s ≤ suc s :=
  match s with
  | suc s => mono (sucLeq s)
  | lim f => limiting (λ a, transLeq (sucLeq (f a)) (mono (cocone a reflLeq)))
  end.

Inductive Acc (s : Size) : Prop :=
acc { accLt : ∀ r, r < s → Acc r }.
Arguments accLt {_}.

Axiom funext : ∀ {A} {B : A → Type} {p q : ∀ x, B x},
  (∀ x, p x = q x) → p = q.

Equations accIsProp {s} (acc1 acc2 : Acc s) : acc1 = acc2 :=
| acc _ p, acc _ q =>
  f_equal _ (funext (λ r, funext (λ rs, accIsProp (p r rs) (q r rs)))).

Lemma accLeq : ∀ r s, r ≤ s → Acc s → Acc r.
Proof.
  intros r s rs acc.
  induction acc as [s p IH].
  exact (acc r (λ t tr, p t (transLeq tr rs))).
Qed.

Theorem wf : ∀ s, Acc s.
Proof.
  intros s.
  induction s as [s IH | A f IH].
  - refine (acc (suc s) (λ r rsucs, accLeq r s _ IH)).
    inversion rsucs as [r' s' rs | |].
    exact rs.
  - refine (acc (lim f) (λ r rlimf, _)).
    inversion rlimf as [| r' A' f' a rfa eqr eqA |].
    dependent destruction H.
    destruct (IH a) as [p].
    exact (p r rfa).
Qed.

Definition wfInd P IH s : P s :=
  let wfInd := fix wfAcc s acc {struct acc} :=
    IH s (λ r rs, wfAcc r (acc.(accLt) r rs))
  in wfInd s (wf s).
\end{minted}

\subsubsection{W Types} \label{app:mechanization:coq:W}

\begin{minted}{coq}
Import SigTNotations.

Inductive W (A : Type) (B : A → Type) (s : Size) : Type :=
| sup : ∀ r, r < s → ∀ a, (B a → W A B r) → W A B s.
Arguments sup {_ _ _}.

Definition ac {A B} a (f : B a → {s & W A B s}) : {s & B a → W A B s} :=
  let f' b :=
    match (f b).2 with
    | sup r rs a f => sup r (transLeq rs (cocone b reflLeq)) a f
    end
  in (lim (λ b, (f b).1) ; f').
\end{minted}

\subsection{Mechanized \CICE Definitions in Agda}

The following type check on Agda 2.6.2.

\subsubsection{Preliminary definitions} \label{app:mechanization:agda:prelim}

\begin{minted}{agda}
{-# OPTIONS --without-K #-}

open import Agda.Primitive using (Level; lsuc)
open import Relation.Binary.PropositionalEquality.Core using (_≡_; cong)
open import Data.Empty using (⊥; ⊥-elim)

variable
  ℓ ℓ′ : Level
  A C : Set ℓ
  B : A → Set ℓ

infix 30 ↑_
infix 30 ⊔_

data Size {ℓ} : Set (lsuc ℓ) where
  ↑_ : Size {ℓ} → Size
  ⊔_ : {A : Set ℓ} → (A → Size {ℓ}) → Size

data _≤_ {ℓ} : Size {ℓ} → Size {ℓ} → Set (lsuc ℓ) where
  ↑s≤↑s : ∀ {r s} → r ≤ s → ↑ r ≤ ↑ s
  s≤⊔f  : ∀ {s} f (a : A) → s ≤ f a → s ≤ ⊔ f
  ⊔f≤s  : ∀ {s} f → (∀ (a : A) → f a ≤ s) → ⊔ f ≤ s

◯ : Size
◯ = ⊔ ⊥-elim

◯≤s : ∀ {s} → ◯ ≤ s
◯≤s = ⊔f≤s ⊥-elim λ ()

s≤s : ∀ {s : Size {ℓ}} → s ≤ s
s≤s {s = ↑ s} = ↑s≤↑s s≤s
s≤s {s = ⊔ f} = ⊔f≤s f (λ a → s≤⊔f f a s≤s)

s≤s≤s : ∀ {r s t : Size {ℓ}} → r ≤ s → s ≤ t → r ≤ t
s≤s≤s (↑s≤↑s r≤s) (↑s≤↑s s≤t) = ↑s≤↑s (s≤s≤s r≤s s≤t)
s≤s≤s r≤s (s≤⊔f f a s≤fa) = s≤⊔f f a (s≤s≤s r≤s s≤fa)
s≤s≤s (⊔f≤s f fa≤s) s≤t = ⊔f≤s f (λ a → s≤s≤s (fa≤s a) s≤t)
s≤s≤s (s≤⊔f f a s≤fa) (⊔f≤s f fa≤t) = s≤s≤s s≤fa (fa≤t a)

s≤↑s : ∀ {s : Size {ℓ}} → s ≤ ↑ s
s≤↑s {s = ↑ s} = ↑s≤↑s s≤↑s
s≤↑s {s = ⊔ f} = ⊔f≤s f (λ a → s≤s≤s s≤↑s (↑s≤↑s (s≤⊔f f a s≤s)))

_<_ : Size {ℓ} → Size {ℓ} → Set (lsuc ℓ)
r < s = ↑ r ≤ s

record Acc (s : Size {ℓ}) : Set (lsuc ℓ) where
  inductive
  pattern
  constructor acc
  field acc< : (∀ r → r < s → Acc r)
open Acc

accIsProp : ∀ {s : Size {ℓ}} → (acc1 acc2 : Acc s) → acc1 ≡ acc2
accIsProp (acc p) (acc q) =
  cong acc (funext p q (λ r → funext (p r) (q r) (λ r<s → accIsProp (p r r<s) (q r r<s))))
  where postulate funext : ∀ (p q : ∀ x → B x) → (∀ x → p x ≡ q x) → p ≡ q

acc≤ : ∀ {r s : Size {ℓ}} → r ≤ s → Acc s → Acc r
acc≤ r≤s (acc p) = acc (λ t t<r → p t (s≤s≤s t<r r≤s))

wf : ∀ (s : Size {ℓ}) → Acc s
wf (↑ s) = acc (λ { _ (↑s≤↑s r≤s) → acc≤ r≤s (wf s) })
wf (⊔ f) = acc (λ { r (s≤⊔f f a r<fa) → (wf (f a)).acc< r r<fa })

wfInd : ∀ (P : Size {ℓ} → Set ℓ′) → (∀ s → (∀ r → r < s → P r) → P s) → ∀ s → P s
wfInd P f s = wfAcc s (wf s)
  where
  wfAcc : ∀ s → Acc s → P s
  wfAcc s (acc p) = f s (λ r r<s → wfAcc r (p r r<s))
\end{minted}

\subsubsection{W Types} \label{app:mechanization:agda:W}

\begin{minted}{agda}
open import Data.Product using (proj₁; proj₂; Σ-syntax; _,_)

data W (A : Set ℓ) (B : A → Set ℓ) (s : Size {ℓ}) : Set (lsuc ℓ) where
  sup : ∀ r → r < s → (a : A) → (B a → W A B r) → W A B s

ac : ∀ a → (B a → Σ[ s ∈ Size ] W A B s) → Σ[ s ∈ Size ] (B a → W A B s)
ac a f = ⊔ (λ b → proj₁ (f b)) , f′
  where
  f′ : _
  f′ b with proj₂ (f b)
  ... | sup r r<s a f = sup r (s≤s≤s r<s (s≤⊔f _ b s≤s)) a f
\end{minted}
\end{document}