\chapter{Introduction} \label{ch:introduction}

A type system is a delicate balance between expressivity and well-behavedness.
More expressivity allows for more programs to be well typed,
while paradoxically allowing less code to be written,
because some of the onus of ensuring a program is valid is shifted
from the programmer to the process of type checking.
On the other hand, too much expressivity can destroy
desirable metatheoretical properties of the type system ---
strings get be treated as integers, booleans can be called like functions,
and the language falls to chaos and ruin.

The balancing act is particularly perilous for type systems
used by proof assistants for mechanized theorem proving,
because a lack of expressivity can make it too much of a burden to be practically useful,
while it's all to easy to add some convenient, seemingly innocuous feature
that turns out to dismantle the logic of the system,
rendering it unusable for proving.

This thesis deals with dependent type theory, which is commonly used for theorem proving,
together with sized types, a feature for inductively defined types
used to increase the expressivity of recursive functions,
and shows that dependent types can be extended with state-of-the-art sized types
without sacrificing key metatheoretical properties.

\section{Background}

Before we begin, I briefly summarize why dependent types are used for theorem proving,
as well as what sized types have to offer in the context of programming with dependent types.

\subsection{Theorem proving with dependent types} \label{tt}

Many contemporary proof assistants are founded on the Curry--Howard correspondence,
where propositions correspond to types,
proofs of these propositions to terms of those types,
and proof verification to type checking.
Different type theories correspond to different logical systems;
dependent type theories, in particular, correspond to predicate logics.
The primary feature of dependent type theory is the dependent function type,
because it encodes universal quantification by allowing types to \emph{depend on} terms:
the type $\funtype{x}{A}{\app{P}{x}}$ can be interpreted as the statement
``for all objects $x$ in $A$, $P$ holds of $x$'',
with the consequent $\app{P}{x}$ depending on the term $x$.
Other logical constructs can be encoded as well, for example:

\begin{itemize}
  \item Falsehood ($\bot$) as $\funtype{P}{\Prop}{P}$,
    (where $\Prop$ is the type of all propositions),
    the statement that \emph{all} propositions hold;
  \item Implication as $\funtype{\any}{A}{B}$
    (or $\arr*{A}{B}$ for short),
    where $B$ holds only when given a proof of $A$;
  \item Negation as $\arr*{A}{\bot}$,
    or that proving $A$ will yield a falsehood; and
  \item Truthhood as $\funtype{P}{\Prop}{\arr*{P}{P}}$\punctstack{,}%
    \footnote{This encoding is slightly simpler than
    the negation of falsehood $\arr*{\bot}{\bot}$,
    particularly when written out in full.}
    which is trivially inhabited by the identity function
    $\fun{P}{\Prop}{\fun{p}{P}{p}}$;
\end{itemize}
and many other logical connectives such as existential quantification, conjunction, disjunction, and equality.

A key metatheoretical property that a type theory must satisfy
for its interpretation as a logical system to be valid is \emph{consistency}:
falsehood cannot be proven.
In other words, there must not be any term whose type is $\funtype{P}{\Prop}{P}$.
Otherwise, any proposition would be provable,
including those we know to be obviously untrue,
such as $0 = 1$.

The two most important components of a formal description of a dependent type theory
are its typing judgement, whose rules describe when a term has a certain type,
and its equality judgement(s), whose rules describe when two terms are equal.
The latter is relevant in dependent types because they allow determining that,
for instance, a proof of \mbox{$\app{P}{(1 + 1)}$}
is in fact also a proof of $\app{P}{2}$.
Naturally, equality is sometimes described in terms of reduction judgements,
governed by the rules of computation.
This means that, in contrast to nondependent type systems,
whether type checking always terminates depends on whether the language is normalizing.
The issue gets even more complicated with more complex forms of reduction like recursion.

\subsection{Recursive programming with sized types} \label{ss}

Inductively defined data and recursive functions on them
are indispensable tools of programming,
allowing the expression of a variety of constructs, programs,
and in the case of dependent types, their properties.
A crucial restriction on recursive functions
in proof assistants such as Agda, Coq, Idris, Lean, and more
is that recursive functions must be \emph{guarded by destructors}\index{guardedness}
and only recur on structurally smaller arguments~\citep{guard}.
This guardedness check is statically done before or during type checking and,
to vastly simplify, ensures that recursive calls occur only on syntactic subarguments:
if the original argument is some natural $\app{\succ*}{n}$ that is the successor of $n$,
then the function recurs only on $n$;
if it's some cons list $\app{\cons*}{\hd}{\tl}$
consisting of an element and the rest of the list $\tl$,
then it recurs only on $\tl$; and so on.

Although this is usually referred to as a syntactic termination check,
and termination is essential in ensuring decidability of type checking,
the key desired property is rather consistency.
Indeed, a type theory may be consistent yet nonnormalizing
(as is the case with an impredicative\index{impredicativity},
definitionally proof-irrelevant $\Prop$ universe~\citep{impred-proof-irrel}),
but unrestricted recursion easily proves an inconsistency:
the recursive function $f$ defined by $\fun{P}{\Prop}{\app{f}{P}}$,
for instance, can be assigned type $\funtype{P}{\Prop}{P}$,
in addition to being nonterminating.

While the guardedness check, being based on the elimination principles of inductive types,
yields terminating functions and a consistent type theory,
it has known disadvantages:

\begin{itemize}
  \item Calling some other function on a subargument prior to a recursive call is disallowed,
    even if that function doesn't change the structure of that subargument,
    because the input of the recursive call is no longer exactly the syntactic subargument.
    For instance, a recursive function on lists cannot first map over the sublist,
    even though the usual mapping function doesn't change the length of a list
    and therefore presents no threat to termination.
  \item Some proof assistants (Coq, for one) will inline functions and even unfold recursive functions
    so that the guardedness check has more information
    and that more recursive functions pass the check.
    However, this too has its disadvantages:
    \begin{itemize}
      \item Programs become nonmodular and noncompositional:
        not only do recursive functions that require inlining
        depend on the implementation of inlined functions,
        they may also depend on their specific implementation details,
        and even a minor syntactic change otherwise equivalent to the original
        can break guardedness~\citep{CIC-hat-minus}.
      \item As aptly summarized by \citet{coqterm},
        \begin{quote}
        \begin{singlespace}
        \textit{{\rm [\ldots]} unfold{\rm [ing]} all the definitions used in the body of the function, do{\rm [ing]} reductions, e.t.c.
        {\rm [\ldots]} makes typechecking extremely slow at times.
        Also, the unfoldings can cause the code to bloat by orders of magnitude and become impossible to debug.}
        \end{singlespace}
        \end{quote}
    \end{itemize}
\end{itemize}

In cases where the guardedness check fails on terminating functions,
the programmer can refactor the function to recur on a different, related argument,
such as the length of a list in the example above,
but this requires additional work and may bloat programs
with extra code solely for satisfying the guardedness check
that detracts from the programs' original intent.
Alternatives to the syntactic guardedness check that avoid these issues are type-based checks,
where the type system itself is augmented so that successful type checking
immediately guarantees termination and consistency.

The one this thesis focuses on is \emph{sized typing}\punctstack{,}%
\footnote{The other common one uses \emph{guarded types}~\citep{guarded-types}.}
where inductive types are annotated with size information,
and constructors construct constructions whose size is larger than that of their subarguments.
In short, the notion of a ``smaller'' subargument is encoded within the types,
no longer requiring syntactic analysis to determine.
Finally, a recursive function is well typed only if it recurs on an argument
whose size is smaller than that of the original argument.

Because well-typedness is the only required condition,
calling some other function before a recursive call is allowed
as long as that function is \emph{size preserving}\index{size preservation}.
For the list mapping example above,
this means that if the sublist of elements of type $\tau$ has size $s$,
then the mapping function must have type $\arr*{\List{\tau}{s}}{\List{\tau}{s}}$.
Notice that the implementation of the mapping function isn't required,
merely its type, restoring modularity and compositionality.

This thesis then introduces Sizey \textsc{mc}Type Theory (\lang),
a sized dependent type theory that I prove to be logically consistent
and therefore suitable for both recursive programming and theorem proving.

\section{Overview}

But what need is there for yet another sized type system?
Since \citet{hughes}, there has been a little over two decades' worth of past work on sized types.
\citet{flationary} notes that it has been the topic of at least five dissertations.
This chapter alone names a dozen different type systems with sized types.
There have been many advancements along the way,
but none quite satisfactory to close the matter,
especially when it comes to dependent types.

\lang and this thesis as a whole does not aim to resolve
all of the remaining problems of sized types left open by these past works.
Instead, the purpose is threefold:

\begin{itemize}
  \item To fill an existing gap in sized dependent type theories.
    Two modern sized type features are higher-rank and bounded size quantification,
    and there exist a few dependent type theories with one or the other,
    but not both, to my knowledge.
    On the other hand, they are currently in Agda's implementation of sized types.
    By proving the consistency of \lang, a novel type theory with these two features,
    I bring the theoretical state-of-the-art closer to practice. \\
  \item To demonstrate the viability of using a syntactic model to prove consistency
    of a sized type theory.
    Doing so from scratch through a set-theoretic model is notoriously difficult,
    and often requires sacrifices to and constraints on the type theory;
    see Sacchini's dissertation~\citep{CIC-hat-minus} or \CIChatstar~\citep{CIC-hat-star},
    for instance.
    Instead, a syntactic model relies on the consistency of the language in which I'm modelling,
    allowing me to focus on only the interesting parts of the sized type theory
    rather than on consistency as a whole.
  \item To reopen the discussion on sized dependent types.
    Since the discovery of the inconsistency of Agda's sized types~\citep{infinity},
    progress appears to have stagnated a little.
    In \cref{ch:discussion}, I examine these difficulties as they might apply to \lang
    and its interpretation in the syntactic model.
\end{itemize}

In this section I describe and justify the design of
the dependent types and the sized types of \lang,
explain how a syntactic model is used to prove its consistency,
and briefly outline the structure of the proof.

\subsection{Dependent types}

As a simple but expressive foundation for dependent types,
I start with the Generalized Calculus of Constructions (\GCC)\index{Calculus of Constructions!Generalized \textasciitilde} \citep{GCC-Coquand},
which has the following features:

\begin{itemize}
  \item \textbf{Dependent function types}, as is standard in the Calculus of Constructions (CC)~\citep{CoC}\index{Calculus of Constructions};
  \item \textbf{Definitions}, \ie locally-named expressions;
  \item \textbf{Universes \ala Russell}\index{universes \ala Russell}, where the types of types are themselves terms
    (as opposed to universes \ala Tarski\index{universes \ala Tarski}, where their \emph{encodings} are terms);
  \item An \textbf{impredicative universe}\index{impredicativity} $\Prop$ such that function types into types in $\Prop$
    are themselves in $\Prop$;
  \item A \textbf{cumulative hierarchy of universes}\index{cumulativity} such that $\Type{i}: \Type{i+1}$,
    any term in $\Type{i}$ is also in $\Type{j}$ given \emph{universe levels}\index{universe level} $i \leq j$,
    and there is a subtyping relation\index{subtyping} on types induced by this inclusion; and
  \item \textbf{Untyped definitional equality} stating when two terms are judgementally equal to one another.
\end{itemize}

These features cover many modern proof assistants.
To name a few, in terms of universes,
Coq and Arend have all of the above;
Lean lacks cumulativity; and
Agda and \Fstar lack cumulativity and an impredicative universe.
On the other hand, these proof assistants all have some form of
\emph{universe level polymorphism},
but this is much more complex and largely orthogonal to sized types
and the syntactic model.

Perhaps the most contentious design decision so far is the use of untyped equality,
as can be found in Coq, over typed equality, as can be found in Agda.
While typed equality is considered to ``have clearer mathematical semantics''~\citep{typed-NbE},
its judgement depends on the typing judgement;
meanwhile, since types can depend on terms,
typing itself depends on equality to check whether one type can be used in place of another.
As we'll see in \cref{sec:syntactic-model},
these mutually-defined judgements would greatly complicate the proofs for the syntactic model,
so I settle for untyped equality instead.

On top of \GCC, I add two inductive definitions featured in Martin--L\"of type theory (MLTT)~\citep{MLTT}\index{Martin--L\"of type theory}:
the \emph{Peano naturals} and \emph{well-founded trees}\punctstack{,}%
\footnote{The types of well-founded trees are also known as \emph{W types}.}
augmented with sizes.
As the simplest nontrivial inductive,
the naturals make it easy to demonstrate intuitive uses of sized inductives.
On the other hand, well-founded trees are an example of \emph{generalized} inductives,
with recursive arguments that are functions that return well-founded trees.
They can encode all (nonnested) inductives,
as well as their induction principles \citep{whynotW} if there are dependent pair types
and a \emph{propositional equality}\index{propositional equality} type.
I don't add inductive types in general in their place
because the syntactic baggage that comes with handling the generalization
obscures the intuition behind sized inductive types and the syntactic model,
while it's easy to see how one \emph{could} go from the naturals and well-founded trees
to inductive types in general.

\subsection{Sized types}\label{sec:sized-types}

Sized types begin with explicit size quantification $\Funtype{\alpha}{\tau}$,
size abstraction $\Fun{\alpha}{e}$, and size application $\App{e}{s}$.
Sizes themselves consist of size variables, a \emph{base size},
and a \emph{size successor operator}.
This is the standard for size expressions in sized type systems.
Some augment the size grammar with addition of or scalar multiplication of size variables,
for instance, increasing expressivity at the expense of added complexity in size checking.
To keep things simple, I don't include these features and stick to successor sizes.
Size expressions here are \emph{not} terms,
and their quantifications, abstractions, and applications
are syntactically distinct from those of terms,
similar to how, in nondependent polymorphic type systems,
types are distinct from terms.

Having explicit sizes differs from some most prior sized type systems where,
extending the type polymorphism analogy,
there is only implicit \emph{rank-1} or
\emph{prenex} size quantification:
size quantifications never appear inside of a type,
and in fact all size abstractions and applications are fully inferred.
Explicit sizes, in contrast, let us express
\emph{higher-rank} size quantification,
which allow for more expressiveness:
for instance, supposing we have cons lists parametrized over some sized type $\tau$,
one could write a size-preserving mapping function over a list
that leaves the size of its elements untouched.
The type of such a function might be

\vspace{-0.25\baselineskip}
$$\Funtype{\alpha}{\arr*{(\Funtype{\beta}{\arr*{\App{\tau}{\beta}}{\App{\tau}{\beta}}})}{\app{\List*}{(\App{\tau}{\alpha})}}{\app{\List*}{(\App{\tau}{\alpha})}}}.$$

Along with higher-rank sizes, I also include \emph{bounded} size quantification $\Funtype<{\alpha}{s}{\tau}$
and abstraction $\Fun<{\alpha}{s}{e}$.
An order on sizes is induced by these bound instantiations and the successor operator;
this order has nothing to do with subtyping,
and in particular we do \emph{not} have subtyping relations between
$\N{\alpha}$ and $\N{\sss{\alpha}}$, for instance.
Fixpoint expressions recur on smaller sizes according to the order,
summarized by the below typing rule.
%
\begin{mathpar}
\inferrule[]{
  \check{\Phi, \alpha; \Gamma, f: \Funtype<{\beta}{\alpha}{\subst{\tau}{\alpha}{\beta}}}{e}{\tau}
}{
  \infer{\Phi; \Gamma}{\fix{f}{\alpha}{\tau}{e}}{\Funtype{\alpha}{\tau}}
}
\end{mathpar}

Bounded sizes were originally introduced to avoid inconsistencies
from pattern-matching on sizes~\citep{MiniAgda}.
Although \lang has no pattern-matching mechanism,
this style of recursion is more elegant because it corresponds neatly to well-founded induction on sizes,
and because it does away with complex \emph{semi-continuity} or approximative \emph{polarity}
requirements on fixpoints' types that would otherwise be needed for consistency.
Indeed, to \citet{flationary},

\begin{quote}
\begin{singlespace}
\textit{A technical condition like semi-continuity can kill a system
as a candidate for the foundation of logics and programming
{\rm [\ldots]} Most systems for type-based termination replace semi-continuity by a rough approximation,
trading expressivity for simplicity}.
\end{singlespace}
\end{quote}

In summary, the sized type features I include are:

\begin{itemize}[noitemsep]
  \item \textbf{Explicit size} quantification $\Funtype{\alpha}{\tau}$,
    abstraction $\Fun{\alpha}{e}$, and
    application $\App{e}{s}$;
  \item A \textbf{simple size grammar} with size variables $\alpha$, a base size $\circ$, and successors $\sss{s}$;
  \item \textbf{Higher-rank sizes}, \eg $\arr*{(\Funtype{\alpha}{\tau})}{\sigma}$;
  \item \textbf{Bounded size} quantification $\Funtype<{\alpha}{s}{\tau}$ and
  abstraction $\Fun<{\alpha}{s}{e}$.
\end{itemize}

Notably, these are all features found in Agda's implementation of sized types.
On the other hand, \lang excludes Agda's \emph{infinite size}\index{infinite size}
that is strictly larger than all sizes,
since this property is known to be inconsistent in Agda.
A demonstration of the inconsistency of a hypothetical infinite size in \lang
is given in \cref{sec:infinity}.
On the other hand, the lack of an infinite size prevents
certain infinitary constructs from being expressed;
\cref{sec:examples:limitations} gives a concrete example.

\subsection{Syntactic model}\label{sec:syntactic-model}

To show that \lang is consistent and therefore a suitable type theory for proofs,
I define for it a \emph{syntactic model}\index{syntactic model}~\citep{syntactic-models}.
This involves defining a translation from \lang into some target type theory
in whose consistency we have more confidence---in this case,
the \emph{Extensional Calculus of Inductive Constructions}\index{Calculus of Inductive Constructions!Extensional \textasciitilde}
(\CICE)~\citep{CICE}.
It, too, has a cumulative hierarchy of universes \ala Russell and an impredicative universe,
augmented with inductive types and \emph{equality reflection}\index{equality reflection},
where a definitional equality between terms can be derived from a propositional one between them.
The most notable difference from \lang, aside from these features and sized types,
is that \CICE uses a typed, declarative equality rather than an untyped, algorithmic equality
to better accommodate equality reflection.
For concision and to avoid confusion, I henceforth refer to the former as \emph{equivalence}\index{equivalence},
and to the latter as \emph{conversion}\index{conversion}.

This translation from \lang to \CICE must be \emph{type preserving}\index{type preservation}%
\footnote{This terminology comes from compilation;
in the context of syntactic modelling,
this is known as \emph{soundness} with respect to the translation target.}:
if some term $e$ is well typed under some environments $\Phi; \Gamma$ with some type $\tau$,
then the translated term $\compile{e}$ must also be well typed
under the translated environment $\compile{\Phi}, \compile{\Gamma}$
with the translated type $\compile{\tau}$.
By the consistency of \CICE and a type-preserving translation to it,
we prove the consistency of \lang.

\begin{postulate}[Consistency of \CICE]\label{fact:consistency-cice}
There exists no term $\eT$ such that
\mbox{$\type{\mt}{\eT}{\funtypeT{\PT}{\PropT}{\PT}}$}.
\end{postulate}

\begin{theorem}[Consistency of \lang]\label{thm:overview:consistency}
Suppose $\compile{\bot} = \funtypeT{\PT}{\PropT}{\PT}$.
Then there exists no term $e$ such that \mbox{$\type{\mt \mathbin{;} \mt}{e}{\bot}$}.
\end{theorem}
\begin{proof}
Suppose that there were such a term $e$.
By the type-preserving translation, we would have that
$\type{\mt}{\compile{e}}{\funtypeT{\PT}{\PropT}{\PT}}$ holds.
However, this contradicts \cref{fact:consistency-cice},
so there must not be such a term.
\end{proof}

What remains, then, is to define $\bot$ and an appropriate translation from \lang to \CICE
such that $\compile{\bot} = \funtypeT{\PT}{\PropT}{\PT}$,
and to show that this translation is type preserving.
I define my translation by induction on the typing derivations of \lang
rather than merely over its syntax,
ensuring that only well-typed terms need to be considered.

Type preservation of the translation is proven by induction on not only the typing derivations,
but also the derivations of the judgements on which the typing rules depend.
In particular, typing depends on subtyping\index{subtyping} $\preccurlyeq$,
which in turn depends on \emph{$\alpha$-cumulativity}\index{$\alpha$-cumulativity}
$\sqsubseteq$~\citep{MetaCoq}
and on the reflexive, transitive closure of reduction $\rhd^*$.
Finally, this closure of reduction depends on the reduction rules $\rhd$,
most of which are defined by substitution.
Below are the relevant rules showing these dependencies and an example reduction rule.
These judgements will be explained in detail shortly in the next chapter;
I present them here only to motivate the proof structure.
%
\begin{mathpar}
\inferrule[]{
  \dots \\\\
  \infer{\Phi; \Gamma}{e}{\sigma} \\\\
  \subtype{\Phi; \Gamma}{\sigma}{\tau}
}{
  \check{\Phi; \Gamma}{e}{\tau}
}

\inferrule[]{
  \acum{\sigma_1}{\sigma_2} \\\\
  \red*{\Phi; \Gamma}{\tau_1}{\sigma_1} \\\\
  \red*{\Phi; \Gamma}{\tau_2}{\sigma_2}
}{
  \subtype{\Phi; \Gamma}{\tau_1}{\tau_2}
}

\inferrule[]{
  \red{\Phi; \Gamma}{e_1}{e_2}
}{
  \red*{\Phi; \Gamma}{e_1}{e_2}
}

\inferrule[]{~}{
  \red{\Phi; \Gamma}{\app{(\fun{x}{\tau}{e})}{e'}}{\subst{e}{x}{e'}}
}
\end{mathpar}

We therefore need some lemmas showing that the translation respects
substitution, reduction, the closure of reduction, $\alpha$-cumulativity, and subtyping
to prove type preservation.
In short, substitution satisfies a \emph{compositionality}\index{compositionality} principle,
reduction of terms are equivalent by the \CICE equivalence judgement
$\defeq{\GammaT}{\eT_1}{\eT_2}{\tauT}$,
and $\alpha$-cumulative terms and subtypes are correspondingly \CICE subtypes.
I summarize them here, omitting some hypotheses and sublemmas,
solely to outline the proof architecture,
which roughly follows the structure of type-preserving compilation by~\citet{wjb}.

\begin{lemma}[Compositionality]\label{lem:overview:compositionality}\hfill
\begin{enumerate}[noitemsep]
  \item $\compile{\subst{e}{x}{e'}} = \subst{\compile{e}}{x}{\compile{e'}}$.
  \item $\compile{\subst{e}{\alpha}{s}} = \subst{\compile{e}}{\alpha}{\compile{s}}$
\end{enumerate}
\end{lemma}

\begin{proof}
By induction on the derivation of $\type{\Phi; \Gamma}{e}{\tau}$.
\end{proof}

\begin{lemma}[Preservation of reduction]\label{lem:overview:pres-red}
If $\red{\Phi; \Gamma}{e}{e'}$ and
$\type{\compile{\Phi}\compile{\Gamma}}{\compile{e}}{\tauT}$
then $\defeq{\compile{\Phi}\compile{\Gamma}}{\compile{e}}{\compile{e'}}{\tauT}$.
\end{lemma}

\begin{proof}
By cases on the derivation of $\red{\Phi; \Gamma}{e}{e'}$,
using inversion on the derivation of $\type{\compile{\Phi}\compile{\Gamma}}{\compile{e}}{\compile{\tau}}$,
\cref{lem:overview:compositionality},
and an application of equality reflection in the case for reduction of fixpoints.
\end{proof}

\begin{lemma}[Preservation of reflexive, transitive closure of reduction]\label{lem:overview:pres-red*}
If $\red*{\Phi; \Gamma}{e}{e'}$ and
$\type{\compile{\Phi}\compile{\Gamma}}{\compile{e}}{\tauT}$
then $\defeq{\compile{\Phi}\compile{\Gamma}}{\compile{e}}{\compile{e'}}{\tauT}$.
\end{lemma}

\begin{proof}
By induction on the derivation of $\red*{\Phi; \Gamma}{e}{e'}$,
using \cref{lem:overview:pres-red},
subject reduction of \lang,
and subject equivalence of \CICE.
\end{proof}

\begin{lemma}[Preservation of $\alpha$-cumulativity]\label{lem:overview:pres-acum}
If $\acum{\tau_1}{\tau_2}$ and
$\type{\compile{\Phi}\compile{\Gamma}}{\compile{\tau_i}}{\UT_i}$
then $\subtype{\compile{\Phi}\compile{\Gamma}}{\compile{\tau_1}}{\compile{\tau_2}}$.
\end{lemma}

\begin{proof}
By induction on the derivation of $\acum{\tau_1}{\tau_2}$,
using inversion on the derivations of $\type{\Phi; \Gamma}{\tau_i}{U_i}$ and
$\type{\compile{\Phi}\compile{\Gamma}}{\compile{\tau_i}}{\UT_i}$.
\end{proof}

\begin{lemma}[Preservation of subtyping]\label{lem:overview:pres-subtyping}
If $\subtype{\Phi; \Gamma}{\tau_1}{\tau_2}$
and $\type{\compile{\Phi}\compile{\Gamma}}{\compile{\tau_i}}{\compile{U}}$
then $\subtype{\compile{\Phi}\compile{\Gamma}}{\compile{\tau_1}}{\compile{\tau_2}}$.
\end{lemma}

\begin{proof}
By cases on the derivation of $\subtype{\Phi; \Gamma}{\tau_1}{\tau_2}$,
using \cref{lem:overview:pres-red*}, \cref{lem:overview:pres-acum},
subject reduction of \lang,
and subject equivalence of \CICE.
\end{proof}

\begin{theorem}[Type preservation]\label{lem:overview:pres-typing}\hfill
\begin{enumerate}[noitemsep]
  \item If $\wf{\Phi}{\Gamma}$ then $\wf{}{\compile{\Phi}\compile{\Gamma}}$.
  \item If $\type{\Phi; \Gamma}{e}{\tau}$ then $\type{\compile{\Phi}\compile{\Gamma}}{\compile{e}}{\compile{\tau}}$.
\end{enumerate}
\end{theorem}

\begin{proof}
By mutual induction on the derivations of $\wf{\Phi}{\Gamma}$ and $\type{\Phi; \Gamma}{e}{\tau}$,
using \cref{lem:overview:pres-subtyping}.
\end{proof}

The most important properties required of \lang are subject reduction\index{subject reduction},
used in \cref{lem:overview:pres-subtyping},
and confluence\index{confluence}, which is used to show that subtyping is transitive,
which in turn is used to prove the inversion principles for the typing judgement.

This proof structure is possible by virtue of the non-mutuality of the judgements of \lang
due to the use of untyped conversion\index{conversion} (and hence untyped reduction).
If it were instead typed, then it would mutually depend on typing and subtyping as well,
and proving the above lemmas would require a seven-part mutual induction on the derivations.

\section{Related Work}

The history and development of sized types spans over two decades of past work,
and that of dependent types far more than that.
I therefore mention here only past work that are directly relevant
or that I refer to again in later chapters.

\subsection{Dependent type theories}

\GCC\index{Calculus of Constructions!Generalized \textasciitilde}
was originally proposed by \citet{GCC-Coquand},
adding a universe hierarchy to CC\index{Calculus of Constructions} with untyped equality
and using \rref{cum} for cumulativity\index{cumulativity}.
The \emph{Extended Calculus of Constructions}\index{Calculus of Constructions!Extended \textasciitilde}
(ECC) by \citet{ECC} adds a dependent pair type
and uses a subtyping relation instead for cumulativity;
\rref{untyped-subtype-prop, untyped-subtype-pi} are the most notable subtyping rules.

\vspace{-\baselineskip}
\begin{mathpar}
\inferrule[\rlabel*{cum}]{
  \type{\Gamma}{e}{\Type{i}}
}{
  \type{\Gamma}{e}{\Type{i+1}}
}
\and
\inferrule[\rlabel{$\preccurlyeq$-prop}{untyped-subtype-prop}]{~}{
  \subtype{\Gamma}{\Prop}{\Type{i}}
}
\and
\inferrule[\rlabel{$\preccurlyeq$-pi}{untyped-subtype-pi}]{
  \Gamma \vdash \sigma_1 \approx \sigma_2 \\
  \subtype{\Gamma, \annot{x}{\sigma_2}}{\tau_1}{\tau_2}
}{
  \subtype{\Gamma}{\funtype{x}{\sigma_1}{\tau_1}}{\funtype{x}{\sigma_2}{\tau_2}}
}
\end{mathpar}

\citet{universes} introduce ways of adding universe polymorphism to \GCC
as well as to a version of \GCC with definitions.
For the purposes of this thesis, I use \GCC to refer to the type theory
with subtyping and definitions.

The \emph{Calculus of Inductive Constructions}\index{Calculus of Inductive Constructions} (CIC) \citep{CIC}
adds inductive types to CC,
while the \emph{Predicative Calculus of Inductive Constructions}\index{Calculus of Inductive Constructions!Predicative \textasciitilde} (pCIC)
further adds definitions and a cumulative universe hierarchy via subtyping\punctstack{.}%
\footnote{More precisely, pCIC refers to Coq's core CIC with the cumulative universe hierarchy
and without an \emph{impredicative $\Set$} universe from version 8 onwards \citep[Chapter~4]{Coq-manual}.
The Coq Reference Manual from version 8.5 onwards refers to it simply as CIC,
as do many others, \eg \citet{CIC-unifier}.
I prefer to use the specific name pCIC,
since most seem to use CIC to refer to whatever Coq's core calculus happens to be based on at the moment,
which is currently pCuIC.}
\citet{pCIC} give a set-theoretic model for a variant of pCIC with typed equality.
The \emph{Predicative Calculus of Cumulative Inductive Constructions}
\index{Predicative Calculus of Cumulative Inductive Constructions} (pCuIC) \citep{pCuIC}
further extends pCIC with additional cumulativity\index{cumulativity} between inductive types,
also providing a set-theoretic model,
and it serves as the modern core calculus of Coq.
Closely related is the MetaCoq project~\citep{MetaCoq}\index{MetaCoq},
which mechanizes pCuIC and various of its metatheorems within Coq itself,
with additional alternate untyped equality and subtyping judgements,
since Coq's implementation of definitional equality is untyped.

\citet{CCE} adds a propositional equality type with
equality reflection\index{equality reflection} to \GCC with \rref{cum} in \CCE,
although transitivity of definitional equality only holds for well-typed terms,
and gives a syntactic model in an extension of CIC.
Following that, \citet{CICE} improve upon the translation,
using ECC with a propositional equality type, typed equality,
and two extensional axioms,
translating from this type theory with equality reflection (ETT) to one without (ITT).
They implement the translation in Template Coq \citep{TemplateCoq},
extending it to include inductive types.

In contrast to these type systems based on CC,
MLTT\index{Martin--L\"of type theory} is a dependent type theory
with a noncumulative universe hierarchy without $\Prop$,
typed equality, and some basic types: the empty type, the unit type,
dependent pair types, disjoint sum types, and a propositional equality type,
as well as the types of naturals and well-founded trees as mentioned.
They can all be implemented as inductive types in pCIC.

The source language \lang and target language \CICE in this thesis aren't exactly
a single type theory from the literature augmented with sized types.
In short, \lang is based on \GCC but with MetaCoq's alternate subtyping rules
and augmented with naturals and well-founded trees,
while \CICE is based on pCIC with extensionality inspired by ETT.

\subsection{Sized type systems}

Sized dependent type systems date all the way back to \CCR by \citet{CCR},
although its sized types are formulated quite differently from ``modern'' sized types.
The first sized dependent type theory recognized as such is \CIChat by \citet{CIC-hat},
which adds prenex sizes
and a size inference algorithm to CIC.
In the intervening years, several nondependent sized type systems were developed,
notably the ML-like type system by \citet{hughes} (independently of \CCR),
\lambdahat~\citep{lambda-hat, lambda-hat-diss},
\Fhat~\citep{F-hat}, and
\Fhattimes~\citep{F-hat-times},
all of which focus on prenex, fully-inferrable sizes.
\Fhatomega~\citep{Abel-diss}, on the other hand,
has higher-rank, explicitly-quantified sizes.
Other sized type systems and more extensive discussions can be found in dissertations by
\citet{lambda-hat-diss} and \citet{Abel-diss}.

\begin{table}[h]
\centering
\iffalse
\begin{tabular}{l c c c l}
Type system & Explicit sizes? & Higher-rank? & Bounded? & Size algebra \\
\hline
\citet{hughes} & \crossmark & \crossmark & \crossmark & $\sss{s}, s + s, n \times s$ \\
\lambdahat~\citep{lambda-hat, lambda-hat-diss} & \crossmark & \crossmark & \crossmark & $\sss{s}$ \\
\Fhat~\citep{F-hat} & \crossmark & \crossmark & \crossmark & $\sss{s}$ \\
\Fhattimes~\citep{F-hat-times} & \crossmark & \crossmark & \crossmark & $\sss{s}, s + s$ \\
\Fhatomega~\citep{Abel-diss} & \checkmark* & \checkmark* & \crossmark & $\sss{s}$ \\
\Fcopomega~\citep{F-omega-cop} & \checkmark* & \checkmark* & \checkmark* & $\sss{s}$ \\
\end{tabular}
\fi

\begin{tabular}{l c c c c}
Type theory & Explicit sizes? & Higher-rank? & Bounded? & Consistent? \\
\hline
\CIChat~\citep{CIC-hat} & \crossmark & \crossmark & \crossmark & \interromark \\
\CIChatminus~\citep{CIC-hat-minus-nat, CIC-hat-minus} & \crossmark & \crossmark & \crossmark & \checkmark* \\
\CChatomega~\citep{CC-hat-omega} & \crossmark & \crossmark & \crossmark & \checkmark* \\
\CIChatl~\citep{CIC-hat-l} & \crossmark & \crossmark & \crossmark & \checkmark* \\
\CIChatsub~\citep{CIC-hat-sub} & \crossmark & \crossmark & \checkmark* & \checkmark* \\
\CIChatstar~\citep{CIC-hat-star} & \crossmark & \crossmark & \crossmark & \interromark \\
\citet{NbE} & \checkmark* & \checkmark* & \crossmark & \checkmark* \\
MiniAgda~\citep{MiniAgda, flationary} & \checkmark* & \checkmark* & \checkmark* & \crossmark \\
\textbf{\lang} & \checkmark* & \checkmark* & \checkmark* & \checkmark*%
\textsuperscript{\labelcref{foot:CICE-consistency}}
\end{tabular}
\caption{Sized dependent type theories and their properties}
\label{tab:sized-types}
\end{table}

\cref{tab:sized-types} lists the sized dependent type theories that follow \CIChat,
along with the main features I focus on:
explicit, higher-rank, bounded size quantification,
and with whether the type theory has been proven consistent.
The type theory of MiniAgda, on which the sized types implemented in Agda are based,
is the only prior one with all three sized type features,
but is known to be inconsistent~\citep{infinity}.
(Interestingly, \Fcopomega~\citep{F-omega-cop}, which extends System F$_\omega$ with (co)inductive types and sized types,
has all three properties \emph{and} has been proven to be strongly normalizing.)
One of the goals of \lang, as mentioned, is to fill in this gap
with a consistent sized dependent type theory with all of the desired features.
However, it's missing two properties that all of these type theories do have:
inductive types in the ``correct'' universes
(as we'll see shortly, the type of naturals and well-founded trees
are one universe higher than is conventional),
and the existence of an infinite size.

% TODO: fix manual footnote number
\footnotetext[6]{\label{foot:CICE-consistency} Under the postulate that \CICE is consistent.}

\begin{center}
\mbox{* * *}
\end{center}
\vspace*{-0.5\baselineskip}

\noindent In the upcoming \cref{ch:sized-dep-types}, I describe the syntax and judgements of \lang in detail
and provide example programs in \lang that use sized types.
Next, in \cref{ch:model} I describe \CICE and define the translation from \lang to \CICE.
Following that, in \cref{ch:proofs} I prove various necessary metatheoretical properties
including confluence and subject reduction,
and elaborate on the proof of type preservation and its associated lemmas.
I conclude in \cref{ch:discussion}, discussing some of the shortcomings of \lang,
namely the universe levels of the inductive types and
decreased expressivity arising from removing the infinite size,
as well directions for future investigation.