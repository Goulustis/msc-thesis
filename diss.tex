%%%%%%%%%%%%%%%%%%%%%%%%%%%%%%%%%%%%%%%%%%%%%%%%%%%%%%%%%%%%%%%%%%%%%%
% Template for a UBC-compliant dissertation
% At the minimum, you will need to change the information found
% after the "Document meta-data"
%
%!TEX TS-program = pdflatex
%!TEX encoding = UTF-8 Unicode

%% The ubcdiss class provides several options:
%%   gpscopy (aka fogscopy)
%%       set parameters to exactly how GPS specifies
%%         * single-sided
%%         * page-numbering starts from title page
%%         * the lists of figures and tables have each entry prefixed
%%           with 'Figure' or 'Table'
%%       This can be tested by `\ifgpscopy ... \else ... \fi'
%%   10pt, 11pt, 12pt
%%       set default font size
%%   oneside, twoside
%%       whether to format for single-sided or double-sided printing
%%   balanced
%%       when double-sided, ensure page content is centred
%%       rather than slightly offset (the default)
%%   singlespacing, onehalfspacing, doublespacing
%%       set default inter-line text spacing; the ubcdiss class
%%       provides \textspacing to revert to this configured spacing
%%   draft
%%       disable more intensive processing, such as including
%%       graphics, etc.
%%

% For submission to GPS
\documentclass[gpscopy,twoside,balanced,onehalfspacing,11pt]{ubcdiss}

% For your own copies (looks nicer)
% \documentclass[balanced,twoside,11pt]{ubcdiss}

%%%%%%%%%%%%%%%%%%%%%%%%%%%%%%%%%%%%%%%%%%%%%%%%%%%%%%%%%%%%%%%%%%%%%%
%%%%%%%%%%%%%%%%%%%%%%%%%%%%%%%%%%%%%%%%%%%%%%%%%%%%%%%%%%%%%%%%%%%%%%
%%
%% FONTS:
%% 
%% The defaults below configures Times Roman for the serif font,
%% Helvetica for the sans serif font, and Courier for the
%% typewriter-style font.  Configuring fonts can be time
%% consuming; we recommend skipping to END FONTS!
%% 
%% If you're feeling brave, have lots of time, and wish to use one
%% your platform's native fonts, see the commented out bits below for
%% XeTeX/XeLaTeX.  This is not for the faint at heart. 
%% (And shouldn't you be writing? :-)
%%

%% NFSS font specification (New Font Selection Scheme)
%\usepackage{times,mathptmx,courier}
%\usepackage[scaled=.92]{helvet}
\usepackage{mlmodern}

%% Math or theory people may want to include the handy AMS macros
%\usepackage{amssymb}
%\usepackage{amsmath}
%\usepackage{amsfonts}

%% The pifont package provides access to the elements in the dingbat font.   
%% Use \ding{##} for a particular dingbat (see p7 of psnfss2e.pdf)
%%   Useful:
%%     51,52 different forms of a checkmark
%%     54,55,56 different forms of a cross (saltyre)
%%     172-181 are 1-10 in open circle (serif)
%%     182-191 are 1-10 black circle (serif)
%%     192-201 are 1-10 in open circle (sans serif)
%%     202-211 are 1-10 in black circle (sans serif)
%% \begin{dinglist}{##}\item... or dingautolist (which auto-increments)
%% to create a bullet list with the provided character.
\usepackage{pifont}

%%%%%%%%%%%%%%%%%%%%%%%%%%%%%%%%%%%%%%%%%%%%%%%%%%%%%%%%%%%%%%%%%%%%%%
%% Configure fonts for XeTeX / XeLaTeX using the fontspec package.
%% Be sure to check out the fontspec documentation.
%\usepackage{fontspec,xltxtra,xunicode}	% required
%\defaultfontfeatures{Mapping=tex-text}	% recommended
%% Minion Pro and Myriad Pro are shipped with some versions of
%% Adobe Reader.  Adobe representatives have commented that these
%% fonts can be used outside of Adobe Reader.
%\setromanfont[Numbers=OldStyle]{Minion Pro}
%\setsansfont[Numbers=OldStyle,Scale=MatchLowercase]{Myriad Pro}
%\setmonofont[Scale=MatchLowercase]{Andale Mono}

%% Other alternatives:
%\setromanfont[Mapping=tex-text]{Adobe Caslon}
%\setsansfont[Scale=MatchLowercase]{Gill Sans}
%\setsansfont[Scale=MatchLowercase,Mapping=tex-text]{Futura}
%\setmonofont[Scale=MatchLowercase]{Andale Mono}
%\newfontfamily{\SYM}[Scale=0.9]{Zapf Dingbats}
%% END FONTS
%%%%%%%%%%%%%%%%%%%%%%%%%%%%%%%%%%%%%%%%%%%%%%%%%%%%%%%%%%%%%%%%%%%%%%
%%%%%%%%%%%%%%%%%%%%%%%%%%%%%%%%%%%%%%%%%%%%%%%%%%%%%%%%%%%%%%%%%%%%%%



%%%%%%%%%%%%%%%%%%%%%%%%%%%%%%%%%%%%%%%%%%%%%%%%%%%%%%%%%%%%%%%%%%%%%%
%%%%%%%%%%%%%%%%%%%%%%%%%%%%%%%%%%%%%%%%%%%%%%%%%%%%%%%%%%%%%%%%%%%%%%
%%
%% Recommended packages
%%
\usepackage[margin=1.25in]{geometry}
\usepackage{checkend}	% better error messages on left-open environments
\usepackage{graphicx}	% for incorporating external images

%% booktabs: provides some special commands for typesetting tables as used
%% in excellent journals.  Ignore the examples in the Lamport book!
\usepackage{booktabs}

%% listings: useful support for including source code listings, with
%% optional special keyword formatting.  The \lstset{} causes
%% the text to be typeset in a smaller sans serif font, with
%% proportional spacing.
\usepackage{listings}
\lstset{basicstyle=\sffamily\scriptsize,showstringspaces=false,fontadjust}

%% The acronym package provides support for defining acronyms, providing
%% their expansion when first used, and building glossaries.  See the
%% example in glossary.tex and the example usage throughout the example
%% document.
%% NOTE: to use \MakeTextLowercase in the \acsfont command below,
%%   we *must* use the `nohyperlinks' option -- it causes errors with
%%   hyperref otherwise.  See Section 5.2 in the ``LaTeX 2e for Class
%%   and Package Writers Guide'' (clsguide.pdf) for details.
\usepackage[printonlyused,nohyperlinks]{acronym}
%% The ubcdiss.cls loads the `textcase' package which provides commands
%% for upper-casing and lower-casing text.  The following causes
%% the acronym package to typeset acronyms in small-caps
%% as recommended by Bringhurst.
\renewcommand{\acsfont}[1]{{\scshape \MakeTextLowercase{#1}}}

%% color: add support for expressing colour models.  Grey can be used
%% to great effect to emphasize other parts of a graphic or text.
%% For an excellent set of examples, see Tufte's "Visual Display of
%% Quantitative Information" or "Envisioning Information".
\usepackage{color}
\definecolor{greytext}{gray}{0.5}

%% comment: provides a new {comment} environment: all text inside the
%% environment is ignored.
%%   \begin{comment} ignored text ... \end{comment}
\usepackage{comment}

%% The natbib package provides more sophisticated citing commands
%% such as \citeauthor{} to provide the author names of a work,
%% \citet{} to produce an author-and-reference citation,
%% \citep{} to produce a parenthetical citation.
%% We use \citeeg{} to provide examples
\usepackage[numbers,sort&compress]{natbib}
\newcommand{\citeeg}[1]{\citep[e.g.,][]{#1}}

%% The titlesec package provides commands to vary how chapter and
%% section titles are typeset.  The following uses more compact
%% spacings above and below the title.  The titleformat that follow
%% ensure chapter/section titles are set in singlespace.
\usepackage[compact]{titlesec}
\titleformat*{\section}{\singlespacing\raggedright\bfseries\Large}
\titleformat*{\subsection}{\singlespacing\raggedright\bfseries\large}
\titleformat*{\subsubsection}{\singlespacing\raggedright\bfseries}
\titleformat*{\paragraph}{\singlespacing\raggedright\itshape}

%% The caption package provides support for varying how table and
%% figure captions are typeset.
\usepackage[format=hang,indention=-1cm,labelfont={bf},margin=1em]{caption}

%% url: for typesetting URLs and smart(er) hyphenation.
%% \url{http://...} 
\usepackage{url}
\urlstyle{sf}	% typeset urls in sans-serif


%%%%%%%%%%%%%%%%%%%%%%%%%%%%%%%%%%%%%%%%%%%%%%%%%%%%%%%%%%%%%%%%%%%%%%
%%%%%%%%%%%%%%%%%%%%%%%%%%%%%%%%%%%%%%%%%%%%%%%%%%%%%%%%%%%%%%%%%%%%%%
%%
%% Possibly useful packages: you may need to explicitly install
%% these from CTAN if they aren't part of your distribution;
%% teTeX seems to ship with a smaller base than MikTeX and MacTeX.
%%
%\usepackage{pdfpages}	% insert pages from other PDF files
%\usepackage{longtable}	% provide tables spanning multiple pages
%\usepackage{chngpage}	% support changing the page widths on demand
%\usepackage{tabularx}	% an enhanced tabular environment

%% enumitem: support pausing and resuming enumerate environments.
%\usepackage{enumitem}

%% rotating: provides two environments, sidewaystable and sidewaysfigure,
%% for typesetting tables and figures in landscape mode.  
%\usepackage{rotating}

%% subfig: provides for including subfigures within a figure,
%% and includes being able to separately reference the subfigures.
%\usepackage{subfig}

%% ragged2e: provides several new new commands \Centering, \RaggedLeft,
%% \RaggedRight and \justifying and new environments Center, FlushLeft,
%% FlushRight and justify, which set ragged text and are easily
%% configurable to allow hyphenation.
%\usepackage{ragged2e}

%% The ulem package provides a \sout{} for striking out text and
%% \xout for crossing out text.  The normalem and normalbf are
%% necessary as the package messes with the emphasis and bold fonts
%% otherwise.
%\usepackage[normalem,normalbf]{ulem}    % for \sout

%%%%%%%%%%%%%%%%%%%%%%%%%%%%%%%%%%%%%%%%%%%%%%%%%%%%%%%%%%%%%%%%%%%%%%
%% HYPERREF:
%% The hyperref package provides for embedding hyperlinks into your
%% document.  By default the table of contents, references, citations,
%% and footnotes are hyperlinked.
%%
%% Hyperref provides a very handy command for doing cross-references:
%% \autoref{}.  This is similar to \ref{} and \pageref{} except that
%% it automagically puts in the *type* of reference.  For example,
%% referencing a figure's label will put the text `Figure 3.4'.
%% And the text will be hyperlinked to the appropriate place in the
%% document.
%%
%% Generally hyperref should appear after most other packages

%% The `pagebackref' causes the references in the bibliography to have
%% back-references to the citing page; `backref' puts the citing section
%% number.  See further below for other examples of using hyperref.
%% 2009/12/09: now use `linktocpage' (Jacek Kisynski): GPS now prefers
%%   that the ToC, LoF, LoT place the hyperlink on the page number,
%%   rather than the entry text.
\ifgpscopy
  % GPS requires that weblinks should be dark blue, which looks a bit
  % odd in printed form.
  % https://www.grad.ubc.ca/current-students/dissertation-thesis-preparation/fonts-print
  \usepackage[bookmarks,bookmarksnumbered,%
     pagebackref,linktocpage,%
     colorlinks=true,%
     linkcolor=black,%
     urlcolor=blue,%
     citecolor=black%
     ]{hyperref}
\else
  %% The following puts hyperlinks in very faint grey boxes (in pdf only).
  \usepackage[bookmarks,bookmarksnumbered,%
    pagebackref,linktocpage,%
    allbordercolors={0.8 0.8 0.8},%
    ]{hyperref}
\fi
%% The following change how the the back-references text is typeset in a
%% bibliography when `backref' or `pagebackref' are used
%%
%% Change \nocitations if you'd like some text shown where there
%% are no citations found (e.g., pulled in with \nocite{xxx})
\newcommand{\nocitations}{\relax}
%%\newcommand{\nocitations}{No citations}
%%
%\renewcommand*{\backref}[1]{}% necessary for backref < 1.33
\renewcommand*{\backrefsep}{,~}%
\renewcommand*{\backreftwosep}{,~}% ', and~'
\renewcommand*{\backreflastsep}{,~}% ' and~'
\renewcommand*{\backrefalt}[4]{%
\textcolor{greytext}{\ifcase #1%
\nocitations%
\or
\(\rightarrow\) page #2%
\else
\(\rightarrow\) pages #2%
\fi}}


%% The following uses most defaults, which causes hyperlinks to be
%% surrounded by colourful boxes; the colours are only visible in
%% PDFs and don't show up when printed:
%\usepackage[bookmarks,bookmarksnumbered]{hyperref}

%% The following disables the colourful boxes around hyperlinks.
%\usepackage[bookmarks,bookmarksnumbered,pdfborder={0 0 0}]{hyperref}

%% The following disables all hyperlinking, but still enabled use of
%% \autoref{}
%\usepackage[draft]{hyperref}

%% The following commands causes chapter and section references to
%% uppercase the part name.
\renewcommand{\chapterautorefname}{Chapter}
\renewcommand{\sectionautorefname}{Section}
\renewcommand{\subsectionautorefname}{Section}
\renewcommand{\subsubsectionautorefname}{Section}

%% If you have long page numbers (e.g., roman numbers in the 
%% preliminary pages for page 28 = xxviii), you might need to
%% uncomment the following and tweak the \@pnumwidth length
%% (default: 1.55em).  See the tocloft documentation at
%% http://www.ctan.org/tex-archive/macros/latex/contrib/tocloft/
% \makeatletter
% \renewcommand{\@pnumwidth}{3em}
% \makeatother

%%%%%%%%%%%%%%%%%%%%%%%%%%%%%%%%%%%%%%%%%%%%%%%%%%%%%%%%%%%%%%%%%%%%%%
%%%%%%%%%%%%%%%%%%%%%%%%%%%%%%%%%%%%%%%%%%%%%%%%%%%%%%%%%%%%%%%%%%%%%%
%%
%% Some special settings that controls how text is typeset
%%
% \raggedbottom		% pages don't have to line up nicely on the last line
% \sloppy		% be a bit more relaxed in inter-word spacing
% \clubpenalty=10000	% try harder to avoid orphans
% \widowpenalty=10000	% try harder to avoid widows
% \tolerance=1000

%% And include some of our own useful macros
\makeatletter
\newenvironment{epigraph}{%
	\begin{flushright}
	\begin{minipage}{\columnwidth-0.75in}
	\begin{flushright}
	\@ifundefined{singlespacing}{}{\singlespacing}%
    }{
	\end{flushright}
	\end{minipage}
	\end{flushright}}
\makeatother

% Foreign phrases
\newcommand{\eg}{\emph{e.g.}\xspace}
\newcommand{\ie}{\emph{i.e.}\xspace}
\newcommand{\etal}{\emph{et al.}\xspace}
\newcommand{\ala}{\emph{\`a la}\xspace}
\newcommand{\TODO}{\texttt{TODO}}

% langs
\newcommand{\GCC}{CC$^\omega$}

% Keywords
\newcommand{\Type}[1]{\mathsf{Type}_{#1}}
\newcommand{\Prop}{\mathsf{Prop}}

%%%%%%%%%%%%%%%%%%%%%%%%%%%%%%%%%%%%%%%%%%%%%%%%%%%%%%%%%%%%%%%%%%%%%%
%%%%%%%%%%%%%%%%%%%%%%%%%%%%%%%%%%%%%%%%%%%%%%%%%%%%%%%%%%%%%%%%%%%%%%
%%
%% Document meta-data: be sure to also change the \hypersetup information
%%

\title{sszz}
%\subtitle{If you want a subtitle}

\author{Jonathan Ho\hspace{0.1em}-\hspace{-0.1em}Wing Chan}

% What is this dissertation for?
\degreetitle{Master of Science}

\institution{The University of British Columbia}
\campus{Vancouver}

\faculty{The Faculty of Graduate and Postdoctoral Studies}
\department{Computer Science}
\submissionmonth{April}
\submissionyear{2022}

% details of your examining committee
\examiningcommittee{William J. Bowman, Computer Science}{Supervisor}
\examiningcommittee{Ronald Garcia, Computer Science}{Supervisory Committee Member}
\examiningcommittee{Alexander J. Summers, Computer Science}{Supervisory Committee Member}
\examiningcommittee{Examiner, Department}{Additional Examiner}

% details of your supervisory committee
\supervisorycommittee{Your Mom, Department}{Supervisory Committee Member}

%% hyperref package provides support for embedding meta-data in .PDF
%% files
\hypersetup{
  pdftitle={sszz},
  pdfauthor={Jonathan Chan},
  pdfkeywords={Your keywords here}
}

%%%%%%%%%%%%%%%%%%%%%%%%%%%%%%%%%%%%%%%%%%%%%%%%%%%%%%%%%%%%%%%%%%%%%%
%%%%%%%%%%%%%%%%%%%%%%%%%%%%%%%%%%%%%%%%%%%%%%%%%%%%%%%%%%%%%%%%%%%%%%
%% 
%% The document content
%%

%% LaTeX's \includeonly commands causes any uses of \include{} to only
%% include files that are in the list.  This is helpful to produce
%% subsets of your thesis (e.g., for committee members who want to see
%% the dissertation chapter by chapter).  It also saves time by 
%% avoiding reprocessing the entire file.
%\includeonly{intro,conclusions}
%\includeonly{discussion}

\begin{document}

%%%%%%%%%%%%%%%%%%%%%%%%%%%%%%%%%%%%%%%%%%%%%%%%%%
%% From Thesis Components: Tradtional Thesis
%% <http://www.grad.ubc.ca/current-students/dissertation-thesis-preparation/order-components>

% Preliminary Pages (numbered in lower case Roman numerals)
%    1. Title page (mandatory)
\maketitle

%    2. Committee page (mandatory): lists supervisory committee and,
%    if applicable, the examining committee
\makecommitteepage

%    3. Abstract (mandatory - maximum 350 words)
\chapter{Abstract}

Many contemporary proof assistants based on dependent type theories such as Coq and Agda
are founded on the types-as-propositions paradigm where type checking a program
corresponds to verifying a proof of some proposition in a higher-order predicate logic.
To ensure decidability of type checking and consistency of the logic,
these proof assistants forbid nonterminating recursive functions
using guard predicates that only allow structurally recursive functions
recurring on syntactically smaller arguments.
However, these guard predicates are sometimes too restrictive
and reject simple terminating functions that aren't otherwise structurally recursive.

An alternative is to use type-based termination checking such as sized types,
where inductively-defined types are annotated with sizes.
Successful type checking guarantees that functions recur only on arguments whose types have smaller sizes,
rather than merely on syntactic subarguments.
Some existing models of sized dependent type theories
support features for more expressive sized types,
namely higher-rank size quantification
(which allows for passing around size-preserving functions)
and bounded size quantification
(which eliminates the need for complex semi-continuity checks),
but unfortunately none support both simultaneously.
Meanwhile, the only implementation of sized types in a major proof assistant, Agda,
does support these features, but is unfortunately logically inconsistent.

In this thesis, I design a sized dependent type theory with higher-rank and bounded sizes (\lang),
and show that it's suitable for theorem proving by proving its consistency with a syntactic model:
by compiling \lang into the Extensional Calculus of Inductive Constructions (\CICE),
a variant of Coq's core type theory,
and showing that this translation is type preserving,
the consistency of \lang follows from the consistency of \CICE.

This approach refutes the existence of an ``infinite'' size strictly greater than all sizes,
which is present in prior sized type systems to overcome the limitations of finitary size expressions,
meaning that some infinitary constructs unfortunately aren't definable in \lang.
Even so, \lang provides a valid foundation for sized types in a proof assistant,
opening the way for future work on recovering expressivity lost from the lack of an infinite size
and on restricting sized types in Agda to be consistent.

%    4. Lay Summary (Effective May 2017, mandatory - maximum 150 words)
\chapter{Lay Summary}

While people communicate to one another by speaking or writing in natural languages,
we communicate with computers via programming languages to tell them to, say, perform a calculation.
Just as what's said or written must be grammatically correct to make any sense,
programs written in these programming languages must be checked to ensure that they behave nicely.
One desirable property of programs can be termination:
we want to be certain that they'll eventually finish running at some point.
It's impossible to devise a check that can always pick out all terminating programs,
but approximate termination checks can be improved upon to accept more and more terminating programs.
The topic of this thesis is using \emph{sized types}, a powerful strategy for termination checking,
in the setting of a programming language for mathematicians to write computer-verified proofs,
and proving that the programs sized typing accepts will actually terminate.

%    5. Preface
%% The following is a directive for TeXShop to indicate the main file
%%!TEX root = diss.tex

\chapter{Preface}

The Preface must include a statement indicating the student's contribution to the following:

\begin{itemize}
  \item Identification and design of the research program,
  \item Performance of the various parts of the research, and
  \item Analysis of the research data.
\end{itemize}

Certain additional elements may also be required, as specified below.

\begin{itemize}
  \item If any of the work presented in the thesis has led to any publications or submissions, all of these must be listed in the Preface. Bibliographic details should include the title of the article and the name of the publisher (ONLY if the article has been accepted or published), and the chapter(s) of the thesis in which the associated work is located.
  \item If the work includes publications or material submitted for publication, the statement described above must detail the relative contributions of all collaborators and co-authors (including supervisors and members of the supervisory committee) and state the proportion of research and writing conducted by the student. For further details, see “Including Published Material in a Thesis or Dissertation”.
  \item If the work includes other scholarly artifacts (such as film and other audio, visual, and graphic representations, and application-oriented documents such as policy briefs, curricula, business plans, computer and web tools, pages, and applications, etc.), all of these must be listed in the Preface (with bibliographical information, if applicable).
  \item If ethics approval was required for the research, the Preface must list the Certificate Number(s) of the Ethics Certificate(s) applicable to the project.
\end{itemize}

The content of the Preface must be verified by the student's supervisor, whose endorsement must appear on the final Thesis/Dissertation Approval form.

%    6. Table of contents (mandatory - list all items in the preliminary pages
%    starting with the abstract, followed by chapter headings and
%    subheadings, bibliographies and appendices)
\tableofcontents
\cleardoublepage	% required by tocloft package

%    7. List of tables (mandatory if thesis has tables)
\listoftables
\cleardoublepage	% required by tocloft package

%    8. List of figures (mandatory if thesis has figures)
\listoffigures
\cleardoublepage	% required by tocloft package

%    9. List of illustrations (mandatory if thesis has illustrations)
%   10. Lists of symbols, abbreviations or other (optional)

%   11. Glossary (optional)
\chapter{Glossary}

This glossary uses the handy acronym package to automatically
maintain the glossary.  It uses the package's \texttt{printonlyused}
option to include only those acronyms explicitly referenced in the
\LaTeX\ source.  To change how the acronyms are rendered, change the
\verb+\acsfont+ definition in \verb+diss.tex+.

% use \acrodef to define an acronym, but no listing
\acrodef{UI}{user interface}
\acrodef{UBC}{University of British Columbia}

% The acronym environment will typeset only those acronyms that were
% *actually used* in the course of the document
\begin{acronym}[ANOVA]
\acro{ANOVA}[ANOVA]{Analysis of Variance\acroextra{, a set of
  statistical techniques to identify sources of variability between groups}}
\acro{API}{application programming interface}
\acro{CTAN}{\acroextra{The }Common \TeX\ Archive Network}
\acro{DOI}{Document Object Identifier\acroextra{ (see
    \url{http://doi.org})}}
\acro{GPS}[GPS]{Graduate and Postdoctoral Studies}
\acro{PDF}{Portable Document Format}
\acro{RCS}[RCS]{Revision control system\acroextra{, a software
    tool for tracking changes to a set of files}}
\acro{TLX}[TLX]{Task Load Index\acroextra{, an instrument for gauging
  the subjective mental workload experienced by a human in performing
  a task}}
\acro{UML}{Unified Modelling Language\acroextra{, a visual language
    for modelling the structure of software artefacts}}
\acro{URL}{Unique Resource Locator\acroextra{, used to describe a
    means for obtaining some resource on the world wide web}}
\acro{W3C}[W3C]{\acroextra{the }World Wide Web Consortium\acroextra{,
    the standards body for web technologies}}
\acro{XML}{Extensible Markup Language}
\end{acronym}

% You can also use \newacro{}{} to only define acronyms
% but without explictly creating a glossary
% 
% \newacro{ANOVA}[ANOVA]{Analysis of Variance\acroextra{, a set of
%   statistical techniques to identify sources of variability between groups.}}
% \newacro{API}[API]{application programming interface}
% \newacro{GOMS}[GOMS]{Goals, Operators, Methods, and Selection\acroextra{,
%   a framework for usability analysis.}}
% \newacro{TLX}[TLX]{Task Load Index\acroextra{, an instrument for gauging
%   the subjective mental workload experienced by a human in performing
%   a task.}}
% \newacro{UI}[UI]{user interface}
% \newacro{UML}[UML]{Unified Modelling Language}
% \newacro{W3C}[W3C]{World Wide Web Consortium}
% \newacro{XML}[XML]{Extensible Markup Language}
	% always input, since other macros may rely on it

\textspacing		% begin one-half or double spacing

%   12. Acknowledgements (optional)
\chapter{Acknowledgements}

I'd like to thank the following people for their support over the past two years:

\begin{itemize}
  \item William Bowman, for your encouragement and mentorship throughout my Bachelor's and Master's theses
    and in navigating academia.
    Your influence on me is likely greater than I currently realize,
    and I wouldn't have it any other way.
    I'll be upholding future advisors to your standards and the next one's got big shoes to fill.
  \item Alex Summers, for all the helpful feedback you provided as the very second reader of this thesis,
    and for introducing to me a bunch of really fun board games.
  \item Jordy Dickinson, for exposing me to all sorts of niches of type theory and constructive mathematics,
    and for giving an ear to the most out-of-context problems I was working on.
    I hope you're doing well.
  \item Hazel Levine, for making \href{https://types.pl/}{types.pl} a reality (and letting me in on the fun).
  \item The folks on PL Twitter, the TYPES, Coq, Agda, Idris, and Cedille mailing lists and Zulips,
    and the Proof Assistants Stack Exchange for answering my many questions
    and for asking far more interesting ones.
  \item Finally and most importantly, all my other splabmates at the Software Practices Lab
    for making grad school a way better experience than I could have hoped for,
    especially during an ongoing global pandemic.
    To name a few (by increasing order of name length):
    \vspace{-0.25\baselineskip}
    \begin{quote} \small
    James Yoo\textsuperscript{\href{https://youtu.be/dQw4w9WgXcQ}{0}}, Chris Chen\punctstack{,}\footnote{honourary splabmate}
    Ron Garcia, Lily Bryant, Braxton Hall, Joey Eremondi, Markus de Medeiros, Paulette Koronkevich, Felipe Ba\~nados Schwerter,
    \end{quote}
    \vspace{-0.5\baselineskip}
    and many more, especially those who brought treats for the lab.
\end{itemize}

\vfill

\noindent {\small This research was supported by the Canada Graduate Scholarships -- Master’s (CGS M) programme.
Cette recherche a \'et\'e financ\'ee par le Programme de bourses d'\'etudes sup\'erieures
du Canada au niveau de la maitrise (BESC M).}

%   13. Dedication (optional)
\chapter{Dedication}

\begin{center}
\textit{Dedicated to the SPL\punctstack{.}}%
\footnote{Special thanks to James Yoo, who took most of these photos.}

\hfill

% TODO: Ask everyone for permission to use
\includegraphics[width=\textwidth]{SPL.jpg}
\end{center}

% Body of Thesis (not all sections may apply)
\mainmatter

\acresetall	% reset all acronyms used so far

%    1. Introduction
\chapter{Introduction}
\label{ch:Introduction}

\begin{epigraph}
    \emph{If I have seen farther it is by standing on the shoulders of
    Giants.} ---~Sir Isaac Newton (1855)
\end{epigraph}

This document provides a quick set of instructions for using the
\class{ubcdiss} class to write a dissertation in \LaTeX. 
Unfortunately this document cannot provide an introduction to using
\LaTeX.  The classic reference for learning \LaTeX\ is
\citeauthor{lamport-1994-ladps}'s
book~\cite{lamport-1994-ladps}.  There are also many freely-available
tutorials online;
\webref{http://www.andy-roberts.net/misc/latex/}{Andy Roberts' online
    \LaTeX\ tutorials}
seems to be excellent.
The source code for this docment, however, is intended to serve as
an example for creating a \LaTeX\ version of your dissertation.

We start by discussing organizational issues, such as splitting
your dissertation into multiple files, in
\autoref{sec:SuggestedThesisOrganization}.
We then cover the ease of managing cross-references in \LaTeX\ in
\autoref{sec:CrossReferences}.
We cover managing and using bibliographies with \BibTeX\ in
\autoref{sec:BibTeX}. 
We briefly describe typesetting attractive tables in
\autoref{sec:TypesettingTables}.
We briefly describe including external figures in
\autoref{sec:Graphics}, and using special characters and symbols
in \autoref{sec:SpecialSymbols}.
As it is often useful to track different versions of your dissertation,
we discuss revision control further in
\autoref{sec:DissertationRevisionControl}. 
We conclude with pointers to additional sources of information in
\autoref{sec:Conclusions}.

%%%%%%%%%%%%%%%%%%%%%%%%%%%%%%%%%%%%%%%%%%%%%%%%%%%%%%%%%%%%%%%%%%%%%%
\section{Suggested Thesis Organization}
\label{sec:SuggestedThesisOrganization}

The \acs{UBC} \acf{GPS} specifies a particular arrangement of the
components forming a thesis.\footnote{See
    \url{http://www.grad.ubc.ca/current-students/dissertation-thesis-preparation/order-components}}
This template reflects that arrangement.

In terms of writing your thesis, the recommended best practice for
organizing large documents in \LaTeX\ is to place each chapter in
a separate file.  These chapters are then included from the main
file through the use of \verb+\include{file}+.  A thesis might
be described as six files such as \file{intro.tex},
\file{relwork.tex}, \file{model.tex}, \file{eval.tex},
\file{discuss.tex}, and \file{concl.tex}.

We also encourage you to use macros for separating how something
will be typeset (\eg bold, or italics) from the meaning of that
something. 
For example, if you look at \file{intro.tex}, you will see repeated
uses of a macro \verb+\file{}+ to indicate file names.
The \verb+\file{}+ macro is defined in the file \file{macros.tex}.
The consistent use of \verb+\file{}+ throughout the text not only
indicates that the argument to the macro represents a file (providing
meaning or semantics), but also allows easily changing how
file names are typeset simply by changing the definition of the
\verb+\file{}+ macro.
\file{macros.tex} contains other useful macros for properly typesetting
things like the proper uses of the latinate \emph{exempli grati\={a}}
and \emph{id est} (\ie \verb+\eg+ and \verb+\ie+), 
web references with a footnoted \acs{URL} (\verb+\webref{url}{text}+),
as well as definitions specific to this documentation
(\verb+\latexpackage{}+).

%%%%%%%%%%%%%%%%%%%%%%%%%%%%%%%%%%%%%%%%%%%%%%%%%%%%%%%%%%%%%%%%%%%%%%
\section{Making Cross-References}
\label{sec:CrossReferences}

\LaTeX\ make managing cross-references easy, and the \latexpackage{hyperref}
package's\ \verb+\autoref{}+ command\footnote{%
    The \latexpackage{hyperref} package is included by default in this
    template.}
makes it easier still. 

A thing to be cross-referenced, such as a section, figure, or equation,
is \emph{labelled} using a unique, user-provided identifier, defined
using the \verb+\label{}+ command.  
The thing is referenced elsewhere using the \verb+\autoref{}+ command.
For example, this section was defined using:
\begin{lstlisting}
    \section{Making Cross-References}
    \label{sec:CrossReferences}
\end{lstlisting}
References to this section are made as follows:
\begin{lstlisting}
    We then cover the ease of managing cross-references in \LaTeX\
    in \autoref{sec:CrossReferences}.
\end{lstlisting}
\verb+\autoref{}+ takes care of determining the \emph{type} of the 
thing being referenced, so the example above is rendered as
\begin{quote}
    We then cover the ease of managing cross-references in \LaTeX\
    in \autoref{sec:CrossReferences}.
\end{quote}

The label is any simple sequence of characters, numbers, digits,
and some punctuation marks such as ``:'' and ``--''; there should
be no spaces.  Try to use a consistent key format: this simplifies
remembering how to make references.  This document uses a prefix
to indicate the type of the thing being referenced, such as \texttt{sec}
for sections, \texttt{fig} for figures, \texttt{tbl} for tables,
and \texttt{eqn} for equations.

For details on defining the text used to describe the type
of \emph{thing}, search \file{diss.tex} and the \latexpackage{hyperref}
documentation for \texttt{autorefname}.


%%%%%%%%%%%%%%%%%%%%%%%%%%%%%%%%%%%%%%%%%%%%%%%%%%%%%%%%%%%%%%%%%%%%%%
\section{Managing Bibliographies with \BibTeX}
\label{sec:BibTeX}

One of the primary benefits of using \LaTeX\ is its companion program,
\BibTeX, for managing bibliographies and citations.  Managing
bibliographies has three parts: (i) describing references,
(ii)~citing references, and (iii)~formatting cited references.

\subsection{Describing References}

\BibTeX\ defines a standard format for recording details about a
reference.  These references are recorded in a file with a
\file{.bib} extension.  \BibTeX\ supports a broad range of
references, such as books, articles, items in a conference proceedings,
chapters, technical reports, manuals, dissertations, and unpublished
manuscripts. 
A reference may include attributes such as the authors,
the title, the page numbers, the \ac{DOI}, or a \ac{URL}.  A reference
can also be augmented with personal attributes, such as a rating,
notes, or keywords.

Each reference must be described by a unique \emph{key}.\footnote{%
    Note that the citation keys are different from the reference
    identifiers as described in \autoref{sec:CrossReferences}.}
A key is a simple sequence of characters, numbers, digits, and some
punctuation marks such as ``:'' and ``--''; there should be no spaces. 
A consistent key format simiplifies remembering how to make references. 
For example:
\begin{quote}
   \fbox{\emph{last-name}}\texttt{-}\fbox{\emph{year}}\texttt{-}\fbox{\emph{contracted-title}}
\end{quote}
where \emph{last-name} represents the last name for the first author,
and \emph{contracted-title} is some meaningful contraction of the
title.  Then \citeauthor{kiczales-1997-aop}'s seminal article on
aspect-oriented programming~\cite{kiczales-1997-aop} (published in
\citeyear{kiczales-1997-aop}) might be given the key
\texttt{kiczales-1997-aop}.

An example of a \BibTeX\ \file{.bib} file is included as
\file{biblio.bib}.  A description of the format a \file{.bib}
file is beyond the scope of this document.  We instead encourage
you to use one of the several reference managers that support the
\BibTeX\ format such as
\webref{http://jabref.sourceforge.net}{JabRef} (multiple platforms) or
\webref{http://bibdesk.sourceforge.net}{BibDesk} (MacOS\,X only). 
These front ends are similar to reference managers such as
EndNote or RefWorks.


\subsection{Citing References}

Having described some references, we then need to cite them.  We
do this using a form of the \verb+\cite+ command.  For example:
\begin{lstlisting}
    \citet{kiczales-1997-aop} present examples of crosscutting 
    from programs written in several languages.
\end{lstlisting}
When processed, the \verb+\citet+ will cause the paper's authors
and a standardized reference to the paper to be inserted in the
document, and will also include a formatted citation for the paper
in the bibliography.  For example:
\begin{quote}
    \citet{kiczales-1997-aop} present examples of crosscutting 
    from programs written in several languages.
\end{quote}
There are several forms of the \verb+\cite+ command (provided
by the \latexpackage{natbib} package), as demonstrated in
\autoref{tbl:natbib:cite}.
Note that the form of the citation (numeric or author-year) depends
on the bibliography style (described in the next section).
The \verb+\citet+ variant is used when the author names form
an object in the sentence, whereas the \verb+\citep+ variant
is used for parenthetic references, more like an end-note.
Use \verb+\nocite+ to include a citation in the bibliography
but without an actual reference.
\nocite{rowling-1997-hpps}
\begin{table}
\caption{Available \texttt{cite} variants; the exact citation style
    depends on whether the bibliography style is numeric or author-year.}
\label{tbl:natbib:cite}
\centering
\begin{tabular}{lp{3.25in}}\toprule
Variant & Result \\
\midrule
% We cheat here to simulate the cite/citep/citet for APA-like styles
\verb+\cite+ & Parenthetical citation (\eg ``\cite{kiczales-1997-aop}''
    or ``(\citeauthor{kiczales-1997-aop} \citeyear{kiczales-1997-aop})'') \\
\verb+\citet+ & Textual citation: includes author (\eg
    ``\citet{kiczales-1997-aop}'' or
    or ``\citeauthor{kiczales-1997-aop} (\citeyear{kiczales-1997-aop})'') \\
\verb+\citet*+ & Textual citation with unabbreviated author list \\
\verb+\citealt+ & Like \verb+\citet+ but without parentheses \\
\verb+\citep+ & Parenthetical citation (\eg ``\cite{kiczales-1997-aop}''
    or ``(\citeauthor{kiczales-1997-aop} \citeyear{kiczales-1997-aop})'') \\
\verb+\citep*+ & Parenthetical citation with unabbreviated author list \\
\verb+\citealp+ & Like \verb+\citep+ but without parentheses \\
\verb+\citeauthor+ & Author only (\eg ``\citeauthor{kiczales-1997-aop}'') \\
\verb+\citeauthor*+ & Unabbreviated authors list 
    (\eg ``\citeauthor*{kiczales-1997-aop}'') \\
\verb+\citeyear+ & Year of citation (\eg ``\citeyear{kiczales-1997-aop}'') \\
\bottomrule
\end{tabular}
\end{table}

\subsection{Formatting Cited References}

\BibTeX\ separates the citing of a reference from how the cited
reference is formatted for a bibliography, specified with the
\verb+\bibliographystyle+ command. 
There are many varieties, such as \texttt{plainnat}, \texttt{abbrvnat},
\texttt{unsrtnat}, and \texttt{vancouver}.
This document was formatted with \texttt{abbrvnat}.
Look through your \TeX\ distribution for \file{.bst} files. 
Note that use of some \file{.bst} files do not emit all the information
necessary to properly use \verb+\citet{}+, \verb+\citep{}+,
\verb+\citeyear{}+, and \verb+\citeauthor{}+.

There are also packages available to place citations on a per-chapter
basis (\latexpackage{bibunits}), as footnotes (\latexpackage{footbib}),
and inline (\latexpackage{bibentry}).
Those who wish to exert maximum control over their bibliography
style should see the amazing \latexpackage{custom-bib} package.

%%%%%%%%%%%%%%%%%%%%%%%%%%%%%%%%%%%%%%%%%%%%%%%%%%%%%%%%%%%%%%%%%%%%%%
\section{Typesetting Tables}
\label{sec:TypesettingTables}

\citet{lamport-1994-ladps} made one grievous mistake
in \LaTeX: his suggested manner for typesetting tables produces
typographic abominations.  These suggestions have unfortunately
been replicated in most \LaTeX\ tutorials.  These
abominations are easily avoided simply by ignoring his examples
illustrating the use of horizontal and vertical rules (specifically
the use of \verb+\hline+ and \verb+|+) and using the
\latexpackage{booktabs} package instead.

The \latexpackage{booktabs} package helps produce tables in the form
used by most professionally-edited journals through the use of
three new types of dividing lines, or \emph{rules}.
% There are times that you don't want to use \autoref{}
Tables~\ref{tbl:natbib:cite} and~\ref{tbl:LaTeX:Symbols} are two
examples of tables typeset with the \latexpackage{booktabs} package.
The \latexpackage{booktabs} package provides three new commands
for producing rules:
\verb+\toprule+ for the rule to appear at the top of the table,
\verb+\midrule+ for the middle rule following the table header,
and \verb+\bottomrule+ for the bottom-most at the end of the table.
These rules differ by their weight (thickness) and the spacing before
and after.
A table is typeset in the following manner:
\begin{lstlisting}
    \begin{table}
    \caption{The caption for the table}
    \label{tbl:label}
    \centering
    \begin{tabular}{cc}
    \toprule
    Header & Elements \\
    \midrule
    Row 1 & Row 1 \\
    Row 2 & Row 2 \\
    % ... and on and on ...
    Row N & Row N \\
    \bottomrule
    \end{tabular}
    \end{table}
\end{lstlisting}
See the \latexpackage{booktabs} documentation for advice in dealing with
special cases, such as subheading rules, introducing extra space
for divisions, and interior rules.

%%%%%%%%%%%%%%%%%%%%%%%%%%%%%%%%%%%%%%%%%%%%%%%%%%%%%%%%%%%%%%%%%%%%%%
\section{Figures, Graphics, and Special Characters}
\label{sec:Graphics}

Most \LaTeX\ beginners find figures to be one of the more challenging
topics.  In \LaTeX, a figure is a \emph{floating element}, to be
placed where it best fits.
The user is not expected to concern him/herself with the placement
of the figure.  The figure should instead be labelled, and where
the figure is used, the text should use \verb+\autoref+ to reference
the figure's label.
\autoref{fig:latex-affirmation} is an example of a figure.
\begin{figure}
    \centering
    % For the sake of this example, we'll just use text
    %\includegraphics[width=3in]{file}
    \Huge{\textsf{\LaTeX\ Rocks!}}
    \caption{Proof of \LaTeX's amazing abilities}
    \label{fig:latex-affirmation}   % label should change
\end{figure}
A figure is generally included as follows:
\begin{lstlisting}
    \begin{figure}
    \centering
    \includegraphics[width=3in]{file}
    \caption{A useful caption}
    \label{fig:fig-label}   % label should change
    \end{figure}
\end{lstlisting}
There are three items of note:
\begin{enumerate}
\item External files are included using the \verb+\includegraphics+
    command.  This command is defined by the \latexpackage{graphicx} package
    and can often natively import graphics from a variety of formats.
    The set of formats supported depends on your \TeX\ command processor.
    Both \texttt{pdflatex} and \texttt{xelatex}, for example, can
    import \textsc{gif}, \textsc{jpg}, and \textsc{pdf}.  The plain
    version of \texttt{latex} only supports \textsc{eps} files.

\item The \verb+\caption+ provides a caption to the figure. 
    This caption is normally listed in the List of Figures; you
    can provide an alternative caption for the LoF by providing
    an optional argument to the \verb+\caption+ like so:
    \begin{lstlisting}
    \caption[nice shortened caption for LoF]{%
	longer detailed caption used for the figure}
    \end{lstlisting}
    \ac{GPS} generally prefers shortened single-line captions
    in the LoF: multiple-line captions are a bit unwieldy.

\item The \verb+\label+ command provides for associating a unique, user-defined,
    and descriptive identifier to the figure.  The figure can be
    can be referenced elsewhere in the text with this identifier
    as described in \autoref{sec:CrossReferences}.
\end{enumerate}
See Keith Reckdahl’s excellent guide for more details,
\webref{http://www.ctan.org/tex-archive/info/epslatex.pdf}{\emph{Using
imported graphics in LaTeX2e}}.

\section{Special Characters and Symbols}
\label{sec:SpecialSymbols}

\LaTeX\ appropriates many common symbols for its own purposes,
with some used for commands (\eg \verb+\{}&%+) and
mathematics (\eg \verb+$^_+), and others are automagically transformed
into typographically-preferred forms (\eg \verb+-`'+) or to
completely different forms (\eg \verb+<>+).
\autoref{tbl:LaTeX:Symbols} presents a list of common symbols and
their corresponding \LaTeX\ commands.  A much more comprehensive list 
of symbols and accented characters is available at:
\url{http://www.ctan.org/tex-archive/info/symbols/comprehensive/}
\begin{table}
\caption{Useful \LaTeX\ symbols}\label{tbl:LaTeX:Symbols}
\centering\begin{tabular}{ccp{0.5cm}cc}\toprule
\LaTeX & Result && \LaTeX & Result \\
\midrule
    \verb+\texttrademark+ & \texttrademark && \verb+\&+ & \& \\
    \verb+\textcopyright+ & \textcopyright && \verb+\{ \}+ & \{ \} \\
    \verb+\textregistered+ & \textregistered && \verb+\%+ & \% \\
    \verb+\textsection+ & \textsection && \verb+\verb!~!+ & \verb!~! \\
    \verb+\textdagger+ & \textdagger && \verb+\$+ & \$ \\
    \verb+\textdaggerdbl+ & \textdaggerdbl && \verb+\^{}+ & \^{} \\
    \verb+\textless+ & \textless && \verb+\_+ & \_ \\
    \verb+\textgreater+ & \textgreater && \\
\bottomrule
\end{tabular}
\end{table}

%%%%%%%%%%%%%%%%%%%%%%%%%%%%%%%%%%%%%%%%%%%%%%%%%%%%%%%%%%%%%%%%%%%%%%
\section{Changing Page Widths and Heights}

The \class{ubcdiss} class is based on the standard \LaTeX\ \class{book}
class~\cite{lamport-1994-ladps} that selects a line-width to carry
approximately 66~characters per line.  This character density is
claimed to have a pleasing appearance and also supports more rapid
reading~\cite{bringhurst-2002-teots}.  I would recommend that you
not change the line-widths!

\subsection{The \texttt{geometry} Package}

Some students are unfortunately saddled with misguided supervisors
or committee members whom believe that documents should have the
narrowest margins possible.  The \latexpackage{geometry} package is
helpful in such cases.  Using this package is as simple as:
\begin{lstlisting}
    \usepackage[margin=1.25in,top=1.25in,bottom=1.25in]{geometry}
\end{lstlisting}
You should check the package's documentation for more complex uses.

\subsection{Changing Page Layout Values By Hand}

There are some miserable students with requirements for page layouts
that vary throughout the document.  Unfortunately the
\latexpackage{geometry} can only be specified once, in the document's
preamble.  Such miserable students must set \LaTeX's layout parameters
by hand:
\begin{lstlisting}
    \setlength{\topmargin}{-.75in}
    \setlength{\headsep}{0.25in}
    \setlength{\headheight}{15pt}
    \setlength{\textheight}{9in}
    \setlength{\footskip}{0.25in}
    \setlength{\footheight}{15pt}

    % The *sidemargin values are relative to 1in; so the following
    % results in a 0.75 inch margin
    \setlength{\oddsidemargin}{-0.25in}
    \setlength{\evensidemargin}{-0.25in}
    \setlength{\textwidth}{7in}       % 1.1in margins (8.5-2*0.75)
\end{lstlisting}
These settings necessarily require assuming a particular page height
and width; in the above, the setting for \verb+\textwidth+ assumes
a \textsc{US} Letter with an 8.5'' width.
The \latexpackage{geometry} package simply uses the page height and
other specified values to derive the other layout values.
The
\href{http://tug.ctan.org/tex-archive/macros/latex/required/tools/layout.pdf}{\texttt{layout}}
package provides a
handy \verb+\layout+ command to show the current page layout
parameters. 


\subsection{Making Temporary Changes to Page Layout}

There are occasions where it becomes necessary to make temporary
changes to the page width, such as to accomodate a larger formula. 
The \latexmiscpackage{chngpage} package provides an \env{adjustwidth}
environment that does just this.  For example:
\begin{lstlisting}
    % Expand left and right margins by 0.75in
    \begin{adjustwidth}{-0.75in}{-0.75in}
    % Must adjust the perceived column width for LaTeX to get with it.
    \addtolength{\columnwidth}{1.5in}
    \[ an extra long math formula \]
    \end{adjustwidth}
\end{lstlisting}


%%%%%%%%%%%%%%%%%%%%%%%%%%%%%%%%%%%%%%%%%%%%%%%%%%%%%%%%%%%%%%%%%%%%%%
\section{Keeping Track of Versions with Revision Control}
\label{sec:DissertationRevisionControl}

Software engineers have used \acf{RCS} to track changes to their
software systems for decades.  These systems record the changes to
the source code along with context as to why the change was required.
These systems also support examining and reverting to particular
revisions from their system's past.

An \ac{RCS} can be used to keep track of changes to things other
than source code, such as your dissertation.  For example, it can
be useful to know exactly which revision of your dissertation was
sent to a particular committee member.  Or to recover an accidentally
deleted file, or a badly modified image.  With a revision control
system, you can tag or annotate the revision of your dissertation
that was sent to your committee, or when you incorporated changes
from your supervisor.

Unfortunately current revision control packages are not yet targetted
to non-developers.  But the Subversion project's
\webref{http://tortoisesvn.net/docs/release/TortoiseSVN_en/}{TortoiseSVN}
has greatly simplified using the Subversion revision control system
for Windows users.  You should consult your local geek.

A simpler alternative strategy is to create a GoogleMail account
and periodically mail yourself zipped copies of your dissertation.

%%%%%%%%%%%%%%%%%%%%%%%%%%%%%%%%%%%%%%%%%%%%%%%%%%%%%%%%%%%%%%%%%%%%%%
\section{Recommended Packages}

The real strength to \LaTeX\ is found in the myriad of free add-on
packages available for handling special formatting requirements.
In this section we list some helpful packages.

\subsection{Typesetting}

\begin{description}
\item[\latexpackage{enumitem}:]
    Supports pausing and resuming enumerate environments.

\item[\latexpackage{ulem}:]
    Provides two new commands for striking out and crossing out text
    (\verb+\sout{text}+ and \verb+\xout{text}+ respectively)
    The package should likely
    be used as follows:
    \begin{verbatim}
    \usepackage[normalem,normalbf]{ulem}
    \end{verbatim}
    to prevent the package from redefining the emphasis and bold fonts.

\item[\latexpackage{chngpage}:]
    Support changing the page widths on demand.

\item[\latexpackage{mhchem}:] 
    Support for typesetting chemical formulae and reaction equations.

\end{description}

Although not a package, the
\webref{http://www.ctan.org/tex-archive/support/latexdiff/}{\texttt{latexdiff}}
command is very useful for creating changebar'd versions of your
dissertation.


\subsection{Figures, Tables, and Document Extracts}

\begin{description}
\item[\latexpackage{pdfpages}:]
    Insert pages from other PDF files.  Allows referencing the extracted
    pages in the list of figures, adding labels to reference the page
    from elsewhere, and add borders to the pages.

\item[\latexpackage{subfig}:]
    Provides for including subfigures within a figure, and includes
    being able to separately reference the subfigures.  This is a
    replacement for the older \texttt{subfigure} environment.

\item[\latexpackage{rotating}:]
    Provides two environments, sidewaystable and sidewaysfigure,
    for typesetting tables and figures in landscape mode.  

\item[\latexpackage{longtable}:]
    Support for long tables that span multiple pages.

\item[\latexpackage{tabularx}:]
    Provides an enhanced tabular environment with auto-sizing columns.

\item[\latexpackage{ragged2e}:]
    Provides several new commands for setting ragged text (\eg forms
    of centered or flushed text) that can be used in tabular
    environments and that support hyphenation.

\end{description}


\subsection{Bibliography Related Packages}

\begin{description}
\item[\latexpackage{bibunits}:]
    Support having per-chapter bibliographies.

\item[\latexpackage{footbib}:]
    Cause cited works to be rendered using footnotes.

\item[\latexpackage{bibentry}:] 
    Support placing the details of a cited work in-line.

\item[\latexpackage{custom-bib}:]
    Generate a custom style for your bibliography.

\end{description}


%%%%%%%%%%%%%%%%%%%%%%%%%%%%%%%%%%%%%%%%%%%%%%%%%%%%%%%%%%%%%%%%%%%%%%
\section{Moving On}
\label{sec:Conclusions}

At this point, you should be ready to go.  Other handy web resources:
\begin{itemize}
\item \webref{http://www.ctan.org}{\ac{CTAN}} is \emph{the} comprehensive
    archive site for all things related to \TeX\ and \LaTeX. 
    Should you have some particular requirement, somebody else is
    almost certainly to have had the same requirement before you,
    and the solution will be found on \ac{CTAN}.  The links to
    various packages in this document are all to \ac{CTAN}.

\item An online
    \webref{http://www.ctan.org/get/info/latex2e-help-texinfo/latex2e.html}{%
	reference to \LaTeX\ commands} provides a handy quick-reference
    to the standard \LaTeX\ commands.

\item The list of 
    \webref{http://www.tex.ac.uk/cgi-bin/texfaq2html?label=interruptlist}{%
	Frequently Asked Questions about \TeX\ and \LaTeX}
    can save you a huge amount of time in finding solutions to
    common problems.

\item The \webref{http://www.tug.org/tetex/tetex-texmfdist/doc/}{te\TeX\
    documentation guide} features a very handy list of the most useful
    packages for \LaTeX\ as found in \ac{CTAN}.

\item The
\webref{http://www.ctan.org/tex-archive/macros/latex/required/graphics/grfguide.pdf}{\texttt{color}}
    package, part of the graphics bundle, provides handy commands
    for changing text and background colours.  Simply changing
    text to various levels of grey can have a very 
    \textcolor{greytext}{dramatic effect}.


\item If you're really keen, you might want to join the
    \webref{http://www.tug.org}{\TeX\ Users Group}.

\end{itemize}

\endinput

Any text after an \endinput is ignored.
You could put scraps here or things in progress.


%    2. Main body
% Generally recommended to put each chapter into a separate file
%\include{relatedwork}
%\chapter{Syntactic Model of \lang} \label{ch:model}

\newcommand{\FigSyntaxCIC}[1]{
  \begin{figure}[h]
  \centering
  \begin{align*}
  \iT, \jT, \kT, \mT, \nT &\Coloneqq \meta{\textrm{naturals}} &
  \GammaT &\Coloneqq \mt \mid \GammaT, \annotT{\xT}{\tauT} \mid \GammaT, \define{\xT}{\tauT}{\eT} \\
  \fT, \gT, \wT, \xT, \yT, \zT, \alphaT, \betaT, \gammaT &\Coloneqq \meta{\textrm{term variables}} &
  \DeltaT &\Coloneqq \mt \mid \DeltaT, \annotT{\xT}{\tauT} \mid \DeltaT, \annotT{\cT}{\tauT} \\
  \XT &\Coloneqq \meta{\textrm{inductive type names}} &
  \UT &\Coloneqq \PropT \mid \TypeT{\iT} \\
  \cT &\Coloneqq \meta{\textrm{inductive constructor names}} &
  \DT &\Coloneqq \app{\dataT{\app{\XT}{\DeltaT}}{\arrT*{\DeltaT}{\UT}}}{\DeltaT} \\
  \eT, \aT, \dT, \pT, \PT, \rT, \sT, \tauT, \sigmaT &\Coloneqq
    \mathrlap{\xT \mid \UT \mid \funtypeT{\xT}{\tauT}{\tauT} \mid \funT{\xT}{\tauT}{\eT} \mid \app{\eT}{\eT} \mid \letinT{\xT}{\tauT}{\eT}{\eT} \mid \fixT{\iT}{\fT}{\tauT}{\eT}} \\
  &\mid \mathrlap{\eqT{\eT}{\tauT}{\eT} \mid \reflT{\eT} \mid \JT{\PT}{\dT}{\pT} \mid \matchT{\eT}{\funT*{\vec{\yT}}{\xT}{\PT}}{(\app{\cT}{\vec{\zT}} \RightarrowT \eT) \seq}}
  \end{align*}
  \caption{Syntax (\CICE)}
  \label{#1}
  \end{figure}
}

\newcommand{\FigEquiv}[1]{
  \begin{figure}[p]
  \centering
  \begin{mathpar}
  \fbox{$\defeq{\GammaT}{\eT}{\eT}{\tauT}$} \hfill \\
  \inferrule[\rlabel{$\equiv$-refl}{equiv-refl}]{
    \type{\GammaT}{\eT}{\tauT}
  }{
    \defeq{\GammaT}{\eT}{\eT}{\tauT}
  }
  \and
  \inferrule[\rlabel{$\equiv$-sym}{equiv-sym}]{
    \defeq{\GammaT}{\eT_2}{\eT_1}{\tauT}
  }{
    \defeq{\GammaT}{\eT_1}{\eT_2}{\tauT}
  }
  \and
  \inferrule[\rlabel{$\equiv$-trans}{equiv-trans}]{
    \defeq{\GammaT}{\eT_1}{\eT_2}{\tauT} \\\\
    \defeq{\GammaT}{\eT_2}{\eT_3}{\tauT}
  }{
    \defeq{\GammaT}{\eT_1}{\eT_3}{\tauT}
  }
  \and
  \inferrule[\rlabel{$\equiv$-conv}{equiv-conv}]{
    \subtype{\GammaT}{\sigmaT}{\tauT} \\\\
    \defeq{\GammaT}{\eT_1}{\eT_2}{\sigmaT}
  }{
    \defeq{\GammaT}{\eT_1}{\eT_2}{\tauT}
  }
  \and
  \inferrule[\rlabel{$\equiv$-reflect}{equiv-reflect}]{
    \type{\GammaT}{\pT}{\eqT{\eT_1}{\tauT}{\eT_2}}
  }{
    \defeq{\GammaT}{\eT_1}{\eT_2}{\tauT}
  }
  \and
  \inferrule[\rlabel{$\equiv$-cong}{equiv-cong}]{
    \textrm{For every $1 \leq i \leq n$:} \\
    \defeq{\GammaT'}{\eT_i}{\eT'_i}{\tauT'}
  }{
    \defeq{\GammaT}
      {\subst{\eT}{\xT_1, \seq, \xT_n}{\eT_1, \seq, \eT_n}}
      {\subst{\eT}{\xT_1, \seq, \xT_n}{\eT'_1, \seq, \eT'_n}}
      {\tauT}
  }
  \and
  \inferrule[\rlabel{$\equiv$-$\delta$}{equiv-delta}]{
    (\defineT{\xT}{\tauT}{\eT}) \in \GammaT
  }{
    \defeq{\GammaT}{\xT}{\eT}{\tauT}
  }
  \and
  \inferrule[\rlabel{$\equiv$-$\beta$}{equiv-beta}]{
    \type{\GammaT}{\sigmaT}{\UT} \\
    \type{\GammaT, \annotT{\xT}{\sigmaT}}{\eT}{\tauT} \\
    \type{\GammaT}{\eT'}{\sigmaT}
  }{
    \defeq{\GammaT}{\app{(\funT{\xT}{\sigmaT}{\eT})}{\eT'}}{\subst{\eT}{\xT}{\eT'}}{\subst{\tauT}{\xT}{\eT'}}
  }
  \and
  \inferrule[\rlabel{$\equiv$-$\eta$}{equiv-eta}]{
    \defeq{\GammaT, \annotT{\xT}{\sigmaT}}{\app{\eT_1}{\xT}}{\app{\eT_2}{\xT}}{\tauT}
  }{
    \defeq{\GammaT}{\eT_1}{\eT_2}{\funtypeT{\xT}{\sigmaT}{\tauT}}
  }
  \and
  \inferrule[\rlabel{$\equiv$-$\zeta$}{equiv-zeta}]{
    \type{\GammaT}{\sigmaT}{\UT} \\
    \type{\GammaT}{\eT'}{\sigmaT} \\
    \type{\GammaT, \defineT{\xT}{\sigmaT}{\eT'}}{\eT}{\tauT}
  }{
    \defeq{\GammaT}{\letinT{\xT}{\sigmaT}{\eT'}{\eT}}{\subst{\eT}{\xT}{\eT'}}{\subst{\tauT}{\xT}{\eT'}}
  }
  \and
  \inferrule[\rlabel{$\equiv$-$\rho$}{equiv-rho}]{
    \type{\GammaT}{\eT}{\tauT} \\
    \type{\GammaT}{\PT}{\funtypeT{\yT}{\tauT}{\funtypeT{\zT}{\eqT{\eT}{\tauT}{\yT}}{\UT}}} \\
    \type{\GammaT}{\dT}{\app{\PT}{\eT}{\reflT{\eT}}}
  }{
    \defeq{\GammaT}{\JT{\PT}{\dT}{\reflT{\eT}}}{\dT}{\app{\PT}{\eT}{\reflT{\eT}}}
  }
  \iffalse
  \and
  \inferrule[\rlabel{$\equiv$-$\iota$}{equiv-iota}]{
    \app{\dataT{\app{\XT}{(\annotT{\vec{\wT}}{\vec{\vphantom{\wT}\any}})}}{\arrT*{\DeltaT_I}{\UT}}}{\DeltaT_c} \\
    \annotT{\vec{\yT}}{\vec{\sigmaT}} = \subst{\DeltaT_I}{\vec{\wT}}{\vec{\pT}} \\
    (\annotT{\cT}{\arrT*{(\annotT{\vec{\zT}}{\vec{\tauT}})}{\app{\XT}{\vec{\pT}}{\vec{\aT}}}}) \in \subst{\DeltaT_c}{\vec{\wT}}{\vec{\pT}} \\
    \vec{\aT}' = \subst{\vec{\aT}}{\vec{\zT}}{\vec{\eT}} \\
    \type{\GammaT}{\app{\cT}{\vec{\pT}}{\vec{\eT}}}{\app{\XT}{\vec{\pT}}{\vec{\aT}'}} \\
    \type{\GammaT, \annotT{\vec{\yT}}{\vec{\sigmaT}}, \annotT{\xT}{\app{\XT}{\vec{\pT}}{\vec{\yT}}}}{\PT}{\UT} \\
    \type{\GammaT, \annotT{\vec{\zT}}{\vec{\tauT}}}{\eT}{\subst{\PT}{\vec{\yT}, \vec{\xT}}{\vec{\aT}, \app{\cT}{\vec{\pT}}{\vec{\zT}}}}
  }{
    \defeq{\GammaT}{\matchT{\app{\cT}{\vec{\pT}}{\vec{\eT}}}{\funT*{\vec{\yT}}{\xT}{\PT}}{\seq(\app{\cT}{\vec{\zT}} \RightarrowT \eT)\seq}}{\subst{\eT}{\vec{\zT}}{\vec{\eT}}}{\subst{\PT}{\vec{\yT}, \xT}{\vec{\aT}', \app{\cT}{\vec{\pT}}{\vec{\eT}}}}
  }
  \fi
  \and
  \inferrule[\rlabel{$\equiv$-$\iota$}{equiv-iota}]{
    \app{\dataT{\app{\XT}{(\annotT{\vec{\wT}}{\vec{\sigmaT}_P})}}{\arrT*{\DeltaT_I}{\UT}}}{\DeltaT_c} \\
    \type{\GammaT}{\vec{\pT}}{\vec{\sigmaT}_P} \\
    \annotT{\vec{\yT}}{\vec{\sigmaT}_I} = \subst{\DeltaT_I}{\vec{\wT}}{\vec{\pT}} \\
    \type{\GammaT, \annotT{\vec{\yT}}{\vec{\sigmaT}_I}, \annotT{\xT}{\app{\XT}{\vec{\pT}}{\vec{\yT}}}}{\PT}{\UT'} \\
    \elim{\XT}{\UT}{\UT'} \\\\
    \card{\DeltaT_c} = n \\
    \textrm{For every $1 \leq i \leq n$:} \\
    (\annotT{\cT_i}{\arrT*{(\annotT{\vec{\zT}_i}{\vec{\tauT}_i})}{\app{\XT}{\vec{\pT}}{\vec{\aT}_i}}}) \in \subst{\DeltaT_c}{\vec{\wT}}{\vec{\pT}} \\
    \vec{\zT}_i \notin \FV{\PT} \\
    \type{\GammaT, \annotT{\vec{\zT}_i}{\vec{\tauT}_i}}{\eT_i}{\subst{\PT}{\vec{\yT}, \vec{\xT}}{\vec{\aT}_i, \app{\cT_i}{\vec{\pT}}{\vec{\zT}_i}}} \\\\
    j \in 1 \seq n \\
    \type{\GammaT}{\vec{\eT}}{\vec{\tauT}_j} \\
    \vec{\aT}'_j = \subst{\vec{\aT}_j}{\vec{\zT}_j}{\vec{\eT}}
  }{
    \defeq{\GammaT}{\matchT{\app{\cT_j}{\vec{\pT}}{\vec{\eT}}}{\funT*{\vec{\yT}}{\xT}{\PT}}{(\app{\cT_1}{\vec{\zT}_1} \RightarrowT \eT_1)\seq(\app{\cT_n}{\vec{\zT}_n} \RightarrowT \eT_n)}}{\subst{\eT_j}{\vec{\zT}_j}{\vec{\eT}}}{\subst{\PT}{\vec{\yT}, \xT}{\vec{\aT}'_j, \app{\cT_j}{\vec{\pT}}{\vec{\eT}}}}
  }
  \and
  \inferrule[\rlabel{$\equiv$-$\mu$}{equiv-mu}]{
    \type{\GammaT}{\tauT}{\UT} \\
    \defeq{\GammaT}{\tauT}{\arr*{\DeltaT}{\funtypeT{\any}{\app{\XT}{\vec{\pT}}{\vec{\aT}}}{\tauT'}}}{\UT} \\
    \card{\DeltaT} + 1 = \nT \\
    \type{\GammaT, \annotT{\fT}{\tauT}}{\eT}{\tauT}
  }{
    \defeq{\GammaT}{\fixT{\nT}{\fT}{\tauT}{\eT}}{\subst{\eT}{\fT}{\fixT{\nT}{\fT}{\tauT}{\eT}}}{\tauT}
  }
  \end{mathpar}
  \caption{Equivalence rules}
  \label{#1}
  \end{figure}
}

\newcommand{\FigTypingCIC}[1]{
  \begin{figure}[p]
  \centering
  \begin{mathpar}
  \fbox{$\wf{}{\GammaT}$} \qquad
  \fbox{$\type{\GammaT}{\eT}{\tauT}$} \hfill \\
  \inferrule[\rlabel*{nil*}]{~}{\wf{}{\mt}}
  \and
  \inferrule[\rlabel*{cons*-ass}]{
    \wf{}{\GammaT} \\\\
    \type{\GammaT}{\tauT}{\UT}
  }{
    \wf{}{\GammaT, \annotT{\xT}{\tauT}}
  }
  \and
  \inferrule[\rlabel*{cons*-def}]{
    \wf{}{\GammaT} \\\\
    \type{\GammaT}{\eT}{\tauT}
  }{
    \wf{}{\GammaT, \defineT{\xT}{\tauT}{\eT}}
  }
  \and
  \inferrule[\rlabel*{conv*}]{
    \type{\GammaT}{\eT}{\sigmaT} \\
    \type{\GammaT}{\sigmaT}{\UT} \\\\
    \type{\GammaT}{\tauT}{\UT} \\
    \subtype{\GammaT}{\sigmaT}{\tauT}
  }{
    \type{\GammaT}{\eT}{\tauT}
  } 
  \and
  \inferrule[\rlabel*{var*}]{
    \wf{}{\GammaT} \\
    (\annotT{\xT}{\tauT}) \in \GammaT \\\\
    \textit{or } (\defineT{\xT}{\tauT}{\eT}) \in \Gamma
  }{
    \type{\GammaT}{\xT}{\tauT}
  }
  \and
  \inferrule[\rlabel*{univ*}]{
    \wf{}{\GammaT}
  }{
    \type{\GammaT}{\UT}{\axioms{\UT}}
  }
  \and
  \inferrule[\rlabel*{pi*}]{
    \type{\GammaT}{\sigmaT}{\UT_1} \\
    \type{\GammaT, \annotT{\xT}{\sigmaT}}{\tauT}{\UT_2}
  }{
    \type{\GammaT}{\funtypeT{\xT}{\sigmaT}{\tauT}}{\rules{\UT_1}{\UT_2}}
  }
  \and
  \inferrule[\rlabel*{lam*}]{
    \type{\GammaT}{\sigmaT}{\UT} \\
    \type{\GammaT, \annotT{\xT}{\sigmaT}}{\eT}{\tauT}
  }{
    \type{\GammaT}{\funT{\xT}{\sigmaT}{\eT}}{\funtypeT{\xT}{\sigmaT}{\tauT}}
  }
  \and
  \inferrule[\rlabel*{app*}]{
    \type{\GammaT}{\eT_1}{\funtypeT{\xT}{\sigmaT}{\tauT}} \\
    \type{\GammaT}{\eT_2}{\sigmaT}
  }{
    \type{\GammaT}{\app{\eT_1}{\eT_2}}{\subst{\tauT}{\xT}{\eT_1}}
  }
  \and
  \inferrule[\rlabel*{let*}]{
    \type{\GammaT}{\sigmaT}{\UT} \\
    \type{\GammaT}{\eT_1}{\sigmaT} \\
    \type{\GammaT, \defineT{\xT}{\sigmaT}{\eT_1}}{\eT_2}{\tauT}
  }{
    \type{\GammaT}{\letinT{\xT}{\sigmaT}{\eT_1}{\eT_2}}{\subst{\tauT}{\xT}{\eT_1}}
  }
  \and
  \inferrule[\rlabel*{eq}]{
    \type{\GammaT}{\tauT}{\UT} \\
    \type{\GammaT}{\eT_1}{\tauT} \\
    \type{\GammaT}{\eT_2}{\tauT}
  }{
    \type{\GammaT}{\eqT{\eT_1}{\tauT}{\eT_2}}{\UT}
  }
  \and
  \inferrule[\rlabel*{refl}]{
    \type{\GammaT}{\eT}{\tauT}
  }{
    \type{\GammaT}{\reflT{\eT}}{\eqT{\eT}{\tauT}{\eT}}
  }
  \and
  \inferrule[\rlabel*{J}]{
    \type{\GammaT}{\pT}{\eqT{\eT_1}{\tauT}{\eT_2}} \\
    \type{\GammaT}{\PT}{\funtypeT{\yT}{\tauT}{\funtypeT{\zT}{\eqT{\eT_1}{\tauT}{\yT}}{\UT}}} \\
    \type{\GammaT}{\dT}{\app{\PT}{\eT_1}{\reflT{\eT_1}}}
  }{
    \type{\GammaT}{\JT{\PT}{\dT}{\pT}}{\app{\PT}{\eT_2}{\pT}}
  }
  \and
  \inferrule[\rlabel*{ind}]{
    \wf{}{\GammaT} \\
    \app{\dataT{\app{\XT}{\DeltaT_P}}{\tauT}}{\any}
  }{
    \type{\GammaT}{\XT}{\arrT*{\DeltaT_P}{\tauT}}
  }
  \and
  \inferrule[\rlabel*{constr}]{
    \wf{}{\GammaT} \\
    \app{\dataT{\app{\XT}{\DeltaT_P}}{\any}}{\DeltaT_c} \\
    (\annotT{\cT}{\tauT}) \in \DeltaT_c
  }{
    \type{\GammaT}{\cT}{\arrT*{\DeltaT_P}{\tauT}}
  }
  \and
  \inferrule[\rlabel*{case}]{
    \app{\dataT{\app{\XT}{(\annotT{\vec{\wT}}{\vec{\vphantom{\wT}\any}})}}{\arrT*{\DeltaT_I}{\UT}}}{\DeltaT_c} \\
    \annotT{\vec{\yT}}{\vec{\sigmaT}} = \subst{\DeltaT_I}{\vec{\wT}}{\vec{\pT}} \\
    \type{\GammaT}{\eT}{\app{\XT}{\vec{\pT}}{\vec{\aT}}} \\
    \type{\GammaT, \annotT{\vec{\yT}}{\vec{\sigmaT}}, \annotT{\xT}{\app{\XT}{\vec{\pT}}{\vec{\yT}}}}{\PT}{\UT'} \\
    \elim{\XT}{\UT}{\UT'} \\\\
    \card{\DeltaT_c} = n \\
    \textrm{For every $1 \leq i \leq n$:} \\
    (\annotT{\cT_i}{\arrT*{(\annotT{\vec{\zT}_i}{\vec{\tauT}_i})}{\app{\XT}{\vec{\pT}}{\vec{\aT}_i}}}) \in \subst{\DeltaT_c}{\vec{\wT}}{\vec{\pT}} \\
    \vec{\zT}_i \notin \FV{\PT} \\
    \type{\GammaT, \annotT{\vec{\zT}_i}{\vec{\tauT}_i}}{\eT_i}{\subst{\PT}{\vec{\yT}, \vec{\xT}}{\vec{\aT}_i, \app{\cT_i}{\vec{\pT}}{\vec{\zT}_i}}}
  }{
    \type{\GammaT}{\matchT{\eT}{\funT*{\vec{\yT}}{\xT}{\PT}}{(\app{\cT_1}{\vec{\zT}_1} \RightarrowT \eT_1)\seq(\app{\cT_n}{\vec{\zT}_n} \RightarrowT \eT_n)}}{\subst{\PT}{\vec{\yT}, \xT}{\vec{\aT}, \eT}}
  }
  \and
  \inferrule[\rlabel*{fix*}]{
    \type{\GammaT}{\tauT}{\UT} \\
    \defeq{\GammaT}{\tauT}{\arr*{\DeltaT}{\funtypeT{\any}{\app{\XT}{\vec{\pT}}{\vec{\aT}}}{\tauT'}}}{\UT} \\
    \card{\DeltaT} + 1 = \nT \\
    \type{\GammaT, \annotT{\fT}{\tauT}}{\eT}{\tauT}
  }{
    \type{\GammaT}{\fixT{\nT}{\fT}{\tauT}{\eT}}{\tauT}
  }
  \end{mathpar}
  \caption{Typing and well-formedness rules (\CICE)}
  \label{#1}
  \end{figure}
}

\newcommand{\FigSubtypingCIC}[1]{
  \begin{figure}[h]
  \centering
  \begin{mathpar}
  \fbox{$\subtype{\GammaT}{\tauT}{\tauT}$} \hfill \\
  \inferrule[\rlabel{$\preccurlyeq$-conv}{subtype-conv}]{
    \defeq{\GammaT}{\tauT_1}{\tauT_2}{\UT}
  }{
    \subtype{\GammaT}{\tauT_1}{\tauT_2}
  }
  \and
  \inferrule[\rlabel{$\preccurlyeq$-trans}{subtype-trans}]{
    \subtype{\GammaT}{\tauT_1}{\tauT_2} \\
    \subtype{\GammaT}{\tauT_2}{\tauT_3}
  }{
    \subtype{\GammaT}{\tauT_1}{\tauT_3}
  }
  \and
  \inferrule[\rlabel{$\preccurlyeq$-prop}{subtype-prop}]{~}{
    \subtype{\GammaT}{\PropT}{\TypeT{\iT}}
  }
  \and
  \inferrule[\rlabel{$\preccurlyeq$-type}{subtype-type}]{
    \iT \leq \jT
  }{
    \subtype{\GammaT}{\TypeT{\iT}}{\TypeT{\jT}}
  }
  \and
  \inferrule[\rlabel{$\preccurlyeq$-pi}{subtype-pi}]{
    \defeq{\GammaT}{\sigmaT_1}{\sigmaT_2}{\UT} \\
    \subtype{\GammaT, \annotT{\xT}{\sigmaT_2}}{\tauT_1}{\tauT_2}
  }{
    \subtype{\GammaT}{\funtypeT{\xT}{\sigmaT_1}{\tauT_1}}{\funtypeT{\xT}{\sigmaT_2}{\tauT_2}}
  }
  \end{mathpar}
  \caption{Subtyping rules (\CICE)}
  \label{#1}
  \end{figure}
}
\newcommand{\FigData}[1]{
  \begin{figure}[h]
  \newlength{\xspacing}
  \setlength{\xspacing}{1.5ex}
  \centering
  \begin{align*}
  &\dataT{\botT}{\PropT} \\[\xspacing]
  &\dataT{\SizeT}{\TypeT{i+1}} \\
  &\quad \annotT{\sucT}{\arrT*{\SizeT}{\SizeT}} \\
  &\quad \annotT{\limT}{\funtypeT{A}{\TypeT{i}}{\arrT*{(\arrT*{A}{\SizeT})}{\SizeT}}} \\
  &\LetT{\baseT}{\SizeT}{\app{\limT}{\botT}{(\funT{\mathit{false}}{\botT}{\matchT{\mathit{false}}{\funT*{\mt}{\any}{\SizeT}}{\mt}})}} \\[\xspacing]
  &\dataT{\any \szleT \any}{\arrT*{\SizeT}{\SizeT}{\TypeT{i+1}}} \\
  &\quad \annotT{\monoT}{\funtypeT{\alpha, \beta}{\SizeT}{\arrT*{\alpha \szleT \beta}{\app{\sucT}{\alpha} \szleT \app{\sucT}{\beta}}}} \\
  &\quad \annotT{\coconeT}{\funtypeT{A}{\TypeT{i}}{\funtypeT{\beta}{\SizeT}{\funtypeT{f}{\arrT*{A}{\SizeT}}{\funtypeT{a}{A}{\arr*{\beta \szleT \app{f}{a}}{\beta \szleT \app{\limT}{A}{f}}}}}}} \\
  &\quad \annotT{\limitT}{\funtypeT{A}{\TypeT{i}}{\funtypeT{\beta}{\SizeT}{\funtypeT{f}{\arrT*{A}{\SizeT}}{\arr*{(\funtypeT{a}{A}{\app{f}{a} \szleT \beta})}{\app{\limT}{A}{f} \szleT \beta}}}}} \\
  &\LetT{\any \szltT \any}{\arrT*{\SizeT}{\SizeT}{\TypeT{i+1}}}{\funT{\alpha, \beta}{\SizeT}{\app{\sucT}{\alpha} \szltT \beta}} \\[\xspacing]
  &\dataT{\app{\AccT}{(\annot{\alpha}{\SizeT})}}{\PropT} \\
  &\quad \annot{\accT}{\arrT*{(\funtypeT{\beta}{\SizeT}{\arrT*{\beta \szltT \alpha}{\app{\AccT}{\beta}}})}{\app{\AccT}{\alpha}}} \\[\xspacing]
  &\dataT{\app{\NatT}{(\annot{\alpha}{\SizeT})}}{\TypeT{1}} \\
  &\quad \annot{\zeroT}{\funtypeT{\beta}{\SizeT}{\arrT*{\beta \szltT \alpha}{\app{\NatT}{\alpha}}}} \\
  &\quad \annot{\succT}{\funtypeT{\beta}{\SizeT}{\arrT*{\beta \szltT \alpha}{\app{\NatT}{\beta}}{\app{\NatT}{\alpha}}}} \\[\xspacing]
  &\dataT{\app{\WT}{(\annot{A}{\TypeT{i}})}{(\annot{B}{\arrT*{A}{\TypeT{i}}})}{(\annot{\alpha}{\SizeT})}}{\TypeT{i+1}} \\
  &\quad \annot{\supT}{\funtypeT{\beta}{\SizeT}{\arrT*{\beta \szltT \alpha}{\funtypeT{a}{A}{\arrT*{(\arrT*{\app{B}{a}}{\app{\WT}{A}{B}{\beta}})}{\app{\WT}{A}{B}{\alpha}}}}}}
  \end{align*}
  \caption{Inductive data definitions}
  \label{#1}
  \end{figure}
}
\newcommand{\FigTransSize}[1]{
  \begin{figure}[h]
  \centering
  \begin{multicols}{2}
  \begin{align*}
  \Aboxed{\compile{s} &= \eT} \\
  \compile{\alpha} &= \alphaT \\
  \compile{\circ} &= \baseT \\
  \compile{\sss{s}} &= \app{\sucT}{\compile{s}}
  \end{align*}

  \begin{align*}
  \Aboxed{\compile{\Phi} &= \GammaT} \\
  \compile{\mt} &= \mt \\
  \compile{\Phi, \alpha} &= \compile{\Phi}, \annot{\alphaT}{\SizeT} \\
  \compile{\Phi, \bound{\alpha}{s}} &= \compile{\Phi}, \annot{\alphaT}{\SizeT}, \annot{\alphaT^*}{\alphaT \szltT \compile{s}}
  \end{align*}
  \end{multicols}
  \caption{Translation of sizes and size environments}
  \label{#1}
  \end{figure}
}

\newcommand{\FigTransSubsize}[1]{
  \begin{figure}[h]
  \centering
  \begin{mathpar}
  \fbox{$\subsizeto{\Phi}{s}{s}{\eT}$} \hfill \\
  \inferrule{
    (\bound{\alpha}{s}) \in \Phi
  }{
    \subsizeto{\Phi}{\sss{\alpha}}{s}{\alphaT^*}
  }
  \and
  \inferrule{~}{
    \subsizeto{\Phi}{\circ}{s}{\app{\baseleq}{\compile{s}}}
  }
  \and
  \inferrule{~}{
    \subsizeto{\Phi}{s}{s}{\app{\reflleq}{\compile{s}}}
  }
  \and
  \inferrule{~}{
    \subsizeto{\Phi}{s}{\sss{s}}{\app{\sucleq}{\compile{s}}}
  }
  \and
  \inferrule{
    \subsizeto{\Phi}{r}{s}{\eT}
  }{
    \subsizeto{\Phi}{\sss{r}}{\sss{s}}{\app{\monoT}{\compile{r}}{\compile{s}}{\eT}}
  }
  \and
  \inferrule{
    \subsizeto{\Phi}{s_1}{s_2}{\eT_{12}} \\
    \subsizeto{\Phi}{s_2}{s_3}{\eT_{23}}
  }{
    \subsizeto{\Phi}{s_1}{s_3}{\app{\transleq}{\compile{s_1}}{\compile{s_2}}{\compile{s_3}}{\eT_{12}}{\eT_{23}}}
  }
  \end{mathpar}
  \caption{Translation of subsizing}
  \label{#1}
  \end{figure}
}

\newcommand{\FigTransTerm}[1]{
  \begin{figure}[ht]
  \centering
  \begin{multicols}{2}
  \begin{align*}
  \Aboxed{\compile{e} &= \eT} \\
  \compile{x} &= \xT \\
  \compile{U} &= \UT \\
  \compile{\funtype{x}{\sigma}{\tau}} &= \funtypeT{\xT}{\compile{\sigma}}{\compile{\tau}_{\annot{x}{\sigma}}} \\
  \compile{\fun{x}{\sigma}{e}} &= \funT{\xT}{\compile{\sigma}}{\compile{e}_{\annot{x}{\sigma}}} \\
  \compile{\app{e_1}{e_2}} &= \app{\compile{e_1}}{\compile{e_2}} \\
  \compile{\Funtype{\alpha}{\tau}} &= \funtypeT{\alphaT}{\SizeT}{\compile{\tau}_{\alpha}} \\
  \compile{\Funtype<{\alpha}{s}{\tau}} &= \funtypeT{\alphaT}{\SizeT}{\funtypeT{\alphaT^*}{\alphaT \szltT \compile{s}}{\compile{\tau}_{\bound{\alpha}{s}}}} \\
  \compile{\Fun{\alpha}{e}} &= \funT{\alphaT}{\SizeT}{\compile{e}_{\alpha}} \\
  \compile{\Fun<{\alpha}{s}{e}} &= \funT{\alphaT}{\SizeT}{\funT{\alphaT^*}{\alphaT \szltT \compile{s}}{\compile{e}_{\bound{\alpha}{s}}}} \\
  \compile{\App{e}{s}} &= \app{\compile{e}}{\compile{s}} \qquad (\textrm{\rref{sapp}}) \\
  \end{align*}

  \begin{align*}
  \Aboxed{\compile{\Gamma} &= \GammaT} \\
  \compile{\mt} &= \mt \\
  \compile{\Gamma, \annot{x}{\tau}} &= \compile{\Gamma}, \annotT{\xT}{\compile{\tau}} \\
  \compile{\Gamma, \define{x}{\tau}{e}} &= \compile{\Gamma}, \defineT{\xT}{\compile{\tau}}{\compile{e}}
  \end{align*}
  \end{multicols}
  \vspace{-2\baselineskip}
  \begin{mathpar}
  \fbox{$\typeto{\Phi; \Gamma}{e}{\tau}{\eT}$} \quad \cdots \hfill \\
  \inferrule[\nameref{conv}]{
    \cdots \\
    \typeto{\Phi; \Gamma}{e}{\sigma}{\eT} \\
    \subtype{\Phi; \Gamma}{\sigma}{\tau}
  }{
    \typeto{\Phi; \Gamma}{e}{\tau}{\eT}
  }
  \and
  \inferrule[\nameref{sapp<}]{
    \subsizeto{\Phi}{\hat{s}}{r}{\eT'} \\
    \typeto{\Phi; \Gamma}{e}{\Funtype<{\alpha}{r}{\tau}}{\eT}
  }{
    \typeto{\Phi; \Gamma}{\App{e}{s}}{\subst{\tau}{\alpha}{s}}{\app{\eT}{\compile{s}}{\eT'}}
  }
  \and
  \inferrule[\nameref{let}]{
    \typeto{\Phi; \Gamma}{\sigma}{\UT}{\sigmaT} \\
    \typeto{\Phi; \Gamma}{e_1}{\sigma}{\eT_1} \\
    \typeto{\Phi; \Gamma, \define{x}{\sigma}{e_1}}{e_2}{\tau}{\eT_2}
  }{
    \typeto{\Phi; \Gamma}{\letin{x}{\sigma}{e_1}{e_2}}{\subst{\tau}{x}{e_1}}{\letinT{\xT}{\sigmaT}{\eT_1}{\eT_2}}
  }
  \end{mathpar}
  \caption{Translation of terms (base \lang) and term environments}
  \label{#1}
  \end{figure}
}

\newcommand{\FigTransInd}[1]{
  \begin{figure}[p]
  \centering
  \begin{align*}
  \Aboxed{\compile{e} &= \eT} \quad \cdots \\
  \compile{\N{s}} &= \app{\NatT}{\compile{s}} \\
  \compile{\W{s}{x}{\sigma}{\tau}} &= \app{\WT}{\compile{\sigma}}{\compile{\fun{x}{\sigma}{\tau}}}{\compile{s}}
  \end{align*}
  \begin{mathpar}
  \setlength{\jot}{-1.5pt}
  \fbox{$\typeto{\Phi; \Gamma}{e}{\tau}{\eT}$} \quad \cdots \hfill \\
  \inferrule[\nameref{zero}]{
    \wf{\Phi}{\Gamma} \\
    \subsizeto{\Phi}{\hat{r}}{s}{\eT'}
  }{
    \typeto{\Phi; \Gamma}{\zero{s}{r}}{\N{s}}{\app{\zeroT}{\compile{s}}{\compile{r}}{\eT'}}
  }
  \and
  \inferrule[\nameref{succ}]{
    \subsizeto{\Phi}{\hat{r}}{s}{\eT'} \\
    \typeto{\Phi; \Gamma}{e}{\N{r}}{\eT}
  }{
    \typeto{\Phi; \Gamma}{\succ{s}{r}{e}}{\N{s}}{\app{\succT}{\compile{s}}{\compile{r}}{\eT'}{\eT}}
  }
  \and
  \inferrule[\nameref{sup}]{
    \subsizeto{\Phi}{\hat{r}}{s}{\eT'} \\
    \typeto{\Phi; \Gamma}{\sigma}{U}{\sigmaT} \\
    \typeto{\Phi; \Gamma, \annot{x}{\sigma}}{\tau}{U}{\tauT} \\\\
    \typeto{\Phi; \Gamma}{e_1}{\sigma}{\eT_1} \\
    \typeto{\Phi; \Gamma}{e_2}{\arr*{\subst{\tau}{x}{e_1}}{\W{x}{\sigma}{\tau}{r}}}{\eT_2}
  }{
    \typeto{\Phi; \Gamma}{\sup{x}{\sigma}{\tau}{s}{r}{e_1}{e_2}}{\W{x}{\sigma}{\tau}{s}}{\app{\supT}{\sigmaT}{(\funT{\xT}{\sigmaT}{\tauT})}{\compile{s}}{\compile{r}}{\eT'}{\eT_1}{\eT_2}}
  }
  \and
  \inferrule[\nameref{case-nat}]{
    \typeto{\Phi; \Gamma}{e}{\N{s}}{\eT} \\
    \typeto{\Phi; \Gamma, \annot{x}{\N{s}}}{P}{U}{\PT} \\\\
    \typeto{\Phi, \bound{\alpha}{s}; \Gamma}{e_z}{\subst{P}{x}{\zero{s}{\alpha}}}{\eT_z} \\
    \typeto{\Phi, \bound{\beta}{s}; \Gamma, \annot{z}{\N{\beta}}}{e_s}{\subst{P}{x}{\succ{s}{\beta}{z}}}{\eT_s}
  }{
    \typeto{\Phi; \Gamma}{
      \begin{aligned}
      &\match{e}{\fun*{x}{P}}{ \\
      &\quad \App{\zero*}{\alpha} \Rightarrow e_z \\
      &\quad \app{\App{\succ*}{\beta}}{z} \Rightarrow e_s}
      \end{aligned}
    }{\subst{P}{x}{e}}{
      \begin{aligned}
      &\matchT{\eT}{\funT*{\mt}{\xT}{\PT}}{\\
      &\quad \app{\zeroT}{\alphaT}{\alphaT^*} \RightarrowT \eT_z \\
      &\quad \app{\succT}{\betaT}{\betaT^*}{\zT} \RightarrowT \eT_s}
      \end{aligned}
    }
  }
  \and
  \inferrule[\nameref{case-wft}]{
    \typeto{\Phi; \Gamma}{e}{\W{y}{\sigma}{\tau}{s}}{\eT} \\
    \typeto{\Phi; \Gamma, \annot{x}{\W{y}{\sigma}{\tau}{s}}}{P}{U}{\PT} \\
    \typeto{\Phi, \bound{\alpha}{s}; \Gamma, \annot{z_1}{\sigma}, \annot{z_2}{\arr*{\subst{\tau}{y}{z_1}}{\W{y}{\tau}{\sigma}{\alpha}}}}{e_s}{\subst{P}{x}{\sup{y}{\sigma}{\tau}{s}{\alpha}{z_1}{z_2}}}{\eT_s}
  }{
    \typeto{\Phi; \Gamma}{
      \begin{aligned}
      &\match{e}{\fun*{x}{P}}{ \\
      &\quad \app{\App{\sup*}{\alpha}}{z_1}{z_2} \Rightarrow e_s}
      \end{aligned}
    }{\subst{P}{x}{e}}{
      \begin{aligned}
      &\matchT{\eT}{\funT*{\mt}{\xT}{\PT}}{ \\
      &\quad \app{\supT}{\alphaT}{\alphaT^*}{\zT_1}{\zT_2} \RightarrowT \eT_s}
      \end{aligned}
    }
  }
  \and
  \inferrule[\nameref{fix}]{
    \typeto{\Phi, \alpha; \Gamma}{\sigma}{U}{\sigmaT} \\
    \typeto{\Phi, \alpha; \Gamma, \annot{f}{\Funtype<{\beta}{\alpha}{\subst{\sigma}{\alpha}{\beta}}}}{e}{\sigma}{\eT}
  }{
    \typeto{\Phi; \Gamma}{\fix{f}{\alpha}{\sigma}{e}}{\Funtype{\alpha}{\sigma}}{
      \begin{aligned}
      &\app{\wfind}{(\funT{\alphaT}{\SizeT}{\sigmaT})}{ \\
      &\quad (\funT{\alphaT}{\SizeT}{\funT{\fT}{\funtypeT{\betaT}{\SizeT}{\arrT*{\betaT \szltT \alphaT}{\subst{\sigmaT}{\alphaT}{\betaT}}}}{\eT}})}
      \end{aligned}
    }
  }
  \end{mathpar}
  \caption[Translation of naturals, well-founded trees, $\kw{case}$, fixpoints]{Translation of naturals, well-founded trees, \\
    and $\kw{case}$ and fixpoint expressions}
  \label{#1}
  \end{figure}
}

As briefly described in \cref{ch:introduction}, I model \lang in
\CICE.\index{Calculus of Inductive Constructions!Extensional \textasciitilde}
The key idea is that sizes in \lang can themselves be represented as an inductive type in \CICE,
and naturals and well-founded trees are then inductives with an additional size parameter.
Sizes are represented as a (generalization of) the Brouwer notation for ordinals in type theory,
and their order as an inductive type indexed by sizes.
The order is \emph{well founded}:
there is no infinite sequence of ever-smaller sizes,
and there is always a ``smallest'' size (or many of them).
This property allows for \emph{well-founded induction}\index{well-founded induction},
where to prove some property on sizes, one supposes that it holds for all strictly smaller sizes.

Every fixpoint expression in \lang is modelled as an instance of well-founded induction in \CICE.
To prove well-foundedness and in turn the induction principle,
I show that sizes satisfy an \emph{accessibility predicate}\index{accessibility predicate}~\citep{accessibility},
which states that a size is accessible if all sizes smaller than it are accessible
and provides an inductive structure for structural recursion that matches the semantic nature of the order.
For the type preservation proof to go through,
definitional proof irrelevance\index{proof irrelevance} of accessibility predicates
is required.
Since the proof irrelevance holds propositionally,
I use an extensional CIC to obtain the definitional equality
from the propositional equality via equality reflection\index{equality reflection}.

The first half of this chapter provides the syntax and judgements of \CICE.
In addition to the notation used in \cref{ch:sized-dep-types},
given variables $\vec{\xT} = \xT_1 \seq \xT_n$,
terms $\vec{\eT} = \eT_1 \seq \eT_n$,
and types $\vec{\tauT} = \tauT_1 \seq \tauT_n$,
\new{$\annotT{\vec{\xT}}{\vec{\tauT}}$} denotes the assumption environment
$\annotT*{\xT_1}{\tauT_1}, \seq, \annotT*{\xT_n}{\tauT_n}$,
\new{$\subst{\eT}{\vec{\xT}}{\vec{\eT}}$} denotes the simultaneous substitution
$\subst{\eT}{\xT_1, \seq, \xT_n}{\eT_1, \seq, \eT_n}$, 
\new{$\funT{\vec{\xT}}{\vec{\tauT}}{\eT}$} denotes the $n$-ary function
$\funT{\xT_1}{\tauT_1}{\seq \funT{\xT_n}{\tauT_n}{\eT}}$, and
\new{$\type{\GammaT}{\vec{\eT}}{\vec{\tauT}}$} denotes the $n$ typing judgements
$(\type{\GammaT}{\eT_1}{\tauT_1})$, \seq, $(\type{\GammaT, \annotT{\eT_1}{\tauT_1}, \seq, \annotT{\eT_{n-1}}{\tauT_{n-1}}}{\eT_n}{\tauT_n})$.
I also assume that shadowing is disallowed and that variable names coincide
just as in \lang for convenience, to avoid renaming as much as possible.

% TODO: clearpage
\clearpage
The second half then describes the translation from \lang to \CICE,
which is a metafunction from typing derivations of \lang to terms of \CICE.
Therefore, the translation is only defined for well-typed \lang terms,
but the type preservation theorem only applies to well-typed terms anyway.

\section{Target Type Theory} \label{sec:target}

\FigSyntaxCIC{fig:syntax-cic}
The syntax of \CICE
is given in \cref{fig:syntax-cic};
differences from \lang include a 1-based index for the recursive argument of fixpoint expressions,
$\tg{case}$ expression motives abstracted over the target's inductive type indices,
and a homogeneous propositional equality\index{propositional equality} type with the reflexivity constructor and $\JT*$ eliminator.
New inductive types are defined using data definitions $\DT$,
whose syntax resembles the informal presentation used in \cref{ch:sized-dep-types}.
Metavariable usage convention is roughly the same as for \lang,
with the addition of $\pT$ for inductive type parameters or proofs of equality
and $\aT$ for inductive type indices.

The well-formedness conditions on inductive data definitions,
such as well-typedness and \emph{strict positivity}\index{strict positivity},
are entirely standard, so I omit them here;
see pCIC\index{Calculus of Inductive Constructions!Predicative \textasciitilde}~\citep{pCIC}
for instance for a full description.
Inductive definitions in their full generality are not needed,
and nonmutual, nonnested inductives suffice.
Indeed, only six inductive definitions are used for the translation
for representing sizes, their order, their well-foundedness,
and the empty type, naturals, and well-founded trees.

The typed equivalence\index{equivalence}, subtyping, and typing judgements are defined mutually:
equivalence depends on typing and subtyping,
subtyping depends on equivalence,
and typing depends on subtyping and equivalence.
I present first the equivalence rules in \cref{fig:equivalence},
with the subtyping and typing rules to follow.

Equivalence is, by definition, an equivalence relation,
satisfying reflexivity, symmetry, and transitivity.
Equivalence is also congruent, using the same summary of congruence rules as for \lang via \rref{equiv-cong};
the full set of rules can similarly be found in \cref{app:cong:equiv}.
An equivalence judgement can be converted to one annotated by a supertype via \rref{equiv-conv}.
The key rule for extensionality is equality reflection\index{equality reflection} in \rref{equiv-reflect},
which definitionally equates two terms whenever there exists some proof of their propositional equality.

\FigEquiv{fig:equivalence}

Typed equivalence is required in the presence of equality reflection since
inconsistencies are derivable when using untyped conversion\index{conversion}.
For instance, supposing that \lang had equality reflection and using \new{$\approx$} to denote conversion,
in the empty environment, freely using transitivity,
%
\begin{align*}
  \const{0'} &\approx \app{(\fun{p}{\eq{\const{0'}}{\N{\hat{\hat{\circ}}}}{\const{1}}}{\const{0'}})}{\refl{\const{0'}}} &&\textrm{by $\beta$-reduction} \\
  &\approx \app{(\fun{p}{\eq{\const{0'}}{\N{\hat{\hat{\circ}}}}{\const{1}}}{\const{1}})}{\refl{\const{0'}}} &&\textrm{by congruence and reflection of $p$} \\
  &\approx \const{1} &&\textrm{by $\beta$-reduction},
\end{align*}
since reduction and therefore conversion occurs even when its terms are ill typed,
as the second and third terms are.
An alternate solution would be to disallow transitivity of conversion~\citep{CCE},
but this is too limiting when trying to prove type preservation,
and equivalence would no longer be an equivalence relation.

Untyped conversion with equality reflection also violates subject reduction
under certain environments.
For example, suppose the environment contains the equality $\eq{\arr*{\N{\sss{\circ}}}{\N{\sss{\circ}}}}{}{\arr*{\N{\sss{\circ}}}{\Prop}}$.
Then $\app{(\fun{x}{\N{\sss{\circ}}}{x})}{\const{0}}$ can be assigned type $\Prop$
by equality reflection and conversion of the function's type,
but this term reduces to $\const{0}$ which \emph{cannot} be assigned type $\Prop$.
This can be resolved by adding
\emph{injectivity of type constructors}\index{injectivity of type constructors},
which would allow deriving the equality $\eq{\N{\sss{\circ}}}{}{\Prop}$
from the above, but doing so is generally undesirable since it's inconsistent with
the axioms of both excluded middle and univalence \citep{unification}.

The remaining equivalence rules are typed versions of the usual reduction rules,
with typing premises to ensure well-typedness of both sides.
Functions have both a $\beta$-equivalence rule and an $\eta$-equivalence rule,
the latter of which is only possible since equivalence is typed.
Equivalence rules for $\tg{let}$ expressions are exactly the same as in \lang.
The $\JT*$ eliminator and $\tg{case}$ expressions reduce when applied to
reflexivity and inductive constructors, respectively.

\rref{equiv-mu} for fixpoint expressions is \emph{unguarded}\index{guarded reduction},
meaning that fixpoints are equivalent to the substitution of itself into its own body
regardless of what they are applied to.
To maintain normalization,
the usual guarded reduction rule in intensional CIC reduces fixpoints
only when applied to a literal constructor in the recursive argument position.
\vspace{-0.25\baselineskip}
\begin{mathpar}
\inferrule[\rlabel{$\equiv$-$\mu$-guarded}{equiv-mu-guarded}]{\cdots \\ \card{\vec{\eT}'} + 1 = \nT}{
  \defeq{\GammaT}{\app{(\fixT{\nT}{\fT}{\tauT}{\eT})}{\vec{\eT}'}{(\app{\cT}{\vec{\aT}})}}{\app{(\subst{\eT}{\fT}{\fixT{\nT}{\fT}{\tauT}{\eT}})}{\vec{\eT}'}{(\app{\cT}{\vec{\aT}})}}{\tauT}
}
\end{mathpar}

Evidently \rref{equiv-mu-guarded} can be derived from \rref{equiv-mu} by congruence.
On the other hand, for any particular inductive type $\XT$,
letting $\tauT$ be $\arr{\vec{\xT}}{\vec{\sigmaT}}{\funtypeT{\xT}{\app{\XT}{\vec{\pT}}{\vec{\aT}}}{\tauT'}}$,
\rref{equiv-mu} can be derived from \rref{equiv-mu-guarded} via reflection of the following provable propositional equality,
freely using transitivity:
%
\begin{align*}
\fixT{\nT}{\fT}{\tauT}{\eT} &\eq{}{}{} \funT{\vec{\xT}}{\vec{\sigmaT}}{\funT{\xT}{\app{\XT}{\vec{\pT}}{\vec{\aT}}}{\app{(\fixT{\nT}{\fT}{\tauT}{\eT})}{\vec{\xT}}{\xT}}}
\qquad \textrm{definitionally by \rref{equiv-eta}} \\
& \eq{}{}{} \funT{\vec{\xT}}{\vec{\sigmaT}}{\funT{\xT}{\app{\XT}{\vec{\pT}}{\vec{\aT}}}{\matchT*{\xT}{\seq(\app{\cT}{\vec{\zT}} \RightarrowT \app{(\fixT{\nT}{\fT}{\tauT}{\eT})}{\vec{\xT}}{(\app{\cT}{\vec{\pT}}{\vec{\zT}})})\seq}}} \\
& \phantom{\eq{}{}{}} \textrm{by congruence and case analysis on $\xT$} \\
& \eq{}{}{} \funT{\vec{\xT}}{\vec{\sigmaT}}{\funT{\xT}{\app{\XT}{\vec{\pT}}{\vec{\aT}}}{\matchT*{\xT}{\seq(\app{\cT}{\vec{\zT}} \RightarrowT \app{(\subst{\eT}{\fT}{\fixT{\nT}{\fT}{\tauT}{\eT}})}{\vec{\xT}}{(\app{\cT}{\vec{\pT}}{\vec{\zT}})})\seq}}} \\
& \phantom{\eq{}{}{}} \textrm{definitionally by \rref{equiv-cong, equiv-mu-guarded}} \\
& \eq{}{}{} \funT{\vec{\xT}}{\vec{\sigmaT}}{\funT{\xT}{\app{\XT}{\vec{\pT}}{\vec{\aT}}}{\app{(\subst{\eT}{\fT}{\fixT{\nT}{\fT}{\tauT}{\eT}})}{\vec{\xT}}{\xT}}} \\
& \phantom{\eq{}{}{}} \textrm{by congruence and case analysis on $\xT$} \\
& \eq{}{}{} \subst{\eT}{\fT}{\fixT{\nT}{\fT}{\tauT}{\eT}}
\qquad \textrm{definitionally by \rref{equiv-eta}}
\end{align*}

Since \rref{equiv-mu} and \rref{equiv-mu-guarded} are metatheoretically equivalent in extensional CIC,
I choose to use \rref{equiv-mu} for its simplicity.

\FigSubtypingCIC{fig:subtyping-cic}
As opposed to \lang, for \CICE I use
pCIC's\index{Calculus of Inductive Constructions!Predicative \textasciitilde}
presentation of subtyping\index{subtyping} in \cref{fig:subtyping-cic},
which has a typed equivalence premise in \rref{subtype-conv}.
It also has an explicit rule for transitivity of subtyping since judgements such as
$\subtype{\mt}{\app{(\funT{P}{\TypeT{\tg{1}}}{P})}{\PropT}}{\TypeT{\tg{0}}}$ would fail to hold otherwise.
Like \rref{acum-pi}, \rref{subtype-pi} is invariant in the domain of function types.

The typing and environment well-formedness rules are in \cref{fig:typing-cic}.
Except for \rref{fix*}, the starred rules are the same as for \lang,
with metafunctions $\axioms{\mt}$ and $\rules{\mt}{\mt}$ operating similarly on universes $\UT$
as in \cref{fig:rules-axioms}.
An additional premise to \rref{fix*} ensures that the $\nT$th argument is indeed an inductive type.
% in addition to checking well-typedness of fixpoint bodies with possible recursive references.

As previously mentioned, fixpoints must also be guarded\index{guardedness}:
recursive calls can only occur on structurally smaller arguments of elements of inductives.
The guard condition is well studied \citep{guard, guard-relax, Coq} and so omitted here.
To justify uses of fixpoint expressions in the translation,
I will provide either a mechanization or present a brief argument of guardedness.

\FigTypingCIC{fig:typing-cic}

The new \rref{eq, refl, J} are for the propositional equality\index{propositional equality} type,
its constructor, and its eliminator.
Given some equality proof $\pT$ of $\eqT{\eT_1}{\tauT}{\eT_2}$
and a motive $\PT$ that depends on some equality $\eqT{\eT_1}{\tauT}{\yT}$ for any $\yT$\punctstack{,}%
\footnote{Occasionally referred to as \emph{Paulin-Mohring}'s equality~\citep{CIC},
as opposed to \emph{Martin-L\"of}'s equality~\citep{MLTT}
where the motive instead has type $\funtypeT{\xT}{\tauT}{\funtypeT{\yT}{\tauT}{\arrT*{\eqT{\xT}{\tauT}{\yT}}{\UT}}}$.}
to prove $\app{\PT}{\eT_2}{\pT}$ it suffices to provide to $\JT*$ a proof that
$\app{\PT}{\eT_1}{\reflT{\eT_1}}$ holds.
In terms of pattern matching, this is equivalent to case analysis of $\pT$ as $\reflT{\eT_1}$.
Other usual functions on proofs of equality can be derived from it,
such as coercion (when the motive is \mbox{$\funT{A}{\UT}{\funT{\any}{\any}{A}}$})
or substitution (when the motive ignores the second argument),
as well as its symmetry, transitivity, and congruence.%
\footnote{The combined presence of $\JT*$ and equality reflection also allows proving the
\emph{uniqueness of identity proofs} (UIP)\index{uniqueness of identity proofs},
or that all proofs of an equality $\eqT{a}{A}{b}$ are themselves equal to one another,
by reflecting $p$ in the term $\JT{(\funT{b}{A}{\funT{p}{\eqT{a}{A}{b}}{\eqT{\reflT{a}}{}{p}}})}{\reflT{\reflT{a}}.}{}$}

\rref{ind, constr, case} assign types to inductive types, their constructors,
and $\tg{case}$ expressions, under the premise that
the relevant inductive data definition exists and is well formed.
Here, the difference between the parameters and the indices of inductive types becomes apparent:
the motive\index{motive} of a $\tg{case}$ expression is abstracted over the indices by $\vec{\yT}$
in addition to the target by $\xT$, while the parameters $\vec{\pT}$ are fixed throughout.
Therefore, when dealing with the types of indices and constructor arguments,
the parameters are first substituted in place of $\vec{\wT}$.

A $\tg{case}$ expression is well typed if its target is,
if its motive is for any indices and target with those indices,
and if each branch is well typed for that branch's constructor arguments,
where its type is the motive with the appropriate indices and reconstructed target.
For notational simplicity, the rule assumes that the binding variable names
$\vec{\yT}$ and $\vec{\zT}$ are those found in the data definition,
but of course these can be renamed at the expense of additional renaming substitutions.

Additionally, the motive of a $\tg{case}$ expression is restricted by the metarelation $\elim{\mt}{\mt}{\mt}$,
which indicates when large elimination\index{large elimination} is allowed.
$\elim{\any}{\TypeT{\iT}}{\any}$ and $\elim{\any}{\PropT}{\PropT}$ always hold,
so that inductives in $\TypeT{}$ can be eliminated to any universe
and inductives in $\PropT$ can be eliminated to $\PropT$,
while $\elim{\XT}{\PropT}{\TypeT{\iT}}$ holds if the inductive $\XT$ in $\PropT$ satisfies further conditions.
For the purposes of the translation, the only relevant conditions are that $\XT$ either have no constructors
or have a single constructor whose arguments are all in $\PropT$.
They can be loosened while still retaining consistency (see \eg \citet{SProp}).

The typing premises of
\rref{equiv-beta, equiv-zeta, equiv-rho, equiv-iota, equiv-mu}
corresponding to some of the reduction rules duplicate the premises found in many of the typing rules,
trivially ensuring \emph{subject equivalence}\index{subject reduction}
(\ie well-typedness of both sides).

\section{Preliminary Definitions}

Before defining the translation from \lang terms to \CICE terms,
in this section I describe how the necessary \CICE terms are constructed,
which comprises the aforementioned six inductive data definitions
and well-founded induction principle
as well as the various properties which the order on sizes satisfies.

\FigData{fig:data-defns}

The inductive definitions are listed in \cref{fig:data-defns}
along with some basic definitions I treat as global.
$\botT$ is the usual empty type.
$\SizeT$ is a generalization of the $\Ord*$ type introduced in \cref{sec:examples},
with the domain of the function passed to the limit size $\limT$ replaced by some arbitrary type.
Although limit sizes aren't strictly necessary for the translation,
since \lang only has successor sizes and size variables,
I include them so that various solutions to the problem of the infinite size
can be explored in \cref{sec:infinity}.
Furthermore, this allows for a simplification of the definition,
since the zero size $\baseT$ can be defined as a limit size rather than as another constructor.

The order on sizes $\mt \szleT \mt$ is defined by the three properties that must hold~\citep{ordinals}:
\begin{itemize}[noitemsep]
  \item $\monoT$: The successor operator $\sucT$ is monotone with respect to the order;
  \item $\coconeT$: The limit operator $\limT$ constructs an upper bound in that
    given some function $f$ returning a size,
    any size smaller than any size returned by $f$ is also smaller than the limit of $f$;
  \item $\limitT$: The limit operator on $f$ constructs a \emph{least} upper bound such that
    if a size is larger than \emph{all} sizes returned by $f$
    then it must also be larger than the limit of $f$.
\end{itemize}

Other properties of the order can be derived by induction from these constructors alone.
A corresponding strict order $\mt \szltT \mt$ is also defined,
and an accessibility predicate\index{accessibility predicate} $\AccT$ is specialized to sizes and the strict order.
Note that it lives in $\PropT$, as does the argument of its sole constructor $\accT$,
so accessibility predicates are intended to be interpreted as proof irrelevant\index{proof irrelevance},
and their large elimination\index{large elimination} is allowed.

\FigDefns{fig:defns}

Before moving on to naturals and well-founded trees,
\cref{fig:defns} lists the names and types of a number of provable definitions.
First is \emph{function extensionality}\index{function extensionality},
asserting that two functions $\annot{f, g}{\funtypeT{x}{A}{\app{B}{x}}}$ are propositionally equal if they are pointwise equal.
In \CICE, they are in fact equivalent (\ie definitionally equal):
given some proof $h$ of their pointwise equality,
under the assumption $\annot{x}{A}$,
the propositional equality $\eq{\app{f}{x}}{}{\app{g}{x}}$ proven by $\app{h}{x}$
can be reflected into the corresponding definitional equality,
which by \rref{equiv-eta} is then a definitional equality of $f$ and $g$.

The remaining definitions have been mechanized in either Agda and Coq in
\cref{app:mechanization:agda:prelim} and \cref{app:mechanization:coq:prelim}, respectively,
under the assumption of function extensionality as an axiom
(which cannot be proven in, but is consistent with, intensional CIC).
The mechanizations don't use any additional type-theoretic features beyond CIC,
and the proofs could theoretically be written in plain CIC,
but they would be far less manageable without the ergonomics provided by the proof assistants.
As an example, the Coq proof for $\accessible$ consists of a dozen lines of tactics,
while the full proof term generated from the tactics is 129 lines long.

The definitions themselves describe properties of the order on sizes
and of the accessibility predicate:
\begin{itemize}[noitemsep]
  \item $\baseleq$, $\reflleq$, $\transleq$, and $\sucleq$:
    The order is reflexive and transitive (\ie a preorder)
    such that $\baseT$ is smaller than or equal to all sizes
    and the successor of a size is greater or equal to itself.
  \item $\accIsProp$: Accessibility predicates are mere propositions\index{mere proposition}:
    any two proofs of accessibility of a size are propositionally equal.
  \item $\accleq$: Any size smaller or equal to an accessible size is itself accessible.
  \item $\accessible$: All sizes are accessible; in other words, the order on sizes is well founded.
  \item $\wfind$ and $\wfacc$: The well-founded induction principle\index{well-founded induction} on sizes with respect to the order,
    proven by structural induction on the accessibility of sizes.
\end{itemize}

The only time equality reflection\index{equality reflection} is needed is to prove $\accIsProp$ (via $\funext$),
which in turn is the only other equality that is reflected for proving type preservation.
Since $\AccT$ is in already $\PropT$, \CICE could be replaced by an intensional CIC
with a universe of \emph{strict propositions}\index{strict proposition} $\SPropT$
of types whose elements are definitionally equal~\citep{SProp},
then placing $\AccT$ in $\SPropT$.
This is disallowed by \opcit because it breaks normalization;
however, it doesn't break consistency,
so it would remain suitable as the target language of a syntactic model.
In any case, I use \CICE because equality reflection is more established in the literature
and it allows me to use \rref{equiv-mu} in place of \rref{equiv-mu-guarded}.

Finally, the naturals and the well-founded trees in \CICE are parametrized by a $\SizeT$.
Their definitions respect the translation in the upcoming section,
so that the types of $\NatT$ and $\WT$ and their constructors are preserved.

\section{Translation}

The key type preservation\index{type preservation} theorem states that well-typed terms of \lang translate to
corresponding well-typed terms of CICE.
However, terms are not the only thing requiring translation:
well-typedness holds under some environment, so term environments need translations;
sizes and their environments translate to terms and term environments as well;
and derivations of size orders translate to terms which represent them.

\FigTransSize{fig:trans:size}
I begin with the translation of sizes and their environments in \cref{fig:trans:size},
which are straightforward recursive metafunctions over their syntax.
I use an asterisk superscript \new{$\mt^\ast$} on a variable $\alphaT$ to represent
a fresh variable uniquely associated with $\alphaT$.
Given some bound size variable $\alphaT$,
$\alphaT^\ast$ represents the proof that the translated size is strictly smaller
than its size bound.

\FigTransSubsize{fig:trans:subsize}
The translation of subsizing judgements, on the other hand,
is a recursive metafunction over the \emph{subsizing derivation}.
\cref{fig:trans:subsize} defines the translation to \CICE terms
by induction on the subsizing rules, which recursively translates subderivations.

The translation of terms and term environments are similarly defined
as recursive metafunctions over the typing and well-formedness derivations,
denoted by
\mbox{$\typeto{\Phi; \Gamma}{e}{\tau}{\eT}$} and \mbox{$\wfto{\Phi}{\Gamma}{\GammaT}$}
respectively.
However, for concision, I use $\compile{e}$ to mean the translation of $e$
when well typed under the current implicit environments,
$\compile{e}_{\Phi}$ or $\compile{e}_{\Gamma}$ to mean the translation of $e$
with the current implicit environments extended with $\Phi$ or $\Gamma$,
and $\compile{\Gamma}$ to mean the translation of $\Gamma$
when well formed under the current implicit size environment.

\FigTransTerm{fig:trans:term}

The translation for base \lang without inductives is given in \cref{fig:trans:term}
in the more concise notation when translated terms follow directly from subderivations
and in the usual notation otherwise.
Terms not involving sizes are translated in a straightforward recursive manner.
Unbounded size quantifications and abstractions translate to quantifications and abstractions over $\SizeT$,
while bounded ones have additional quantifications and abstractions over a proof of $\szltT$.
Size applications translate to an additional application to a proof of $\szltT$
when the size abstraction applied is bounded.

Finally, \cref{fig:trans:ind} gives the translation for naturals, well-founded trees,
$\kw{case}$ expressions, and fixpoint expressions from their typing derivations
to a \CICE term (omitting irrelevant side conditions).
Aside from fixpoints, the translations are fairly straightforward,
with an additional proof term from subsizing for constructors
and dually an additional abstraction over such terms in the branches of $\tg{case}$ expressions.

\FigTransInd{fig:trans:ind}

Fixpoints in \lang are not translated as fixpoints in \CICE.
Doing so while preserving types would mean that \lang fixpoints
need to be subject to the same guard conditions as \CICE fixpoints,
which would defeat the purpose of using sized types.
Instead, every single \lang fixpoint, regardless of the inductive on which they recur,
is translated to well-founded induction\index{well-founded induction},
which is defined via a \CICE fixpoint on accessibility predicates\index{accessibility predicate}.
Intuitively, the return type of a fixpoint corresponds to the motive of well-founded induction,
while recursion on a strictly smaller size corresponds to the induction hypothesis,
where the motive holds for all strictly smaller sizes.

In the next chapter, I prove that the translation is type preserving and,
along the way, that the translations of convertible terms are equivalent.
The main challenge is showing that the translations of a fixpoint
and its $\mu$-reduction are equivalent in \CICE,
because this requires showing that the computational behaviour of well-founded induction
is equivalent to that of the fixpoint.
Once that has been established,
showing type preservation of the translation is more or less going through the motions of the proof,
since the remaining \lang terms translate almost directly to their syntactically corresponding terms.
%\include{impl}
%\chapter{Discussion and Conclusions}\label{ch:discussion}

\lang is neither flawless nor complete:
the nature of the syntactic model requires that inductives live in a universe
higher than that in which their corresponding unsized types would live,
and it's missing features such as an infinite size and coinductives.
In this chapter, I discuss some of these shortcomings
and give possible directions for future work.

\section{The Infinite Size} \label{sec:infinity}

In prior sized type systems, the infinite size\index{infinite size} $\infty$ is applied to sized inductive types
to represent a \emph{full} inductive type encompassing that inductive at all sizes,
which essentially corresponds to the usual unsized inductive.
Whereas an inductive of some size $s$ can be thought of as the type of elements
with at most $s$ many layers of constructors,
the full inductive can be thought of as the type of elements with any number of layers of constructors.

The infinite size is characterized by its subsizing behaviour:
$\subsize*{s}{\infty}$ holds for \emph{any} $s$.
This includes its own successor, \ie $\subsize*{\sss{\infty}}{\infty}$,
leading to non--well-founded sequences of strictly ``decreasing'' sizes:
$\dots < \infty < \infty < \infty$.
Naturally, there's no way to model the infinite size as an element of $\SizeT$
given that I've shown that $\SizeT$s \emph{are} well founded.
If there were, then it'd be possible to prove an inconsistency as follows.
Let $\inftyT$ be the translation of $\infty$,
and let $\inftyltinfty$ be the translation of $\subsize*{\sss{\infty}}{\infty}$.
\begin{align*}
&\LetT{\tg{{\neg}wf\inftyT}}{\arrT*{\app{\AccT}{\inftyT}}{\botT}}{\funT{\mathit{acc}}{\app{\AccT}{\inftyT}}{\matchT*{\mathit{acc}}{\app{\accT}{p} \RightarrowT \app{p}{\inftyT}{\inftyltinfty}}}} \\
&\LetT{\tg{false}}{\botT}{\app{\tg{{\neg}wf\inftyT}}{(\app{\accessible}{\inftyT})}}
\end{align*}

In set-theoretic models of sized type systems with an infinite size,
where sizes are modelled as set-theoretic (transfinite) ordinals,
the infinite size isn't modelled as a single ordinal;
instead, for each use of the infinite size,
its set-theoretic interpretation is an ordinal that is ``large enough'' in that context.
For instance, the interpretation of the infinite size of $\N{\infty}$
is the first limit ordinal $\omega$.

This strategy doesn't adapt well to \lang with its size abstractions and syntactic model\index{syntactic model},
since it requires a non-local translation of sizes.
For instance, given the size application $\App{e}{\infty}$,
what $\infty$ translates to would hypothetically depend on what $e$ translates to,
and likely require further static analysis of $\compile{e}$ beyond a simple translation
over typing derivations.

Since the motivation for having the infinite size is specifically for representing full inductives,
one alternative could be to define the full inductive separately
and provide functions to and from the corresponding sized inductive,
such as the following for $\W*$.
\begin{align*}
& \data{\App{\app{\W*}{(\annot{A}{\Type{i}})}{(\annot{B}{\arr*{A}{\Type{i}}})}}{\infty}}{\Type{i+1}} \\
& \quad \annot{\constr{sup\infty}}{\arr{x}{A}{\arr*{(\arr*{\app{B}{x}}{\app{\App{\W*}{\infty}}{A}{B}})}{\App{\app{\W*}{A}{B}}{\infty}}}}
\end{align*}

Defining a function from $\W{x}{A}{B}{s}$ to $\W{x}{A}{B}{\infty}$ is trivial,
since we're discarding size information.
What about going from $\W{x}{A}{B}{\infty}$ to $\W{x}{A}{B}{s}$?
What should $s$ be?
The size algebra could be augmented to be able to represent transfinite ordinals
so that $s$ is again a size that is ``large enough'',
but we can hardly demand programmers to manipulate ordinals,
and I conjecture that we would lose any hope of deciding $\subsize*{}{}$
without any user intervention.

The key insight is that what's important about an element of a full inductive
isn't its precise size and depth of constructors,
but merely that it has \emph{some} unknown size.
Another alternative to the infinite size, then, could be to represent a full inductive
as an existentially-quantified sized inductive,
\ie $\Pairtype{\alpha}{\N{\alpha}}$ and $\Pairtype{\alpha}{\W{x}{A}{B}{\alpha}}$.
We've already seen existential sizes in action: they're in the return types of $\qsort$ and $\msort$.

There is still a limitation similar to that in \cref{sec:examples:limitations}
when trying to represent the constructors of full inductives.
For $\Pairtype{\alpha}{\W{x}{A}{B}{\alpha}}$, we need a ``constructor'' of the following type.
\begin{align*}
\annot{\const{sup'}}{\arr{x}{A}{\arr*{(\arr*{B}{\Pairtype{\alpha}{\W{x}{A}{B}{\alpha}}})}{\Pairtype{\alpha}{\W{x}{A}{B}{\alpha}}}}}
\end{align*}
All we need is a function
$$\annot{\const{ac}}{\arr{x}{A}{\arr*{(\arr*{B}{\Pairtype{\alpha}{\W{x}{A}{B}{\alpha}}})}{\Pairtype{\alpha}{(\arr*{B}{\W{x}{A}{B}{\alpha}})}}}}$$
and we're good to go.
\begin{align*}
\app{\const{sup'}}{x}{f} =
\unpair*{\alpha}{f'}{\app{\const{ac}}{x}{f}}{\Pair{\sss{\alpha}}{\sup{x}{A}{B}{\sss{\alpha}}{\alpha}{x}{f'}}}
\end{align*}

Unfortunately, as the name might suggest,
$\const{ac}$ is an instance of the axiom of choice\punctstack{,}%
\footnote{The ``choice'' made here is the existentially-quantified size in the consequent:
the axiom asserts that there's always a way to choose a size such that
all of the well-founded trees returned have that size.}
which for weak existentials\index{weak dependent pair}
(\ie pairs whose elements we can match on but not project out)
is nonconstructive,
so there's no hope of implementing $\const{ac}$.
Even if size expressions were terms and
projections out to some primitive type $\Size$ of sizes were allowed,
implementing $\const{ac}$ still requires some notion of limit sizes $\slim{A}{f}$ on functions
$\annot{f}{\arr*{A}{\Size}}$ and the below subsizing derivation rules,
which would be undecidable in general if $A$ has uncountably many elements
from which the first rule must select an $a$.
%
\begin{mathpar}
\inferrule{
\subsize{\Phi; \Gamma}{s}{\app{f}{a}}
}{
\subsize{\Phi; \Gamma}{s}{\slim{A}{f}}
}
\and
\inferrule{
\subsize{\Phi; \Gamma, \annot{x}{A}}{\app{f}{x}}{s}
}{
\subsize{\Phi; \Gamma}{\slim{A}{f}}{s}
}
\end{mathpar}

Even so, taking weak existentials as a primitive of \lang
rather than defining them by an encoding,
and modelling them by strong dependent pairs in \CICE,
the \emph{translation} of $\const{ac}$ can be implemented as a function,
making use of $\limT$
In other words, the syntactic model\index{syntactic model} justifies the axiom $\const{ac}$ in the source.
The mechanization of $\compile{\const{ac}}$ in Agda and Coq are given in
\cref{app:mechanization:agda:W} and \cref{app:mechanization:coq:W}, respectively.

Nevertheless, $\const{ac}$ remains a noncomputing axiom
unless the size algebra is augmented to include a limit operator
and size expressions consequently treated as regular terms.
There is no way to represent generalized full inductives as
sized inductives existentially quantified by sizes
and faithfully define their constructors without losing either
decidability of subsizing, due to exposing the underlying ordinal representation of sizes,
or canonicity, due to the inclusion of a noncomputing axiom.
Representing ordinary full inductives (namely naturals) this way,
however, has been previously explored~\citep{guarded, modal-sizes}.

\section{Universe Levels of Inductives} \label{sec:universe-levels}

The universes in which the type of natural numbers and W types of \lang live
are, in a sense, one level\index{universe level} higher than is usually expected;
their typing rules are reproduced below.
$\N*$ is typically in $\Type{0}$,
while $\W*$ is typically in $U$ rather than $\axioms{U}$.
\begin{mathpar}
\inferrule[\rref*{nat}]{
  \wf{\Phi}{\Gamma} \\
  \wf{\Phi}{s}
}{
  \infer{\Phi; \Gamma}{\N{s}}{\Type{1}}
}
\and
\inferrule[\rref*{wft}]{
  \wf{\Phi}{s} \\
  \infer{\Phi; \Gamma}{\sigma}{U} \\\\
  \infer{\Phi; \Gamma, \annot{x}{\sigma}}{\tau}{U}
}{
  \infer{\Phi; \Gamma}{\W{x}{\sigma}{\tau}{s}}{\axioms{U}}
}
\end{mathpar}

This is due to how the translation is defined:
$\NatT$ and $\WT$ have a $\SizeT$ as parameter,
the $\limT$ operator quantifies over the type of the domain of its function argument,
and that type must be ``large'' enough to accommodate the correct type.
For $\app{\WT}{\sigmaT}{\tauT}{\sT}$,
the domain is $\app{\tauT}{\aT}$ for some $\aT$\punctstack{,}%
\footnote{For generalized inductives, given a recursive argument in the form of a function,
the domain of $\limT$ should encompass the domain of that function.}
as is the case for the definition of $\compile{\const{ac}}$,
so if its universe is $\TypeT{\iT}$,
then the universe of the type of $\sT$ must be $\TypeT{\tg{i+1}}$,
according to the definition of $\SizeT$, reproduced below.
\begin{align*}
&\dataT{\SizeT}{\TypeT{\tg{i+1}}} \\
&\quad \annotT{\sucT}{\arrT*{\SizeT}{\SizeT}} \\
&\quad \annotT{\limT}{\funtypeT{A}{\TypeT{\iT}}{\arrT*{(\arrT*{A}{\SizeT})}{\SizeT}}}
\end{align*}

This is a nonnegotiable condition:
defining $\SizeT$ to be in the same universe as that over which $\limT$ quantifies
leads to $\szltT$ no longer being well founded,
since in this hypothetical scenario, $\SizeT$ itself could be applied to $\limT$
to define an $\inftyT$ size.
\begin{align*}
&\LetT{\inftyT}{\SizeT}{\app{\limT}{\SizeT}{(\funT{\xT}{\SizeT}{\xT})}} \\
&\LetT{\inftyltinfty}{\inftyT \szltT \inftyT}{\app{\coconeT}{\SizeT}{(\app{\sucT}{\inftyT})}{(\funT{\xT}{\SizeT}{\xT})}{(\app{\sucT}{\inftyT})}{(\app{\reflleq}{(\app{\sucT}{\inftyT})})}}
\end{align*}

One way to ``shrink'' the universe of $\SizeT$, so to speak,
could be to parametrize it over the type over which $\limT$ currently quantifies,
yielding the following inductive definition.
\begin{align*}
&\dataT{\app{\SizeT}{(\annotT{A}{\TypeT{\iT}})}}{\TypeT{\iT}} \\
&\quad \annotT{\sucT}{\arrT*{\SizeT}{\SizeT}} \\
&\quad \annotT{\limT}{\arrT*{(\arrT*{A}{\SizeT})}{\SizeT}}
\end{align*}

The problem with this alternative is that for $\app{\WT}{\sigmaT}{\tauT}$,
the parameter of its size parameter would be $\app{\tauT}{\aT}$,
where $\annotT{\aT}{\sigmaT}$ is the third formal argument to its constructor $\supT$,
but this argument is only part of the constructor, not the type.
The intuition is that the parametrized $\SizeT$ is too restrictive
and there aren't ``enough'' sizes to cover all well-founded trees of any particular type.

In a sense, it's reasonable to expect that $\SizeT$ needs to live in a larger universe.
The r\^ole of $\SizeT$ is to represent all of the possible sizes of a given full inductive,
so we would expect $\SizeT$ to be just as large as the inductive itself.
Aside from situations where the inductive is impredicative\index{impredicativity}, \ie in $\PropT$,
parametrizing over $\SizeT$ then necessarily requires moving up a universe
so that it's not essentially quantifying over itself.

If having inductive types in the correct universe were more important than
the potential expressibility of infinitude,
we could eliminate $\limT$ altogether and model sizes as ordinary naturals.
Then $\SizeT$ would live in $\Type{0}$, and $\NatT$ and $\WT$ would live in the correct universes,
as would $\N*$ and $\W*$.
Although this would affect the judgements of both \lang and \CICE,
this wouldn't affect the proofs, since the changes are the same for both languages.
However, this would mean giving up $\const{ac}$
and likely any hope of defining any infinitary constructs like $\const{\omega}$
when using this model.

\section{Streams and Coinductives}

Dual to inductive types, coinductive types allow for constructing
potentially infinitely large elements in a principled manner,
and are present in many proof assistants such as Coq, Agda, and Idris.
Conventionally, while fixpoints are guarded by destructors\index{guardedness}
and destruct elements of inductives by recurring only on syntactically smaller elements,
cofixpoints are guarded by constructors and occur only as a syntactic argument to a constructor~\citep{guard}.
Since coinductives aren't present in pCIC\index{Calculus of Inductive Constructions!Predicative \textasciitilde},
upon which \CICE is based, I don't include them in \lang either,
as they would lack a translation in the syntactic model.
Nevertheless, we can speculate on how they might appear and interact with sizes.
Here, I define the classic example of coinductive streams,
composed of a head element and a tail stream.
\begin{mathpar}
\inferrule[\rlabel*{stream}]{
  \wf{\Phi}{s} \\
  \infer{\Phi; \Gamma}{\tau}{U}
}{
  \infer{\Phi; \Gamma}{\Stream{\tau}{s}}{U}
}
\and
\inferrule[\rlabel*{hd}]{
  \infer{\Phi; \Gamma}{e}{\Stream{\tau}{s}}
}{
  \infer{\Phi; \Gamma}{\shd{e}}{\tau}
}
\and
\inferrule[\rlabel*{tl}]{
  \infer{\Phi; \Gamma}{e}{\Stream{\tau}{s}}
}{
  \infer{\Phi; \Gamma}{\stl{e}}{\Funtype<{\alpha}{s}{\Stream{\tau}{\alpha}}}
}
\and
\inferrule[\rlabel*{scons}]{
  \infer{\Phi; \Gamma}{e_1}{\tau} \\
  \infer{\Phi, \bound{\alpha}{s}; \Gamma}{e_2}{\Stream{\tau}{\alpha}}
}{
  \infer{\Phi; \Gamma}{\scons{\alpha}{s}{e_1}{\alpha}{s}{e_2}}{\Stream{\tau}{s}}
}
\and
\inferrule[\rlabel*{cofix}]{
  \infer{\Phi, \alpha; \Gamma}{\sigma}{U} \\
  \fresh{\beta} \\
  \check{\Phi, \alpha; \Gamma, \annot{f}{\Funtype<{\beta}{\alpha}{\subst{\sigma}{\alpha}{\beta}}}}{e}{\sigma}
}{
  \infer{\Phi; \Gamma}{\cofix{f}{\alpha}{\sigma}{e}}{\Funtype{\alpha}{\sigma}}
}
\end{mathpar}

Just as the size of an inductive informally indicates at most how many layers of constructors its elements contain,
the size of a coinductive informally indicates \emph{at least} how many layers,
and in the case of $\app{\Stream*}{A}$, at least how many elements of $A$ it contains.
Of course, streams may contain an infinite number of elements,
so the analogy works better in the relative sense:
if a stream contains at least $s$ elements,
then its tail must contain at least $\bound{\alpha}{s}$, strictly fewer, elements.
The reduction rules of streams, omitted here, operate as expected:
$\shd{}$ projects out the head and $\stl{}$ projects out the tail.

To actually construct infinite streams, we need cofixpoint expressions.
The natural analogue of guardedness\index{guardedness} in sized types
is that cofixpoint bodies must construct coinductives larger than
their own occurrences inside the body,
which coincides with the structure of \rref{fix}.
Cofixpoints also reduce when applied to a size that has some subsize,
just like fixpoints.
Interestingly, this suggests that cofixpoints might also be translated in the syntactic model
to well-founded induction\index{well-founded induction} on sizes despite constructing a coinductive type.

With cofixpoints, we could then construct an infinite stream of some single element.
Let $A$ in the definitions to follow be some type.
\begin{align*}
& \Let{\dup}{\arr*{A}{\Funtype{\alpha}{\Stream{A}{\alpha}}}}{\\
& \quad \fun{a}{A}{\cofix{\dup*}{\alpha}{\Stream{A}{\alpha}}{\\
& \qquad \scons{\beta}{\alpha}{a}{\beta}{\alpha}{\App{\dup*}{\beta}}}}}
\end{align*}

Notice that the stream produced has \emph{any} size,
which can be interpreted as having ``at least'' any number of elements---a
truly infinite stream.
More generally, a coinductive universally quantified by size corresponds to a full coinductive,
dual to an inductive existentially quantified by a size corresponding to a full inductive.

We could manipulate sized streams as expected,
such as taking only the odd elements of a stream
or interleaving two streams by alternation.
\begin{align*}
& \Let{\odds}{\Funtype{\alpha}{\arr*{(\Funtype{\beta}{\Stream{A}{\beta}})}{\Stream{A}{\alpha}}}}{ \\
& \quad \cofix{\odds*}{\alpha}{\arr*{(\Funtype{\beta}{\Stream{A}{\beta}})}{\Stream{A}{\alpha}}}{ \\
& \qquad \fun{s}{\Funtype{\beta}{\Stream{A}{\beta}}}{\scons{\beta}{\alpha}{\shd{(\App{s}{\alpha})}}{\beta}{\alpha}{\app{\App{\odds*}{\beta}}{(\Fun{\gamma}{\App{\stl{(\App{\stl{(\App{s}{\sss{\sss{\gamma}}})}}{\sss{\gamma}})}}{\gamma}})}}}}} \\
& \Let{\interleave}{\Funtype{\alpha}{\arr*{\Stream{A}{\alpha}}{\Stream{A}{\alpha}}{\Stream{A}{\alpha}}}}{ \\
& \quad \cofix{\interleave*}{\alpha}{\arr*{\Stream{A}{\alpha}}{\Stream{A}{\alpha}}{\Stream{A}{\alpha}}}{ \\
& \qquad \fun{s_1}{\Stream{A}{\alpha}}{\fun{s_2}{\Stream{A}{\alpha}}{\scons{\beta}{\alpha}{\shd{s_1}}{\beta}{\alpha}{ \\
& \qquad \quad \scons{\gamma}{\beta}{\shd{s_2}}{\gamma}{\beta}{\app{\App{\interleave*}{\alpha}}{(\App{\stl{s_1}}{\gamma})}{(\App{\stl{s_2}}{\gamma})}}}}}}}
\end{align*}

Just as with lists, we could write size-preserving functions on streams
and use them to define cofixpoints that would otherwise not pass syntactic guard checking.
One common example from Haskell is defining a stream of Fibonacci numbers using,
in our case, a size-preserving stream zipping function.
Suppose we have a function $\annot{f}{\arr*{A}{A}{A}}$ and some element $\annot{a}{A}$.
Then we can define the zipping function and a Fibonacci-like stream as follows.
(For the actual Fibonacci numbers, $A$ would be the naturals, $f$ would be addition, and $a$ would be 1.)

\begin{align*}
& \Let{\zip}{\Funtype{\alpha}{\arr*{\Stream{A}{\alpha}}{\Stream{A}{\alpha}}{\Stream{A}{\alpha}}}}{ \\
& \quad \cofix{\zip*}{\alpha}{\arr*{\Stream{A}{\alpha}}{\Stream{A}{\alpha}}{\Stream{A}{\alpha}}}{ \\
& \qquad \fun{s_1}{\Stream{A}{\alpha}}{\fun{s_2}{\Stream{A}{\alpha}}{ \\
& \qquad \quad \scons{\beta}{\alpha}{\app{f}{\shd{s_1}}{\shd{s_2}}}{\beta}{\alpha}{\app{\App{\zip*}{\beta}}{(\App{\stl{s_1}}{\beta})}{(\App{\stl{s_2}}{\beta})}}}}}} \\
&\Let{\fibs}{\Funtype{\alpha}{\Stream{A}{\alpha}}}{ \\
& \quad \cofix{\fibs*}{\alpha}{\Stream{A}{\alpha}}{ \\
& \qquad \scons{\beta}{\alpha}{a}{\beta}{\alpha}{\scons{\gamma}{\beta}{a}{\gamma}{\beta}{\app{\App{\zip}{\gamma}}{(\App{\fibs*}{\gamma})}{(\App{\stl{(\App{\fibs*}{\beta})}}{\gamma})}}}}}
\end{align*}

Coinductive types in general can be implemented as record definitions
just as inductive types can be implemented as data definitions.
However, sized coinductives in Agda's standard library are implemented
in terms of a single coinductive $\Thunk{F}{s}$ record%
\footnote{See \textsf{Codata.Sized.Thunk} from the Agda standard library, URL \url{https://agda.github.io/agda-stdlib/Codata.Sized.Thunk.html}.}
with sized parameter $F$ and size $s$,
and all other coinductives are implemented as nested inductives,
where occurrences of the type in its constructors' types are wrapped around a $\Thunk*$.
In essence, the $\Thunk*$ delays evaluation of corecursive occurrences until they're $\force*$d to.
\begin{mathpar}
\inferrule[\rlabel*{thunk}]{
  \wf{\Phi}{s} \\
  \infer{\Phi; \Gamma}{F}{\Funtype{\alpha}{U}}
}{
  \infer{\Phi; \Gamma}{\Thunk{F}{s}}{U}
}
\and
\inferrule[\rlabel*{force}]{
  \infer{\Phi; \Gamma}{e}{\Thunk{F}{s}}
}{
  \infer{\Phi; \Gamma}{\force{e}}{\Funtype<{\alpha}{s}{\App{F}{\alpha}}}
}
\and
\inferrule[\rlabel*{think}]{
  \infer{\Phi, \bound{\alpha}{s}; \Gamma}{e}{\App{F}{\alpha}}
}{
  \infer{\Phi; \Gamma}{\thunk{\alpha}{s}{e}}{\Thunk{F}{s}}
}
\end{mathpar}

This alternate technique is equally expressive,
since it allows encoding M types, the dual of W types, not just streams.
\begin{align*}
& \data{\Stream{(\annot{A}{\Type{i}})}{\alpha}}{\Type{i+1}} \\
& \quad \annot{\scons*}{\arr*{A}{\Thunk{(\app{\Stream*}{A})}{\alpha}}{\Stream{A}{\alpha}}} \\
& \data{\App{\app{\M*}{(\annot{A}{\Type{i}})}{(\annot{B}{\arr*{A}{\Type{i}}})}}{\alpha}}{\Type{i+1}} \\
& \quad \annot{\inf*}{\arr{a}{A}{\arr*{(\arr*{\app{B}{a}}{\Thunk{(\app{\M*}{A}{B})}{\alpha}})}{\App{\app{\M*}{A}{B}}{\alpha}}}}
\end{align*}

\section{Designing for Syntactic Modelling} \label{sec:design}

Choosing to use a syntactic model imposes some constraints on the design
of the source and target languages, as well as on the proof architecture.
Most notably, proofs of correctness of syntactic models proceed
by induction on the source derivations,
so \lang uses untyped conversion to avoid a dependency between typing and definitional equality,
which would otherwise require a large and complex mutually inductive proof of type preservation.

\iffalse
Strictly speaking, such a proof would still be possible,
since the compositionality lemmas are merely up to syntactic equality.
If they were instead up to typed equivalence
(as is the case for the translations in \citet{wjb}),
then the base cases would fail:
to show
$\defeq{\compile{\Phi}, \compile{\Gamma}}{\yT}{\yT}{\compile{\tau}}$
via \rref{equiv-refl} for instance,
$\type{\compile{\Phi}, \compile{\Gamma}}{\yT}{\compile{\tau}}$
would be required,
but this can't be constructed even from
$\type{\Phi; \Gamma}{y}{\tau}$
as it's not a premise and therefore the mutual type preservation induction hypothesis can't be applied.
Still, I use untyped conversion to avoid any other potential complications with the mutual induction.
\fi

On the other hand, induction isn't done on the derivations of the target,
but since equality reflection is needed for translation-specific reasons,
\CICE uses typed equivalence instead to avoid the inconsistency and loss of subject equivalence
arising from reflecting propositional equality into transitive, untyped conversion,
as detailed in \cref{sec:target}.

Due to the incongruity between untyped conversion in the source and typed equivalence in the target,
the na\"ive proof attempts at preservation using only the relevant source derivation
and well-typedness of the source terms would be missing crucial typing information
to construct the target derivations, especially for equivalence.
This is why, in all of the preservation lemmas, well-typedness of the translated terms
is also required as premises, essentially replacing what would have been mutual or circular appeals
to type preservation with just the right amount of typing information.

Quite unexpectedly, requiring well-typedness of translated terms
also has an effect on whether conversion in the source is declarative or algorithmic.
If conversion were declarative like \CICE's equivalence, with no reduction rules or their closures,
then transitivity of subtyping would need to be an explicit rule rather than a theorem.
%
\begin{mathpar}
\inferrule{
  \subtype{\Phi; \Gamma}{\tau_1}{\tau_2} \\
  \subtype{\Phi; \Gamma}{\tau_2}{\tau_3}
}{
  \subtype{\Phi; \Gamma}{\tau_1}{\tau_3}
}
\end{mathpar}
%
Recall that preservation of subtyping states that if $\subtype{\Phi; \Gamma}{\tau_1}{\tau_3}$ holds
and $\tau_i, \compile{\tau_i}$ for $i = 1, 2$ are well typed,
then $\subtype{\compile{\Phi}, \compile{\Gamma}}{\compile{\tau_1}}{\compile{\tau_3}}$ holds.
To apply the induction hypotheses in the above transitivity case,
we would need well-typedness of all of $\compile{\tau_1}, \compile{\tau_2}, \compile{\tau_3}$;
for the first and last, they're provided as premises.
But where would we derive the well-typedness of $\compile{\tau_2}$?
There's no ``subject subtyping'' as there is for subject reduction, and indeed,
$\subtype{\mt; \mt}{\Type{1}}{\App{(\fix{f}{\alpha}{\N{\alpha}}{\Type{2}})}{\sss{\circ}}}$
holds while the right-hand side, which is convertible to $\Type{2}$, isn't even well typed.

So simply by deciding on a syntactic model into an extensional type theory,
definitional equality in the source must be untyped and algorithmic,
while in the target it must be typed,
lest we create headaches for ourselves with these complex yet avoidable issues.

These consequences would have further effects on the design of the source syntax,
particularly concerning type annotations, if we also took into account
the other required metatheoretical properties.
Proving subject reduction in the case of $\delta$-reduction requires that
definitions in the environment have type annotations;
otherwise, even if $\Gamma = (\define*{x}{e}), (\annot{x}{\tau})$ were well formed,
we wouldn't be able to guarantee that $x$ reduces to $e$ of the \emph{same} type $\tau$,
only that $e$ has \emph{some} type.
Type-annotated definitions in turn force locally-named expressions
in $\kw{let}$ expressions to have annotations as well.
In the congruence rule that reduces the body of a $\kw{let}$ expression,
the named expression needs to be added as a definition to the environment,
but since reduction is untyped, its type can't be derived:
it must be provided by the $\kw{let}$ expression itself.

Although not directly related to the type preservation proofs,
proving confluence requires that the notion of $\rhd^*$ be
exactly the reflexive, transitive closure of $\rhd$,
so the congruence rules need to be a part of $\rhd$ and not of $\rhd^*$.

Some of these constraints wouldn't be so serious:
$\kw{let}$ expressions without type annotations, for instance,
can be easily elaborated into a core calculus with annotations.
Other constraints would require more work to handle:
although a variant of \lang with typed equivalence would instinctively be equivalent
to \lang without, proving that nothing has been broken by the introduction of typed equivalence
requires yet another syntactic model, which would have the same mutual induction issues.
In other words, a syntactic model constrains us to a particular presentation of the type theory
and its judgement rules, and changing the presentation (\eg for implementation in a type checker)
requires either more proof work or a certain amount of trust in the correctness of the changes.

\section{Future Directions}

There is still a long way to go towards a sized dependent type system
that is expressive, consistent, and useable.
We've seen from \cref{sec:examples} that working without an infinite size
or full inductives is verbose at best, in the case of $\msort$ and $\qsort$,
and impossible at worse, in the case of $\const{\omega}$,
while attempts at including them either involve inconsistency, undecidability, or noncanonicity.
Alternate approaches need to be investigated to circumvent these problems.

While the issue with the universe levels of inductive types appears to be cosmetic
and not affect any important metatheoretical properties,
it might be a mere symptom of a larger issue with how sizes are modelled,
especially since prior sized type systems shown to be consistent
haven't required any similar restrictions.
One possibility is to take inspiration from Agda and
make $\SizeT$ live in a universe $\SizeUnivT$ independent of $\TypeT{}$,
but the consistency of adding $\SizeUnivT$ hasn't been established,
especially when it needs to be slightly impredicative:
$\arrT*{\TypeT{}}{\SizeT}$ would live in $\SizeUnivT$
(as would be the case for the type of $\limT$)
while $\funtypeT{\alphaT}{\SizeT}{\app{\NatT}{\alphaT}}$ would remain in $\TypeT{\tg{0}}$
(as would be the case for the type of $\zeroT$).

The main barrier to a syntactic model\index{syntactic model} of \lang with any coinductive types
is the absence of an established target type theory with coinductive types
with nice properties.
Whereas pCIC\index{Calculus of Inductive Constructions!Predicative \textasciitilde}
and pCuIC\index{Predicative Calculus of Cumulative Inductive Constructions} lack them,
MetaCoq\index{MetaCoq} mechanizes Coq-style coinductive types,
but necessarily cannot prove its own consistency.
Meanwhile, older CICs with coinductive types and cofixpoints guarded by constructors\index{guardedness}
(\eg \citet{guard}) have issues with subject reduction~\citep{coind-SR}\index{subject reduction},
which is a necessary property for proving type preservation\index{type preservation} in \cref{ch:proofs}.
Another potential issue is the translation of cofixpoints into applications of
a well-founded induction principle, which seems suspicious and also requires further investigation.

Finally, \lang is missing some important but comparatively simple features
from the perspective of the syntactic model, namely inductive types in general.
They can be added directly to \lang as a generalization of the existing inductives,
and I conjecture that the type preservation proofs would be a straightforward extension.
Alternatively, since there already are well-founded trees,
\lang could instead be augmented with dependent pairs and an equality type,
both of which I also conjecture to be straightforward;
then mutual inductives, indexed inductives~\citep{whynotW}, some nested inductives~\citep{barras},
and even some inductive--inductive types~\citep{ind-ind}
could be encoded, along with their induction principles, using W types and propositional equality~\citep{whynotW}.

\subsection{Metatheory}

\lang is missing proofs of two desirable metatheoretical properties:
strong normalization and decidability of type checking.
Because \nameref{lem:pres-red} only preserves reduction up to equivalence in \CICE,
not reduction, normalization doesn't hold immediately from the translation.
In particular, the proof of \nameref{lem:pres-red} uses equality reflection
to obtain an equivalence between two proofs of accessibility\index{accessibility predicate} of a size,
and \citet{SProp} show that this equivalence can lead to nonnormalization in
certain inconsistent environments,
as is the case when evaluating $\tg{false}$.

Even so, I conjecture that the translation of an \lang term would never lead to
a \CICE term with a subterm in an inconsistent environment that leads to nonnormalization,
because the translation will never yield an assumption of the form $\sT \szltT \sT$.
By inspection of the translation, the only fitting subsizing relation possible is of the form $\alphaT \szltT \alphaT$,
and the only possible sources $\Funtype<{\alpha}{\alpha}{\tau}$, $\Fun<{\alpha}{\alpha}{e}$ are ill-scoped.
Since fixpoints only reduce when applied to sizes that have a subsize
and we shouldn't be able to obtain infinitely decreasing chains of subsizes,
they wouldn't unfold endlessly and must stop eventually.

Decidability of type checking, meanwhile, rests not only on strong normalization,
but also on decidability of subsizing, \ie deciding whether $\bound{r}{s}$ holds.
The trickiest rule is transitivity of subsizing,
but I conjecture that subsizing, too, is decidable,
as the size environment only contains a finite number of bounded size variables,
and sizes can only have a finite number of successor operators applied.
Moreover, Agda has transitive subsizing, and so far hasn't had issues with its decidability.

\subsection{Implementation}

The consistency of \lang shows that sized types for inductives in Agda
would likely be consistent were the infinite size\index{infinite size} removed.
This would immediately pose a significant problem to Agda's standard library,
as 20 of the 33 files that use sized types in version 1.7.1 of the standard library use the infinite size.
Furthermore, 30 of those files deal with coinductives rather than inductives,
and it remains to be seen whether \lang with coinductives really is consistent.
The addition of the $\const{ac}$ axiom to recover some of the expressivity of the infinite size
wouldn't ideal for a proof assistant either, since it would block computation.

\section{Conclusion}

In this thesis, I introduced \lang, a sized dependent type theory
with higher-rank size quantification and bounded size quantification,
but without an infinite size that is strictly greater than itself,
along with sized naturals and well-founded trees.
I gave examples of programming with sized inductives to write recursive programs
that would have otherwise not passed the usual syntactic guard checks,
as well as limitations of programming without the infinite size.
I then reduced the consistency of \lang to that of \CICE,
an extensional dependent type theory, by translating \lang terms to \CICE terms
and proving that the translation is type preserving.
Finally, I discussed the tradeoffs in the design of \lang for this syntactic model to work,
namely the lack of an infinite size and raising the universe levels of the inductives,
as well as some potential extensions and future work.
%\include{conclusions}

%    3. Notes
%    4. Footnotes

%    5. Bibliography
\begin{singlespace}
\raggedright
\bibliographystyle{abbrvnat}
\bibliography{biblio}
\end{singlespace}

\appendix
%    6. Appendices (including copies of all required UBC Research
%       Ethics Board's Certificates of Approval)
%\include{reb-coa}	% pdfpages is useful here
\chapter{Mechanized \CICE Definitions}

\section{Coq}

\singlespacing
\begin{minted}{coq}
From Equations Require Import Equations.
Require Import Coq.Program.Equality.
Require Import Coq.Unicode.Utf8_core.

Reserved Notation "r ≤ s" (at level 70, no associativity).

Inductive Size : Type :=
| suc : Size → Size
| lim : ∀ {A : Type}, (A → Size) → Size.

Inductive Leq : Size → Size → Type :=
| mono : ∀ {r s}, r ≤ s → suc r ≤ suc s
| cocone : ∀ {s A f} (a : A), s ≤ f a → s ≤ lim f
| limiting : ∀ {s A f}, (∀ (a : A), f a ≤ s) → lim f ≤ s
where "r ≤ s" := (Leq r s).

Definition Lt (r s : Size) : Type := suc r ≤ s.
Notation "r < s" := (Lt r s).

Definition base : Size := lim (False_rect Size).

Definition baseLeq (s : Size) : base ≤ s :=
  limiting (λ a, (False_rect (_ ≤ s) a)).

Fixpoint reflLeq (s : Size) : s ≤ s :=
  match s with
  | suc s => mono (reflLeq s)
  | lim f => limiting (λ a, cocone a (reflLeq (f a)))
  end.

Property transLeq {r s t : Size} (rs : r ≤ s) (st : s ≤ t) : r ≤ t.
Admitted.

(*
Derive NoConfusion for Size.
Equations transLeq {r s t : Size} (rs : r ≤ s) (st : s ≤ t) : r ≤ t :=
  transLeq (mono rs) (mono st) := mono (transLeq rs st);
  transLeq rs (cocone a sfa) := cocone a (transLeq rs sfa);
  transLeq (limiting fas) st := limiting (λ a, transLeq (fas a) st);
  transLeq (cocone a rfa) (limiting fat) := transLeq rfa (fat a).
*)

Fixpoint sucLeq (s : Size) : s ≤ suc s :=
  match s with
  | suc s => mono (sucLeq s)
  | lim f => limiting (λ a, transLeq (sucLeq (f a)) (mono (cocone a (reflLeq (f a)))))
  end.

Inductive Acc (s : Size) : Prop :=
| acc : (∀ r, r < s → Acc r) → Acc s.

Axiom funext : ∀ {A} {B : A → Type} {p q : ∀ x, B x},
  (∀ x, p x = q x) → p = q.

Equations accIsProp {s} (acc1 acc2 : Acc s) : acc1 = acc2 :=
| acc _ p, acc _ q =>
  f_equal _ (funext (λ r, funext (λ rs, accIsProp (p r rs) (q r rs)))).

Lemma accLeq : ∀ r s, r ≤ s → Acc s → Acc r.
Proof.
  intros r s rs acc.
  induction acc as [s p IH].
  exact (acc r (λ t tr, p t (transLeq tr rs))).
Qed.

Theorem wf : ∀ s, Acc s.
Proof.
  intros s.
  induction s as [s IH | A f IH].
  - refine (acc (suc s) (λ r rsucs, accLeq r s _ IH)).
    inversion rsucs as [r' s' rs | |].
    exact rs.
  - refine (acc (lim f) (λ r rlimf, _)).
    inversion rlimf as [| r' A' f' a rfa eqr eqA |].
    dependent destruction H.
    destruct (IH a) as [p].
    exact (p r rfa).
Qed.

Fixpoint wfAcc (P : Size → Type) (IH : ∀ s, (∀ r, r < s → P r) → P s) (s : Size) (acc : Acc s) : P s :=
  let p r rs := wfAcc P IH r (match acc return Acc r with | acc _ p => p r rs end)
  in IH s p.

Definition wfInd (P : Size → Type) (IH : ∀ s, (∀ r, r < s → P r) → P s) (s : Size) : P s :=
  wfAcc P IH s (wf s).
\end{minted}
\textspacing

\section{Agda}

\singlespacing
\begin{minted}{agda}
{-# OPTIONS --without-K #-}

open import Agda.Primitive using (Level; lsuc)
open import Agda.Builtin.Equality
open import Data.Empty using (⊥; ⊥-elim)

variable
  ℓ ℓ′ : Level
  A C : Set ℓ
  B : A → Set ℓ

infix 30 ↑_
infix 30 ⊔_

data Size {ℓ} : Set (lsuc ℓ) where
  ↑_ : Size {ℓ} → Size
  ⊔_ : {A : Set ℓ} → (A → Size {ℓ}) → Size

data _≤_ {ℓ} : Size {ℓ} → Size {ℓ} → Set (lsuc ℓ) where
  ↑s≤↑s : ∀ {r s} → r ≤ s → ↑ r ≤ ↑ s
  s≤⊔f  : ∀ {s} f (a : A) → s ≤ f a → s ≤ ⊔ f
  ⊔f≤s  : ∀ {s} f → (∀ (a : A) → f a ≤ s) → ⊔ f ≤ s

◯ : Size
◯ = ⊔ ⊥-elim

◯≤s : ∀ {s} → ◯ ≤ s
◯≤s = ⊔f≤s ⊥-elim λ ()

s≤s : ∀ {s : Size {ℓ}} → s ≤ s
s≤s {s = ↑ s} = ↑s≤↑s s≤s
s≤s {s = ⊔ f} = ⊔f≤s f (λ a → s≤⊔f f a s≤s)

s≤s≤s : ∀ {r s t : Size {ℓ}} → r ≤ s → s ≤ t → r ≤ t
s≤s≤s (↑s≤↑s r≤s) (↑s≤↑s s≤t) = ↑s≤↑s (s≤s≤s r≤s s≤t)
s≤s≤s r≤s (s≤⊔f f a s≤fa) = s≤⊔f f a (s≤s≤s r≤s s≤fa)
s≤s≤s (⊔f≤s f fa≤s) s≤t = ⊔f≤s f (λ a → s≤s≤s (fa≤s a) s≤t)
s≤s≤s (s≤⊔f f a s≤fa) (⊔f≤s f fa≤t) = s≤s≤s s≤fa (fa≤t a)

s≤↑s : ∀ {s : Size {ℓ}} → s ≤ ↑ s
s≤↑s {s = ↑ s} = ↑s≤↑s s≤↑s
s≤↑s {s = ⊔ f} = ⊔f≤s f (λ a → s≤s≤s s≤↑s (↑s≤↑s (s≤⊔f f a s≤s)))

_<_ : Size {ℓ} → Size {ℓ} → Set (lsuc ℓ)
r < s = ↑ r ≤ s

record Acc (s : Size {ℓ}) : Set (lsuc ℓ) where
  inductive
  pattern
  constructor acc
  field
    acc< : (∀ r → r < s → Acc r)
open Acc

accIsProp : ∀ {s : Size {ℓ}} → (acc1 acc2 : Acc s) → acc1 ≡ acc2
accIsProp (acc p) (acc q) =
  cong acc (funext p q (λ r → funext (p r) (q r) (λ r<s → accIsProp (p r r<s) (q r r<s))))
  where
    cong : ∀ (f : A → C) {a b} → a ≡ b → f a ≡ f b
    cong f refl = refl
    postulate funext : ∀ (p q : ∀ x → B x) → (∀ x → p x ≡ q x) → p ≡ q

acc≤ : ∀ {r s : Size {ℓ}} → r ≤ s → Acc s → Acc r
acc≤ r≤s (acc p) = acc (λ t t<r → p t (s≤s≤s t<r r≤s))

wf : ∀ (s : Size {ℓ}) → Acc s
wf (↑ s) = acc (λ { _ (↑s≤↑s r≤s) → acc≤ r≤s (wf s) })
wf (⊔ f) = acc (λ { r (s≤⊔f f a r<fa) → (wf (f a)).acc< r r<fa })

wfInd : ∀ (P : Size {ℓ} → Set ℓ′) → (∀ s → (∀ r → r < s → P r) → P s) → ∀ s → P s
wfInd P f s = wfAcc s (wf s)
  where
  wfAcc : ∀ s → Acc s → P s
  wfAcc s (acc p) = f s (λ r r<s → wfAcc r (p r r<s))
\end{minted}
\textspacing

\backmatter
%    7. Index
% See the makeindex package: the following page provides a quick overview
% <http://www.image.ufl.edu/help/latex/latex_indexes.shtml>


\end{document}
