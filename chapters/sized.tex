\chapter{Sized Dependent Types} \label{ch:sized-dep-types}

\newcommand{\FigSyntax}[1]{
  \begin{figure}[h]
  \centering
  \begin{align*}
  i, j, k, m, n &\Coloneqq \meta{\textsf{naturals}} &
  \Gamma &\Coloneqq \mt \mid \Gamma, \annot{x}{\tau} \mid \Gamma, \define{x}{e} &
  r, s &\Coloneqq \alpha \mid \sss{s} \mid \circ \\
  f, g, x, y, z &\Coloneqq \meta{\textsf{term variables}} &
  \Delta &\Coloneqq \mt \mid \Delta, \annot{x}{\tau} &
  U &\Coloneqq \Prop \mid \Type{i} \\
  \alpha, \beta &\Coloneqq \meta{\textsf{size variables}} &
  \Phi &\Coloneqq \mt \mid \Phi, \alpha \mid \Phi, \bound{\alpha}{s} \\
  e, p, P, \tau, \sigma &\Coloneqq \mathrlap{x \mid U \mid \letin{x}{\tau}{e}{e} \mid \eq{e}{\tau}{e} \mid \refl{e} \mid \J{P}{d}{p}} \\
  &\mid \mathrlap{\funtype{x}{\tau}{\tau} \mid \fun{x}{\tau}{e} \mid \app{e}{e} \mid \Funtype{\alpha}{\tau} \mid \Funtype<{\alpha}{s}{\tau} \mid \Fun{\alpha}{e} \mid \Fun<{\alpha}{s}{e} \mid \App{e}{s}} \\
  \end{align*}
  \caption{Syntax (base \lang)}
  \label{#1}
  \end{figure}
}

\newcommand{\FigRed}[1]{
  \begin{figure}[h]
  \centering
  \begin{mathpar}
  \fbox{$\red{\Phi; \Gamma}{e}{e}$} \qquad
  \fbox{$\red*{\Phi; \Gamma}{e}{e}$} \hfill \\
  \inferrule[]{
    (\annot{x}{\tau}) \in \Gamma \\\\
    (\define{x}{e}) \in \Gamma
  }{
    \red{\Phi; \Gamma}{x}{e}
  }
  \and \inferrule[]{~}{\red{\Phi; \Gamma}{\app{(\fun{x}{\tau}{e})}{e'}}{\subst{e}{x}{e'}}}
  \and \inferrule[]{~}{\red{\Phi; \Gamma}{\letin{x}{\tau}{e'}{e}}{\subst{e}{x}{e'}}}
  \and \inferrule[]{~}{\red{\Phi; \Gamma}{\App{(\Fun{\alpha}{e})}{s}}{\subst{e}{\alpha}{s}}}
  \and \inferrule[]{~}{\red{\Phi; \Gamma}{\App{(\Fun<{\alpha}{r}{e})}{s}}{\subst{e}{\alpha}{s}}}
  \and \inferrule[]{~}{\red{\Phi; \Gamma}{\J{}{d}{\refl{}}}{d}}
  \\\\
  \inferrule[\rlabel{$\rhd^*$-refl}{red*-refl}]{~}{\red*{\Phi; \Gamma}{e}{e}}
  \and
  \inferrule[\rlabel{$\rhd^*$-trans}{red*-trans}]{
    \red{\Phi; \Gamma}{e_1}{}{e_2} \\
    \red*{\Phi; \Gamma}{e_2}{e_3}
  }{
    \red*{\Phi; \Gamma}{e_1}{e_3}
  }
  \and
  \inferrule[\rlabel{$\rhd^*$-cong}{red*-cong}]{
    \text{For every $1 \leq i \leq n$:} \and
    \red*{\Phi'; \Gamma'}{e_i}{e'_i}
  }{
    \red*{\Phi; \Gamma}{\subst{e}{x_1, \seq, x_n}{e_1, \seq, e_n}}{\subst{e}{x_1, \seq, x_n}{e'_1, \seq, e'_n}}
  }
  \end{mathpar}
  \caption{Reduction rules (base \lang)}
  \label{#1}
  \end{figure}
}

\newcommand{\FigSubtype}[1]{
  \begin{figure}
  \centering
  \begin{mathpar}
  \fbox{$\subtype{\Phi; \Gamma}{\tau}{\tau}$} \qquad
  \fbox{$\acum{e}{e}$} \hfill \\
  \inferrule[$\preccurlyeq$-red]{
    \red*{\Phi; \Gamma}{\tau_1}{\sigma_1} \\
    \red*{\Phi; \Gamma}{\tau_2}{\sigma_2} \\
    \acum{\sigma_1}{\sigma_2}
  }{
    \subtype{\Phi; \Gamma}{\tau_1}{\tau_2}
  }
  \and
  \inferrule[\rlabel{$\sqsubseteq$-prop}{acum-prop}]{~}{\acum{\Prop{}}{\Type{i}}}
  \and
  \inferrule[\rlabel{$\sqsubseteq$-type}{acum-type}]{i \leq j}{\acum{\Type{i}}{\Type{j}}}
  \and
  \inferrule[\rlabel{$\sqsubseteq$-refl}{acum-refl}]{
  }{
      \acum{e}{e}
  }
  \and
  \inferrule[\rlabel{$\sqsubseteq$-pi}{acum-pi}]{
    \acum{\tau_1}{\tau_2}
  }{
    \acum{\fun{x}{\sigma}{\tau_1}}{\fun{x}{\sigma}{\tau_2}}
  }
  \and
  \inferrule[\rlabel{$\sqsubseteq$-forall}{acum-forall}]{
    \acum{\tau_1}{\tau_2}
  }{
    \acum{\Funtype{\alpha}{\tau_1}}{\Funtype{\alpha}{\tau_2}}
  }
  \and
  \inferrule[\rlabel{$\sqsubseteq$-forall$<$}{acum-forall<}]{
    \acum{\tau_1}{\tau_2}
  }{
    \acum{\Funtype<{\alpha}{s}{\tau_1}}{\Funtype<{\alpha}{s}{\tau_2}}
  }
  \end{mathpar}
  \caption{Subtyping and $\alpha$-cumulativity rules}
  \label{#1}
  \end{figure}
}

\newcommand{\FigSubsize}[1]{
  \begin{figure}
  \centering
  \begin{mathpar}
  \fbox{$\wf{\Phi}{s}$} \qquad
  \fbox{$\subsize{\Phi}{s}{s}$} \hfill \\
  \inferrule[]{
    \wf{}{\Phi} \\
    \alpha \in \Phi
    \textit{ or }
    (\bound{\alpha}{s}) \in \Phi
  }{
    \wf{\Phi}{\alpha}
  }
  \and
  \inferrule[]{\wf{}{\Phi}}{\wf{\Phi}{\circ}}
  \and
  \inferrule[]{
    \wf{\Phi}{s}
  }{
    \wf{\Phi}{\sss{s}}
  }
  \and
  \inferrule[]{
    \wf{}{\Phi} \\
    (\bound{\alpha}{s}) \in \Phi
  }{
    \subsize{\Phi}{\sss{\alpha}}{s}
  }
  \\\\
  \inferrule[]{
    \wf{\Phi}{s}
  }{
    \subsize{\Phi}{\circ}{s}
  }
  \and
  \inferrule[]{
    \wf{\Phi}{s}
  }{
    \subsize{\Phi}{s}{s}
  }
  \and
  \inferrule[]{
    \wf{\Phi}{s}
  }{
    \subsize{\Phi}{s}{\sss{s}}
  }
  \and
  \inferrule[]{
    \subsize{\Phi}{r}{s}
  }{
    \subsize{\Phi}{\sss{r}}{\sss{s}}
  }
  \and
  \inferrule[]{
    \subsize{\Phi}{s_1}{s_2} \\\\
    \subsize{\Phi}{s_2}{s_3}
  }{
    \subsize{\Phi}{s_1}{s_3}
  }
  % \and \inferrule[]{~}{\subsize{\Gamma}{s}{\infty}}
  \end{mathpar}
  \caption{Size and subsizing rules}
  \label{#1}
  \end{figure}
}

\newcommand{\FigWF}[1]{
  \begin{figure}
  \centering
  \begin{mathpar}
  \fbox{$\wf{}{\Phi}$} \qquad
  \fbox{$\wf{\Phi}{\Gamma}$} \hfill \\
  \inferrule[\rlabel{nil}{nil}]{~}{\wf{}{\mt}}
  \and
  \inferrule[\rlabel{cons-size}{cons-size}]{
    \wf{}{\Phi}
  }{
    \wf{}{\Phi, \alpha}
  }
  \and
  \inferrule[\rlabel{cons-size$<$}{cons-size<}]{
    \wf{}{\Phi} \\\\
    \wf{\Phi}{s}
  }{
    \wf{}{\Phi, \bound{\alpha}{s}}
  }
  \and
  \inferrule[nil]{\wf{}{\Phi}}{\wf{\Phi}{\mt}}
  \and
  \inferrule[\rlabel{cons-ass}{cons-ass}]{
    \wf{\Phi}{\Gamma} \\\\
    \infer{\Phi; \Gamma}{\tau}{U}
  }{
    \wf{\Phi}{\Gamma, \annot{x}{\tau}}
  }
  \and
  \inferrule[\rlabel{cons-def}{cons-def}]{
    \wf{\Phi}{\Gamma} \\\\
    \infer{\Phi; \Gamma}{e}{\tau}
  }{
    \wf{\Phi}{\Gamma, \define{x}{e}}
  }
  \end{mathpar}
  \caption{Wellformedness rules}
  \label{#1}
  \end{figure}
}
\newcommand{\FigSyntaxInd}[1]{
  \begin{figure}[h]
  \centering
  \begin{align*}
  c &\Coloneqq \zero* \mid \succ* \mid \sup* &&&&& \\
  %t &\Coloneqq e \mid \mathopen{[} s \mathclose{]} &
  %\Theta &\Coloneqq \mt \mid \Theta, \annot{x}{\tau} \mid \Theta, \alpha \mid \Theta, \bound{\alpha}{s} & \\
  e &\Coloneqq \mathrlap{\cdots \mid \N{s} \mid \zero{s}{s} \mid \succ{s}{s}{e} \mid \W{x}{\tau}{\tau}{s} \mid \sup{x}{\tau}{\tau}{s}{s}{e}{e}} \\
  &\mid \mathrlap{\match{e}{\fun*{x}{P}}{(\app{\App{c}{\alpha}}{z_1 \seq z_m} \Rightarrow e) \seq} \mid \fix{}{f}{\alpha}{\tau}{e}}
  \end{align*}
  \caption{Syntax (naturals, well-founded trees, \\ and $\kw{case}$ and fixpoint expressions)}
  \label{#1}
  \end{figure}
}

\newcommand{\FigRedInd}[1]{
  \begin{figure}[h]
  \centering
  \begin{mathpar}
  \vspace{-2ex}
  \fbox{$\red{\Phi; \Gamma}{e}{e}$} \quad \cdots \hfill \\
  \inferrule[]{~}{
    \red[]{\Phi; \Gamma}{\match*{\App{\zero*_{\any}}{s}}{(\App{\zero*}{\alpha} \Rightarrow e_z) (\app{\App{\succ*}{\any}}{\any} \Rightarrow \any)}}{\subst{e_z}{\alpha}{s}}
  }
  \and
  \inferrule[]{~}{
    \red[]{\Phi; \Gamma}{\match*{\app{\App{\succ*_{\any}}{s}}{e}}{(\App{\zero*}{\any} \Rightarrow \any) (\app{\App{\succ*}{\alpha}}{z} \Rightarrow e_s)}}{\subst{e_s}{\alpha, z}{s, e}}
  }
  \and
  \inferrule[]{~}{
    \red[]{\Phi; \Gamma}{\match*{\app{\App{\sup*_{\any}}{s}}{e_1}{e_2}}{(\app{\App{\sup*}{\alpha}}{z_1}{z_2} \Rightarrow e)}}{\subst{e}{\alpha, z_1, z_2}{s, e_1, e_2}}
  }
  \and
  \inferrule[]{
    \subsize{\Phi}{\sss{r}}{s}
  }{
    \red[]{\Phi; \Gamma}{
      \App{(\fix{}{f}{\alpha}{\sigma}{e})}{s}
    }{
      \subst{e}{\alpha, f}{s, \Fun<{\beta}{s}{\App{(\fix{}{f}{\alpha}{\sigma}{e})}{\beta}}}
    }
  }
  \end{mathpar}
  \caption{Reduction rules ($\kw{case}$ and fixpoint expressions)}
  \label{#1}
  \end{figure}
}

\newcommand{\FigTypingInd}[1]{
  \begin{figure}[h]
  \begin{mathpar}
  \fbox{$\type{\Phi; \Gamma}{e}{\tau}$} \quad \cdots \hfill \\
  \inferrule[\rlabel*{nat}]{
    \wf{\Phi}{\Gamma} \\
    \wf{\Phi}{s}
  }{
    \infer{\Phi; \Gamma}{\N{s}}{\Type{1}}
  }
  \and
  \inferrule[\rlabel*{zero}]{
    \wf{\Phi}{\Gamma} \\
    \subsize{\Phi}{\hat{r}}{s}
  }{
    \infer{\Phi; \Gamma}{\zero{s}{r}}{\N{s}}
  }
  \and
  \inferrule[\rlabel*{succ}]{
    \subsize{\Phi}{\hat{r}}{s} \\
    \check{\Phi; \Gamma}{e}{\N{r}}
  }{
    \infer{\Phi; \Gamma}{\succ{s}{r}{e}}{\N{s}}
  }
  \and
  \inferrule[\rlabel*{wft}]{
    \wf{\Phi}{s} \\
    \infer{\Phi; \Gamma}{\sigma}{U} \\\\
    \infer{\Phi; \Gamma, \annot{x}{\sigma}}{\tau}{U}
  }{
    \infer{\Phi; \Gamma}{\W{x}{\sigma}{\tau}{s}}{\axioms{U}}
  }
  \and
  \inferrule[\rlabel*{sup}]{
    \subsize{\Phi}{\hat{r}}{s} \\
    \check{\Phi; \Gamma}{e_1}{\sigma} \\\\
    \check{\Phi; \Gamma}{e_2}{\arr*{\subst{\tau}{x}{e_1}}{\W{x}{\sigma}{\tau}{r}}}
  }{
    \infer{\Phi; \Gamma}{\sup{x}{\sigma}{\tau}{s}{r}{e_1}{e_2}}{\W{x}{\sigma}{\tau}{s}}
  }
  \and
  \inferrule[\rlabel*{case-nat}]{
    \infer{\Phi; \Gamma}{e}{\N{s}} \\
    z \notin \FV{P} \\
    \infer{\Phi; \Gamma, \annot{x}{\N{s}}}{P}{U} \\\\
    \check{\Phi, \bound{\alpha}{s}; \Gamma}{e_z}{\subst{P}{x}{\zero{s}{\alpha}}} \\
    \check{\Phi, \bound{\beta}{s}; \Gamma, \annot{z}{\N{\beta}}}{e_s}{\subst{P}{x}{\succ{s}{\beta}{z}}}
  }{
    \infer{\Phi; \Gamma}{\match{e}{\fun*{x}{P}}{(\App{\zero*}{\alpha} \Rightarrow e_z)(\app{\App{\succ*}{\beta}}{z} \Rightarrow e_s)}}{\subst{P}{x}{e}}
  }
  \and
  \inferrule[\rlabel*{case-sup}]{
    \infer{\Phi; \Gamma}{e}{\W{y}{\sigma}{\tau}{s}} \\
    z_1, z_2 \notin \FV{P} \\
    \infer{\Phi; \Gamma, \annot{x}{\W{y}{\sigma}{\tau}{s}}}{P}{U} \\
    \check{\Phi, \bound{\alpha}{s}; \Gamma, \annot{z_1}{\sigma}, \annot{z_2}{\arr*{\subst{\tau}{y}{z_1}}{\W{y}{\tau}{\sigma}{\alpha}}}}{e_s}{\subst{P}{x}{\sup{y}{\sigma}{\tau}{s}{\alpha}{z_1}{z_2}}}
  }{
    \infer{\Phi; \Gamma}{\match{e}{\fun*{x}{P}}{(\app{\App{\sup*}{\alpha}}{z_1}{z_2} \Rightarrow e_s)}}{\subst{P}{x}{e}}
  }
  \and
  \inferrule[\rlabel*{fix}]{
    \infer{\Phi, \alpha; \Gamma}{\sigma}{U} \\
    \fresh{\beta} \\
    \check{\Phi, \alpha; \Gamma, \annot{f}{\Funtype<{\beta}{\alpha}{\subst{\sigma}{\alpha}{\beta}}}}{e}{\sigma}
  }{
    \infer{\Phi; \Gamma}{\fix{}{f}{\alpha}{\sigma}{e}}{\Funtype{\alpha}{\sigma}}
  }
  \end{mathpar}
  \caption{Typing rules (naturals, well-founded trees, \\ and $\kw{case}$ and fixpoint expressions)}
  \label{#1}
  \end{figure}
}

This chapter is divided into two halves:
the first gives the formal description of the syntax, judgements, and rules of \lang,
while the second provides a number of examples written in \lang to show how sized types can be used.
They are kept separate so that the examples don't detract from the formal description
and vice versa, but can be read in parallel;
the examples provide intuition for understanding the rules,
while the rules ensure proper understanding of why the examples are correct.

In the sections to follow,
I use ellipses \new{$\seq$} as metanotation for denoting a repeated sequence of some syntactic construct,
overlines \new{$\vec{\phantom{I}}$} for sequences of variables or terms specifically (\eg $\vec{z}$, $\vec{p}$),
and \new{$\mt$} for denoting an empty sequence (particularly in the context of environments).
Irrelevant constructs are omitted using an underscore \new{$\any$}.
Metafunctions are introduced in the prose as needed.

\section{Base \lang}

Although sized types are quite pointless without any inductive types to be sized,
I present in this subsection the sublanguage of \lang without naturals or well-founded trees
to not only get the preliminary details out of the way first,
but also to show that the sublanguage is independent of the chosen inductive types.

\FigSyntax{fig:syntax}
\cref{fig:syntax} gives its syntax, consisting of universes $U$,
sizes $s$, and terms $e$, which include term functions and size abstractions.
Note that the hierarchy of universes above $\Prop$ start at $\Type{1}$, not $\Type{0}$.
Sizes consist of size variables (distinct from term variables),
a base size $\circ$, and successors of sizes $\sss{s}$.

Most judgements use two environments: a term environment $\Gamma$ with assumptions $\annot{x}{\tau}$
and type-annotated definitions $\define{x}{\tau}{e}$,
and a size environment $\Phi$ with unbounded and bounded size variables.
I also use the assumption environment%
\footnote{These are conventionally called \emph{telescopes}\index{telescopes} due to \citet{telescope}.}
$\Delta$ as a shorthand when writing nested expressions with assumptions;
in particular, letting for instance $\Delta_{xy} = \annot{x}{\sigma_1}, \annot{y}{\sigma_2}$,
I use \new{$\arr*{\Delta_{xy}}{\tau}$} to mean $\funtype{x}{\sigma_1}{\funtype{y}{\sigma_2}{\tau}}$.
Similarly, I use \new{$\arr*{\sigma}{\tau}$} to mean the nondependent function type $\funtype{\any}{\sigma}{\tau}$
As a loose convention, I use $\tau, \sigma$ for type-like terms,
$P$ for the motive of eliminators,
$z$ for variables representing constructor arguments, and
$f, g$ for variables representing functions.

To avoid tedious complications with variables and renaming,
I assume that shadowing is disallowed
and that variable names coincide where needed instead of explicitly stating $\alpha$-equivalence.
This lets me write fewer variable substitutions and checks for variable freshness.

As mentioned in \cref{ch:introduction}, the judgement forms of \lang include
reduction, its various closures, $\alpha$-cumulativity, subtyping, and typing.
On top of those, there are also judgement forms for subsizing, sizes,
and well-formedness of size and term environments.

\FigRed{fig:reduction}
The reduction rules and their reflexive, transitive closure are described in \cref{fig:reduction}.
\new{$\subst{e}{x}{e'}$} denotes capture-avoiding substitution of $x$ for $e'$ within $e$,
and correspondingly \new{$\subst{e}{x_1, \seq, x_n}{e_1, \seq, e_n}$} denotes simultaneous substitution.
For every syntactic form of a composite term there is a corresponding congruence rule,
which is summarized by \rref{red-cong} using substitution;
the full set of rules can be found in \cref{app:cong:red}.
By convention, the reduction rules for function applications are also referred to as $\beta$-reduction,
for $\kw{let}$ expressions as $\zeta$-reduction,
and for defined variables as $\delta$-reduction.

These rules can be thought of as a description of nondeterministic evaluation of open terms,
``running'' from the left term to the right.
Separating them into reduction rules that only step once somewhere in the term
and their reflexive, transitive closure,
as opposed to the convention used by \eg \citet{wjb}
where $\rhd^*$ corresponds to the reflexive, transitive, \emph{compatible} closure,
allows for the usual proof techniques to prove confluence.

\FigSubtype{fig:subtyping}
Rather than a single subtyping\index{subtyping} judgement like in
\GCC,\index{Calculus of Constructions!Generalized \textasciitilde}
I use the same presentation as MetaCoq\index{MetaCoq}
and split it into a subtyping judgement
and a separate $\alpha$-cumulativity\index{$\alpha$-cumulativity} judgement,
listed in \cref{fig:subtyping}.
This overcomes some technical proof complications specific to the syntactic model
that appear in the single-judgement presentation due to the transitivity rule.
Aside from the expected cumulativity of universes,
the function type and size quantification are covariant in the codomain
(\ie are $\alpha$-cumulative when their codomains are),
while remaining invariant in the domain\punctstack{.}%
\footnote{The domain could be made contravariant instead if function type subtyping
in \CICE were similarly contravariant,
but to my knowledge, there is no such variant of CIC with untyped conversion\index{conversion}
and without $\eta$-conversion that has been proven consistent.}
All other types are invariant, as reflected by \rref{acum-refl}.
A term is then a subtype of another if they are confluent up to $\alpha$-cumulativity.
% Conversion can be defined as the confluence of two terms,
% but conversion isn't needed for any judgements so I exclude the definition.

Notably, \lang does \emph{not} have a notion of $\eta$-conversion\index{$\eta$-conversion}
in either the reduction rules or in the subtyping rules,
which would otherwise allow conversion between $\fun{x}{\tau}{\app{f}{x}}$ and $f$.
Mixing $\eta$-conversion and untyped conversion is notoriously difficult~\citep{eta},
and remains to date an unresolved problem in MetaCoq\index{MetaCoq}, so I exclude it here.

\FigSubsize{fig:subsizing}
\cref{fig:subsizing} describes a preorder on sizes such that
the successor operator is monotonic with respect to the order.
The base size $\circ$ is smaller than all sizes,
and the strict preorder $\bound{\alpha}{s}$ arising from bounded quantification or abstraction
is defined as $\sss{\alpha} \mathrel{\leqslant} s$.
An additional size judgement ensures well-scopedness of sizes.
The size environment must be well formed as well;
its rules are listed in \cref{fig:wf},
along with those for well-formedness of term environments.
\FigWF{fig:wf}

\FigRulesAxioms{fig:rules-axioms}
Finally, the typing rules for the base \lang are given in \cref{fig:typing}.
They use the metafunctions $\axioms{\mt}$ for the type of a universe and
$\rules{\mt}{\mt}$ for the universe of a function type,
defined in \cref{fig:rules-axioms}.

\FigTyping[h]{fig:typing}
\rref{var, univ, let} are the usual rules for variables, universes, and $\kw{let}$ expressions
in \GCC,\index{Calculus of Constructions!Generalized \textasciitilde}
while \rref{pi, lam, app} are the usual ones for functions.
The $\Prop$ universe is impredicative\index{impredicativity} because by the first case of $\rules*$,
a function type quantifying over any type is itself in $\Prop$ as long as its codomain is as well,
allowing a restricted form of circularity.
For example, a function $\id \mathrel{\coloneqq} \fun{\tau}{\Prop}{\fun{x}{\tau}{x}}$
can be assigned the type $\Id \mathrel{\coloneqq} \funtype{\tau}{\Prop}{\arr*{\tau}{\tau}}$ in $\Prop$
and applied to its own type and itself to yield $\app{\id}{\Id}{\id}$
of type $\Id$.

\rref{conv} uses the subtyping judgement and essentially allows casting a term
from one type to a supertype as needed.
If we were working in a bidirectional presentation,
where the typing judgement is replaced by either
\emph{type checking}, which checks an input term against an input type,
or \emph{type synthesis}, which synthesizes an output type for an input term,
this rule for $\type{\Phi; \Gamma}{e}{\tau}$ would be the sole checking rule,
first synthesizing types $\sigma$ and $U$ for $e$ and $\tau$ respectively,
then checking $\sigma$ against $U$,
and lastly asserting that $\sigma$ is indeed a subtype of $\tau$.
This is why, despite $\type{\Phi; \Gamma}{\sigma}{U}$ being derivable,
I choose to retain it as a premise.

\rref{forall, forall<, slam, slam<, sapp, sapp<} are the new rules relevant to sized types,
describing bound and unbound size quantification, abstraction, and application,
which work similarly to functions.
Of note is the bounded size application rule,
which only allows applications to smaller sizes following the subsizing judgement.

\iffalse
Lastly are \rref{eq, refl, J} for propositional equality.
The constructor $\refl{e}$ is a reflexive proof of $\eq{e}{\tau}{e}$,
that $e$ of type $\tau$ is equal to itself.
Given some equality proof $p$ of $\eq{e_1}{\tau}{e_2}$
and a motive\index{motive} $P$ taking some $y$ of type $\tau$ and a proof that $\eq{e_1}{\tau}{y}$,
the $\J*$ eliminator is a proof of $\app{P}{e_2}{p}$ when provided a proof of $\app{P}{e_1}{\refl{e_1}}$.
Other usual functions on proofs of equality can be derived from it,
such as coercion (when the motive is a constant function on types in its first argument)
or substitution (when the motive ignores the second argument).
$\J*$ is only well typed when fully applied;
it can be manually uncurried for a specific universe $U$ as the function
\marginnote{The type annotation for the equality type and argument to $\refl{}$ may be omitted when evident from context.}
$$\fun{\tau}{U}{\fun{e_1}{\tau}{\fun{e_2}{\tau}{\fun{P}{(\funtype{y}{\tau}{\funtype{z}{\eq{e_1}{}{y}}{U}})}{\fun{d}{\app{P}{e_1}{\refl{}}}{\fun{p}{\eq{e_1}{}{e_2}}{\J{P}{d}{p}}}}}}},$$
and similarly for $\refl{}$.
\fi

\section{Inductive Types: Naturals and Well-Founded Trees}\label{sec:ind-types}

\vspace{-\baselineskip}
\FigSyntaxInd{fig:syntax-ind}
\cref{fig:syntax-ind} extends the grammar with sized naturals, sized well-founded trees,
$\kw{case}$ expressions, and fixpoint expressions.
Informally, borrowing syntax from the definition of general inductives,
sized naturals and well-founded trees can be thought of as being defined by the following:
%
\begin{align*}
&\data{\N{\alpha}}{\Type{1}} && \data{\App{\app{\W*}{(\annot{A}{\Type{i}})}{(\annot{B}{\arr*{A}{\Type{i}}})}}{\alpha}}{\Type{i+1}} \\
&\quad \annot{\zero*}{\Funtype<{\beta}{\alpha}{\N{\alpha}}} && \quad \annot{\sup*}{\Funtype<{\beta}{\alpha}{\arr{x}{A}{\arr*{(\arr*{\app{B}{x}}{\app{\App{\W*}{\beta}}{A}{B}})}{\App{\app{\W*}{A}{B}}{\alpha}}}}} \\
&\quad \annot{\succ*}{\Funtype<{\beta}{\alpha}{\arr*{\N{\beta}}{\N{\alpha}}}}
\end{align*}

The types of naturals and well-founded trees can then be considered to be (nonuniformly) parametrized by a size,
and constructing an element of that type requires providing a strictly smaller size,
which is the size%
\footnote{The ``size of'' some construction is more precisely the size by which its \emph{inductive type} is parametrized.}
of the constructor's recursive arguments.
Constructors therefore always construct elements whose sizes are larger than their arguments'.
Since constructor argument sizes are bounded by above,
the argument can have \emph{any} smaller size,
and the size of an inductive represents the size its elements can \emph{at most} have;
consequently, an element of a sized inductive can always be lifted to an inductive with a larger size,
but never lowered to a smaller one.
This is demonstrated concretely in \cref{subsec:concrete}.

In \lang, the constructors are annotated with their types
since the parameter-like sizes cannot otherwise be synthesized.
For instance, $\succ{s}{r}{e}$ constructs a natural of size $s$,
while taking as argument a natural of size $r < s$.
Although $\zero*$ has no recursive arguments,
it still takes a smaller size argument solely for uniformity with $\succ*$.
Additionally, W types have explicit binders for the first type parameter for convenience,
much like dependent function types:
the variable $x$ is bound within $\tau$ in the type $\W{x}{\sigma}{\tau}{s}$.

The dependent $\kw{case}$ expression, or the destructor for an inductive element,
consists of three parts:
\begin{itemize}
  \item The target being destructed, either a natural or a well-founded tree in \lang;
  \item A motive with a free size variable for what's being destructed,
    representing the return type of the entire expression when substituted by the target
    and the types of each of the branches when substituted by the constructors; and
  \item A branch for each constructor of the inductive,
    with a free size variable for the size argument
    and free term variables for the other constructor arguments.
\end{itemize}

The purpose of the motive will become clearer shortly when looking at how the expression is typed;
the whole expression otherwise behaves like an ordinary nondependent $\kw{case}$ expression on different constructors.
The reduction rules of $\kw{case}$ expressions for each constructor are given in \cref{fig:reduction-ind},
along with the reduction rule for fixpoint expressions.
By convention, the reduction rules for $\kw{case}$ expressions
are also referred to as $\iota$-reduction,
and for fixpoint expressions as $\mu$-reduction.
\FigRedInd{fig:reduction-ind}

Fixpoints reduce when applied to some size $s$ by substitution of itself into its own body.
Fixpoints' bodies are well typed when recursive applications occur only on smaller sizes,
so the substitution wraps itself in a bound size abstraction.
Most importantly, they reduce only when there exists some size strictly smaller than $s$;
intuitively, this restriction prevents fixpoints from reducing indefinitely
because subsizing is well founded, and there cannot be an infinite chain of smaller sizes.

This reduction strategy supersedes the usual restriction that fixpoints only reduce
when applied to a constructor,
since all sized constructors carry a smaller size argument
that will satisfy the subsizing premise.
Furthermore, reduction can also occur when the fixpoint is applied to a successor size
by reflexivity of subsizing, \ie $\App{(\fix{f}{\alpha}{\tau}{e})}{\sss{s}}$ will reduce to
$\subst{e}{\alpha, f}{\sss{s}, \Fun<{\beta}{s}{\App{(\fix{f}{\alpha}{\tau}{e})}{\beta}}}$
since $s < \sss{s}$.

\FigTypingInd{fig:typing-ind}

The typing rules for all new constructs are given in \cref{fig:typing-ind},
using two additional metafunctions: $\fresh{\seq}$ asserts the freshness of the given variables,
and $\FV{\mt}$ produces the free variables in the given term.
The types of naturals and well-founded trees are well typed
when the sizes they are applied to are well formed, and
their constructors are well typed when applied to smaller sizes.

$\kw{case}$ expressions match on these size arguments in addition to the usual term arguments,
asserting within their branches that they are strictly smaller than the target's size.
The entire expression can be thought of as an inversion principle where the motive $P$ is the goal:
one proves $P$ for some natural $x$, for instance, when one can prove it
for the case where $x$ is $\zero*$
and the case where $x$ is a $\succ*$.
Alternatively, it can be thought of as an applied destructor:
informally borrowing syntax from dependent pattern matching, the destructor
$$\fun{y}{\N{s}}{\match{y}{\fun*{x}{P}}{(\App{\zero*}{\alpha} \Rightarrow e_z)(\app{\App{\succ*}{\beta}}{z} \Rightarrow e_s)}}$$
on naturals is equivalent to a definition by case analysis
%
\begin{align*}
\Let{&\const{destruct}}{\funtype{y}{\N{s}}{\subst{P}{x}{y}}}{\\
&\app{\const{destruct}}{\app{(\zero{s}{\alpha})}{\phantom{z}}} = e_z \\
&\app{\const{destruct}}{(\succ{s}{\beta}{z}) = e_s}}
\end{align*}
where the branches must have types $\subst{P}{x}{\zero{s}{\alpha}}$
and $\subst{P}{x}{\succ{s}{\beta}{z}}$, respectively,
depending on which branch is being filled in.

As discussed, the body of a fixpoint is only well typed
when the fixpoint is recursively applied to a smaller size,
as enforced by its type in the environment when type checking the body.

Sizes aside, the only difference from regular, unsized naturals and well-founded trees
is that their types live in a universe one level higher than they usually are.
$\N{s}$ lives in $\Type{1}$ rather than in $\Type{0}$,
and $\W{x}{\sigma}{\tau}{s}$ lives in $\axioms{U}$ rather than in $U$.
This is a direct consequence of the way that the translation is defined,
and is necessary to maintain its type preservation\index{type preservation} properties.
While not incorrect, inductive types living in the ``wrong'' universe is aesthetically unpleasant
and removes some of the impredicativity\index{impredicativity} of their parameters
by preventing them from quantifying over the inductive types themselves.
I discuss potential methods of circumventing this undesirable trait to varying degrees of success
later in \cref{sec:universe-levels}.

\section{Examples}\label{sec:examples}

Now that the rules of \lang have been established,
this section presents examples of using \lang for programming.
Although it only has naturals and well-founded trees,
I also use other sized inductive types as examples,
informally defining them similarly to \cref{sec:ind-types}.
Additionally, I omit the type annotation of $\kw{let}$-bound expressions
when the type is evident or deducible from context.

\subsection{Concrete natural numbers} \label{subsec:concrete}

Concrete numbers can be constructed using the base size $\circ$ as a starting point
and its successors as strictly larger sizes.
For convenience, I use the notation \new{$s+n$} for some constant $n$ to mean the $n$th size successor of $s$.
The same number can be represented as terms of different types by changing the annotation since,
for instance, both $\circ+2$ and $\circ+3$ are strictly larger than $\sss{\circ}$.
Similarly, two different naturals can have the same size
as the size of an inductive type represents \emph{at most}
how many layers of constructors an element has.
%
\begin{alignat*}{4}
&\Let{\const{0}&&}{\N{\sss{\circ}}&&}{\zero{\hat{\circ}}{\circ}} \\
&\Let{\const{0'}&&}{\N{\circ+2}&&}{\zero{\circ+2}{\circ}} \\
&\Let{\const{1}&&}{\N{\circ+2}&&}{\succ{\circ+2}{\hat{\circ}}{\const{0}}} \\
&\Let{\const{1'}&&}{\N{\circ+3}&&}{\succ{\circ+3}{\hat{\circ}}{\const{0}}}
\end{alignat*}

Two terms representing the same number might not be convertible,
as shown by the terms $\const{0}$ and $\const{0'}$,
or by $\const{1}$ and $\const{1'}$;
this requires \emph{shape irrelevance}\index{shape irrelevance}~\citep{NbE},
which is beyond the scope of this thesis.
However, any natural can be ``lifted'' to a larger size
simply by recursively reconstructing the natural with different sizes.
%
\begin{align*}
&\Let{\liftN}{\Funtype{\alpha}{\Funtype<{\beta}{\alpha}{\arr*{\N{\beta}}{\N{\alpha}}}}}{\\
&\fix{\lift}{\alpha}{\Funtype<{\beta}{\alpha}{\arr*{\N{\beta}}{\N{\alpha}}}}{\\
&\quad \Fun<{\beta}{\alpha}{\fun{n}{\N{\beta}}{\\
&\quad \match{n}{\fun*{\any}{\N{\alpha}}}{\\
&\qquad \App{\zero*}{\gamma} \Rightarrow \zero{\alpha}{\beta} \\
&\qquad \app{\App{\succ*}{\gamma}}{z} \Rightarrow \succ{\alpha}{\beta}{(\app{\App{\lift}{\beta}{\gamma}}{z})}}}}}}
\end{align*}

A lifting function can also be defined for well-founded trees
(and for strictly-positive sized inductive types in general).
%
\begin{align*}
&\Let{\liftW}{\funtype{A}{\Type{i}}{\funtype{B}{\arr*{A}{\Type{i}}}{\Funtype{\alpha}{\Funtype<{\beta}{\alpha}{\arr*{\W{x}{A}{\app{B}{x}}{\beta}}{\W{x}{A}{\app{B}{x}}{\alpha}}}}}}}{\\
&\fun{A}{\Type{i}}{\fun{B}{\arr*{A}{\Type{i}}}{\\
&\fix{\lift}{\alpha}{\Funtype<{\beta}{\alpha}{\arr*{\W{x}{A}{\app{B}{x}}{\beta}}{\W{x}{A}{\app{B}{x}}{\alpha}}}}{\\
&\quad \Fun<{\beta}{\alpha}{\fun{w}{\W{x}{A}{\app{B}{x}}{\beta}}{\\
&\quad \match{w}{\fun*{\any}{\W{x}{A}{\app{B}{x}}{\alpha}}}{\\
&\qquad \app{\App{\sup*}{\gamma}}{a}{f} \Rightarrow \sup{x}{A}{\app{B}{x}}{\alpha}{\beta}{a}{(\fun{x}{A}{\app{\App{\lift}{\beta}{\gamma}}{(\app{f}{x})}})}}}}}}}}
\end{align*}

\iffalse
\begin{align*}
\Let{&\liftW}{\funtype{A}{\Type{i}}{\funtype{B}{\arr*{A}{\Type{i}}}{\Funtype{\alpha}{\Funtype<{\beta}{\alpha}{\arr*{\W{x}{A}{\app{B}{x}}{\beta}}{\W{x}{A}{\app{B}{x}}{\alpha}}}}}}}{\\
&\liftW \: A \: B \: [\alpha] \: [\beta] \: (\app{\App{\sup*}{\gamma}}{a}{f}) = \sup* \: [\beta] \: a \: (\fun{x}{A}{\liftW \: A \: B \: [\beta] \: [\gamma] \: (\app{f}{x})})}
\end{align*}
\fi

\subsection{Size-preserving functions}

One of the most important uses of sized types is the ability to define
\emph{size-preserving}\index{size preservation} functions,
where the sizes of the input and output are the same.
This guarantees that the output is never larger than the input,
and size-preserving functions can be used in recursive calls of fixpoints.
For instance, the predecessor function $\pred$ which computes
$\maximum{0, n - 1}$ for some number $n$ can be size preserving.
%
\begin{align*}
&\Let{\pred}{\Funtype{\alpha}{\arr*{\N{\alpha}}{\N{\alpha}}}}{ \\
&\Fun{\alpha}{\fun{n}{\N{\alpha}}{\match*{n}{ \\
&\quad \App{\zero*}{\beta} \Rightarrow \zero{\alpha}{\beta} \\
&\quad \app{\App{\succ*}{\beta}}{m} \Rightarrow \app{\App{\liftN}{\alpha}{\beta}}{m}}}}}
\end{align*}

The $\monus$ function, which computes $\maximum{0, n - m}$ given numbers $n, m$,
is similarly size preserving in its first argument,
since $n - m$ is never greater than $n$.
%
\begin{align*}
&\Let{\monus}{\Funtype{\alpha}{\Funtype{\beta}{\arr*{\N{\beta}}{\N{\alpha}}{\N{\beta}}}}}{ \\
&\fix{\monus*}{\alpha}{\Funtype{\beta}{\arr*{\N{\beta}}{\N{\alpha}}{\N{\beta}}}}{ \\
&\quad \Fun{\beta}{\fun{n}{\N{\beta}}{\fun{m}{\N{\alpha}}{\match*{m}{ \\
&\qquad \App{\zero*}{\gamma} \Rightarrow n \\
&\qquad \app{\App{\succ*}{\gamma}}{k} \Rightarrow \app{\App{\monus*}{\gamma}{\beta}}{(\app{\App{\pred}{\beta}}{n})}{k}}}}}}}
\end{align*}

We see the benefit of size preservation with $\divv$,
which computes Euclidean division of $n$ by $m$, or $\left\lceil\frac{n}{m+1}\right\rceil$.
This is computed recursively by subtracting $m$ from the numerator using $\monus$
until $\zero*$ is reached, and counting the number of times the subtraction is performed.
The recursive call is performed on the result of $\monus$;
$\divv$ then only type checks because $\monus$ is size preserving.
Because the first argument of the recursive call to $\divv$ isn't \emph{structurally}
a subterm with the call to $\monus$ in the way,
a guardedness check wouldn't accept the corresponding unsized function,
as is the case for CIC.
%
\begin{align*}
&\Let{\divv}{\Funtype{\alpha}{\Funtype{\beta}{\arr*{\N{\alpha}}{\N{\beta}}{\N{\alpha}}}}}{ \\
&\fix{\divv*}{\alpha}{\Funtype{\beta}{\arr*{\N{\alpha}}{\N{\beta}}{\N{\alpha}}}}{ \\
&\quad \Fun{\beta}{\fun{n}{\N{\alpha}}{\fun{m}{\N{\beta}}{\match*{n}{ \\
&\qquad \App{\zero*}{\gamma} \Rightarrow \zero{\alpha}{\gamma} \\
&\qquad \app{\App{\succ*}{\gamma}}{k} \Rightarrow
\succ{\alpha}{\gamma}{
  (\underbrace{\app{\App{\divv*}{\gamma}{\beta}}{
    (\underbrace{\app{\App{\monus}{\beta}{\gamma}}{k}{
      (\underbrace{\app{\App{\pred}{\beta}}{m}}_{\N{\beta}})
    }}_{\N{\gamma}})
  }{m}}_{\N{\gamma}})
}}}}}}}
\end{align*}

\subsection{Large sized functions}

So far, none of these examples have made use of dependent types:
their unsized variants could have all been written in System F.
We therefore now turn to the $n$-ary size-preserving function type,
which does \emph{large elimination}\index{large elimination},
where the return type lives in a larger universe
than that of the input type.
(In this case, $n$ is a natural whose type lives in $\Type{1}$,
while the return type must live in $\Type{2}$).
This example is due to \citet{MiniAgda}.
%
\begin{align*}
&\Let{\Nary}{\Funtype{\alpha}{\arr*{\N{\alpha}}{\Funtype{\beta}{\Type{1}}}}}{\\
&\fix{\nary}{\alpha}{\arr*{\N{\alpha}}{\Funtype{\beta}{\Type{1}}}}{\\
&\quad \fun{n}{\N{\alpha}}{\Fun{\beta}{\match{n}{\fun*{\any}{\Type{1}}}{\\
&\qquad \App{\zero*}{\gamma} \Rightarrow \N{\beta} \\
&\qquad \app{\App{\succ*}{\gamma}}{m} \Rightarrow \arr*{\N{\beta}}{\App{\app{\App{\nary}{\gamma}}{m}}{\beta}}}}}}}
\end{align*}

Intuitively, $\Nary$ constructs the function type

$$\App{\app{\App{\Nary}{\alpha}}{n}}{\beta} = \arr*{\underbrace{\arr*{\N{\beta}}{\seq}{\N{\beta}}}_{\text{$n$ arguments}}}{\N{\beta}}.$$

$n$-ary size-preserving functions can then be typed using $\Nary$,
such as a $\MaX$ function that takes the maximum of $n+1$ numbers
all of the same size.
For concision, I define $\maX$ in a pattern-matching style of syntax,
but it can easily be translated to proper \lang as a fixpoint and two $\kw{case}$ expressions.
%
\begin{align*}
\Let{&\maX}{\Funtype{\alpha}{\arr*{\N{\alpha}}{\N{\alpha}}{\N{\alpha}}}}{ \\
&\maX \: \sqbr{\alpha} \: n \: (\App{\zero*}{\beta}) = n \\
&\maX \: \sqbr{\alpha} \: (\App{\zero*}{\beta}) \: m = m \\
&\maX \: \sqbr{\alpha} \: (\app{\App{\succ*}{\beta}}{n'}) \: (\app{\App{\succ*}{\beta}}{m'}) = \succ{\alpha}{\beta}{(\app{\App{\max}{\beta}}{n'}{m'})}}
\end{align*}
\iffalse
\begin{align*}
&\Let{\maX}{\Funtype{\alpha}{\arr*{\N{\alpha}}{\N{\alpha}}{\N{\alpha}}}}{ \\
&\fix{\mathit{max}}{\alpha}{\arr*{\N{\alpha}}{\N{\alpha}}{\N{\alpha}}}{ \\
&\quad \fun{n}{\N{\alpha}}{\fun{m}{\N{\alpha}}{ \\
&\quad \match*{n}{ \\
&\qquad \App{\zero*}{\beta} \Rightarrow m \\
&\qquad \app{\App{\succ*}{\beta}}{n'} \Rightarrow \\
&\qquad \quad \match*{m}{ \\
&\qquad \qquad \App{\zero*}{\beta} \Rightarrow n \\
&\qquad \qquad \app{\App{\succ*}{\beta}}{m'} \Rightarrow \succ{\alpha}{\beta}{(\app{\App{\mathit{max}}{\beta}}{n'}{m'})}}}}}}}
\end{align*}
\fi
%
\begin{align*}
&\Let{\MaX}{\Funtype{\alpha}{\funtype{k}{\N{\alpha}}{\Funtype{\beta}{\arr*{\N{\beta}}{\App{\app{\App{\Nary}{\alpha}}{k}}{\beta}}}}}}{\\
&\fix{\MaX*}{\alpha}{\funtype{k}{\N{\alpha}}{\Funtype{\beta}{\arr*{\N{\beta}}{\App{\app{\App{\Nary}{\alpha}}{k}}{\beta}}}}}{\\
&\quad \fun{k}{\N{\alpha}}{\Fun{\beta}{\fun{n}{\N{\beta}}{\match{k}{\fun*{x}{\App{\app{\App{\Nary}{\alpha}}{x}}{\beta}}}{\\
&\qquad \App{\zero*}{\gamma} \Rightarrow n \\
&\qquad \app{\App{\succ*}{\gamma}}{k'} \Rightarrow \fun{m}{\N{\beta}}{\app{\App{\app{\App{\MaX*}{\gamma}}{k'}}{\beta}}{(\app{\App{\max}{\beta}}{n}{m})}}}}}}}}
\end{align*}

\subsection{Sized lists}

Implementing sized quicksort~\citep{term-check} and mergesort~\citep{Abel-diss} are perhaps the most classic examples,
demonstrating their ease of programming compared to na\"ive unsized implementations that otherwise
impose a significant termination proof burden to the programmer.
Although \lang has no lists for simplicity's sake,
they are no more complicated to support than naturals.
Using the same informal syntax from \cref{sec:ind-types},
they can be defined by the following:
%
\begin{align*}
&\data{\List{(\annot{A}{\Type{i}})}{\alpha}}{\Type{i+1}} \\
&\quad \annot{\nil*}{\Funtype<{\beta}{\alpha}{\List{A}{\alpha}}} \\
&\quad \annot{\cons*}{\Funtype<{\beta}{\alpha}{\arr*{A}{\List{A}{\beta}}{\List{A}{\alpha}}}}
\end{align*}

A similar lifting function for lists can be defined, whose type is

$$\annot{\liftL}{\funtype{A}{\Type{i}}{\Funtype{\alpha}{\Funtype<{\beta}{\alpha}{\arr*{\List{A}{\beta}}{\List{A}{\alpha}}}}}}.$$

I also use an encoding for \emph{weak dependent pairs}\index{weak dependent pair}
to express pairing a particular size with a type of that size
in situations where the size expressions of \lang aren't expressive enough
to specify the exact size.
%
\begin{align*}
\Pairtype{\alpha}{\tau} &= \funtype{\sigma}{\Type{i}}{\arr*{(\Funtype{\alpha}{(\arr*{\tau}{\sigma})})}{\sigma}} \\
\Pair{s}{e}_{\Pairtype{\alpha}{\tau}} &= \fun{\sigma}{\Type{i}}{\fun{f}{\Funtype{\alpha}{(\arr*{\tau}{\sigma})}}{\app{\App{f}{s}}{e}}} \\
\unpair{\alpha}{x}{\alpha}{\tau}{e_1}{\annot{e_2}{\sigma}} &= \app{e_1}{\sigma}{(\Fun{\alpha}{\fun{x}{\tau}{e_2}})}
\end{align*}

This encoding is used in the $\append$ function, written in pattern-matching style,
where the precise size of the resulting list should be the sum of the sizes of the input lists,
but \lang has no addition operator for sizes.
Suppose that $A$ is some type.
%
\begin{align*}
\Let{&\append}{\Funtype{\alpha}{\Funtype{\beta}{\arr*{\List{A}{\alpha}}{\List{A}{\beta}}{\Pairtype{\gamma}{\List{A}{\gamma}}}}}}{ \\
&\append \: \sqbr{\alpha} \: \sqbr{\beta} \: (\nil* \: \sqbr{\alpha'}) \: r = \Pair{\beta}{r} \\
&\append \: \sqbr{\alpha} \: \sqbr{\beta} \: (\cons* \: \sqbr{\alpha'} \: \hd \: \tl) \: r =
  \unpair*{\gamma}{x}{\append \: \sqbr{\alpha'} \: \sqbr{\beta} \: \tl \: r}{
    \cons* \: \sqbr{\gamma} \: hd \: x
  }}
\end{align*}

$\append$ retains the order of the elements of the lists.
On the other hand, if given some ordering function on $A$
$$\annot{\ifleq}{\funtype{C}{\Type{i}}{\arr*{A}{A}{C}{C}{C}}}$$
which selects one of two branches $C$ based on the order of the elements of $A$,
we can define a similar $\mrg$ function that merges two lists such that
the resulting list is sorted with respect to $\ifleq$ if the input lists were.
In pattern-matching style, the recursion appears lexicographical,
but it can be translated to nested fixpoints~\citep{Abel-diss}.
%
\begin{align*}
\Let{&\mrg}{\Funtype{\alpha}{\Funtype{\beta}{\arr*{\List{A}{\alpha}}{\List{A}{\beta}}{\Pairtype{\gamma}{\List{A}{\gamma}}}}}}{ \\
&\mrg \: \sqbr{\alpha} \: \sqbr{\beta} \: (\nil* \: \sqbr{\alpha'}) \: r = \Pair{\beta}{r} \\
&\mrg \: \sqbr{\alpha} \: \sqbr{\beta} \: l \: (\nil* \: \sqbr{\beta'}) = \Pair{\alpha}{l} \\
&\mrg \: \sqbr{\alpha} \: \sqbr{\beta} \: (\cons* \: \sqbr{\alpha'} \: a \: l) \: (\cons* \: \sqbr{\beta'} \: b \: r) = \\
&\quad \ifleq \: (\List{A}{\gamma}) \: a \: b \\
&\qquad (\unpair*{\gamma}{x}{\mrg \: \sqbr{\alpha'} \: \sqbr{\beta} \: l \: (\cons* \: \sqbr{\beta'} \: b \: r)}{\Pair{\sss{\gamma}}{\cons* \: \sqbr{\gamma} \: a \: x}}) \\
&\qquad (\unpair*{\gamma}{x}{\mrg \: \sqbr{\alpha} \: \sqbr{\beta'} \: (\cons* \: \sqbr{\alpha'} \: a \: l) \: r}{\Pair{\sss{\gamma}}{\cons* \: \sqbr{\gamma} \: b \: x}})}
\end{align*}

Quicksort and mergesort are both divide-and-conquer algorithms,
so we need some way of dividing up a list.
The simplest to implement recursively is to $\spt$ the list into two lists of alternating elements,
written in continuation-passing style since \lang has no pairs.
%
\begin{align*}
\Let{&\spt}{\Funtype{\alpha}{\funtype{C}{\Type{i}}{\arr*{\List{A}{\alpha}}{(\arr*{\List{A}{\alpha}}{\List{A}{\alpha}}{C})}{C}}}}{ \\
&\spt \: \sqbr{\alpha} \: C \: (\nil* \: \sqbr{\beta}) \: k = \app{k}{(\nil* \: \sqbr{\beta})}{(\nil* \: \sqbr{\beta})} \\
&\spt \: \sqbr{\alpha} \: C \: (\cons* \: \sqbr{\beta} \: \hd \: \tl) \: k = \\
&\quad \spt \: \sqbr{\beta} \: C \: \tl \:
  (\fun{l}{\List{A}{\beta}}{\fun{r}{\List{A}{\beta}}{
    \app{k}{(\cons* \: \sqbr{\beta} \: \hd \: r)}{(\liftL \: A \: \sqbr{\alpha} \: \sqbr{\beta} \: l)}
  }})}
\end{align*}

Alternatively, the list can be $\partition$ed relative to some pivot element
into two according to $\ifleq$,
such that one list contains only elements that are smaller or equal to the pivot
and the other contains elements larger than it.
%
\begin{align*}
\Let{&\partition}{\Funtype{\alpha}{\funtype{C}{\Type{i}}{\arr*{A}{\List{A}{\alpha}}{(\arr*{\List{A}{\alpha}}{\List{A}{\alpha}}{C})}{C}}}}{ \\
&\partition \: \sqbr{\alpha} \: C \: a \: (\nil* \: \sqbr{\beta}) \: k = \app{k}{(\nil* \: \sqbr{\beta})}{(\nil* \: \sqbr{\beta})} \\
&\partition \: \sqbr{\alpha} \: C \: a \: (\cons* \: \sqbr{\beta} \: \hd \: \tl) \: k = \\
&\quad \partition \: \sqbr{\beta} \: C \: a \: \tl \:
  (\fun{l}{\List{A}{\beta}}{\fun{r}{\List{A}{\beta}}{ \\
    &\qquad \ifleq \: C \: \hd \: a \\
    &\qquad \quad (\app{k}{(\cons* \: \sqbr{\beta} \: \hd \: l)}{(\liftL \: A \: \sqbr{\alpha} \: \sqbr{\beta} \: r)}) \\
    &\qquad \quad (\app{k}{(\liftL \: A \: \sqbr{\alpha} \: \sqbr{\beta} \: l)}{(\cons* \: \sqbr{\beta} \: \hd \: r)})
  }})}
\end{align*}

Quicksort sorts a list by picking a pivot element,
$\partition$ing the rest of the list into sublists of elements smaller and greater than the pivot,
recursively sorting them,
then $\append$ing the sorted lists with the pivot in the middle.
Meanwhile, mergesort sorts a list by $\spt$ting the list into sublists,
recursively sorting them,
then $\mrg$ing the sorted lists while maintaining ordering.
Both involve recursive calls on lists that are not structurally smaller,
but do have smaller sizes,
and they differ in where $\ifleq$ is applied ($\partition$ or $\mrg$).

\vspace{-2\baselineskip}
\begin{multicols}{2}
\begin{align*}
&\Let{\qsort}{\Funtype{\alpha}{\arr*{\List{A}{\alpha}}{\Pairtype{\gamma}{\List{A}{\gamma}}}}}{ \\
&\fix{\qsort*}{\alpha}{\arr*{\List{A}{\alpha}}{\Pairtype{\gamma}{\List{A}{\gamma}}}}{ \\
&\quad \fun{l}{\List{A}{\alpha}}{\match*{l}{ \\
&\qquad \App{\nil*}{\beta} \Rightarrow \Pair{\alpha}{l} \\
&\qquad \app{\App{\cons*}{\beta}}{\hd}{\tl \Rightarrow \\
  &\qquad \quad \app{\App{\partition}{\beta}}{(\Pairtype{\gamma}{\List{A}{\gamma}})}{\hd}{\tl}{ \\
    &\qquad \quad (\fun{l}{\List{A}{\beta}}{\fun{r}{\List{A}{\beta}}{ \\
      &\qquad \qquad \unpair{\delta}{x}{\gamma}{\List{A}{\gamma}}{\app{\App{\qsort*}{\beta}}{l}}{ \\
      &\qquad \qquad \unpair{\varepsilon}{y}{\gamma}{\List{A}{\gamma}}{\app{\App{\qsort*}{\beta}}{r}}{ \\
      &\qquad \qquad \app{\App{\append}{\delta}{\sss{\varepsilon}}}{x}{(\cons{A}{\sss{\varepsilon}}{\varepsilon}{\hd}{y})}}}
    }})}}}}}}
\end{align*}

\begin{align*}
&\Let{\msorted}{\Funtype{\alpha}{\arr*{A}{\List{A}{\alpha}}{\Pairtype{\gamma}{\List{A}{\gamma}}}}}{ \\
&\fix{\msort*}{\alpha}{\arr*{A}{\List{A}{\alpha}}{\Pairtype{\gamma}{\List{A}{\gamma}}}}{ \\
&\quad \fun{a}{A}{\fun{l}{\List{A}{\alpha}}{\match*{l}{ \\
&\qquad \App{\nil*}{\beta} \Rightarrow \Pair{\sss{\alpha}}{\cons{A}{\sss{\alpha}}{\alpha}{a}{l}} \\
&\qquad \app{\App{\cons*}{\beta}}{b}{\tl} \Rightarrow \\
&\qquad \quad \app{\App{\spt}{\beta}}{(\Pairtype{\gamma}{\List{A}{\gamma}})}{\tl}{ \\
&\qquad \quad (\fun{l}{\List{A}{\beta}}{\fun{r}{\List{A}{\beta}}{ \\
&\qquad \qquad \unpair{\delta}{x}{\gamma}{\List{A}{\gamma}}{\app{\App{\msort*}{\beta}}{a}{l}}{ \\
&\qquad \qquad \unpair{\varepsilon}{y}{\gamma}{\List{A}{\gamma}}{\app{\App{\msort*}{\beta}}{b}{r}}{ \\
&\qquad \qquad \app{\App{\mrg}{\delta}{\varepsilon}}{x}{y}}}}})}}}}}} \\\\
&\Let{\msort}{\Funtype{\alpha}{\arr*{\List{A}{\alpha}}{\Pairtype{\gamma}{\List{A}{\gamma}}}}}{ \\
&\Fun{\alpha}{\fun{l}{\List{A}{\alpha}}{\match*{l}{ \\
&\quad \App{\nil*}{\beta} \Rightarrow \Pair{\alpha}{l} \\
&\quad \app{\App{\cons*}{\beta}}{\hd}{\tl \Rightarrow \app{\App{\msorted}{\beta}}{\hd}{\tl}}}}}}
\end{align*}
\end{multicols}

\subsection{Higher-rank sizes}

The final example, demonstrating higher-rank size quantification,
is a traversal of a well-founded tree with a size-preserving function,
adapted from the rose tree traversal example by \citet{NbE}.
Given some size-preserving function on a well-founded tree,
the function is applied to subtrees prior to traversal.
Since the function is also size polymorphic,
it can be applied at any point in the traversal.
Suppose that $B$ here is some type operator on $A$.
%
\begin{align*}
&\Let{\traverseW}{\arr*{(\Funtype{\gamma}{\arr*{\W{x}{A}{\app{B}{x}}{\gamma}}{\W{x}{A}{\app{B}{x}}{\gamma}}})}{ \\
&\phantom{\kw{let} \: \traverseW \mathrel{:} \phantom{}} \Funtype{\alpha}{\arr*{\W{x}{A}{\app{B}{x}}{\alpha}}{\W{x}{A}{\app{B}{x}}{\alpha}}}}}{ \\
&\quad \fun{f}{\Funtype{\gamma}{\arr*{\W{x}{A}{\app{B}{x}}{\gamma}}{\W{x}{A}{\app{B}{x}}{\gamma}}}}{ \\
&\quad \fix{\traverse}{\alpha}{\arr*{\W{x}{A}{\app{B}{x}}{\alpha}}{\W{x}{A}{\app{B}{x}}{\alpha}}}{ \\
&\qquad \fun{w}{\W{x}{A}{\app{B}{x}}{\alpha}}{\match*{w}{ \\
&\qquad \quad \app{\App{\sup*}{\beta}}{a}{g} \Rightarrow \sup{x}{A}{\app{B}{x}}{\alpha}{\beta}{a}{(\fun{b}{\app{B}{a}}{\app{\App{\traverse}{\beta}}{(\app{\App{f}{\beta}}{(\app{g}{b})})}})}}}}}}
\end{align*}

\subsection{Limitations} \label{sec:examples:limitations}

As expressive as bounded, higher-rank sized types are,
there still exist limitations to what can be expressed in comparison to ordinary inductive types
or to sized type theories which have an infinite size\index{infinite size},
such as the ones listed in \cref{sec:sized-types}.
Limitations typically arise when dealing with inductive definitions where
recursive arguments appear as the return type of a function.
Consider the following inductive definition
representing the Brouwer notation for ordinals (see \eg \citet{ordinals})
with zero, successor, and limit ordinals:
%
\begin{align*}
&\data{\Ord{\alpha}}{\Type{1}} \\
&\quad \annot{\zord*}{\Funtype<{\beta}{\alpha}{\Ord{\alpha}}} \\
&\quad \annot{\sord*}{\Funtype<{\beta}{\alpha}{\arr*{\Ord{\beta}}{\Ord{\alpha}}}} \\
&\quad \annot{\lord*}{\Funtype<{\beta}{\alpha}{\arr*{(\Funtype<{\gamma}{\beta}{\arr*{\N{\gamma}}{\Ord{\beta}}})}{\Ord{\alpha}}}}
\end{align*}

The limit ordinal, taking some function on naturals returning an ordinal,
constructs an ordinal meant to be ``larger'' than any of the returned ordinals.
Such a function should be able to return an ordinal larger than any natural,
hence the bounded size quantification in its domain.
Conversely, the naturals embed quite naturally within the ordinals.
%
\begin{align*}
&\Let{\natOrd}{\Funtype{\alpha}{\arr*{\N{\alpha}}{\Ord{\alpha}}}}{ \\
&\fix{\natOrd*}{\alpha}{\arr*{\N{\alpha}}{\Ord{\alpha}}}{ \\
&\quad \fun{n}{\N{\alpha}}{\match*{n}{ \\
&\qquad \App{\zero*}{\beta} \Rightarrow \zord{\alpha}{\beta} \\
&\qquad \app{\App{\succ*}{\beta}}{m} \Rightarrow \sord{\alpha}{\beta}{(\app{\App{\natOrd*}{\beta}}{m})}}}}}
\end{align*}

In an unsized type theory, supposing that $\const{natOrd'}$ is an unsized version of $\natOrd$,
the first limit ordinal $\const{\omega'}$ is easily defined as
$\Let{\const{\omega'}}{\const{Ord}}{\app{\lord*}{\const{natOrd'}}}$.
However, in \lang, things are not so easy; the na\"ive attempt yields the following:
%
\begin{align*}
&\Let{\liftO}{\Funtype{\alpha}{\Funtype<{\beta}{\alpha}{\arr*{\Ord{\beta}}{\Ord{\alpha}}}}}{\seq} \\
&\Let{\omegaOrd}{\Ord{\sss{\hole}}}{\lord{\sss{\hole}}{\hole}{(\Fun<{\gamma}{\hole}{\fun{n}{\N{\gamma}}{\app{\App{\liftO}{\hole}{\gamma}}{(\app{\App{\natOrd}{\gamma}}{n})}}})}}
\end{align*}

Aside from the additional bounds and size lifting, there is one crucial problem: what size fills in the hole \new{$\hole$}?
Intuitively, this size must be larger than the size of \emph{any} natural.
Therefore, a corresponding ``limit size'' is needed;
the infinite size in other sized type theories can fill this r\^ole since it's larger than all sizes
and therefore the limit of \emph{all} sizes.
I further discuss the infinite size and its absence from \lang in \cref{sec:infinity}.