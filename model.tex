\chapter{Syntactic Model of \lang}

\newcommand{\FigSyntaxCIC}[1]{
  \begin{figure}[h]
  \centering
  \begin{align*}
  \iT, \jT, \kT, \mT, \nT &\Coloneqq \meta{\textrm{naturals}} &
  \GammaT &\Coloneqq \mt \mid \GammaT, \annotT{\xT}{\tauT} \mid \GammaT, \define{\xT}{\tauT}{\eT} \\
  \fT, \gT, \wT, \xT, \yT, \zT, \alphaT, \betaT, \gammaT &\Coloneqq \meta{\textrm{term variables}} &
  \DeltaT &\Coloneqq \mt \mid \DeltaT, \annotT{\xT}{\tauT} \mid \DeltaT, \annotT{\cT}{\tauT} \\
  \XT &\Coloneqq \meta{\textrm{inductive type names}} &
  \UT &\Coloneqq \PropT \mid \TypeT{\iT} \\
  \cT &\Coloneqq \meta{\textrm{inductive constructor names}} &
  \DT &\Coloneqq \app{\dataT{\app{\XT}{\DeltaT}}{\arrT*{\DeltaT}{\UT}}}{\DeltaT} \\
  \eT, \aT, \dT, \pT, \PT, \rT, \sT, \tauT, \sigmaT &\Coloneqq
    \mathrlap{\xT \mid \UT \mid \funtypeT{\xT}{\tauT}{\tauT} \mid \funT{\xT}{\tauT}{\eT} \mid \app{\eT}{\eT} \mid \letinT{\xT}{\tauT}{\eT}{\eT} \mid \fixT{\iT}{\fT}{\tauT}{\eT}} \\
  &\mid \mathrlap{\eqT{\eT}{\tauT}{\eT} \mid \reflT{\eT} \mid \JT{\PT}{\dT}{\pT} \mid \matchT{\eT}{\funT*{\vec{\yT}}{\xT}{\PT}}{(\app{\cT}{\vec{\zT}} \RightarrowT \eT) \seq}}
  \end{align*}
  \caption{Syntax (\CICE)}
  \label{#1}
  \end{figure}
}

\newcommand{\FigEquiv}[1]{
  \begin{figure}[p]
  \centering
  \begin{mathpar}
  \fbox{$\defeq{\GammaT}{\eT}{\eT}{\tauT}$} \hfill \\
  \inferrule[\rlabel{$\equiv$-refl}{equiv-refl}]{
    \type{\GammaT}{\eT}{\tauT}
  }{
    \defeq{\GammaT}{\eT}{\eT}{\tauT}
  }
  \and
  \inferrule[\rlabel{$\equiv$-sym}{equiv-sym}]{
    \defeq{\GammaT}{\eT_2}{\eT_1}{\tauT}
  }{
    \defeq{\GammaT}{\eT_1}{\eT_2}{\tauT}
  }
  \and
  \inferrule[\rlabel{$\equiv$-trans}{equiv-trans}]{
    \defeq{\GammaT}{\eT_1}{\eT_2}{\tauT} \\\\
    \defeq{\GammaT}{\eT_2}{\eT_3}{\tauT}
  }{
    \defeq{\GammaT}{\eT_1}{\eT_3}{\tauT}
  }
  \and
  \inferrule[\rlabel{$\equiv$-conv}{equiv-conv}]{
    \subtype{\GammaT}{\sigmaT}{\tauT} \\\\
    \defeq{\GammaT}{\eT_1}{\eT_2}{\sigmaT}
  }{
    \defeq{\GammaT}{\eT_1}{\eT_2}{\tauT}
  }
  \and
  \inferrule[\rlabel{$\equiv$-reflect}{equiv-reflect}]{
    \type{\GammaT}{\pT}{\eqT{\eT_1}{\tauT}{\eT_2}}
  }{
    \defeq{\GammaT}{\eT_1}{\eT_2}{\tauT}
  }
  \and
  \inferrule[\rlabel{$\equiv$-cong}{equiv-cong}]{
    \textrm{For every $1 \leq i \leq n$:} \\
    \defeq{\GammaT'}{\eT_i}{\eT'_i}{\tauT'}
  }{
    \defeq{\GammaT}
      {\subst{\eT}{\xT_1, \seq, \xT_n}{\eT_1, \seq, \eT_n}}
      {\subst{\eT}{\xT_1, \seq, \xT_n}{\eT'_1, \seq, \eT'_n}}
      {\tauT}
  }
  \and
  \inferrule[\rlabel{$\equiv$-$\delta$}{equiv-delta}]{
    (\defineT{\xT}{\tauT}{\eT}) \in \GammaT
  }{
    \defeq{\GammaT}{\xT}{\eT}{\tauT}
  }
  \and
  \inferrule[\rlabel{$\equiv$-$\beta$}{equiv-beta}]{
    \type{\GammaT}{\sigmaT}{\UT} \\
    \type{\GammaT, \annotT{\xT}{\sigmaT}}{\eT}{\tauT} \\
    \type{\GammaT}{\eT'}{\sigmaT}
  }{
    \defeq{\GammaT}{\app{(\funT{\xT}{\sigmaT}{\eT})}{\eT'}}{\subst{\eT}{\xT}{\eT'}}{\subst{\tauT}{\xT}{\eT'}}
  }
  \and
  \inferrule[\rlabel{$\equiv$-$\eta$}{equiv-eta}]{
    \defeq{\GammaT, \annotT{\xT}{\sigmaT}}{\app{\eT_1}{\xT}}{\app{\eT_2}{\xT}}{\tauT}
  }{
    \defeq{\GammaT}{\eT_1}{\eT_2}{\funtypeT{\xT}{\sigmaT}{\tauT}}
  }
  \and
  \inferrule[\rlabel{$\equiv$-$\zeta$}{equiv-zeta}]{
    \type{\GammaT}{\sigmaT}{\UT} \\
    \type{\GammaT}{\eT'}{\sigmaT} \\
    \type{\GammaT, \defineT{\xT}{\sigmaT}{\eT'}}{\eT}{\tauT}
  }{
    \defeq{\GammaT}{\letinT{\xT}{\sigmaT}{\eT'}{\eT}}{\subst{\eT}{\xT}{\eT'}}{\subst{\tauT}{\xT}{\eT'}}
  }
  \and
  \inferrule[\rlabel{$\equiv$-$\rho$}{equiv-rho}]{
    \type{\GammaT}{\eT}{\tauT} \\
    \type{\GammaT}{\PT}{\funtypeT{\yT}{\tauT}{\funtypeT{\zT}{\eqT{\eT}{\tauT}{\yT}}{\UT}}} \\
    \type{\GammaT}{\dT}{\app{\PT}{\eT}{\reflT{\eT}}}
  }{
    \defeq{\GammaT}{\JT{\PT}{\dT}{\reflT{\eT}}}{\dT}{\app{\PT}{\eT}{\reflT{\eT}}}
  }
  \iffalse
  \and
  \inferrule[\rlabel{$\equiv$-$\iota$}{equiv-iota}]{
    \app{\dataT{\app{\XT}{(\annotT{\vec{\wT}}{\vec{\vphantom{\wT}\any}})}}{\arrT*{\DeltaT_I}{\UT}}}{\DeltaT_c} \\
    \annotT{\vec{\yT}}{\vec{\sigmaT}} = \subst{\DeltaT_I}{\vec{\wT}}{\vec{\pT}} \\
    (\annotT{\cT}{\arrT*{(\annotT{\vec{\zT}}{\vec{\tauT}})}{\app{\XT}{\vec{\pT}}{\vec{\aT}}}}) \in \subst{\DeltaT_c}{\vec{\wT}}{\vec{\pT}} \\
    \vec{\aT}' = \subst{\vec{\aT}}{\vec{\zT}}{\vec{\eT}} \\
    \type{\GammaT}{\app{\cT}{\vec{\pT}}{\vec{\eT}}}{\app{\XT}{\vec{\pT}}{\vec{\aT}'}} \\
    \type{\GammaT, \annotT{\vec{\yT}}{\vec{\sigmaT}}, \annotT{\xT}{\app{\XT}{\vec{\pT}}{\vec{\yT}}}}{\PT}{\UT} \\
    \type{\GammaT, \annotT{\vec{\zT}}{\vec{\tauT}}}{\eT}{\subst{\PT}{\vec{\yT}, \vec{\xT}}{\vec{\aT}, \app{\cT}{\vec{\pT}}{\vec{\zT}}}}
  }{
    \defeq{\GammaT}{\matchT{\app{\cT}{\vec{\pT}}{\vec{\eT}}}{\funT*{\vec{\yT}}{\xT}{\PT}}{\seq(\app{\cT}{\vec{\zT}} \RightarrowT \eT)\seq}}{\subst{\eT}{\vec{\zT}}{\vec{\eT}}}{\subst{\PT}{\vec{\yT}, \xT}{\vec{\aT}', \app{\cT}{\vec{\pT}}{\vec{\eT}}}}
  }
  \fi
  \and
  \inferrule[\rlabel{$\equiv$-$\iota$}{equiv-iota}]{
    \app{\dataT{\app{\XT}{(\annotT{\vec{\wT}}{\vec{\sigmaT}_P})}}{\arrT*{\DeltaT_I}{\UT}}}{\DeltaT_c} \\
    \type{\GammaT}{\vec{\pT}}{\vec{\sigmaT}_P} \\
    \annotT{\vec{\yT}}{\vec{\sigmaT}_I} = \subst{\DeltaT_I}{\vec{\wT}}{\vec{\pT}} \\
    \type{\GammaT, \annotT{\vec{\yT}}{\vec{\sigmaT}_I}, \annotT{\xT}{\app{\XT}{\vec{\pT}}{\vec{\yT}}}}{\PT}{\UT'} \\
    \elim{\XT}{\UT}{\UT'} \\\\
    \card{\DeltaT_c} = n \\
    \textrm{For every $1 \leq i \leq n$:} \\
    (\annotT{\cT_i}{\arrT*{(\annotT{\vec{\zT}_i}{\vec{\tauT}_i})}{\app{\XT}{\vec{\pT}}{\vec{\aT}_i}}}) \in \subst{\DeltaT_c}{\vec{\wT}}{\vec{\pT}} \\
    \vec{\zT}_i \notin \FV{\PT} \\
    \type{\GammaT, \annotT{\vec{\zT}_i}{\vec{\tauT}_i}}{\eT_i}{\subst{\PT}{\vec{\yT}, \vec{\xT}}{\vec{\aT}_i, \app{\cT_i}{\vec{\pT}}{\vec{\zT}_i}}} \\\\
    j \in 1 \seq n \\
    \type{\GammaT}{\vec{\eT}}{\vec{\tauT}_j} \\
    \vec{\aT}'_j = \subst{\vec{\aT}_j}{\vec{\zT}_j}{\vec{\eT}}
  }{
    \defeq{\GammaT}{\matchT{\app{\cT_j}{\vec{\pT}}{\vec{\eT}}}{\funT*{\vec{\yT}}{\xT}{\PT}}{(\app{\cT_1}{\vec{\zT}_1} \RightarrowT \eT_1)\seq(\app{\cT_n}{\vec{\zT}_n} \RightarrowT \eT_n)}}{\subst{\eT_j}{\vec{\zT}_j}{\vec{\eT}}}{\subst{\PT}{\vec{\yT}, \xT}{\vec{\aT}'_j, \app{\cT_j}{\vec{\pT}}{\vec{\eT}}}}
  }
  \and
  \inferrule[\rlabel{$\equiv$-$\mu$}{equiv-mu}]{
    \type{\GammaT}{\tauT}{\UT} \\
    \defeq{\GammaT}{\tauT}{\arr*{\DeltaT}{\funtypeT{\any}{\app{\XT}{\vec{\pT}}{\vec{\aT}}}{\tauT'}}}{\UT} \\
    \card{\DeltaT} + 1 = \nT \\
    \type{\GammaT, \annotT{\fT}{\tauT}}{\eT}{\tauT}
  }{
    \defeq{\GammaT}{\fixT{\nT}{\fT}{\tauT}{\eT}}{\subst{\eT}{\fT}{\fixT{\nT}{\fT}{\tauT}{\eT}}}{\tauT}
  }
  \end{mathpar}
  \caption{Equivalence rules}
  \label{#1}
  \end{figure}
}

\newcommand{\FigTypingCIC}[1]{
  \begin{figure}[p]
  \centering
  \begin{mathpar}
  \fbox{$\wf{}{\GammaT}$} \qquad
  \fbox{$\type{\GammaT}{\eT}{\tauT}$} \hfill \\
  \inferrule[\rlabel*{nil*}]{~}{\wf{}{\mt}}
  \and
  \inferrule[\rlabel*{cons*-ass}]{
    \wf{}{\GammaT} \\\\
    \type{\GammaT}{\tauT}{\UT}
  }{
    \wf{}{\GammaT, \annotT{\xT}{\tauT}}
  }
  \and
  \inferrule[\rlabel*{cons*-def}]{
    \wf{}{\GammaT} \\\\
    \type{\GammaT}{\eT}{\tauT}
  }{
    \wf{}{\GammaT, \defineT{\xT}{\tauT}{\eT}}
  }
  \and
  \inferrule[\rlabel*{conv*}]{
    \type{\GammaT}{\eT}{\sigmaT} \\
    \type{\GammaT}{\sigmaT}{\UT} \\\\
    \type{\GammaT}{\tauT}{\UT} \\
    \subtype{\GammaT}{\sigmaT}{\tauT}
  }{
    \type{\GammaT}{\eT}{\tauT}
  } 
  \and
  \inferrule[\rlabel*{var*}]{
    \wf{}{\GammaT} \\
    (\annotT{\xT}{\tauT}) \in \GammaT \\\\
    \textit{or } (\defineT{\xT}{\tauT}{\eT}) \in \Gamma
  }{
    \type{\GammaT}{\xT}{\tauT}
  }
  \and
  \inferrule[\rlabel*{univ*}]{
    \wf{}{\GammaT}
  }{
    \type{\GammaT}{\UT}{\axioms{\UT}}
  }
  \and
  \inferrule[\rlabel*{pi*}]{
    \type{\GammaT}{\sigmaT}{\UT_1} \\
    \type{\GammaT, \annotT{\xT}{\sigmaT}}{\tauT}{\UT_2}
  }{
    \type{\GammaT}{\funtypeT{\xT}{\sigmaT}{\tauT}}{\rules{\UT_1}{\UT_2}}
  }
  \and
  \inferrule[\rlabel*{lam*}]{
    \type{\GammaT}{\sigmaT}{\UT} \\
    \type{\GammaT, \annotT{\xT}{\sigmaT}}{\eT}{\tauT}
  }{
    \type{\GammaT}{\funT{\xT}{\sigmaT}{\eT}}{\funtypeT{\xT}{\sigmaT}{\tauT}}
  }
  \and
  \inferrule[\rlabel*{app*}]{
    \type{\GammaT}{\eT_1}{\funtypeT{\xT}{\sigmaT}{\tauT}} \\
    \type{\GammaT}{\eT_2}{\sigmaT}
  }{
    \type{\GammaT}{\app{\eT_1}{\eT_2}}{\subst{\tauT}{\xT}{\eT_1}}
  }
  \and
  \inferrule[\rlabel*{let*}]{
    \type{\GammaT}{\sigmaT}{\UT} \\
    \type{\GammaT}{\eT_1}{\sigmaT} \\
    \type{\GammaT, \defineT{\xT}{\sigmaT}{\eT_1}}{\eT_2}{\tauT}
  }{
    \type{\GammaT}{\letinT{\xT}{\sigmaT}{\eT_1}{\eT_2}}{\subst{\tauT}{\xT}{\eT_1}}
  }
  \and
  \inferrule[\rlabel*{eq}]{
    \type{\GammaT}{\tauT}{\UT} \\
    \type{\GammaT}{\eT_1}{\tauT} \\
    \type{\GammaT}{\eT_2}{\tauT}
  }{
    \type{\GammaT}{\eqT{\eT_1}{\tauT}{\eT_2}}{\UT}
  }
  \and
  \inferrule[\rlabel*{refl}]{
    \type{\GammaT}{\eT}{\tauT}
  }{
    \type{\GammaT}{\reflT{\eT}}{\eqT{\eT}{\tauT}{\eT}}
  }
  \and
  \inferrule[\rlabel*{J}]{
    \type{\GammaT}{\pT}{\eqT{\eT_1}{\tauT}{\eT_2}} \\
    \type{\GammaT}{\PT}{\funtypeT{\yT}{\tauT}{\funtypeT{\zT}{\eqT{\eT_1}{\tauT}{\yT}}{\UT}}} \\
    \type{\GammaT}{\dT}{\app{\PT}{\eT_1}{\reflT{\eT_1}}}
  }{
    \type{\GammaT}{\JT{\PT}{\dT}{\pT}}{\app{\PT}{\eT_2}{\pT}}
  }
  \and
  \inferrule[\rlabel*{ind}]{
    \wf{}{\GammaT} \\
    \app{\dataT{\app{\XT}{\DeltaT_P}}{\tauT}}{\any}
  }{
    \type{\GammaT}{\XT}{\arrT*{\DeltaT_P}{\tauT}}
  }
  \and
  \inferrule[\rlabel*{constr}]{
    \wf{}{\GammaT} \\
    \app{\dataT{\app{\XT}{\DeltaT_P}}{\any}}{\DeltaT_c} \\
    (\annotT{\cT}{\tauT}) \in \DeltaT_c
  }{
    \type{\GammaT}{\cT}{\arrT*{\DeltaT_P}{\tauT}}
  }
  \and
  \inferrule[\rlabel*{case}]{
    \app{\dataT{\app{\XT}{(\annotT{\vec{\wT}}{\vec{\vphantom{\wT}\any}})}}{\arrT*{\DeltaT_I}{\UT}}}{\DeltaT_c} \\
    \annotT{\vec{\yT}}{\vec{\sigmaT}} = \subst{\DeltaT_I}{\vec{\wT}}{\vec{\pT}} \\
    \type{\GammaT}{\eT}{\app{\XT}{\vec{\pT}}{\vec{\aT}}} \\
    \type{\GammaT, \annotT{\vec{\yT}}{\vec{\sigmaT}}, \annotT{\xT}{\app{\XT}{\vec{\pT}}{\vec{\yT}}}}{\PT}{\UT'} \\
    \elim{\XT}{\UT}{\UT'} \\\\
    \card{\DeltaT_c} = n \\
    \textrm{For every $1 \leq i \leq n$:} \\
    (\annotT{\cT_i}{\arrT*{(\annotT{\vec{\zT}_i}{\vec{\tauT}_i})}{\app{\XT}{\vec{\pT}}{\vec{\aT}_i}}}) \in \subst{\DeltaT_c}{\vec{\wT}}{\vec{\pT}} \\
    \vec{\zT}_i \notin \FV{\PT} \\
    \type{\GammaT, \annotT{\vec{\zT}_i}{\vec{\tauT}_i}}{\eT_i}{\subst{\PT}{\vec{\yT}, \vec{\xT}}{\vec{\aT}_i, \app{\cT_i}{\vec{\pT}}{\vec{\zT}_i}}}
  }{
    \type{\GammaT}{\matchT{\eT}{\funT*{\vec{\yT}}{\xT}{\PT}}{(\app{\cT_1}{\vec{\zT}_1} \RightarrowT \eT_1)\seq(\app{\cT_n}{\vec{\zT}_n} \RightarrowT \eT_n)}}{\subst{\PT}{\vec{\yT}, \xT}{\vec{\aT}, \eT}}
  }
  \and
  \inferrule[\rlabel*{fix*}]{
    \type{\GammaT}{\tauT}{\UT} \\
    \defeq{\GammaT}{\tauT}{\arr*{\DeltaT}{\funtypeT{\any}{\app{\XT}{\vec{\pT}}{\vec{\aT}}}{\tauT'}}}{\UT} \\
    \card{\DeltaT} + 1 = \nT \\
    \type{\GammaT, \annotT{\fT}{\tauT}}{\eT}{\tauT}
  }{
    \type{\GammaT}{\fixT{\nT}{\fT}{\tauT}{\eT}}{\tauT}
  }
  \end{mathpar}
  \caption{Typing and well-formedness rules (\CICE)}
  \label{#1}
  \end{figure}
}

\newcommand{\FigSubtypingCIC}[1]{
  \begin{figure}[h]
  \centering
  \begin{mathpar}
  \fbox{$\subtype{\GammaT}{\tauT}{\tauT}$} \hfill \\
  \inferrule[\rlabel{$\preccurlyeq$-conv}{subtype-conv}]{
    \defeq{\GammaT}{\tauT_1}{\tauT_2}{\UT}
  }{
    \subtype{\GammaT}{\tauT_1}{\tauT_2}
  }
  \and
  \inferrule[\rlabel{$\preccurlyeq$-trans}{subtype-trans}]{
    \subtype{\GammaT}{\tauT_1}{\tauT_2} \\
    \subtype{\GammaT}{\tauT_2}{\tauT_3}
  }{
    \subtype{\GammaT}{\tauT_1}{\tauT_3}
  }
  \and
  \inferrule[\rlabel{$\preccurlyeq$-prop}{subtype-prop}]{~}{
    \subtype{\GammaT}{\PropT}{\TypeT{\iT}}
  }
  \and
  \inferrule[\rlabel{$\preccurlyeq$-type}{subtype-type}]{
    \iT \leq \jT
  }{
    \subtype{\GammaT}{\TypeT{\iT}}{\TypeT{\jT}}
  }
  \and
  \inferrule[\rlabel{$\preccurlyeq$-pi}{subtype-pi}]{
    \defeq{\GammaT}{\sigmaT_1}{\sigmaT_2}{\UT} \\
    \subtype{\GammaT, \annotT{\xT}{\sigmaT_2}}{\tauT_1}{\tauT_2}
  }{
    \subtype{\GammaT}{\funtypeT{\xT}{\sigmaT_1}{\tauT_1}}{\funtypeT{\xT}{\sigmaT_2}{\tauT_2}}
  }
  \end{mathpar}
  \caption{Subtyping rules (\CICE)}
  \label{#1}
  \end{figure}
}

\section{Target Type Theory}

\FigSyntaxCIC{fig:syntax-cic}
The syntax of \CICE is given in \cref{fig:syntax-cic};
differences from \lang include a 0-based index for the recursive argument of fixpoint expressions,
$\tg{case}$ expression motives abstracted over the target's inductive type indices,
and a \emph{propositional equality}\index{propositional equality} type with the reflexivity constructor and $\JT*$ eliminator.
New inductive types are defined using data definitions $\DT$,
whose syntax resembles the informal presentation used in \cref{ch:sized-dep-types}.
Metavariable usage convention is roughly the same as for \lang,
with the addition of $\pT$ for inductive type parameters or proofs of equality
and $\aT$ for inductive type indices.

The well-formedness conditions on inductive data definitions,
such as well-typedness and \emph{strict positivity}\index{strict positivity},
are entirely standard, so I don't detail them here;
see pCuIC~\citep{pCuIC} for instance for a full description.
Inductive definitions in their full generality are not needed,
and nonmutual, nonnested inductives suffice.
Indeed, only five inductive definitions are used for the translation,
for representing sizes, their order, their well-foundedness,
and naturals and well-founded trees.

\FigEquiv{fig:equivalence}